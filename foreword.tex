
\chapter{Foreword}

By David Beazley

As an educator, researcher, and book author, I am delighted to see the
completion of this book.  Python is a fun and extremely easy-to-use
programming language that has steadily gained in popularity over the
last few years.  Developed over ten years ago by Guido van Rossum,
Python's simple syntax and overall feel is largely derived from ABC, a
teaching language that was developed in the 1980's.  However, Python
was also created to solve real problems and it borrows a wide variety
of features from programming languages such as C++, Java, Modula-3,
and Scheme.  Because of this, one of Python's most remarkable features
is its broad appeal to professional software developers, scientists,
researchers, artists, and educators.

Despite Python's appeal to many different communities, you may still
wonder ``why Python?'' or ``why teach programming with Python?''
Answering these questions is no simple task---especially when popular
opinion is on the side of more masochistic alternatives such
as C++ and Java.  However, I think the most direct answer is that
programming in Python is simply a lot of fun and more productive.

When I teach computer science courses, I want to cover important
concepts in addition to making the material interesting and engaging
to students.  Unfortunately, there is a tendency for introductory
programming courses to focus far too much attention on mathematical
abstraction and for students to become frustrated with annoying
problems related to low-level details of syntax, compilation, and the
enforcement of seemingly arcane rules.  Although such abstraction and
formalism is important to professional software engineers and students
who plan to continue their study of computer science, taking such an
approach in an introductory course mostly succeeds in making computer
science boring.  When I teach a course, I don't want to have a room of
uninspired students.  I would much rather see them trying to solve
interesting problems by exploring different ideas, taking
unconventional approaches, breaking the rules, and learning from their
mistakes. In doing so, I don't want to waste half of the semester
trying to sort out obscure syntax problems, unintelligible compiler
error messages, or the several hundred ways that a program might
generate a general protection fault.

One of the reasons why I like Python is that it provides a really nice
balance between the practical and the conceptual.  Since Python is
interpreted, beginners can pick up the language and start doing
neat things almost immediately without getting lost in the problems of
compilation and linking.  Furthermore, Python comes with a large
library of modules that can be used to do all sorts of tasks ranging
from web-programming to graphics.  Having such a practical focus is a
great way to engage students and it allows them to complete
significant projects.  However, Python can also serve as an excellent
foundation for introducing important computer science concepts.  Since
Python fully supports procedures and classes, students can be
gradually introduced to topics such as procedural abstraction, data
structures, and object-oriented programming---all of which are
applicable to later courses on Java or C++.  Python even borrows a
number of features from functional programming languages and can be
used to introduce concepts that would be covered in more detail in
courses on Scheme and Lisp.

In reading Jeffrey's preface, I am struck by his comments that Python
allowed him to see a ``higher level of success and a lower level of
frustration'' and that he was able to ``move faster with better
results.''  Although these comments refer to his introductory course, I
sometimes use Python for these exact same reasons in advanced graduate
level computer science courses at the University of Chicago.  In these
courses, I am constantly faced with the daunting task of covering a
lot of difficult course material in a blistering nine week quarter.
Although it is certainly possible for me to inflict a lot of pain and
suffering by using a language like C++, I have often found this
approach to be counterproductive---especially when the course is about
a topic unrelated to just ``programming.''  I find that using Python
allows me to better focus on the actual topic at hand while allowing
students to complete substantial class projects.

Although Python is still a young and evolving language, I believe that
it has a bright future in education.  This book is an important step in
that direction.

\vspace{0.25in}
\begin{flushleft}
David Beazley \\
University of Chicago \\
Author of the {\em Python Essential Reference} \\
\end{flushleft}

   



