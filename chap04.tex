% LaTeX source for textbook ``How to think like a computer scientist''
% Copyright (c)  2001  Allen B. Downey, Jeffrey Elkner, and Chris Meyers.

% Permission is granted to copy, distribute and/or modify this
% document under the terms of the GNU Free Documentation License,
% Version 1.1  or any later version published by the Free Software
% Foundation; with the Invariant Sections being "Contributor List",
% with no Front-Cover Texts, and with no Back-Cover Texts. A copy of
% the license is included in the section entitled "GNU Free
% Documentation License".

% This distribution includes a file named fdl.tex that contains the text
% of the GNU Free Documentation License.  If it is missing, you can obtain
% it from www.gnu.org or by writing to the Free Software Foundation,
% Inc., 59 Temple Place - Suite 330, Boston, MA 02111-1307, USA.
\chapter{Conditionals and recursion}

\section{The modulus operator}
\index{modulus operator}
\index{operator!modulus}

The {\bf modulus operator} works on integers (and integer expressions)
and yields the remainder when the first operand is divided by the
second.  In Python, the modulus operator is a percent sign ({\tt
\%}).  The syntax is the same as for other operators:

\beforeverb
\begin{verbatim}
>>> quotient = 7 / 3
>>> print quotient
2
>>> remainder = 7 % 3
>>> print remainder
1
\end{verbatim}
\afterverb
%
So 7 divided by 3 is 2 with 1 left over.

The modulus operator turns out to be surprisingly useful.  For
example, you can check whether one number is divisible by another---if
{\tt x \% y} is zero, then {\tt x} is divisible by {\tt y}.

Also, you can extract the right-most digit
or digits from a number.  For example, {\tt x \% 10} yields the
right-most digit of {\tt x} (in base 10).  Similarly {\tt x \% 100}
yields the last two digits.

\adjustpage{1}

\section{Boolean expressions}
\index{boolean expression}
\index{expression!boolean}
\index{logical operator}
\index{operator!logical}

A {\bf boolean expression} is an expression that is either true
or false.  One way to write a boolean expression is to use the
operator {\tt ==}, which compares two values and produces a boolean
value:

\beforeverb
\begin{verbatim}
>>> 5 == 5
True
>>> 5 == 6
False
\end{verbatim}
\afterverb
%
In the first statement, the two operands are equal, so the value of
the expression is {\tt True}; in the second statement, 5 is not equal
to 6, so we get {\tt False}.  {\tt True} and {\tt False} are special
values that are built into Python.

The {\tt ==} operator is one of the {\bf comparison operators}; the
others are:

\beforeverb
\begin{verbatim}
      x != y               # x is not equal to y
      x > y                # x is greater than y
      x < y                # x is less than y
      x >= y               # x is greater than or equal to y
      x <= y               # x is less than or equal to y
\end{verbatim}
\afterverb
%
Although these operations are probably familiar to you, the Python
symbols are different from the mathematical symbols.  A common error
is to use a single equal sign ({\tt =}) instead of a double equal sign
({\tt ==}).  Remember that {\tt =} is an assignment operator and
{\tt ==} is a comparison operator.   Also, there is no such thing as
{\tt =<} or {\tt =>}.


\section {Logical operators}
\index{logical operator}
\index{operator!logical}

There are three {\bf logical operators}: {\tt and}, {\tt
or}, and {\tt not}.  The semantics (meaning) of these operators is
similar to their meaning in English.  For example,
{\tt x > 0 and x < 10} is true only if {\tt x} is greater than 0
{\em and} less than 10.

{\tt n\%2 == 0 or n\%3 == 0} is true if {\em either} of the conditions
is true, that is, if the number is divisible by 2 {\em or} 3.

Finally, the {\tt not} operator negates a boolean
expression, so {\tt not(x > y)} is true if {\tt (x > y)} is false,
that is, if {\tt x} is less than or equal to {\tt y}.

Strictly speaking, the operands of the logical operators should be
boolean expressions, but Python is not very strict.
Any nonzero number is interpreted as ``true.''

\beforeverb
\begin{verbatim}
>>>  x = 5
>>>  x and 1
1
>>>  y = 0
>>>  y and 1
0
\end{verbatim}
\afterverb
%
In general, this sort of thing is not considered good style.  If you
want to compare a value to zero, you should do it explicitly.


\section{Conditional execution}
\label{conditional execution}
\index{conditional branching}
\index{conditional execution}

In order to write useful programs, we almost always need the ability
to check conditions and change the behavior of the program
accordingly.  {\bf Conditional statements} give us this ability.  The
simplest form is the {\tt if} statement:

\beforeverb
\begin{verbatim}
if x > 0:
  print "x is positive"
\end{verbatim}
\afterverb
%
The boolean expression after the {\tt if} statement is
called the {\bf condition}.  If it is true, then the indented
statement gets executed.  If not, nothing happens.

\index{compound statement}
\index{compound statement!header}
\index{compound statement!body}
\index{compound statement!statement block}
\index{statement!compound}

Like other compound statements, the
{\tt if} statement is made up of a header and a block
of statements:

\beforeverb
\begin{verbatim}
HEADER:
  FIRST STATEMENT
  ...
  LAST STATEMENT
\end{verbatim}
\afterverb
%
The header begins on a new line and ends with a colon (:).  The
indented statements that follow are called a {\bf block}.
The first unindented statement marks the end of the block.
A statement block inside a compound statement is called the {\bf body}
of the statement.

\index{block}
\index{statement!block}
\index{body}

There is no limit on the number of statements that can appear in
the body of an if statement, but there has to be at least one.
Occasionally, it is useful to have a body with no statements (usually
as a place keeper for code you haven't written yet).  In that
case, you can use the {\tt pass} statement, which does nothing.

\index{pass statement}
\index{statement!pass}



\section{Alternative execution}
\label{alternative execution}

A second form of the {\tt if} statement is alternative execution,
in which there are two possibilities and the condition determines
which one gets executed.  The syntax looks like this:

\beforeverb
\begin{verbatim}
if x%2 == 0:
  print x, "is even"
else:
  print x, "is odd"
\end{verbatim}
\afterverb
%
If the remainder when {\tt x} is divided by 2 is 0, then we
know that {\tt x} is even, and the program displays a message to that
effect.  If the condition is false, the second set of statements is
executed.  Since the condition must be true or false, exactly one of
the alternatives will be executed.  The alternatives are called
{\bf branches}, because they are branches in the flow of execution.

\index{branch}

As an aside, if you need to check the parity (evenness or
oddness) of numbers often, you might ``wrap'' this code in a
function:

\beforeverb
\begin{verbatim}
def printParity(x):
  if x%2 == 0:
    print x, "is even"
  else:
    print x, "is odd"
\end{verbatim}
\afterverb
%
For any value of {\tt x}, {\tt printParity} displays an
appropriate message.
When you call it, you can provide any integer expression
as an argument.

\beforeverb
\begin{verbatim}
>>> printParity(17)
17 is odd
>>> y = 17
>>> printParity(y+1)
18 is even
\end{verbatim}
\afterverb
%


\section{Chained conditionals}
\index{chained conditional}
\index{conditional!chained}

Sometimes there are more than two possibilities and we need more than
two branches.  One way to express a computation like that is a {\bf
chained conditional}:

\beforeverb
\begin{verbatim}
if x < y:
  print x, "is less than", y
elif x > y:
  print x, "is greater than", y
else:
  print x, "and", y, "are equal"
\end{verbatim}
\afterverb
%
{\tt elif} is an abbreviation of ``else if.''  Again, exactly one
branch will be executed.  There is no limit of the number of {\tt
elif} statements, but the last branch has to be an {\tt else}
statement:

\adjustpage{3}
\beforeverb
\begin{verbatim}
if choice == 'A':
  functionA()
elif choice == 'B':
  functionB()
elif choice == 'C':
  functionC()
else:
  print "Invalid choice."
\end{verbatim}
\afterverb
%
Each condition is checked in order.  If the first is false,
the next is checked, and so on.  If one of them is
true, the corresponding branch executes, and the statement
ends.  Even if more than one condition is true, only the
first true branch executes.  

\begin{quote}
\begin{quote}
{\em As an exercise, wrap these examples in functions
called {\tt compare(x, y)} and {\tt dispatch(choice)}.}
\end{quote}
\end{quote}


\section{Nested conditionals}

One conditional can also be nested within another.  We could have
written the trichotomy example as follows:

\beforeverb
\begin{verbatim}
if x == y:
  print x, "and", y, "are equal"
else:
  if x < y:
    print x, "is less than", y
  else:
    print x, "is greater than", y
\end{verbatim}
\afterverb
%
The outer conditional contains two branches.  The
first branch contains a simple output statement.  The second branch
contains another {\tt if} statement, which has two branches of its
own.  Those two branches are both output statements,
although they could have been conditional statements as well.

Although the indentation of the statements makes the structure
apparent, nested conditionals become difficult to read very
quickly. In general, it is a good idea to avoid them when you can.

Logical operators often provide a way to simplify nested conditional
statements.  For example, we can rewrite the following code using a
single conditional:

\beforeverb
\begin{verbatim}
if 0 < x:
  if x < 10:
    print "x is a positive single digit."
\end{verbatim}
\afterverb
%
The {\tt print} statement is executed only if we make it past both the
conditionals, so we can use the {\tt and} operator:

\beforeverb
\begin{verbatim}
if 0 < x and x < 10:
  print "x is a positive single digit."
\end{verbatim}
\afterverb
%
These kinds of conditions are common, so Python provides an
alternative syntax that is similar to mathematical notation:

\beforeverb
\begin{verbatim}
if 0 < x < 10:
  print "x is a positive single digit."
\end{verbatim}
\afterverb
%
This condition is semantically the same as
the compound boolean expression and the nested conditional.


\section{The {\tt return} statement}
\index{return statement}
\index{statement!return}

The {\tt return} statement allows you to terminate the execution of a
function before you reach the end.  One reason to use it is if you
detect an error condition:

\beforeverb
\begin{verbatim}
import math

def printLogarithm(x):
  if x <= 0:
    print "Positive numbers only, please."
    return

  result = math.log(x)
  print "The log of x is", result
\end{verbatim}
\afterverb
%
The function {\tt printLogarithm} has a
parameter named {\tt x}.  The first thing it does is check whether
{\tt x} is less than or equal to 0, in which case it displays an
error message and then uses {\tt return} to exit the function. The
flow of execution immediately returns to the caller, and the remaining
lines of the function are not executed.

Remember that to use a function from the math
module, you have to import it.


\section{Recursion}
\label{recursion}
\index{recursion}

We mentioned that it is legal for one function to call another, and
you have seen several examples of that.  We neglected to mention that
it is also legal for a function to call itself.  It may not be obvious
why that is a good thing, but it turns out to be one of the most
magical and interesting things a program can do.
For example, look at the following function:

\beforeverb
\begin{verbatim}
def countdown(n):
  if n == 0:
    print "Blastoff!"
  else:
    print n
    countdown(n-1)
\end{verbatim}
\afterverb
%
{\tt countdown} expects the parameter, {\tt n}, to be a positive
integer.  If {\tt n} is 0, it outputs the word, ``Blastoff!''
Otherwise, it outputs {\tt n} and then calls a function named
{\tt countdown}---itself---passing {\tt n-1} as an argument.

What happens if we call this function like this:

\beforeverb
\begin{verbatim}
>>> countdown(3)
\end{verbatim}
\afterverb
%
The execution of {\tt countdown} begins with {\tt n=3}, and since
{\tt n} is not 0, it outputs the value 3, and then calls itself...

\begin{quote}
The execution of {\tt countdown} begins with {\tt n=2}, and since
{\tt n} is not 0, it outputs the value 2, and then calls itself...

\begin{quote}
The execution of {\tt countdown} begins with {\tt n=1}, and since
{\tt n} is not 0, it outputs the value 1, and then calls itself...

\begin{quote}
The execution of {\tt countdown} begins with {\tt n=0}, and since
{\tt n} is 0, it outputs the word, ``Blastoff!'' and then returns.
\end{quote}

The {\tt countdown} that got {\tt n=1} returns.
\end{quote}

The {\tt countdown} that got {\tt n=2} returns.
\end{quote}

The {\tt countdown} that got {\tt n=3} returns.

And then you're back in {\tt \_\_main\_\_} (what a trip).  So, the
total output looks like this:

\beforeverb
\begin{verbatim}
3
2
1
Blastoff!
\end{verbatim}
\afterverb
%
As a second example, look again at the functions {\tt newLine} and
{\tt threeLines}:

\beforeverb
\begin{verbatim}
def newline():
  print

def threeLines():
  newLine()
  newLine()
  newLine()
\end{verbatim}
\afterverb
%
Although these work, they would not be much help if we wanted to output 2
newlines, or 106.  A better alternative would be this:

\beforeverb
\begin{verbatim}
def nLines(n):
  if n > 0:
    print
    nLines(n-1)
\end{verbatim}
\afterverb
%
This program is similar to {\tt countdown}; as long as {\tt n} is
greater than 0, it outputs one newline and then calls itself to
output {\tt n-1} additional newlines.  Thus, the total number of
newlines is {\tt 1 + (n - 1)} which, if you do your algebra right, comes
out to {\tt n}.

The process of a function calling itself is {\bf recursion}, and
such functions are said to be recursive.

\index{recursion}
\index{function!recursive}


\section{Stack diagrams for recursive functions}
\index{stack diagram}
\index{function frame}
\index{frame}

In Section~\ref{stackdiagram}, we used a stack diagram to represent
the state of a program during a function call.  The same kind of
diagram can help interpret a recursive function.

Every time a function gets called, Python creates a new function
frame, which contains the function's local variables and parameters.
For a recursive function, there might be more than one frame on the
stack at the same time.

This figure shows a stack diagram for {\tt countdown} called with
{\tt n = 3}:

\beforefig
\centerline{\psfig{figure=illustrations/stack2.eps}}
\afterfig

As usual, the top of the stack is the frame for {\tt \_\_main\_\_}.
It is empty because we did not create any variables in {\tt
\_\_main\_\_} or pass any arguments to it.

The four {\tt countdown} frames have different values for the
parameter {\tt n}.  The bottom of the stack, where {\tt n=0}, is
called the {\bf base case}.  It does not make a recursive call, so
there are no more frames.

\adjustpage{2}
\begin{quote}
{\em As an exercise, draw a stack diagram for {\tt nLines} called with
{\tt n=4}.}
\end{quote}

\index{base case}
\index{recursion!base case}


\section{Infinite recursion}
\index{infinite recursion}
\index{recursion!infinite}
\index{runtime error}
\index{error!runtime}
\index{traceback}

If a recursion never reaches a base case, it goes on making
recursive calls forever, and the program never terminates.  This is
known as {\bf infinite recursion}, and it is generally not considered
a good idea.  Here is a minimal program with an infinite recursion:

\beforeverb
\begin{verbatim}
def recurse():
  recurse()
\end{verbatim}
\afterverb
%
In most programming environments, a program with infinite recursion
does not really run forever.  Python reports an error
message when the maximum recursion depth is reached:

\beforeverb
\begin{verbatim}
  File "<stdin>", line 2, in recurse
  (98 repetitions omitted)
  File "<stdin>", line 2, in recurse
RuntimeError: Maximum recursion depth exceeded
\end{verbatim}
\afterverb
%
This traceback is a little bigger than the one we saw in the
previous chapter.  When the error occurs, there are 100
{\tt recurse} frames on the stack!

\begin{quote}
{\em As an exercise, write a function with infinite recursion and run
it in the Python interpreter.}
\end{quote}


\section{Keyboard input}

The programs we have written so far are a bit rude in the sense that
they accept no input from the user.  They just do the same thing every
time.

Python provides built-in functions that get input from the keyboard.
The simplest is called {\tt raw\_input}.  When this function is
called, the program stops and waits for the user to type something.
When the user presses Return or the Enter key, the program resumes and
{\tt raw\_input} returns what the user typed as a {\bf string}:

\beforeverb
\begin{verbatim}
>>> input = raw_input ()
What are you waiting for?
>>> print input
What are you waiting for?
\end{verbatim}
\afterverb
%
Before calling {\tt raw\_input}, it is a good idea to print a message
telling the user what to input.  This message is called a {\bf
prompt}.  We can supply a prompt as an argument to {\tt raw\_input}:

\index{prompt}

\beforeverb
\begin{verbatim}
>>> name = raw_input ("What...is your name? ")
What...is your name? Arthur, King of the Britons!
>>> print name
Arthur, King of the Britons!
\end{verbatim}
\afterverb
%
If we expect the response to be an integer, we can use the
{\tt input} function:

\beforeverb
\begin{verbatim}
prompt = "What...is the airspeed velocity of an unladen swallow?\n"
speed = input(prompt)
\end{verbatim}
\afterverb
%
The sequence \verb+\n+ at the end of the string represents a newline,
so the user's input appears below the prompt.

If the user types a string of digits, it is converted to an
integer and assigned to {\tt speed}.  Unfortunately, if the user
types a character that is not a digit, the program crashes:

\beforeverb
\begin{verbatim}
>>> speed = input (prompt)
What...is the airspeed velocity of an unladen swallow?
What do you mean, an African or a European swallow?
SyntaxError: invalid syntax
\end{verbatim}
\afterverb
%
To avoid this kind of error, it is generally a good idea to
use {\tt raw\_input} to get a string and then use conversion
functions to convert to other types.


\section{Glossary}

\begin{description}

\item[modulus operator:]  An operator, denoted with a percent sign
({\tt \%}), that works on integers and yields the remainder when one
number is divided by another.

\item[boolean expression:]  An expression that is either true or false.

\item[comparison operator:] One of the operators that compares two
values: {\tt ==}, {\tt !=}, {\tt >}, {\tt <}, {\tt >=}, and {\tt <=}.

\item[logical operator:] One of the operators that combines boolean
expressions: {\tt and}, {\tt or}, and {\tt not}.

\item[conditional statement:]  A statement that controls the flow of
execution depending on some condition.

\item[condition:] The boolean expression in a conditional statement
that determines which branch is executed.

\item[compound statement:]  A statement that consists of a header
and a body.  The header ends with a colon (:).  The body is indented
relative to the header.

\item[block:] A group of consecutive statements with the same
indentation.

\item[body:] The block in a compound statement that follows the
header.

\item[nesting:]  One program structure within another, such as a
conditional statement inside a branch of another conditional
statement.

\item[recursion:]  The process of calling the function that is
currently executing.

\item[base case:]  A branch of the conditional statement in a
recursive function that does not result in a recursive call.

\item[infinite recursion:]  A function that calls itself recursively
without ever reaching the base case.  Eventually, an infinite recursion
causes a runtime error.

\item[prompt:]  A visual cue that tells the user to input data.

\index{modulus operator}
\index{boolean expression}
\index{expression!boolean}
\index{conditional statement}
\index{statement!conditional}
\index{condition}
\index{compound statement}
\index{branch}
\index{body}
\index{block}
\index{nesting}
\index{recursion}
\index{base case}
\index{infinite recursion}
\index{prompt}

\end{description}
