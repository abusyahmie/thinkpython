% LaTeX source for textbook ``How to think like a computer scientist''
% Copyright (c)  2001  Allen B. Downey, Jeffrey Elkner, and Chris Meyers.

% Permission is granted to copy, distribute and/or modify this document under
% the terms of the GNU Free Documentation License, Version 1.1  or any later
% version published by the Free Software Foundation; with the Invariant
% Sections being "Contributor List", "Forward", and "Preface", with no
% Front-Cover Texts, and with no Back-Cover Texts. A copy of the license is
% included in the section entitled "GNU Free Documentation License".

% This distribution includes a file named fdl.tex that contains the text of the
% GNU Free Documentation License.  If it is missing, you can obtain it from
% www.gnu.org or by writing to the Free Software Foundation, Inc.,
% 59 Temple Place - Suite 330, Boston, MA 02111-1307, USA.
%
\chapter{Preface}

By Jeff Elkner

This book owes its existence to the collaboration made possible by the
Internet and the free software movement.  Its three authors---a
college professor, a high school teacher, and a professional
programmer---have yet to meet face to face, but we have been able to
work closely together and have been aided by many wonderful folks who
have donated their time and energy to helping make this book better.

We think this book is a testament to the benefits and future
possibilities of this kind of collaboration, the framework for which
has been put in place by Richard Stallman and the Free Software
Foundation.


\section*{How and why I came to use Python}

In 1999, the College Board's Advanced Placement (AP) Computer Science
exam was given in C++ for the first time.  As in many high schools
throughout the country, the decision to change languages had a direct
impact on the computer science curriculum at Yorktown High School in
Arlington, Virginia, where I teach.  Up to this point, Pascal was the
language of instruction in both our first-year and AP courses.  In
keeping with past practice of giving students two years of exposure to
the same language, we made the decision to switch to C++ in the
first-year course for the 1997-98 school year so that we would be in
step with the College Board's change for the AP course the following
year.

Two years later, I was convinced that C++ was a poor choice to use for
introducing students to computer science. While it is certainly a very
powerful programming language, it is also an extremely difficult
language to learn and teach.  I found myself constantly fighting with
C++'s difficult syntax and multiple ways of doing things, and I was
losing many students unnecessarily as a result. Convinced there had to
be a better language choice for our first-year class, I went looking
for an alternative to C++.

I needed a language that would run on the machines in our Linux lab as
well as on the Windows and Macintosh platforms most students have at
home.  I wanted it to be free and available electronically, so that
students could use it at home regardless of their income.  I wanted a
language that was used by professional programmers, and one that had
an active developer community around it.  It had to support both
procedural and object-oriented programming.  And most importantly, it
had to be easy to learn and teach.  When I investigated the choices
with these goals in mind, Python stood out as the best candidate for
the job.

I asked one of Yorktown's talented students, Matt Ahrens, 
to give Python a try.  In two months he not only learned the language
but wrote an application called pyTicket that enabled our staff to
report technology problems via the Web.  I knew that Matt could not
have finished an application of that scale in so short a time in C++,
and this accomplishment, combined with Matt's positive assessment of
Python, suggested that Python was the solution I was looking for.


\section*{Finding a textbook}

Having decided to use Python in both of my introductory computer
science classes the following year, the most pressing problem was the
lack of an available textbook.

Free content came to the rescue.  Earlier in the year, Richard
Stallman had introduced me to Allen Downey.  Both of us had written to
Richard expressing an interest in developing free educational
content.  Allen had already written a first-year computer science
textbook, {\em How to Think Like a Computer Scientist}.  When I read
this book, I knew immediately that I wanted to use it in my class.  It
was the clearest and most helpful computer science text I had seen.
It emphasized the processes of thought involved in programming rather
than the features of a particular language.  Reading it immediately
made me a better teacher.

{\em How to Think Like a Computer Scientist} was not just an excellent
book, but it had been released under a GNU public license, which meant
it could be used freely and modified to meet the needs of its user.
Once I decided to use Python, it occurred to me that I could translate
Allen's original Java version of the book into the new language.
While I would not have been able to write a textbook on my own, having
Allen's book to work from made it possible for me to do so, at the
same time demonstrating that the cooperative development model used so
well in software could also work for educational content.

Working on this book for the last two years has been rewarding for
both my students and me, and my students played a big part in the
process. Since I could make instant changes whenever someone found a
spelling error or difficult passage, I encouraged them to look for
mistakes in the book by giving them a bonus point each time they made
a suggestion that resulted in a change in the text. This had the
double benefit of encouraging them to read the text more carefully and
of getting the text thoroughly reviewed by its most important critics,
students using it to learn computer science.

% I fixed this paragraph too:

For the second half of the book on object-oriented programming, I knew
that someone with more real programming experience than I had would be
needed to do it right.  The book sat in an unfinished state for the better
part of a year until the free software community once again provided the
needed means for its completion.

I received an email from Chris Meyers expressing interest in the book.
Chris is a professional programmer who started teaching a programming
course last year using Python at Lane Community College in Eugene,
Oregon.  The prospect of teaching the course had led Chris to the
book, and he started helping out with it immediately.  By the end of
the school year he had created a companion project on our website at
{\tt http://www.ibiblio.org/obp} called {\em Python for Fun} and was
working with some of my most advanced students as a master teacher,
guiding them beyond where I could take them.


\section*{Introducing programming with Python}

The process of translating and using {\em How to Think Like a Computer
Scientist} for the past two years has confirmed Python's suitability
for teaching beginning students.  Python greatly simplifies
programming examples and makes important programming ideas easier to
teach.

The first example from the text illustrates this point.
It is the traditional ``hello, world'' program, which in the C++
version of the book looks like this:

\begin{verbatim}
   #include <iostream.h>

   void main()
   {
     cout << "Hello, world." << endl;
   }
\end{verbatim}

in the Python version it becomes:

\begin{verbatim}
   print "Hello, World!"
\end{verbatim}

Even though this is a trivial example, the advantages of Python stand
out.  Yorktown's Computer Science I course has no prerequisites, so
many of the students seeing this example are looking at their first
program.  Some of them are undoubtedly a little nervous, having heard
that computer programming is difficult to learn. The C++ version has
always forced me to choose between two unsatisfying options: either to
explain {\tt \#include}, {\tt void main()}, \{, and \},
and risk confusing or intimidating some of the students right at the
start, or to tell them, ``Just don't worry about all of that stuff
now; we will talk about it later,'' and risk the same thing.  The
educational objectives at this point in the course are to introduce
students to the idea of a programming language and to get them to
write their first program, thereby introducing them to the programming
environment.  The Python program has exactly what is needed to do these
things, and nothing more.

Comparing the explanatory text of the program in each version of
the book further illustrates what this means to the beginning student.
There are thirteen paragraphs of explanation of ``Hello, world!'' in the
C++ version; in the Python version, there are only two. More importantly,
the missing eleven paragraphs do not deal with the ``big ideas'' in
computer programming but with the minutia of C++ syntax.  I found this
same thing happening throughout the book.  Whole paragraphs simply
disappear from the Python version of the text because Python's much
clearer syntax renders them unnecessary.

Using a very high-level language like Python allows a teacher to
postpone talking about low-level details of the machine until students
have the background that they need to better make sense of the
details.  It thus creates the ability to put ``first things first''
pedagogically.  One of the best examples of this is the way in which
Python handles variables. In C++ a variable is a name for a place that
holds a thing.  Variables have to be declared with types at least in
part because the size of the place to which they refer needs to be
predetermined. Thus, the idea of a variable is bound up with the
hardware of the machine. The powerful and fundamental concept of a
variable is already difficult enough for beginning students (in both
computer science and algebra).  Bytes and addresses do not help the
matter. In Python a variable is a name that refers to a thing. This is
a far more intuitive concept for beginning students and is much closer
to the meaning of ``variable'' that they learned in their math
courses. I had much less difficulty teaching variables this year than
I did in the past, and I spent less time helping students with
problems using them.

Another example of how Python aids in the teaching and learning of
programming is in its syntax for functions.  My students have always
had a great deal of difficulty understanding functions. The main
problem centers around the difference between a function definition
and a function call, and the related distinction between a parameter
and an argument. Python comes to the rescue with syntax that is
nothing short of beautiful. Function definitions begin with the
keyword {\tt def}, so I simply tell my students, ``When you define a
function, begin with {\tt def}, followed by the name of the function
that you are defining; when you call a function, simply call (type)
out its name.'' Parameters go with definitions; arguments go with
calls. There are no return types, parameter types, or reference and
value parameters to get in the way, so I am now able to teach
functions in less than half the time that it previously took me, with
better comprehension.

Using Python has improved the effectiveness of our computer science
program for all students.  I see a higher general level of success and
a lower level of frustration than I experienced during the two years I
taught C++.  I move faster with better results.  More students leave
the course with the ability to create meaningful programs and with the
positive attitude toward the experience of programming that this
engenders.


\section*{Building a community}

% Jeff: the following sentence sounds funny to me.  You mention
% every continent and then Korea specifically.  How about just
% the every continent part and take out the Korea?

% done.

I have received email from all over the globe from people using
this book to learn or to teach programming.  A user community has
begun to emerge, and many people have been contributing to the
project by sending in materials for the companion website at
{\tt http://www.thinkpython.com}.

With the publication of the book in print form, I expect the growth in
the user community to continue and accelerate.  The emergence of this
user community and the possibility it suggests for similar
collaboration among educators have been the most exciting parts of
working on this project for me.  By working together, we can increase
the quality of materials available for our use and save valuable time.
I invite you to join our community and look forward to hearing from
you.  Please write to the authors at {\tt feedback@thinkpython.com}.

\vspace{0.25in}
\begin{flushleft}
Jeffrey Elkner\\
Yorktown High School\\
Arlington, Virginia\\
\end{flushleft}
