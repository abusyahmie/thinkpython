% LaTeX source for textbook ``How to think like a computer scientist''
% Copyright (c)  2001,2002  Allen B. Downey.

% Permission is granted to copy, distribute and/or modify this
% document under the terms of the GNU Free Documentation License,
% Version 1.1  or any later version published by the Free Software
% Foundation; with the Invariant Sections being "Contributor List",
% with no Front-Cover Texts, and with no Back-Cover Texts. A copy of
% the license is included in the section entitled "GNU Free
% Documentation License".

% This distribution includes a file named fdl.tex that contains the text
% of the GNU Free Documentation License.  If it is missing, you can obtain
% it from www.gnu.org or by writing to the Free Software Foundation,
% Inc., 59 Temple Place - Suite 330, Boston, MA 02111-1307, USA.
%
\documentclass[10pt]{book}
\usepackage[paperwidth=6.75in,paperheight=9.25in,
  width=4.9in,height=7.2in,
  hmarginratio=3:2,vmarginratio=3:2]{geometry}
\usepackage{fancyhdr}
\usepackage{psfig}
\usepackage{makeidx}
\pssilent

\sloppy
% dimensions for b5paper
%\setlength{\topmargin}{-0.375in}
%\setlength{\oddsidemargin}{0.0in}
%\setlength{\evensidemargin}{0.0in}

% dimensions for 8.5 x 11
%\setlength{\topmargin}{0.625in}
%\setlength{\oddsidemargin}{0.875in}
%\setlength{\evensidemargin}{0.875in}

\setlength{\headsep}{3ex}
\setlength{\textheight}{7.2in}

\setlength{\parindent}{0.0in}
\setlength{\parskip}{1.7ex plus 0.5ex minus 0.5ex}
\renewcommand{\baselinestretch}{1.02}

% see LaTeX Companion page 62
\setlength{\topsep}{-0.0\parskip}
\setlength{\partopsep}{-0.5\parskip}
\setlength{\itemindent}{0.0in}
\setlength{\listparindent}{0.0in}

% see LaTeX Companion page 26
% these are copied from /usr/local/teTeX/share/texmf/tex/latex/base/book.cls
% all I changed is afterskip

\makeatletter
\renewcommand{\section}{\@startsection 
    {section} {1} {0mm}%
    {-3.5ex \@plus -1ex \@minus -.2ex}%
    {0.7ex \@plus.2ex}%
    {\normalfont\Large\bfseries}}
\renewcommand\subsection{\@startsection {subsection}{2}{0mm}%
    {-3.25ex\@plus -1ex \@minus -.2ex}%
    {0.3ex \@plus .2ex}%
    {\normalfont\large\bfseries}}
\renewcommand\subsubsection{\@startsection {subsubsection}{3}{0mm}%
    {-3.25ex\@plus -1ex \@minus -.2ex}%
    {0.3ex \@plus .2ex}%
    {\normalfont\normalsize\bfseries}}
\makeatother

\newcommand{\beforefig}{\vspace{1.3\parskip}}
\newcommand{\afterfig}{\vspace{-0.2\parskip}}

\newcommand{\beforeverb}{\vspace{0.6\parskip}}
\newcommand{\afterverb}{\vspace{0.6\parskip}}

\newcommand{\adjustpage}[1]{\enlargethispage{#1\baselineskip}}

% Note: the following command seems to cause problems for Acroreader
% on Windows, so for now I am overriding it.
\newcommand{\clearemptydoublepage}{\newpage{\pagestyle{empty}\cleardoublepage}}
\renewcommand{\clearemptydoublepage}{\cleardoublepage}

\newcommand{\blankpage}{\pagestyle{empty}\vspace*{1in}\newpage}
\renewcommand{\blankpage}{\vspace*{1in}\newpage}

\pagestyle{fancyplain}

\renewcommand{\chaptermark}[1]{\markboth{#1}{}}
\renewcommand{\sectionmark}[1]{\markright{\thesection\ #1}{}}

\lhead[\fancyplain{}{\bfseries\thepage}]%
      {\fancyplain{}{\bfseries\rightmark}}
\rhead[\fancyplain{}{\bfseries\leftmark}]%
      {\fancyplain{}{\bfseries\thepage}}
\cfoot{}

% turn off the rule under the header
%\setlength{\headrulewidth}{0pt}

% the following is a brute-force way to prevent the headers
% from getting transformed into all-caps
\renewcommand\MakeUppercase{}

\makeindex

\begin{document}

\frontmatter

%-half title--------------------------------------------------
\thispagestyle{empty}

\begin{flushright}
\vspace*{2.5in}

{\huge How to Think Like a Computer Scientist}

\vspace{1in}

{\LARGE Learning with Python}

\vfill

\end{flushright}

%--verso------------------------------------------------------

\clearemptydoublepage
%\pagebreak
%\thispagestyle{empty}
%\vspace*{6in}

%--title page--------------------------------------------------
\pagebreak
\thispagestyle{empty}

\begin{flushright}
\vspace*{2.5in}

{\huge How to Think Like a Computer Scientist}

\vspace{0.25in}

{\LARGE Learning with Python}

\vspace{1in}

{\Large
Allen Downey\\
Jeffrey Elkner\\
Chris Meyers\\
}


\vspace{1in}

{\Large Green Tea Press}

{\small Wellesley, Massachusetts}

%\psfig{figure=illustrations/logo1.eps,width=1in}
\vfill

\end{flushright}


%--copyright--------------------------------------------------
\pagebreak
\thispagestyle{empty}

{\small
Copyright \copyright ~2002 Allen Downey, Jeffrey Elkner, and Chris Meyers.

Edited by Shannon Turlington and Lisa Cutler.  Cover design by Rebecca Gimenez.

Printing history:

\begin{description}

\item[April 2002:] First edition.

\item[August 2008:] Second printing.

\end{description}

\vspace{0.2in}

\begin{flushleft}
Green Tea Press       \\
1 Grove St.           \\
P.O. Box 812901       \\
Wellesley, MA 02482   \\
\end{flushleft}

Permission is granted to copy, distribute, and/or modify this document
under the terms of the GNU Free Documentation License, Version 1.1 or
any later version published by the Free Software Foundation; with the
Invariant Sections being ``Foreword,'' ``Preface,'' and ``Contributor
List,'' with no Front-Cover Texts, and with no Back-Cover Texts. A
copy of the license is included in the appendix entitled ``GNU Free
Documentation License.''

The GNU Free Documentation License is available from {\tt www.gnu.org}
or by writing to the Free Software Foundation, Inc., 59 Temple Place,
Suite 330, Boston, MA 02111-1307, USA.

The original form of this book is \LaTeX\ source code.  Compiling this
\LaTeX\ source has the effect of generating a device-independent
representation of a textbook, which can be converted to other formats
and printed.

The \LaTeX\ source for this book is available from
{\tt http://www.thinkpython.com}

\vspace{0.2in}

Publisher's Cataloging-in-Publication (provided by Quality Books, Inc.)

\begin{tabbing}
Downey, Allen\\
\qquad  How to think like a computer scientist : learning \\
\quad  with Python / Allen Downey, Jeffrey Elkner, Chris       \\
\quad  Meyers. -- 1st ed. \\
\qquad  p. cm.            \\
\qquad Includes index.    \\
\qquad ISBN 0-9716775-0-6 \\
\qquad LCCN 2002100618    \\
\\
\qquad  1. Python (Computer program language)  I. Elkner, \\
\quad  Jeffrey.  II. Meyers, Chris.  III. Title           \\
\\
\quad QA76.73.P98D69 2002 \qquad \qquad \quad 005.13'3          \\
\qquad \qquad \qquad \qquad \qquad \qquad \qquad QBI02-200031    \\
\end{tabbing}

} % end small

%-----------------------------------------------------------------


\chapter{Foreword}

By David Beazley

As an educator, researcher, and book author, I am delighted to see the
completion of this book.  Python is a fun and extremely easy-to-use
programming language that has steadily gained in popularity over the
last few years.  Developed over ten years ago by Guido van Rossum,
Python's simple syntax and overall feel is largely derived from ABC, a
teaching language that was developed in the 1980's.  However, Python
was also created to solve real problems and it borrows a wide variety
of features from programming languages such as C++, Java, Modula-3,
and Scheme.  Because of this, one of Python's most remarkable features
is its broad appeal to professional software developers, scientists,
researchers, artists, and educators.

Despite Python's appeal to many different communities, you may still
wonder ``why Python?'' or ``why teach programming with Python?''
Answering these questions is no simple task---especially when popular
opinion is on the side of more masochistic alternatives such
as C++ and Java.  However, I think the most direct answer is that
programming in Python is simply a lot of fun and more productive.

When I teach computer science courses, I want to cover important
concepts in addition to making the material interesting and engaging
to students.  Unfortunately, there is a tendency for introductory
programming courses to focus far too much attention on mathematical
abstraction and for students to become frustrated with annoying
problems related to low-level details of syntax, compilation, and the
enforcement of seemingly arcane rules.  Although such abstraction and
formalism is important to professional software engineers and students
who plan to continue their study of computer science, taking such an
approach in an introductory course mostly succeeds in making computer
science boring.  When I teach a course, I don't want to have a room of
uninspired students.  I would much rather see them trying to solve
interesting problems by exploring different ideas, taking
unconventional approaches, breaking the rules, and learning from their
mistakes. In doing so, I don't want to waste half of the semester
trying to sort out obscure syntax problems, unintelligible compiler
error messages, or the several hundred ways that a program might
generate a general protection fault.

One of the reasons why I like Python is that it provides a really nice
balance between the practical and the conceptual.  Since Python is
interpreted, beginners can pick up the language and start doing
neat things almost immediately without getting lost in the problems of
compilation and linking.  Furthermore, Python comes with a large
library of modules that can be used to do all sorts of tasks ranging
from web-programming to graphics.  Having such a practical focus is a
great way to engage students and it allows them to complete
significant projects.  However, Python can also serve as an excellent
foundation for introducing important computer science concepts.  Since
Python fully supports procedures and classes, students can be
gradually introduced to topics such as procedural abstraction, data
structures, and object-oriented programming---all of which are
applicable to later courses on Java or C++.  Python even borrows a
number of features from functional programming languages and can be
used to introduce concepts that would be covered in more detail in
courses on Scheme and Lisp.

In reading Jeffrey's preface, I am struck by his comments that Python
allowed him to see a ``higher level of success and a lower level of
frustration'' and that he was able to ``move faster with better
results.''  Although these comments refer to his introductory course, I
sometimes use Python for these exact same reasons in advanced graduate
level computer science courses at the University of Chicago.  In these
courses, I am constantly faced with the daunting task of covering a
lot of difficult course material in a blistering nine week quarter.
Although it is certainly possible for me to inflict a lot of pain and
suffering by using a language like C++, I have often found this
approach to be counterproductive---especially when the course is about
a topic unrelated to just ``programming.''  I find that using Python
allows me to better focus on the actual topic at hand while allowing
students to complete substantial class projects.

Although Python is still a young and evolving language, I believe that
it has a bright future in education.  This book is an important step in
that direction.

\vspace{0.25in}
\begin{flushleft}
David Beazley \\
University of Chicago \\
Author of the {\em Python Essential Reference} \\
\end{flushleft}

   




\clearemptydoublepage

% LaTeX source for textbook ``How to think like a computer scientist''
% Copyright (c)  2001  Allen B. Downey, Jeffrey Elkner, and Chris Meyers.

% Permission is granted to copy, distribute and/or modify this document under
% the terms of the GNU Free Documentation License, Version 1.1  or any later
% version published by the Free Software Foundation; with the Invariant
% Sections being "Contributor List", "Forward", and "Preface", with no
% Front-Cover Texts, and with no Back-Cover Texts. A copy of the license is
% included in the section entitled "GNU Free Documentation License".

% This distribution includes a file named fdl.tex that contains the text of the
% GNU Free Documentation License.  If it is missing, you can obtain it from
% www.gnu.org or by writing to the Free Software Foundation, Inc.,
% 59 Temple Place - Suite 330, Boston, MA 02111-1307, USA.
%
\chapter{Preface}

By Jeff Elkner

This book owes its existence to the collaboration made possible by the
Internet and the free software movement.  Its three authors---a
college professor, a high school teacher, and a professional
programmer---have yet to meet face to face, but we have been able to
work closely together and have been aided by many wonderful folks who
have donated their time and energy to helping make this book better.

We think this book is a testament to the benefits and future
possibilities of this kind of collaboration, the framework for which
has been put in place by Richard Stallman and the Free Software
Foundation.


\section*{How and why I came to use Python}

In 1999, the College Board's Advanced Placement (AP) Computer Science
exam was given in C++ for the first time.  As in many high schools
throughout the country, the decision to change languages had a direct
impact on the computer science curriculum at Yorktown High School in
Arlington, Virginia, where I teach.  Up to this point, Pascal was the
language of instruction in both our first-year and AP courses.  In
keeping with past practice of giving students two years of exposure to
the same language, we made the decision to switch to C++ in the
first-year course for the 1997-98 school year so that we would be in
step with the College Board's change for the AP course the following
year.

Two years later, I was convinced that C++ was a poor choice to use for
introducing students to computer science. While it is certainly a very
powerful programming language, it is also an extremely difficult
language to learn and teach.  I found myself constantly fighting with
C++'s difficult syntax and multiple ways of doing things, and I was
losing many students unnecessarily as a result. Convinced there had to
be a better language choice for our first-year class, I went looking
for an alternative to C++.

I needed a language that would run on the machines in our Linux lab as
well as on the Windows and Macintosh platforms most students have at
home.  I wanted it to be free and available electronically, so that
students could use it at home regardless of their income.  I wanted a
language that was used by professional programmers, and one that had
an active developer community around it.  It had to support both
procedural and object-oriented programming.  And most importantly, it
had to be easy to learn and teach.  When I investigated the choices
with these goals in mind, Python stood out as the best candidate for
the job.

I asked one of Yorktown's talented students, Matt Ahrens, 
to give Python a try.  In two months he not only learned the language
but wrote an application called pyTicket that enabled our staff to
report technology problems via the Web.  I knew that Matt could not
have finished an application of that scale in so short a time in C++,
and this accomplishment, combined with Matt's positive assessment of
Python, suggested that Python was the solution I was looking for.


\section*{Finding a textbook}

Having decided to use Python in both of my introductory computer
science classes the following year, the most pressing problem was the
lack of an available textbook.

Free content came to the rescue.  Earlier in the year, Richard
Stallman had introduced me to Allen Downey.  Both of us had written to
Richard expressing an interest in developing free educational
content.  Allen had already written a first-year computer science
textbook, {\em How to Think Like a Computer Scientist}.  When I read
this book, I knew immediately that I wanted to use it in my class.  It
was the clearest and most helpful computer science text I had seen.
It emphasized the processes of thought involved in programming rather
than the features of a particular language.  Reading it immediately
made me a better teacher.

{\em How to Think Like a Computer Scientist} was not just an excellent
book, but it had been released under a GNU public license, which meant
it could be used freely and modified to meet the needs of its user.
Once I decided to use Python, it occurred to me that I could translate
Allen's original Java version of the book into the new language.
While I would not have been able to write a textbook on my own, having
Allen's book to work from made it possible for me to do so, at the
same time demonstrating that the cooperative development model used so
well in software could also work for educational content.

Working on this book for the last two years has been rewarding for
both my students and me, and my students played a big part in the
process. Since I could make instant changes whenever someone found a
spelling error or difficult passage, I encouraged them to look for
mistakes in the book by giving them a bonus point each time they made
a suggestion that resulted in a change in the text. This had the
double benefit of encouraging them to read the text more carefully and
of getting the text thoroughly reviewed by its most important critics,
students using it to learn computer science.

% I fixed this paragraph too:

For the second half of the book on object-oriented programming, I knew
that someone with more real programming experience than I had would be
needed to do it right.  The book sat in an unfinished state for the better
part of a year until the free software community once again provided the
needed means for its completion.

I received an email from Chris Meyers expressing interest in the book.
Chris is a professional programmer who started teaching a programming
course last year using Python at Lane Community College in Eugene,
Oregon.  The prospect of teaching the course had led Chris to the
book, and he started helping out with it immediately.  By the end of
the school year he had created a companion project on our website at
{\tt http://www.ibiblio.org/obp} called {\em Python for Fun} and was
working with some of my most advanced students as a master teacher,
guiding them beyond where I could take them.


\section*{Introducing programming with Python}

The process of translating and using {\em How to Think Like a Computer
Scientist} for the past two years has confirmed Python's suitability
for teaching beginning students.  Python greatly simplifies
programming examples and makes important programming ideas easier to
teach.

The first example from the text illustrates this point.
It is the traditional ``hello, world'' program, which in the C++
version of the book looks like this:

\begin{verbatim}
   #include <iostream.h>

   void main()
   {
     cout << "Hello, world." << endl;
   }
\end{verbatim}

in the Python version it becomes:

\begin{verbatim}
   print "Hello, World!"
\end{verbatim}

Even though this is a trivial example, the advantages of Python stand
out.  Yorktown's Computer Science I course has no prerequisites, so
many of the students seeing this example are looking at their first
program.  Some of them are undoubtedly a little nervous, having heard
that computer programming is difficult to learn. The C++ version has
always forced me to choose between two unsatisfying options: either to
explain {\tt \#include}, {\tt void main()}, \{, and \},
and risk confusing or intimidating some of the students right at the
start, or to tell them, ``Just don't worry about all of that stuff
now; we will talk about it later,'' and risk the same thing.  The
educational objectives at this point in the course are to introduce
students to the idea of a programming language and to get them to
write their first program, thereby introducing them to the programming
environment.  The Python program has exactly what is needed to do these
things, and nothing more.

Comparing the explanatory text of the program in each version of
the book further illustrates what this means to the beginning student.
There are thirteen paragraphs of explanation of ``Hello, world!'' in the
C++ version; in the Python version, there are only two. More importantly,
the missing eleven paragraphs do not deal with the ``big ideas'' in
computer programming but with the minutia of C++ syntax.  I found this
same thing happening throughout the book.  Whole paragraphs simply
disappear from the Python version of the text because Python's much
clearer syntax renders them unnecessary.

Using a very high-level language like Python allows a teacher to
postpone talking about low-level details of the machine until students
have the background that they need to better make sense of the
details.  It thus creates the ability to put ``first things first''
pedagogically.  One of the best examples of this is the way in which
Python handles variables. In C++ a variable is a name for a place that
holds a thing.  Variables have to be declared with types at least in
part because the size of the place to which they refer needs to be
predetermined. Thus, the idea of a variable is bound up with the
hardware of the machine. The powerful and fundamental concept of a
variable is already difficult enough for beginning students (in both
computer science and algebra).  Bytes and addresses do not help the
matter. In Python a variable is a name that refers to a thing. This is
a far more intuitive concept for beginning students and is much closer
to the meaning of ``variable'' that they learned in their math
courses. I had much less difficulty teaching variables this year than
I did in the past, and I spent less time helping students with
problems using them.

Another example of how Python aids in the teaching and learning of
programming is in its syntax for functions.  My students have always
had a great deal of difficulty understanding functions. The main
problem centers around the difference between a function definition
and a function call, and the related distinction between a parameter
and an argument. Python comes to the rescue with syntax that is
nothing short of beautiful. Function definitions begin with the
keyword {\tt def}, so I simply tell my students, ``When you define a
function, begin with {\tt def}, followed by the name of the function
that you are defining; when you call a function, simply call (type)
out its name.'' Parameters go with definitions; arguments go with
calls. There are no return types, parameter types, or reference and
value parameters to get in the way, so I am now able to teach
functions in less than half the time that it previously took me, with
better comprehension.

Using Python has improved the effectiveness of our computer science
program for all students.  I see a higher general level of success and
a lower level of frustration than I experienced during the two years I
taught C++.  I move faster with better results.  More students leave
the course with the ability to create meaningful programs and with the
positive attitude toward the experience of programming that this
engenders.


\section*{Building a community}

% Jeff: the following sentence sounds funny to me.  You mention
% every continent and then Korea specifically.  How about just
% the every continent part and take out the Korea?

% done.

I have received email from all over the globe from people using
this book to learn or to teach programming.  A user community has
begun to emerge, and many people have been contributing to the
project by sending in materials for the companion website at
{\tt http://www.thinkpython.com}.

With the publication of the book in print form, I expect the growth in
the user community to continue and accelerate.  The emergence of this
user community and the possibility it suggests for similar
collaboration among educators have been the most exciting parts of
working on this project for me.  By working together, we can increase
the quality of materials available for our use and save valuable time.
I invite you to join our community and look forward to hearing from
you.  Please write to the authors at {\tt feedback@thinkpython.com}.

\vspace{0.25in}
\begin{flushleft}
Jeffrey Elkner\\
Yorktown High School\\
Arlington, Virginia\\
\end{flushleft}

\clearemptydoublepage

% LaTeX source for textbook ``How to think like a computer scientist''
% Copyright (c)  2001  Allen B. Downey, Jeffrey Elkner, and John Dewey.

% Permission is granted to copy, distribute and/or modify this
% document under the terms of the GNU Free Documentation License,
% Version 1.1  or any later version published by the Free Software
% Foundation; with the Invariant Sections being "Contributor List",
% with no Front-Cover Texts, and with no Back-Cover Texts. A copy of
% the license is included in the section entitled "GNU Free
% Documentation License".

% This distribution includes a file named fdl.tex that contains the text
% of the GNU Free Documentation License.  If it is missing, you can obtain
% it from www.gnu.org or by writing to the Free Software Foundation,
% Inc., 59 Temple Place - Suite 330, Boston, MA 02111-1307, USA.

\chapter{Contributor List}

To paraphrase the philosophy of the Free Software Foundation, this
book is free like free speech, but not necessarily free like free
pizza.  It came about because of a collaboration that would not have
been possible without the GNU Free Documentation License.  So we
thank the Free Software Foundation for developing this license
and, of course, making it available to us.

We also thank the more than 100 sharp-eyed and
thoughtful readers who have sent us suggestions and corrections over
the past few years.  In the spirit of free software, we decided to
express our gratitude in the form of a contributor list.  Unfortunately,
this list is not complete, but we are doing our best to keep it
up to date.

If you have a chance to look through the list, you should
realize that each person here has spared you and all subsequent
readers from the confusion of a technical error or a
less-than-transparent explanation, just by sending us a note.

Impossible as it may seem after so many corrections, there may still
be errors in this book.  If you should stumble across one, please
check the online version of the book at {\tt http://thinkpython.com},
which is the most up-to-date version.  If the error has not been
corrected, please take a minute to send us email at {\tt
feedback@thinkpython.com}.  If we make a change due to your
suggestion, you will appear in the next version of the contributor
list (unless you ask to be omitted).  Thank you!


\begin{itemize}

\item Lloyd Hugh Allen sent in a correction to Section 8.4.
%He can be reached at: {\tt lha2@columbia.edu}

\item Yvon Boulianne sent in a correction of a semantic error in
Chapter 5.
%She can be reached at: {\tt mystic@monuniverse.net}

\item Fred Bremmer submitted a correction in Section 2.1.
%He can be reached at:  {\tt Fred.Bremmer@ubc.cu}

\item Jonah Cohen wrote the Perl scripts to convert the
LaTeX source for this book into beautiful HTML.

%His Web page is {\tt jonah.ticalc.org}
%and his email is {\tt JonahCohen@aol.com}

\item Michael Conlon sent in a grammar correction in Chapter 2
and an improvement in style in Chapter 1, and he initiated discussion
on the technical aspects of interpreters.

%Michael can be reached at: {\tt michael.conlon@sru.edu}

\item Benoit Girard sent in a
correction to a humorous mistake in Section 5.6.

%Benoit can be reached at:
%{\tt benoit.girard@gouv.qc.ca}

\item Courtney Gleason and Katherine Smith wrote {\tt horsebet.py},
which was used as a case study in an earlier version of the book.  Their
program can now be found on the website.

%Courtney can be reached at: {\tt
%orion1558@aol.com}

\item Lee Harr submitted more corrections than we have room to list
here, and indeed he should be listed as one of the principal editors
of the text.

%He can be reached at: {\tt missive@linuxfreemail.com}

\item James Kaylin is a student using the text. He has submitted
numerous corrections.

%James can be reached by email at: {\tt Jamarf@aol.com}

\item David Kershaw fixed the broken {\tt catTwice} function in Section
3.10.

%He can be reached at: {\tt david\_kershaw@merck.com}

\item Eddie Lam has sent in numerous corrections to Chapters 
1, 2, and 3.
He also fixed the Makefile so that it creates an index the first time it is
run and helped us set up a versioning scheme.  

%Eddie can be reached at:
%{\tt nautilus@yoyo.cc.monash.edu.au}

\item Man-Yong Lee sent in a correction to the example code in
Section 2.4.  

%He can be reaced at: {\tt yong@linuxkorea.co.kr}

\item David Mayo pointed out that the word ``unconsciously"
in Chapter 1 needed
to be changed to ``subconsciously".

%David can be reached at:{\tt bdbear44@netscape.net}

\item Chris McAloon sent in several corrections to Sections 3.9 and
3.10.

%He can be reached at: {\tt cmcaloon@ou.edu}

\item Matthew J. Moelter has been a long-time contributor who sent
in numerous corrections and suggestions to the book.  

%He can be reached at:
%{\tt mmoelter@calpoly.edu}

\item Simon Dicon Montford reported a missing function definition and
several typos in Chapter 3.  He also found errors in the {\tt increment}
function in Chapter 13.

%He can be reached at: {\tt dicon@bigfoot.com}

\item John Ouzts corrected the definition of ``return value"
in Chapter 3.

%He can be reached at: {\tt jouzts@bigfoot.com}

\item Kevin Parks sent in valuable comments and suggestions as to how
to improve the distribution of the book.

%He can be reached at: {\tt cpsoct@lycos.com}

\item David Pool sent in a typo in the glossary of Chapter 1, as well
as kind words of encouragement.

%He can be reached at: {\tt pooldavid@hotmail.com}

\item Michael Schmitt sent in a correction to the chapter on files
and exceptions.

%He can be reached at: {\tt ipv6\_128@yahoo.com}

\item Robin Shaw pointed out an error in Section 13.1, where the
printTime function was used in an example without being defined.

%Robin can be reached at: {\tt randj@iowatelecom.net}

\item Paul Sleigh found an error in Chapter 7 and a bug in Jonah Cohen's
Perl script that generates HTML from LaTeX.

%He can be reached at: {\tt bat@atdot.dotat.org}

%\item Christopher Smith is a computer science teacher at the Blake
%School in Minnesota who teaches Python to his beginning students.

%He can be reached at: {\tt csmith@blakeschool.org or smiles@saysomething.com}

\item Craig T. Snydal is testing the text in a course at Drew
University.  He has contributed several valuable suggestions and corrections.

%and can be reached at: {\tt csnydal@drew.edu}

\item Ian Thomas and his students are using the text in a programming
course.  They are the first ones to test the chapters in the latter half
of the book, and they have made numerous corrections and suggestions.

%Ian can be reached at: {\tt ithomas@sd70.bc.ca}

\item Keith Verheyden sent in a correction in Chapter 3.

%He can be reached at: {\tt kverheyd@glam.ac.uk}

\item Peter Winstanley let us know about a longstanding error in
our Latin in Chapter 3.

%He can be reached at:{\tt Peter.Winstanley@scotland.gsi.gov.uk} 

\item Chris Wrobel made corrections to the code in the chapter on
file I/O and exceptions. 

%He can be reached at: {\tt ferz980@yahoo.com}

\item Moshe Zadka has made invaluable contributions to this project.
In addition to writing the first draft of the chapter on Dictionaries, he
provided continual guidance in the early stages of the book.

%He can be reached at: {\tt moshez@math.huji.ac.il}

\item Christoph Zwerschke sent several corrections and
pedagogic suggestions, and explained the difference between {\em gleich}
and {\em selbe}.

\item James Mayer sent us a whole slew of spelling and
typographical errors, including two in the contributor list.

% james.mayer@acm.org

\item Hayden McAfee caught a potentially confusing inconsistency
between two examples.
%hayden.mcafee@mindspring.com

\item Angel Arnal is part of an international team of translators
working on the Spanish version of the text.  He has also found several
errors in the English version.

\item Tauhidul Hoque and Lex Berezhny created the illustrations
in Chapter 1 and improved many of the other illustrations.

\item Dr. Michele Alzetta caught an error in Chapter 8 and sent
some interesting pedagogic comments and suggestions about Fibonacci
and Old Maid.
%mikalzet@libero.it

\item Andy Mitchell caught a typo in Chapter 1 and a broken example
in Chapter 2.
%phantom917@hotmail.com

\item Kalin Harvey suggested a clarification in Chapter 7 and
caught some typos.
%kalin@metamuscle.net

\item Christopher P. Smith caught several typos and is helping us
prepare to update the book for Python 2.2.
%csmith@blakeschool.org

\item David Hutchins caught a typo in the Foreword.
%jsdah2@uas.alaska.edu

\item Gregor Lingl is teaching Python at a high school in Vienna,
Austria.  He is working on a German translation of the book,
and he caught a couple of bad errors in Chapter 5.
%glingl@aon.at

%Sean McShane sent us a very nice note
%sean.mcshane@sheridanc.on.ca

\item Julie Peters caught a typo in the Preface.
%jkpeters@dmacc.cc.ia.us

\item Florin Oprina sent in an improvement in {\tt makeTime},
a correction in {\tt printTime}, and a nice typo.
%oprina@student.uit.no 

\item D.~J.~Webre suggested a clarification in Chapter 3.
%d_webre@yahoo.com

% \item 
% jkane@broadlink.com

\item Ken found a fistful of errors in Chapters 8, 9 and 11.
%ken@codeweavers.com

\item Ivo Wever caught a typo in Chapter 5 and suggested a clarification
in Chapter 3.
% I.J.W.Wever@student.tnw.tudelft.nl

% rbeumer@knijnenberg.nl

\item Curtis Yanko suggested a clarification in Chapter 2.
% YankoC@gspinc.com

\item Ben Logan sent in a number of typos and problems with translating
the book into HTML.
%ben@wblogan.net

%\item XXX suggested a clarification in Chapter 7, but prefers not
% to be included here.
%ejykfy@comcast.net

%\item Florian Thiel caught an inconsistency in Chapter 2.
%noroute@web.de

\item Jason Armstrong saw the missing word in Chapter 2.
%jarmstrong@technicacorp.com

\item Louis Cordier noticed a spot in Chapter 16 where the code
didn't match the text.
% lcordier@dsp.sun.ac.za

\item Brian Cain suggested several clarifications in Chapters 2 and 3.
% Brian.Cain@motorola.com

\item Rob Black sent in a passel of corrections, including some
changes for Python 2.2.
% Rob.Black@static2358.com

\item Jean-Philippe Rey at Ecole Centrale
Paris sent a number of patches, including some updates for Python 2.2
and other thoughtful improvements.
%<jean-philippe.rey@ecp.fr>

\item Jason Mader at George Washington University made a number
of useful suggestions and corrections.
%Jason Mader <jason@ncac.gwu.edu>

\item Jan Gundtofte-Bruun reminded us that ``a error'' is an error.
% Jan Gundtofte-Bruun <jan@g-b.dk>

\item Abel David and Alexis Dinno reminded us that the plural of
``matrix'' is ``matrices'', not ``matrixes''.  This error was in the
book for years, but two readers with the same initials reported it on
the same day.  Weird.
% Abel David <abel.david@gmail.com>, lexy-lou@doyenne.com

\item Charles Thayer encouraged us to get rid of the semi-colons
we had put at the ends of some statements and to clean up our
use of ``argument'' and ``parameter''.
% Charles Thayer <catintp@yahoo.com>

\item Roger Sperberg pointed out a twisted piece of logic in Chapter 3.
%<rsperberg@gmail.com>

\item Sam Bull pointed out a confusing paragraph in Chapter 2.
%Sam Bull <dreamsorcerer@gmail.com>

\item Andrew Cheung pointed out two instances of ``use before def.''
%cheunga@u.washington.edu

\item Hans Batra found an error in Chapter 16.

\item Chris Seberino suggested some improvements in the Preface.

\item Yuri Takhteyev pointed out a problem with single and double quotes.

\end{itemize}


% correspondents

% Python version

% Sam
% "sgasster@muon.com" <sgasster@muon.com>
% suggestions about syntax glossary

% James Pomeroy
% moodykre8r@earthlink.net
% suggestion about PEMDAS

% Joel Jensen  << joel@ens.net >>
% translated the book into MS html


% Gregg Boggs <boggs+@pitt.edu>
% correction to Pythagorean theorem

% John P. Withers
% jp_withers@yahoo.com
% question about latex

% Michael Brownfield
% mmc81@airmail.net
% general kind words

% wolfgang teschner
% wtr@hannover.sgh-net.de
% general kind words

% Jeff
% jca@po.cwru.edu
% problem with pdf, and nice comments

% d_webre@yahoo.com
% confusing comment about calculation of pi

% Michael Wheatfill
% mwheatfill@tse-us.com
% general good comments and question about histograms

% Java version

% Vladimir
% "pisemsky@pisem.net"
% general praise

% Yong
% ybakos69@yahoo.com" <ybakos69@yahoo.com>
% clarification of interface and abstract class






\clearemptydoublepage

% The following lines add a little extra space to the column
% in which the Section numbers appear in the table of contents
\makeatletter
\renewcommand{\l@section}{\@dottedtocline{1}{1.5em}{3.0em}}
\makeatother
\setcounter{tocdepth}{1}

\tableofcontents
\clearemptydoublepage

\mainmatter
% LaTeX source for textbook ``How to think like a computer scientist''
% Copyright (c)  2001  Allen B. Downey, Jeffrey Elkner, and Chris Meyers.

% Permission is granted to copy, distribute and/or modify this
% document under the terms of the GNU Free Documentation License,
% Version 1.1  or any later version published by the Free Software
% Foundation; with the Invariant Sections being "Contributor List",
% with no Front-Cover Texts, and with no Back-Cover Texts. A copy of
% the license is included in the section entitled "GNU Free
% Documentation License".

% This distribution includes a file named fdl.tex that contains the text
% of the GNU Free Documentation License.  If it is missing, you can obtain
% it from www.gnu.org or by writing to the Free Software Foundation,
% Inc., 59 Temple Place - Suite 330, Boston, MA 02111-1307, USA.
\chapter{The way of the program}

The goal of this book is to teach you to think like a
computer scientist. This way of thinking combines some of the best features
of mathematics, engineering, and natural science.  Like mathematicians,
computer scientists use formal languages to denote ideas (specifically
computations).  Like engineers, they design things, assembling components
into systems and evaluating tradeoffs among alternatives.  Like scientists,
they observe the behavior of complex systems, form hypotheses, and test
predictions.

The single most important skill for a computer scientist is {\bf
problem solving}.  Problem solving means the ability to formulate
problems, think creatively about solutions, and express a solution clearly
and accurately.  As it turns out, the process of learning to program is an
excellent opportunity to practice problem-solving skills.  That's why
this chapter is called, ``The way of the program.''

On one level, you will be learning to program, a useful
skill by itself.  On another level, you will use programming as a means to
an end.  As we go along, that end will become clearer.

\section{The Python programming language}
\index{programming language}
\index{language!programming}

The programming language you will be learning is Python. Python is
an example of a {\bf high-level language}; other high-level languages
you might have heard of are C, C++, Perl, and Java.

As you might infer from the name ``high-level language,'' there are
also {\bf low-level languages}, sometimes referred to as ``machine
languages'' or ``assembly languages.''  Loosely speaking, computers
can only execute programs written in low-level languages.  Thus,
programs written in a high-level language have to be processed before
they can run.  This extra processing takes some time, which is a small
disadvantage of high-level languages.

\index{portable}
\index{high-level language}
\index{low-level language}
\index{language!high-level}
\index{language!low-level}

But the advantages are enormous.  First, it is much easier to program
in a high-level language. Programs written in a high-level language
take less time to write, they are shorter and easier to read, and they
are more likely to be correct.  Second, high-level languages are {\bf
portable}, meaning that they can run on different kinds of computers
with few or no modifications.  Low-level programs can run on only one
kind of computer and have to be rewritten to run on another.

Due to these advantages, almost all programs are written in high-level
languages.  Low-level languages are used only for a few specialized
applications.

\index{compile}
\index{interpret}

Two kinds of programs process high-level languages
into low-level languages: {\bf interpreters} and {\bf compilers}.
An interpreter reads a high-level program and executes it, meaning that it
does what the program says.  It processes the program a little at a time,
alternately reading lines and performing computations.

\beforefig
\centerline{\psfig{figure=illustrations/interpret.eps,height=0.77in}}
\afterfig

A compiler reads the program and translates it completely before the
program starts running.  In this case, the high-level program is
called the {\bf source code}, and the translated program is called the
{\bf object code} or the {\bf executable}.  Once a program is
compiled, you can execute it repeatedly without further translation.

\beforefig
\centerline{\psfig{figure=illustrations/compile.eps,height=0.77in}}
\afterfig

Python is considered an interpreted language because Python
programs are executed by an interpreter.  There are two ways to
use the interpreter: command-line mode and script mode. In
command-line mode, you type Python programs and the interpreter
prints the result:

\adjustpage{-2}
\pagebreak
\beforeverb
\begin{verbatim}
$ python
Python 2.4.1 (#1, Apr 29 2005, 00:28:56)
Type "help", "copyright", "credits" or "license" for more information.
>>> print 1 + 1
2
\end{verbatim}
\afterverb
%
The first line of this example is the command that starts the
Python interpreter.  The next two lines are messages from the
interpreter.  The third line starts with {\tt >>>}, which is the
prompt the interpreter uses to indicate that it is ready.  We typed
{\tt print 1 + 1}, and the interpreter replied {\tt 2}.

Alternatively, you can write a program in a file and use the
interpreter to execute the contents of the file.  Such a file is
called a {\bf script}.  For example, we used a text editor to
create a file named {\tt latoya.py} with the following contents:


\beforeverb
\begin{verbatim}
print 1 + 1
\end{verbatim}
\afterverb
%
By convention, files that contain Python programs have names that
end with {\tt .py}.

To execute the program, we have to tell the interpreter the name of
the script:


\beforeverb
\begin{verbatim}
$ python latoya.py
2
\end{verbatim}
\afterverb
%
In other development environments, the details of executing programs
may differ.  Also, most programs are more interesting than this one.

Most of the examples in this book are executed on the command line.
Working on the command line is convenient for program development and
testing, because you can type programs and execute them
immediately.  Once you have a working program, you should store
it in a script so you can execute or modify it in the future.


\section{What is a program?}

A {\bf program} is a sequence of instructions that specifies how to
perform a computation.  The computation might be something
mathematical, such as solving a system of equations or finding the
roots of a polynomial, but it can also be a symbolic computation, such
as searching and replacing text in a document or (strangely enough)
compiling a program.

The details look different in
different languages, but a few basic instructions
appear in just about every language:

\begin{description}

\item[input:] Get data from the keyboard, a file, or some
other device.

\item[output:] Display data on the screen or send data to a
file or other device.

\item[math:] Perform basic mathematical operations like addition and
multiplication.

\item[conditional execution:] Check for certain conditions and
execute the appropriate sequence of statements.

\item[repetition:] Perform some action repeatedly, usually with
some variation.

\end{description}

Believe it or not, that's pretty much all there is to it.  Every
program you've ever used, no matter how complicated, is made up of
instructions that look more or less like these.  Thus, we can
describe programming as the process of breaking a large, complex task
into smaller and smaller subtasks until the subtasks are
simple enough to be performed with one of these basic instructions.

That may be a little vague, but we will come back to this topic later
when we talk about {\bf algorithms}.

\section{What is debugging?}
\index{debugging}
\index{bug}

Programming is a complex process, and because it is done by human beings,
it often leads to errors.  For whimsical reasons, programming errors are
called {\bf bugs} and the process of tracking them down and correcting them
is called {\bf debugging}.

Three kinds of errors can occur in a program: syntax errors, runtime 
errors, and semantic errors. It is useful
to distinguish between them in order to track them down more quickly.

\subsection{Syntax errors}
\index{syntax error}
\index{error!syntax}

Python can only execute a program if the program is syntactically
correct; otherwise, the process fails and returns an error message.
{\bf Syntax} refers to the structure of a program and the rules about
that structure. \index{syntax} For example, in English, a sentence must
begin with a capital letter and end with a period.  this sentence contains
a {\bf syntax error}.  So does this one

For most readers, a few syntax errors are not a significant problem,
which is why we can read the poetry of e. e. cummings without spewing error
messages.  Python is not so forgiving.  If there is a single syntax error
anywhere in your program, Python will print an error message and quit,
and you will not be able to run your program. During the first few weeks
of your programming career, you will probably spend a lot of time tracking
down syntax errors.  As you gain experience, though, you will make fewer
errors and find them faster.

\subsection{Runtime errors}
\label{runtime}
\index{runtime error}
\index{error!runtime}
\index{exception}
\index{safe language}
\index{language!safe}

The second type of error is a runtime error, so called because the
error does not appear until you run the program.  These errors are also
called {\bf exceptions} because they usually indicate that something
exceptional (and bad) has happened.

Runtime errors are rare in the simple programs you will see in the
first few chapters, so it might be a while before you encounter one.


\subsection{Semantic errors}
\index{semantics}
\index{semantic error}
\index{error!semantic}

The third type of error is the {\bf semantic error}.  If there is a
semantic error in your program, it will run successfully, in the sense
that the computer will not generate any error messages, but it will
not do the right thing.  It will do something else.  Specifically, it
will do what you told it to do.

The problem is that the program you wrote is not the program you
wanted to write.  The meaning of the program (its semantics) is wrong.
Identifying semantic errors can be tricky because it requires you to work
backward by looking at the output of the program and trying to figure
out what it is doing.

\subsection{Experimental debugging}

One of the most important skills you will acquire is
debugging.  Although it can be frustrating, debugging is one of the
most intellectually rich, challenging, and interesting parts of programming.

In some ways, debugging is like detective work.  You are confronted
with clues, and you have to infer the processes and events that led
to the results you see.

Debugging is also like an experimental science.  Once you have an idea
what is going wrong, you modify your program and try again.  If your
hypothesis was correct, then you can predict the result of the
modification, and you take a step closer to a working program.  If
your hypothesis was wrong, you have to come up with a new one.  As
Sherlock Holmes pointed out, ``When you have eliminated the
impossible, whatever remains, however improbable, must be the truth.''
(A. Conan Doyle, {\em The Sign of Four})

\index{Holmes, Sherlock}
\index{Doyle, Arthur Conan}

For some people, programming and debugging are the
same thing.  That is, programming is the process of gradually
debugging a program until it does what you want.  The idea
is that you should start with a program that
does {\em something} and make small modifications, debugging
them as you go, so that you always have a working program.

For example, Linux is an operating system that contains thousands of
lines of code, but it started out as a simple program Linus Torvalds
used to explore the Intel 80386 chip.  According to Larry Greenfield,
``One of Linus's earlier projects was a program that would switch
between printing AAAA and BBBB.  This later evolved to Linux.''
({\em The Linux Users' Guide} Beta Version 1)

\index{Linux}

Later chapters will make more suggestions about debugging and other
programming practices.

\section{Formal and natural languages}
\index{formal language}
\index{natural language}
\index{language!formal}
\index{language!natural}

{\bf Natural languages} are the languages that people speak,
such as English, Spanish, and French.  They were not designed
by people (although people try to impose some order on them);
they evolved naturally.

{\bf Formal languages} are languages that are designed by people for
specific applications.  For example, the notation that mathematicians
use is a formal language that is particularly good at denoting
relationships among numbers and symbols.  Chemists use a formal
language to represent the chemical structure of molecules.  And
most importantly:

\begin{quote}
{\bf Programming languages are formal languages that have been
designed to express computations.}
\end{quote}

Formal languages tend to have strict rules about syntax.  For example,
$3+3=6$ is a syntactically correct mathematical statement, but
{\tt 3=+6\$} is not.  $H_2O$ is a syntactically correct chemical name,
but $_2Zz$ is not.

Syntax rules come in two flavors, pertaining to {\bf tokens} and structure.
Tokens are the basic elements of the language, such as words, numbers,
and chemical elements.  One of the problems with {\tt 3=+6\$} is that
{\tt \$} is not a legal token in mathematics (at least as far as we
know).  Similarly, $_2Zz$ is not legal because there is no element with
the abbreviation $Zz$.

The second type of syntax error pertains to the structure of a
statement---that is, the way the tokens are arranged.  The statement
{\tt 3=+6\$} is structurally illegal because you can't place a plus
sign immediately after an equal sign.  Similarly, molecular formulas
have to have subscripts after the element name, not before.

\begin{quote}
{\em As an exercise, create what appears to be a well-structured English
sentence with unrecognizable tokens in it.  Then write another sentence
with all valid tokens but with invalid structure.}
\end{quote}

When you read a sentence in English or a statement in a formal
language, you have to figure out what the structure of the sentence is
(although in a natural language you do this subconsciously).  This
process is called {\bf parsing}.

\index{parse}

For example, when you hear the sentence, ``The other shoe fell,'' you
understand that ``the other shoe'' is the subject and ``fell'' is the
predicate.  Once you have parsed a sentence, you can figure out what it
means, or the semantics of the sentence.  Assuming that you know
what a shoe is and what it means to fall, you will understand the
general implication of this sentence.

Although formal and natural languages have many features in
common---tokens, structure, syntax, and semantics---there are many
differences:

\index{ambiguity}
\index{redundancy}
\index{literalness}

\begin{description}

\item[ambiguity:] Natural languages are full of ambiguity, which
people deal with by using contextual clues and other information.
Formal languages are designed to be nearly or completely unambiguous,
which means that any statement has exactly one meaning,
regardless of context.

\item[redundancy:] In order to make up for ambiguity and reduce
misunderstandings, natural languages employ lots of
redundancy.  As a result, they are often verbose.  Formal languages
are less redundant and more concise.

\item[literalness:] Natural languages are full of idiom and
metaphor.  If I say, ``The other shoe fell,'' there is probably
no shoe and nothing falling.  Formal languages mean
exactly what they say.

\end{description}

People who grow up speaking a natural language---everyone---often have a
hard time adjusting to formal languages.  In some ways, the difference
between formal and natural language is like the difference between
poetry and prose, but more so:

\index{poetry}
\index{prose}

\begin{description}

\item[Poetry:] Words are used for their sounds as well as for
their meaning, and the whole poem together creates an effect or
emotional response.  Ambiguity is not only common but often
deliberate.

\item[Prose:] The literal meaning of words is more important,
and the structure contributes more meaning.  Prose is more amenable to
analysis than poetry but still often ambiguous.

\item[Programs:] The meaning of a computer program is unambiguous
and literal, and can be understood entirely by analysis of the
tokens and structure.

\end{description}

Here are some suggestions for reading programs (and other formal
languages).  First, remember that formal languages are much more dense
than natural languages, so it takes longer to read them.  Also, the
structure is very important, so it is usually not a good idea to read
from top to bottom, left to right.  Instead, learn to parse the
program in your head, identifying the tokens and interpreting the
structure.  Finally, the details matter.  Little things
like spelling errors and bad punctuation, which you can get away
with in natural languages, can make a big difference in a formal
language.

\section{The first program}
\label{hello}
\index{hello world}

Traditionally, the first program written in a new language
is called ``Hello, World!'' because all it does is display the
words, ``Hello, World!''  In Python, it looks like this:


\beforeverb
\begin{verbatim}
print "Hello, World!"
\end{verbatim}
\afterverb
%
This is an example of a {\bf print statement}, which doesn't
actually print anything on paper.  It displays a value on the
screen.  In this case, the result is the words


\beforeverb
\begin{verbatim}
Hello, World!
\end{verbatim}
\afterverb
%
The quotation marks in the program mark the beginning and end
of the value; they don't appear in the result.

\index{print statement}
\index{statement!print}

Some people judge the quality of a programming language by the
simplicity of the ``Hello, World!'' program.  By this standard, Python
does about as well as possible.

\section{Glossary}

\begin{description}

\item[problem solving:]  The process of formulating a problem, finding
a solution, and expressing the solution.

\item[high-level language:]  A programming language like Python that
is designed to be easy for humans to read and write.

\item[low-level language:]  A programming language that is designed
to be easy for a computer to execute; also called ``machine language'' or
``assembly language.''

\item[portability:]  A property of a program that can run on more
than one kind of computer.

\item[interpret:]  To execute a program in a high-level language
by translating it one line at a time.

\item[compile:]  To translate a program written in a high-level language
into a low-level language all at once, in preparation for later
execution.

\item[source code:]  A program in a high-level language before
being compiled.

\item[object code:]  The output of the compiler after it translates
the program.

\item[executable:]  Another name for object code that is ready
to be executed.

\item[script:] A program stored in a file (usually one that will be
interpreted).

\item[program:] A set of instructions that specifies a computation.

\item[algorithm:]  A general process for solving a category of
problems.

\item[bug:]  An error in a program.

\item[debugging:]  The process of finding and removing any of the
three kinds of programming errors.

\item[syntax:]  The structure of a program.

\item[syntax error:]  An error in a program that makes it impossible
to parse (and therefore impossible to interpret).

\item[runtime error:]  An error that does not occur until the program
has started to execute but that prevents the program from continuing.

\item[exception:]  Another name for a runtime error.

\item[semantic error:]   An error in a program that makes it do something
other than what the programmer intended.

\item[semantics:]  The meaning of a program.

\item[natural language:]  Any one of the languages that people speak that
evolved naturally.

\item[formal language:]  Any one of the languages that people have designed
for specific purposes, such as representing mathematical ideas or
computer programs; all programming languages are formal languages.

\item[token:]  One of the basic elements of the syntactic structure of
a program, analogous to a word in a natural language.

\item[parse:]  To examine a program and analyze the syntactic structure.

\item[print statement:]  An instruction that causes the Python
interpreter to display a value on the screen.

\index{program}
\index{problem-solving}
\index{high-level language}
\index{low-level language}
\index{portability}
\index{interpret}
\index{compile}
\index{source code}
\index{object code}
\index{executable}
\index{algorithm}
\index{bug}
\index{debugging}
\index{syntax}
\index{semantics}
\index{syntax error}
\index{runtime error}
\index{exception}
\index{semantic error}
\index{formal language}
\index{natural language}
\index{parse}
\index{token}
\index{script}
\index{print statement}
\index{statement!print}

\end{description}

\clearemptydoublepage
% LaTeX source for textbook ``How to think like a computer scientist''
% Copyright (c)  2001  Allen B. Downey, Jeffrey Elkner, and Chris Meyers.

% Permission is granted to copy, distribute and/or modify this
% document under the terms of the GNU Free Documentation License,
% Version 1.1  or any later version published by the Free Software
% Foundation; with the Invariant Sections being "Contributor List",
% with no Front-Cover Texts, and with no Back-Cover Texts. A copy of
% the license is included in the section entitled "GNU Free
% Documentation License".

% This distribution includes a file named fdl.tex that contains the text
% of the GNU Free Documentation License.  If it is missing, you can obtain
% it from www.gnu.org or by writing to the Free Software Foundation,
% Inc., 59 Temple Place - Suite 330, Boston, MA 02111-1307, USA.
\chapter{Variables, expressions and statements}

\section{Values and types}
\index{value}
\index{type}
\index{string}

A {\bf value} is one of the fundamental things---like a letter or a
number---that a program manipulates.  The values we have seen so far
are {\tt 2} (the result when we added {\tt 1 + 1}), and
{\tt 'Hello, World!'}.

These values belong to different {\bf types}:
{\tt 2} is an integer, and {\tt 'Hello, World!'} is a {\bf string},
so-called because it contains a ``string'' of letters.
You (and the interpreter) can identify
strings because they are enclosed in quotation marks.

The print statement also works for integers.

\beforeverb
\begin{verbatim}
>>> print 4
4
\end{verbatim}
\afterverb
%
If you are not sure what type a value has,
the interpreter can tell you.

\beforeverb
\begin{verbatim}
>>> type('Hello, World!')
<type 'str'>
>>> type(17)
<type 'int'>
\end{verbatim}
\afterverb
%
Not surprisingly, strings belong to the type {\tt str} and
integers belong to the type {\tt int}.  Less obviously, numbers
with a decimal point belong to a type called {\tt float},
because these numbers are represented in a
format called {\bf floating-point}.

\index{type}
\index{string}
\index{type!str}
\index{int}
\index{type!int}
\index{float}
\index{type!float}

\beforeverb
\begin{verbatim}
>>> type(3.2)
<type 'float'>
\end{verbatim}
\afterverb
%
What about values like {\tt '17'} and {\tt '3.2'}?
They look like numbers, but they are in quotation marks like
strings.

\beforeverb
\begin{verbatim}
>>> type('17')
<type 'str'>
>>> type('3.2')
<type 'str'>
\end{verbatim}
\afterverb
%
They're strings.

When you type a large integer, you might be tempted to use commas
between groups of three digits, as in {\tt 1,000,000}.  This is not a
legal integer in Python, but it is a legal expression:

\beforeverb
\begin{verbatim}
>>> print 1,000,000
1 0 0
\end{verbatim}
\afterverb
%
Well, that's not what we expected at all!  Python interprets {\tt
1,000,000} as a comma-separated list of three integers, which it
prints consecutively.  This is the first example we have seen of a
semantic error: the code runs without producing an error message, but
it doesn't do the ``right'' thing.


\section{Variables}
\index{variable}
\index{assignment}
\index{statement!assignment}

One of the most powerful features of a programming language is the
ability to manipulate {\bf variables}.  A variable is a name that
refers to a value.

The {\bf assignment statement} creates new variables and gives
them values:

\beforeverb
\begin{verbatim}
>>> message = "What's up, Doc?"
>>> n = 17
>>> pi = 3.14159
\end{verbatim}
\afterverb
%
This example makes three assignments.  The first assigns the string
{\tt "What's up, Doc?"} to a new variable named {\tt message}.
The second gives the integer {\tt 17} to {\tt n}, and the third
gives the floating-point number {\tt 3.14159} to {\tt pi}.

Notice that the first statement uses double quotes to enclose the
string.  In general, single and double quotes do the same thing, but
if the string contains a single quote (or an apostrophe, which is
the same character), you have to use double quotes to enclose it.

\index{state diagram}

A common way to represent variables on paper is to write the name with
an arrow pointing to the variable's value.  This kind of figure is
called a {\bf state diagram} because it shows what state each of the
variables is in (think of it as the variable's state of mind).
This diagram shows the result of the assignment statements:

\beforefig
\centerline{\psfig{figure=illustrations/state2.eps}}
\afterfig

The print statement also works with variables.

\beforeverb
\begin{verbatim}
>>> print message
What's up, Doc?
>>> print n
17
>>> print pi
3.14159
\end{verbatim}
\afterverb
%
In each case the result is the value of the variable.
Variables also have types; again, we can ask the
interpreter what they are.

\beforeverb
\begin{verbatim}
>>> type(message)
<type 'str'>
>>> type(n)
<type 'int'>
>>> type(pi)
<type 'float'>
\end{verbatim}
\afterverb
%
The type of a variable is the type of the value it
refers to.


\section{Variable names and keywords}
\index{keyword}

Programmers generally choose names for their variables that
are meaningful---they document what the variable is used for.

Variable names can be arbitrarily long.  They can contain
both letters and numbers, but they have to begin with a letter.
Although it is legal to use uppercase letters, by convention
we don't.  If you do, remember that case matters.  {\tt Bruce}
and {\tt bruce} are different variables.

The underscore character ({\tt \_}) can appear in a name.
It is often used in names with multiple words, such as
{\tt my\_name} or {\tt price\_of\_tea\_in\_china}.

\index{underscore character}

If you give a variable an illegal name, you get a syntax error:

\adjustpage{-2}
\pagebreak
\beforeverb
\begin{verbatim}
>>> 76trombones = 'big parade'
SyntaxError: invalid syntax
>>> more$ = 1000000
SyntaxError: invalid syntax
>>> class = 'Computer Science 101'
SyntaxError: invalid syntax
\end{verbatim}
\afterverb
%
{\tt 76trombones} is illegal because it does not begin with a letter.
{\tt more\$} is illegal because it contains an illegal character, the dollar
sign.  But what's wrong with {\tt class}?

It turns out that {\tt class} is one of the Python {\bf keywords}.
Keywords define the language's rules and structure, and they cannot be
used as variable names.

\index{keyword}

Python has twenty-nine keywords:

\beforeverb
\begin{verbatim}
and       def       exec      if        not       return
assert    del       finally   import    or        try
break     elif      for       in        pass      while
class     else      from      is        print     yield
continue  except    global    lambda    raise
\end{verbatim}
\afterverb
%
You might want to keep this list handy.  If the interpreter complains
about one of your variable names and you don't know why, see if it
is on this list.


\section{Statements}

A statement is an instruction that the Python interpreter can
execute.  We have seen two kinds of statements: print
and assignment.

When you type a statement on the command line, Python
executes it and displays the result, if there is one.  The
result of a print statement is a value.  Assignment statements
don't produce a result.

A script usually contains a sequence of statements.  If there
is more than one statement, the results appear one at a time
as the statements execute.

For example, the script

\beforeverb
\begin{verbatim}
print 1
x = 2
print x
\end{verbatim}
\afterverb
%
produces the output

\beforeverb
\begin{verbatim}
1
2
\end{verbatim}
\afterverb
%
Again, the assignment statement produces no output.



\section{Evaluating expressions}

An expression is a combination of values, variables, and operators.
If you type an expression on the command line, the interpreter
{\bf evaluates} it and displays the result:

\beforeverb
\begin{verbatim}
>>> 1 + 1
2
\end{verbatim}
\afterverb
%
Although expressions contain values, variables, and operators,
not every expression contains all of these elements.
A value all by itself is considered an expression, and so is
a variable.

\beforeverb
\begin{verbatim}
>>> 17
17
>>> x
2
\end{verbatim}
\afterverb
%
Confusingly, evaluating an expression is not quite the same
thing as printing a value.

\beforeverb
\begin{verbatim}
>>> message = 'Hello, World!'
>>> message
'Hello, World!'
>>> print message
Hello, World!
\end{verbatim}
\afterverb
%
When the Python interpreter displays the value of an expression, it
uses the same format you would use to enter a value.  In the case of
strings, that means that it includes the quotation marks.  But if
you use a print statement, Python displays the contents of the string
without the quotation marks.

In a script, an expression all by itself is a legal statement, but it
doesn't do anything.  The script

\beforeverb
\begin{verbatim}
17
3.2
'Hello, World!'
1 + 1
\end{verbatim}
\afterverb
%
produces no output at all.  How would you change the script to
display the values of these four expressions?


\section{Operators and operands}
\index{operator}
\index{operand}
\index{expression}

{\bf Operators} are special symbols that represent computations
like addition and multiplication.  The values the operator uses are
called {\bf operands}.

The following are all legal Python expressions whose meaning is more or
less clear:
\adjustpage{2}
\beforeverb
\begin{verbatim}
20+32   hour-1   hour*60+minute   minute/60   5**2   (5+9)*(15-7)
\end{verbatim}
\afterverb
%
The symbols {\tt +}, {\tt -}, and {\tt /}, and the use of parenthesis for
grouping, mean in Python what they mean in mathematics.  The asterisk
({\tt *}) is the symbol for multiplication, and {\tt **} is the symbol
for exponentiation.

When a variable name appears in the place of an operand, it
is replaced with its value before the operation is
performed.

Addition, subtraction, multiplication, and exponentiation all do what
you expect, but you might be surprised by division.  The following
operation has an unexpected result:

\beforeverb
\begin{verbatim}
>>> minute = 59
>>> minute/60
0
\end{verbatim}
\afterverb
%
The value of {\tt minute} is 59, and in conventional arithmetic 59
divided by 60 is 0.98333, not 0.  The reason for the discrepancy is
that Python is performing {\bf integer division}.

\index{integer division}

When both of the operands are integers, the result must also be an integer,
and by convention, integer division always rounds {\em down}, even in cases
like this where the next integer is very close.

A possible solution to this problem is to calculate a percentage
rather than a fraction:

\beforeverb
\begin{verbatim}
>>> minute*100/60
98
\end{verbatim}
\afterverb
%
Again the result is rounded down, but at least now the answer is
approximately correct.  Another alternative is to use floating-point
division, which we get to in Chapter~\ref{floatchap}.


\section{Order of operations}
\index{order of operations}
\index{rules of precedence}

When more than one operator appears in an expression, the order of
evaluation depends on the {\bf rules of precedence}.  Python follows
the same precedence rules for its mathematical operators that
mathematics does.  The acronym {\bf PEMDAS} is a useful way to
remember the order of operations:

\begin{itemize}

\item {\bf P}arentheses have the highest precedence and can be used 
to force an expression to evaluate in the order you want. Since
expressions in parentheses are evaluated first, {\tt 2 * (3-1)} is 4,
and {\tt (1+1)**(5-2)} is 8. You can also use parentheses to make an
expression easier to read, as in {\tt (minute * 100) / 60}, even
though it doesn't change the result.

\item {\bf E}xponentiation has the next highest precedence, so
{\tt 2**1+1} is 3 and not 4, and {\tt 3*1**3} is 3 and not 27.

\item {\bf M}ultiplication and {\bf D}ivision have the same precedence,
which is higher than {\bf A}ddition and {\bf S}ubtraction, which also
have the same precedence.  So {\tt 2*3-1} yields 5 rather than 4, and
{\tt 2/3-1} is {\tt -1}, not {\tt 1} (remember that in integer
division, {\tt 2/3=0}).

\item Operators with the same precedence are evaluated from left to 
right.  So in the expression {\tt minute*100/60}, the multiplication
happens first, yielding {\tt 5900/60}, which in turn yields {\tt 98}.
If the operations had been evaluated from right to left, the result
would have been {\tt 59*1}, which is {\tt 59}, which is wrong.

\end{itemize}


\section{Operations on strings}
\index{string operation}

In general, you cannot perform mathematical operations on strings, even
if the strings look like numbers.  The following are illegal (assuming
that {\tt message} has type {\tt string}):

\beforeverb
\begin{verbatim}
 message-1   'Hello'/123   message*'Hello'   '15'+2
\end{verbatim}
\afterverb
%
Interestingly, the {\tt +} operator does work with strings, although it
does not do exactly what you might expect.  For strings, the {\tt +} operator
represents {\bf concatenation}, which means joining the two operands by
linking them end-to-end.  For example:

\index{concatenation}

\beforeverb
\begin{verbatim}
fruit = 'banana'
bakedGood = ' nut bread'
print fruit + bakedGood
\end{verbatim}
\afterverb
%
The output of this program is {\tt banana nut bread}.  The space
before the word {\tt nut} is part of the string, and is necessary
to produce the space between the concatenated strings.

The {\tt *} operator also works on strings; it performs repetition.
For example, {\tt 'Fun'*3} is {\tt 'FunFunFun'}.  One of the operands
has to be a string; the other has to be an integer.

On one hand, this interpretation of {\tt +} and {\tt *} makes sense by
analogy with addition and multiplication.  Just as {\tt 4*3} is
equivalent to {\tt 4+4+4}, we expect {\tt 'Fun'*3} to be the same as
{\tt 'Fun'+'Fun'+'Fun'}, and it is.  On the other hand, there is a
significant way in which string concatenation and repetition are
different from integer addition and multiplication.
Can you think of a property that addition and multiplication have
that string concatenation and repetition do not?


\section{Composition}
\index{composition}

So far, we have looked at the elements of a program---variables,
expressions, and statements---in isolation, without talking about how to
combine them.

One of the most useful features of programming languages is their
ability to take small building blocks and {\bf compose} them.  For
example, we know how to add numbers and we know how to print; it turns
out we can do both at the same time:

\beforeverb
\begin{verbatim}
>>>  print 17 + 3
20
\end{verbatim}
\afterverb
%
In reality, the
addition has to happen before the printing, so the actions aren't 
actually happening at the same time. The point is that any
expression involving numbers, strings, and variables can be used inside a
print statement.  You've already seen an example of this:

\beforeverb
\begin{verbatim}
print 'Number of minutes since midnight: ', hour*60+minute
\end{verbatim}
\afterverb
%
You can also put arbitrary expressions on the right-hand side of an
assignment statement:

\beforeverb
\begin{verbatim}
percentage = (minute * 100) / 60
\end{verbatim}
\afterverb
%
This ability may not seem impressive now, but you will see other examples
where composition makes it possible to express complex computations neatly and
concisely.

Warning: There are limits on where you can use certain expressions.  For
example, the left-hand side of an assignment statement has to be a
{\em variable} name, not an expression.  So, the following is illegal:
{\tt minute+1 = hour}.


\section{Comments}
\index{comment}

As programs get bigger and more complicated, they get more difficult to
read.  Formal languages are dense, and it is often difficult to look
at a piece of
code and figure out what it is doing, or why.

For this reason, it is a good idea to add notes to your programs to explain
in natural language what the program is doing.  These notes are called
{\bf comments}, and they are marked with the {\tt \#} symbol:

\beforeverb
\begin{verbatim}
# compute the percentage of the hour that has elapsed
percentage = (minute * 100) / 60
\end{verbatim}
\afterverb
%
In this case, the comment appears on a line by itself.  You can also put
comments at the end of a line:

\beforeverb
\begin{verbatim}
percentage = (minute * 100) / 60     # caution: integer division
\end{verbatim}
\afterverb
%
Everything from the {\tt \#} to the end of the line is ignored---it
has no effect on the program.  The message is intended for the programmer or
for future programmers who might use this code.  In this case, it
reminds the reader about the ever-surprising behavior of integer division.

\index{integer division}

This sort of comment is less necessary if you use the integer division
operation, \verb+//+.  It has the same effect as the division
operator\footnote{For now.  The behavior of the division operator may
change in future versions of Python.}, but it signals that the effect
is deliberate.

\beforeverb
\begin{verbatim}
percentage = (minute * 100) // 60 
\end{verbatim}
\afterverb
%
The integer division operator is like a comment that says, ``I know
this is integer division, and I like it that way!''

\section{Glossary}

\begin{description}

\item[value:]  A number or string (or other thing to be named later)
that can be stored in a variable or computed in an expression.

\item[type:]  A set of values.  The type of a value determines how
it can be used in expressions.  So far, the types you have seen are integers
(type {\tt int}), floating-point numbers (type {\tt float}),
and strings (type {\tt string}).

\item[floating-point:] A format for representing numbers with fractional
parts.

\item[variable:]  A name that refers to a value.

\item[statement:]  A section of code that represents a command or action.  So
far, the statements you have seen are assignments and print statements.

\item[assignment:]  A statement that assigns a value to a variable.

\item[state diagram:]  A graphical representation of a set of variables and the
values to which they refer.

\item[keyword:]  A reserved word that is used by the compiler to parse a
program; you cannot use keywords like {\tt if}, {\tt  def}, and {\tt while} as
variable names.

\item[operator:]  A special symbol that represents a simple computation like
addition, multiplication, or string concatenation.

\item[operand:]  One of the values on which an operator operates.

\item[expression:]  A combination of variables, operators, and values that
represents a single result value.

\item[evaluate:]  To simplify an expression by performing the operations
in order to yield a single value.

\item[integer division:]  An operation that divides one integer by
another and yields an integer.  Integer division yields only the
whole number of times that the numerator is divisible by the
denominator and discards any remainder.

\item[rules of precedence:]  The set of rules governing the order in which
expressions involving multiple operators and operands are evaluated.

\item[concatenate:]  To join two operands end-to-end.

\item[composition:]  The ability to combine simple expressions and statements
into compound statements and expressions in order to represent complex
computations concisely.

\item[comment:]  Information in a program that is meant for other
programmers (or anyone reading the source code) and has no effect on the
execution of the program.

\index{value}
\index{floating-point}
\index{variable}
\index{type}
\index{keyword}
\index{statement}
\index{assignment}
\index{comment}
\index{state diagram}
\index{expression}
\index{operator}
\index{operand}
\index{integer division}
\index{rules of precedence}
\index{precedence}
\index{concatenation}
\index{composition}

\end{description}

\clearemptydoublepage
% LaTeX source for textbook ``How to think like a computer scientist''
% Copyright (c)  2001  Allen B. Downey, Jeffrey Elkner, and Chris Meyers.

% Permission is granted to copy, distribute and/or modify this
% document under the terms of the GNU Free Documentation License,
% Version 1.1  or any later version published by the Free Software
% Foundation; with the Invariant Sections being "Contributor List",
% with no Front-Cover Texts, and with no Back-Cover Texts. A copy of
% the license is included in the section entitled "GNU Free
% Documentation License".

% This distribution includes a file named fdl.tex that contains the text
% of the GNU Free Documentation License.  If it is missing, you can obtain
% it from www.gnu.org or by writing to the Free Software Foundation,
% Inc., 59 Temple Place - Suite 330, Boston, MA 02111-1307, USA.

\chapter{Functions}
\label{floatchap}

\section{Function calls}
\label{functionchap}
\index{function call}
\index{call!function}

You have already seen one example of a {\bf function call}:

\beforeverb
\begin{verbatim}
>>> type("32")
<type 'str'>
\end{verbatim}
\afterverb
%
The name of the function is {\tt type}, and it displays the type of
a value or variable.  The value or variable, which is called the
{\bf argument} of the function, has to be enclosed in parentheses.
It is common to say that a function ``takes'' an argument and ``returns''
a result.  The result is called the {\bf return value}.

\index{argument}
\index{return value}

Instead of printing the return value, we could assign it to a variable:

\beforeverb
\begin{verbatim}
>>> betty = type("32")
>>> print betty
<type 'str'>
\end{verbatim}
\afterverb
%
As another example, the {\tt id} function takes a value or a variable and
returns an integer that acts as a unique identifier for the value:

\beforeverb
\begin{verbatim}
>>> id(3)
134882108
>>> betty = 3
>>> id(betty)
134882108
\end{verbatim}
\afterverb
%
Every value has an {\tt id}, which is a unique number related to where
it is stored in the memory of the computer.  The {\tt id} of a
variable is the {\tt id} of the value to which it refers.



\section{Type conversion}
\index{conversion!type}
\index{type conversion}

Python provides a collection of built-in functions that convert values
from one type to another.  The {\tt int} function takes any value and
converts it to an integer, if possible, or complains otherwise:

\beforeverb
\begin{verbatim}
>>> int("32")
32
>>> int("Hello")
ValueError: invalid literal for int(): Hello
\end{verbatim}
\afterverb
%
{\tt int} can also convert floating-point values to integers, but
remember that it truncates the fractional part:

\beforeverb
\begin{verbatim}
>>> int(3.99999)
3
>>> int(-2.3)
-2
\end{verbatim}
\afterverb
%
The {\tt float} function converts integers and strings to floating-point
numbers:

\beforeverb
\begin{verbatim}
>>> float(32)
32.0
>>> float("3.14159")
3.14159
\end{verbatim}
\afterverb
%
Finally, the {\tt str} function converts to
type {\tt string}:

\beforeverb
\begin{verbatim}
>>> str(32)
'32'
>>> str(3.14149)
'3.14149'
\end{verbatim}
\afterverb
%
It may seem odd that Python distinguishes the integer value {\tt 1}
from the floating-point value {\tt 1.0}.  They may represent the same
number, but they belong to different types.  The reason is that they
are represented differently inside the computer.



\section{Type coercion}
\index{type coercion}
\index{coercion!type}
\index{integer division}
\index{division!integer}

Now that we can convert between types, we have another way to deal
with integer division.  Returning to the example from the previous
chapter, suppose we want to calculate the fraction of an hour that has
elapsed.  The most obvious expression, {\tt minute / 60}, does integer
arithmetic, so the result is always 0, even at 59 minutes
past the hour.

One solution is to convert {\tt minute} to floating-point
and do floating-point division:

\beforeverb
\begin{verbatim}
>>> minute = 59
>>> float(minute) / 60
0.983333333333
\end{verbatim}
\afterverb
%
Alternatively, we can take advantage of the rules for
automatic type conversion, which is called {\bf type coercion}.
For the mathematical operators, if either operand is a {\tt float},
the other is automatically converted to a {\tt float}:

\beforeverb
\begin{verbatim}
>>> minute = 59
>>> minute / 60.0
0.983333333333
\end{verbatim}
\afterverb
%
By making the denominator a {\tt float}, we force Python to do
floating-point division.


\section{Math functions}
\index{math function}
\index{function!math}

In mathematics, you have probably seen functions like {\tt sin} and
{\tt log}, and you have learned to evaluate expressions like {\tt
sin(pi/2)} and {\tt log(1/x)}.  First, you evaluate the expression in
parentheses (the argument).  For example, {\tt pi/2} is approximately
1.571, and {\tt 1/x} is 0.1 (if {\tt x} happens to be 10.0).

Then, you evaluate the function itself, either by looking it up in a
table or by performing various computations.  The {\tt sin} of 1.571
is 1, and the {\tt log} of 0.1 is -1 (assuming that {\tt log}
indicates the logarithm base 10).

This process can be applied repeatedly to evaluate more complicated
expressions like {\tt log(1/sin(pi/2))}.  First, you evaluate the
argument of the innermost function, then evaluate the function, and so
on.

Python has a math module that provides most of the familiar
mathematical functions.  A {\bf module} is a file that contains a
collection of related functions grouped together.

\index{module}

Before we can use the functions from a module, we have to import them:

\beforeverb
\begin{verbatim}
>>> import math
\end{verbatim}
\afterverb
%
To call one of the functions, we have to specify the name of the
module and the name of the function, separated by a dot, also
known as a period.  This format is called {\bf dot notation}.

\index{dot notation}

\beforeverb
\begin{verbatim}
>>> decibel = math.log10 (17.0)
>>> angle = 1.5
>>> height = math.sin(angle)
\end{verbatim}
\afterverb
%
The first statement sets {\tt decibel} to the logarithm of 17, base
{\tt 10}.  There is also a function called {\tt log} that takes logarithm
base {\tt e}.

The third statement finds the sine of the value of the variable {\tt
angle}.  {\tt sin} and the other trigonometric functions ({\tt cos},
{\tt tan}, etc.)  take arguments in radians. To convert from degrees
to radians, divide by 360 and multiply by {\tt 2*pi}.  For example, to
find the sine of 45 degrees, first calculate the angle in radians and
then take the sine:

\beforeverb
\begin{verbatim}
>>> degrees = 45
>>> angle = degrees * 2 * math.pi / 360.0
>>> math.sin(angle)
0.707106781187
\end{verbatim}
\afterverb
%
The constant {\tt pi} is also part of the math module.  If you know
your geometry, you can check the previous result by comparing it to
the square root of two divided by two:

\beforeverb
\begin{verbatim}
>>> math.sqrt(2) / 2.0
0.707106781187
\end{verbatim}
\afterverb
%

\section{Composition}
\index{composition}
\index{function!composition}

Just as with mathematical functions, Python functions can be composed,
meaning that you use one expression as part of another. For example, you can
use any expression as an argument to a function:

\beforeverb
\begin{verbatim}
>>> x = math.cos(angle + math.pi/2)
\end{verbatim}
\afterverb
%
This statement takes the value of {\tt pi}, divides it by 2, and adds
the result to the value of {\tt angle}.  The sum is then passed as an
argument to the {\tt cos} function.

You can also take the result of one function and pass it as an argument to
another:

\beforeverb
\begin{verbatim}
>>> x = math.exp(math.log(10.0))
\end{verbatim}
\afterverb
%
This statement finds the log base {\tt e} of 10 and then raises {\tt e} to
that power. The result gets assigned to {\tt x}.


\section{Adding new functions}

So far, we have only been using the functions that come with Python,
but it is also possible to add new functions.  Creating new functions
to solve your particular problems is one of the most useful things
about a general-purpose programming language.

In the context of programming, a {\bf function} is a named sequence of
statements that performs a desired operation.  This operation is specified
in a {\bf function definition}.  The functions we have been using so far have
been defined for us, and these definitions have been hidden.  This is a good
thing, because it allows us to use the functions without worrying about the
details of their definitions.

\index{function}
\index{function definition}
\index{definition!function}

The syntax for a function definition is:

\beforeverb
\begin{verbatim}
def NAME( LIST OF PARAMETERS ):
  STATEMENTS
\end{verbatim}
\afterverb
%
You can make up any names you want for the functions you create, except that
you can't use a name that is a Python keyword.  The list of parameters
specifies what information, if any, you have to provide in order to
use the new function.

There can be any number of statements inside the function, but they
have to be indented from the left margin.  In the examples in this
book, we will use an indentation of two spaces.

The first couple of functions we are going to write have no parameters,
so the syntax looks like this:

\beforeverb
\begin{verbatim}
def newLine():
  print
\end{verbatim}
\afterverb
%
This function is named {\tt newLine}.  The empty parentheses indicate
that it has no parameters.  It contains only a single statement, which
outputs a newline character. (That's what happens when you use a {\tt print}
command without any arguments.)

The syntax for calling the new function is the same as the syntax
for built-in functions:

\beforeverb
\begin{verbatim}
print "First Line."
newLine()
print "Second Line."
\end{verbatim}
\afterverb
%
The output of this program is:

\beforeverb
\begin{verbatim}
First line.

Second line.
\end{verbatim}
\afterverb
%
Notice the extra space between the two lines.  What if we wanted
more space between the lines?  We could call the same function
repeatedly:

\beforeverb
\begin{verbatim}
print "First Line."
newLine()
newLine()
newLine()
print "Second Line."
\end{verbatim}
\afterverb
%
Or we could write a new function named {\tt threeLines} that prints
three new lines:

\beforeverb
\begin{verbatim}
def threeLines():
  newLine()
  newLine()
  newLine()

print "First Line."
threeLines()
print "Second Line."
\end{verbatim}
\afterverb
%
This function contains three statements, all of which are indented by two
spaces.  Since the next statement is not indented, Python knows that it is
not part of the function.

You should notice a few things about this program:

\begin{enumerate}

\item You can call the same procedure repeatedly.  In fact, it is quite common
and useful to do so.

\item You can have one function call another function; in this case
{\tt threeLines} calls {\tt newLine}.

\end{enumerate}

So far, it may not be clear why it is worth the trouble to create all of these
new functions.  Actually, there are a lot of reasons, but this example
demonstrates two:

\begin{itemize}

\item Creating a new function gives you an opportunity to name a
group of statements.  Functions can simplify a program by hiding a complex
computation behind a single command and by using English words in place of
arcane code.

\item Creating a new function can make a program smaller by eliminating
repetitive code.  For example, a short way to print nine consecutive new
lines is to call {\tt threeLines} three times.

\end{itemize}

\begin{quote}
{\em As an exercise, write a function called {\tt nineLines} that uses
{\tt threeLines} to print nine blank lines. How would you print 
twenty-seven new lines?}
\end{quote}


\section{Definitions and use}

Pulling together the code fragments from Section 3.6, the
whole program looks like this:

\beforeverb
\begin{verbatim}
def newLine():
  print

def threeLines():
  newLine()
  newLine()
  newLine()

print "First Line."
threeLines()
print "Second Line."
\end{verbatim}
\afterverb
%
This program contains two function definitions: {\tt newLine} and
{\tt threeLines}.  Function definitions get executed just like other
statements, but the effect is to create the new function.  The statements
inside the function do not get executed until the function is called, and
the function definition generates no output.

As you might expect, you have to create a function before you can
execute it.  In other words, the function definition has to be
executed before the first time it is called.

\begin{quote}
{\em As an exercise, move the last three lines of this program
to the top, so the function calls appear before the definitions. Run 
the program and see what error
message you get.}
\end{quote}

\begin{quote}
{\em As another exercise, start with the working version of the program
and move the definition of {\tt newLine} after the definition of
{\tt threeLines}.  What happens when you run this program?}
\end{quote}

\section{Flow of execution}
\index{flow of execution}

In order to ensure that a function is defined before its first use,
you have to know the order in which statements are executed, which is
called the {\bf flow of execution}.

Execution always begins at the first statement of the program.  Statements are
executed one at a time, in order from top to bottom.

Function definitions do not alter the flow of execution of the program, but
remember that statements inside the function are not executed until the
function is called.  Although it is not common, you can define one function
inside another.  In this case, the inner definition isn't executed until the
outer function is called.

Function calls are like a detour in the flow of execution. Instead of going to
the next statement, the flow jumps to the first line of the called function,
executes all the statements there, and then comes back to pick up where it left
off.

That sounds simple enough, until you remember that one function can call
another.  While in the middle of one function, the program might have to
execute the statements in another function. But while executing that
new function, the program might have to execute yet another function!

Fortunately, Python is adept at keeping track of where it is, so each time a
function completes, the program picks up where it left off in the function that
called it.  When it gets to the end of the program, it terminates.

What's the moral of this sordid tale?  When you read a program, don't read from
top to bottom.  Instead, follow the flow of execution.


\section{Parameters and arguments}
\label{parameters}
\index{parameter}
\index{function!parameter}
\index{argument}
\index{function!argument}

Some of the built-in functions you have used require arguments, the
values that control how the function does its job.  For example, if
you want to find the sine of a number, you have to indicate what the
number is.  Thus, {\tt sin} takes a numeric value as an argument.

Some functions take more than one argument. For example, {\tt pow}
takes two arguments, the base and the exponent.  Inside the function,
the values that are passed get assigned to variables called {\bf parameters}.

Here is an example of a user-defined function that has a parameter:

\beforeverb
\begin{verbatim}
def printTwice(bruce):
  print bruce, bruce
\end{verbatim}
\afterverb
%
This function takes a single argument and assigns it to a parameter
named {\tt bruce}.  The value of the parameter (at this point we
have no idea what it will be) is printed twice, followed by a newline.
The name {\tt bruce} was chosen to suggest that the name you give a
parameter is up to you, but in general, you want to choose something
more illustrative than {\tt bruce}.

The function {\tt printTwice} works for any type that can be printed:

\beforeverb
\begin{verbatim}
>>> printTwice('Spam')
Spam Spam
>>> printTwice(5)
5 5
>>> printTwice(3.14159)
3.14159 3.14159
\end{verbatim}
\afterverb
%
In the first function call, the argument is a string. In the second,
it's an integer. In the third, it's a {\tt float}.

The same rules of composition that apply to built-in functions also
apply to user-defined functions, so we can use any kind of expression
as an argument for {\tt printTwice}:

\beforeverb
\begin{verbatim}
>>> printTwice('Spam'*4)
SpamSpamSpamSpam SpamSpamSpamSpam
>>> printTwice(math.cos(math.pi))
-1.0 -1.0
\end{verbatim}
\afterverb
%
As usual, the expression is evaluated before the function is run, so
{\tt printTwice} prints {\tt SpamSpamSpamSpam SpamSpamSpamSpam} instead of
{\tt 'Spam'*4 'Spam'*4}.

\begin{quote}
{\em As an exercise, write a call to {\tt printTwice} that does print
{\tt 'Spam'*4 'Spam'*4}.  Hint: strings can be enclosed in either single or 
double quotes, and the type of quote not used to enclose the string can be used
inside it as part of the string.}
\end{quote}

We can also use a variable as an argument:

\beforeverb
\begin{verbatim}
>>> michael = 'Eric, the half a bee.'
>>> printTwice(michael)
Eric, the half a bee. Eric, the half a bee.
\end{verbatim}
\afterverb
%
Notice something very important here. The name of the variable we pass
as an argument ({\tt michael}) has nothing to do with the name of the
parameter ({\tt bruce}).  It doesn't matter what the value was
called back home (in the caller); here in {\tt printTwice}, we call
everybody {\tt bruce}.


\section{Variables and parameters are local}
\index{local variable}
\index{variable!local}

When you create a {\bf local variable} inside a function, it only exists inside
the function, and you cannot use it outside.  For example:

\beforeverb
\begin{verbatim}
def catTwice(part1, part2):
  cat = part1 + part2
  printTwice(cat)
\end{verbatim}
\afterverb
%
This function takes two arguments, concatenates them, and then prints
the result twice.
We can call the function with two strings:

\beforeverb
\begin{verbatim}
>>> chant1 = "Pie Jesu domine, "
>>> chant2 = "Dona eis requiem."
>>> catTwice(chant1, chant2)
Pie Jesu domine, Dona eis requiem. Pie Jesu domine, Dona eis requiem.
\end{verbatim}
\afterverb
%
When {\tt catTwice} terminates, the variable {\tt cat} is destroyed.
If we try to print it, we get an error:

\beforeverb
\begin{verbatim}
>>> print cat
NameError: cat
\end{verbatim}
\afterverb
%
Parameters are also local.
For example, outside the function {\tt printTwice}, there is no
such thing as {\tt bruce}.  If you try to use it, Python will
complain.


\section{Stack diagrams}
\label{stackdiagram}
\index{stack diagram}
\index{function frame}
\index{frame}

To keep track of which variables can be used where, it is sometimes
useful to draw a {\bf stack diagram}.  Like state diagrams, stack
diagrams show the value of each variable, but they also show the
function to which each variable belongs.

Each function is represented by a {\bf frame}.  A frame is a box
with the name of a function
beside it and the parameters and variables of the function inside it.
The stack diagram for the
previous example looks like this:

\adjustpage{-4}
\beforefig
\centerline{\psfig{figure=illustrations/stack.eps}}
\afterfig

The order of the stack shows the flow of execution.  {\tt printTwice}
was called by {\tt catTwice}, and {\tt catTwice} was called by {\tt
\_\_main\_\_}, which is a special name for the topmost function.  When
you create a variable outside of any function, it belongs to {\tt
\_\_main\_\_}.

Each parameter refers to the same value as its corresponding
argument.  So, {\tt part1} has the same value as
{\tt chant1}, {\tt part2} has the same value as {\tt chant2},
and {\tt bruce} has the same value as {\tt cat}.

If an error occurs during a function call, Python prints the
name of the function, and the name of the function that called
it, and the name of the function that called {\em that}, all the
way back to {\tt \_\_main\_\_}.

For example, if we try to access {\tt cat} from within {\tt
printTwice}, we get a {\tt NameError}:

\beforeverb
\begin{verbatim}
Traceback (innermost last):
  File "test.py", line 13, in __main__
    catTwice(chant1, chant2)
  File "test.py", line 5, in catTwice
    printTwice(cat)
  File "test.py", line 9, in printTwice
    print cat
NameError: cat
\end{verbatim}
\afterverb
%
This list of functions is called a {\bf traceback}.  It tells you what
program file the error occurred in, and what line, and what functions
were executing at the time.  It also shows the line of code that
caused the error.

\index{traceback}

Notice the similarity between the traceback and the
stack diagram.  It's not a coincidence.


\section{Functions with results}

You might have noticed by now that some of the functions we are using,
such as the math functions, yield results.  Other functions, like
{\tt newLine}, perform an action but don't return a value.  That raises
some questions:

\begin{enumerate}

\item What happens if you call a function and you don't do anything
with the result (i.e., you don't assign it to a variable or use it as
part of a larger expression)?

\item What happens if you use a function without a result as part of
an expression, such as {\tt newLine() + 7}?

\item Can you write functions that yield results, or are you stuck with
simple function like {\tt newLine} and {\tt printTwice}?

\end{enumerate}

The answer to the last question is that you can write functions that
yield results, and we'll do it in Chapter 5.

\begin{quote}
{\em As an exercise, answer the other two questions by trying them
out.  When you have a question about what is legal or illegal in
Python, a good way to find out is to ask the interpreter.}
\end{quote}


\section{Glossary}

\begin{description}

\item[function call:]  A statement that executes a function. It consists of
the name of the function followed by a list of arguments enclosed in
parentheses.

\item[argument:]  A value provided to a function when the function is called.
This value is assigned to the corresponding parameter in the function.

\item[return value:]  The result of a function.  If a function call
is used as an expression, the return value is the value of
the expression.

\item[type conversion:] An explicit statement that takes a value
of one type and computes a corresponding value of another type.

\item[type coercion:]  A type conversion that happens automatically
according to Python's coercion rules.

\item[module:]  A file that contains a collection of related functions and
classes.

\item[dot notation:]  The syntax for calling a function in another
module, specifying the module name followed by a dot (period) and
the function name.

\item[function:]  A named sequence of statements that performs some useful
operation.  Functions may or may not take arguments and may or may not
produce a result.

\item[function definition:]  A statement that creates a new function,
specifying its name, parameters, and the statements it executes.

\item[flow of execution:]  The order in which statements are executed during
a program run.

\item[parameter:]  A name used inside a function to refer to the value passed
as an argument.

\item[local variable:]  A variable defined inside a function.  A local
variable can only be used inside its function.

\item[stack diagram:]  A graphical representation of a stack of functions,
their variables, and the values to which they refer.

\item[frame:]  A box in a stack diagram that represents a function call.
It contains the local variables and parameters of the function.

\item[traceback:]  A list of the functions that are executing,
printed when a runtime error occurs.

\index{function call}
\index{return value}
\index{argument}
\index{coercion}
\index{module}
\index{dot notation}
\index{function}
\index{function definition}
\index{flow of execution}
\index{parameter}
\index{local variable}
\index{stack diagram}
\index{function frame}
\index{frame}
\index{traceback}

\end{description}

\clearemptydoublepage
% LaTeX source for textbook ``How to think like a computer scientist''
% Copyright (c)  2001  Allen B. Downey, Jeffrey Elkner, and Chris Meyers.

% Permission is granted to copy, distribute and/or modify this
% document under the terms of the GNU Free Documentation License,
% Version 1.1  or any later version published by the Free Software
% Foundation; with the Invariant Sections being "Contributor List",
% with no Front-Cover Texts, and with no Back-Cover Texts. A copy of
% the license is included in the section entitled "GNU Free
% Documentation License".

% This distribution includes a file named fdl.tex that contains the text
% of the GNU Free Documentation License.  If it is missing, you can obtain
% it from www.gnu.org or by writing to the Free Software Foundation,
% Inc., 59 Temple Place - Suite 330, Boston, MA 02111-1307, USA.
\chapter{Conditionals and recursion}

\section{The modulus operator}
\index{modulus operator}
\index{operator!modulus}

The {\bf modulus operator} works on integers (and integer expressions)
and yields the remainder when the first operand is divided by the
second.  In Python, the modulus operator is a percent sign ({\tt
\%}).  The syntax is the same as for other operators:

\beforeverb
\begin{verbatim}
>>> quotient = 7 / 3
>>> print quotient
2
>>> remainder = 7 % 3
>>> print remainder
1
\end{verbatim}
\afterverb
%
So 7 divided by 3 is 2 with 1 left over.

The modulus operator turns out to be surprisingly useful.  For
example, you can check whether one number is divisible by another---if
{\tt x \% y} is zero, then {\tt x} is divisible by {\tt y}.

Also, you can extract the right-most digit
or digits from a number.  For example, {\tt x \% 10} yields the
right-most digit of {\tt x} (in base 10).  Similarly {\tt x \% 100}
yields the last two digits.

\adjustpage{1}

\section{Boolean expressions}
\index{boolean expression}
\index{expression!boolean}
\index{logical operator}
\index{operator!logical}

A {\bf boolean expression} is an expression that is either true
or false.  One way to write a boolean expression is to use the
operator {\tt ==}, which compares two values and produces a boolean
value:

\beforeverb
\begin{verbatim}
>>> 5 == 5
True
>>> 5 == 6
False
\end{verbatim}
\afterverb
%
In the first statement, the two operands are equal, so the value of
the expression is {\tt True}; in the second statement, 5 is not equal
to 6, so we get {\tt False}.  {\tt True} and {\tt False} are special
values that are built into Python.

The {\tt ==} operator is one of the {\bf comparison operators}; the
others are:

\beforeverb
\begin{verbatim}
      x != y               # x is not equal to y
      x > y                # x is greater than y
      x < y                # x is less than y
      x >= y               # x is greater than or equal to y
      x <= y               # x is less than or equal to y
\end{verbatim}
\afterverb
%
Although these operations are probably familiar to you, the Python
symbols are different from the mathematical symbols.  A common error
is to use a single equal sign ({\tt =}) instead of a double equal sign
({\tt ==}).  Remember that {\tt =} is an assignment operator and
{\tt ==} is a comparison operator.   Also, there is no such thing as
{\tt =<} or {\tt =>}.


\section {Logical operators}
\index{logical operator}
\index{operator!logical}

There are three {\bf logical operators}: {\tt and}, {\tt
or}, and {\tt not}.  The semantics (meaning) of these operators is
similar to their meaning in English.  For example,
{\tt x > 0 and x < 10} is true only if {\tt x} is greater than 0
{\em and} less than 10.

{\tt n\%2 == 0 or n\%3 == 0} is true if {\em either} of the conditions
is true, that is, if the number is divisible by 2 {\em or} 3.

Finally, the {\tt not} operator negates a boolean
expression, so {\tt not(x > y)} is true if {\tt (x > y)} is false,
that is, if {\tt x} is less than or equal to {\tt y}.

Strictly speaking, the operands of the logical operators should be
boolean expressions, but Python is not very strict.
Any nonzero number is interpreted as ``true.''

\beforeverb
\begin{verbatim}
>>>  x = 5
>>>  x and 1
1
>>>  y = 0
>>>  y and 1
0
\end{verbatim}
\afterverb
%
In general, this sort of thing is not considered good style.  If you
want to compare a value to zero, you should do it explicitly.


\section{Conditional execution}
\label{conditional execution}
\index{conditional branching}
\index{conditional execution}

In order to write useful programs, we almost always need the ability
to check conditions and change the behavior of the program
accordingly.  {\bf Conditional statements} give us this ability.  The
simplest form is the {\tt if} statement:

\beforeverb
\begin{verbatim}
if x > 0:
  print "x is positive"
\end{verbatim}
\afterverb
%
The boolean expression after the {\tt if} statement is
called the {\bf condition}.  If it is true, then the indented
statement gets executed.  If not, nothing happens.

\index{compound statement}
\index{compound statement!header}
\index{compound statement!body}
\index{compound statement!statement block}
\index{statement!compound}

Like other compound statements, the
{\tt if} statement is made up of a header and a block
of statements:

\beforeverb
\begin{verbatim}
HEADER:
  FIRST STATEMENT
  ...
  LAST STATEMENT
\end{verbatim}
\afterverb
%
The header begins on a new line and ends with a colon (:).  The
indented statements that follow are called a {\bf block}.
The first unindented statement marks the end of the block.
A statement block inside a compound statement is called the {\bf body}
of the statement.

\index{block}
\index{statement!block}
\index{body}

There is no limit on the number of statements that can appear in
the body of an if statement, but there has to be at least one.
Occasionally, it is useful to have a body with no statements (usually
as a place keeper for code you haven't written yet).  In that
case, you can use the {\tt pass} statement, which does nothing.

\index{pass statement}
\index{statement!pass}



\section{Alternative execution}
\label{alternative execution}

A second form of the {\tt if} statement is alternative execution,
in which there are two possibilities and the condition determines
which one gets executed.  The syntax looks like this:

\beforeverb
\begin{verbatim}
if x%2 == 0:
  print x, "is even"
else:
  print x, "is odd"
\end{verbatim}
\afterverb
%
If the remainder when {\tt x} is divided by 2 is 0, then we
know that {\tt x} is even, and the program displays a message to that
effect.  If the condition is false, the second set of statements is
executed.  Since the condition must be true or false, exactly one of
the alternatives will be executed.  The alternatives are called
{\bf branches}, because they are branches in the flow of execution.

\index{branch}

As an aside, if you need to check the parity (evenness or
oddness) of numbers often, you might ``wrap'' this code in a
function:

\beforeverb
\begin{verbatim}
def printParity(x):
  if x%2 == 0:
    print x, "is even"
  else:
    print x, "is odd"
\end{verbatim}
\afterverb
%
For any value of {\tt x}, {\tt printParity} displays an
appropriate message.
When you call it, you can provide any integer expression
as an argument.

\beforeverb
\begin{verbatim}
>>> printParity(17)
17 is odd
>>> y = 17
>>> printParity(y+1)
18 is even
\end{verbatim}
\afterverb
%


\section{Chained conditionals}
\index{chained conditional}
\index{conditional!chained}

Sometimes there are more than two possibilities and we need more than
two branches.  One way to express a computation like that is a {\bf
chained conditional}:

\beforeverb
\begin{verbatim}
if x < y:
  print x, "is less than", y
elif x > y:
  print x, "is greater than", y
else:
  print x, "and", y, "are equal"
\end{verbatim}
\afterverb
%
{\tt elif} is an abbreviation of ``else if.''  Again, exactly one
branch will be executed.  There is no limit of the number of {\tt
elif} statements, but the last branch has to be an {\tt else}
statement:

\adjustpage{3}
\beforeverb
\begin{verbatim}
if choice == 'A':
  functionA()
elif choice == 'B':
  functionB()
elif choice == 'C':
  functionC()
else:
  print "Invalid choice."
\end{verbatim}
\afterverb
%
Each condition is checked in order.  If the first is false,
the next is checked, and so on.  If one of them is
true, the corresponding branch executes, and the statement
ends.  Even if more than one condition is true, only the
first true branch executes.  

\begin{quote}
\begin{quote}
{\em As an exercise, wrap these examples in functions
called {\tt compare(x, y)} and {\tt dispatch(choice)}.}
\end{quote}
\end{quote}


\section{Nested conditionals}

One conditional can also be nested within another.  We could have
written the trichotomy example as follows:

\beforeverb
\begin{verbatim}
if x == y:
  print x, "and", y, "are equal"
else:
  if x < y:
    print x, "is less than", y
  else:
    print x, "is greater than", y
\end{verbatim}
\afterverb
%
The outer conditional contains two branches.  The
first branch contains a simple output statement.  The second branch
contains another {\tt if} statement, which has two branches of its
own.  Those two branches are both output statements,
although they could have been conditional statements as well.

Although the indentation of the statements makes the structure
apparent, nested conditionals become difficult to read very
quickly. In general, it is a good idea to avoid them when you can.

Logical operators often provide a way to simplify nested conditional
statements.  For example, we can rewrite the following code using a
single conditional:

\beforeverb
\begin{verbatim}
if 0 < x:
  if x < 10:
    print "x is a positive single digit."
\end{verbatim}
\afterverb
%
The {\tt print} statement is executed only if we make it past both the
conditionals, so we can use the {\tt and} operator:

\beforeverb
\begin{verbatim}
if 0 < x and x < 10:
  print "x is a positive single digit."
\end{verbatim}
\afterverb
%
These kinds of conditions are common, so Python provides an
alternative syntax that is similar to mathematical notation:

\beforeverb
\begin{verbatim}
if 0 < x < 10:
  print "x is a positive single digit."
\end{verbatim}
\afterverb
%
This condition is semantically the same as
the compound boolean expression and the nested conditional.


\section{The {\tt return} statement}
\index{return statement}
\index{statement!return}

The {\tt return} statement allows you to terminate the execution of a
function before you reach the end.  One reason to use it is if you
detect an error condition:

\beforeverb
\begin{verbatim}
import math

def printLogarithm(x):
  if x <= 0:
    print "Positive numbers only, please."
    return

  result = math.log(x)
  print "The log of x is", result
\end{verbatim}
\afterverb
%
The function {\tt printLogarithm} has a
parameter named {\tt x}.  The first thing it does is check whether
{\tt x} is less than or equal to 0, in which case it displays an
error message and then uses {\tt return} to exit the function. The
flow of execution immediately returns to the caller, and the remaining
lines of the function are not executed.

Remember that to use a function from the math
module, you have to import it.


\section{Recursion}
\label{recursion}
\index{recursion}

We mentioned that it is legal for one function to call another, and
you have seen several examples of that.  We neglected to mention that
it is also legal for a function to call itself.  It may not be obvious
why that is a good thing, but it turns out to be one of the most
magical and interesting things a program can do.
For example, look at the following function:

\beforeverb
\begin{verbatim}
def countdown(n):
  if n == 0:
    print "Blastoff!"
  else:
    print n
    countdown(n-1)
\end{verbatim}
\afterverb
%
{\tt countdown} expects the parameter, {\tt n}, to be a positive
integer.  If {\tt n} is 0, it outputs the word, ``Blastoff!''
Otherwise, it outputs {\tt n} and then calls a function named
{\tt countdown}---itself---passing {\tt n-1} as an argument.

What happens if we call this function like this:

\beforeverb
\begin{verbatim}
>>> countdown(3)
\end{verbatim}
\afterverb
%
The execution of {\tt countdown} begins with {\tt n=3}, and since
{\tt n} is not 0, it outputs the value 3, and then calls itself...

\begin{quote}
The execution of {\tt countdown} begins with {\tt n=2}, and since
{\tt n} is not 0, it outputs the value 2, and then calls itself...

\begin{quote}
The execution of {\tt countdown} begins with {\tt n=1}, and since
{\tt n} is not 0, it outputs the value 1, and then calls itself...

\begin{quote}
The execution of {\tt countdown} begins with {\tt n=0}, and since
{\tt n} is 0, it outputs the word, ``Blastoff!'' and then returns.
\end{quote}

The {\tt countdown} that got {\tt n=1} returns.
\end{quote}

The {\tt countdown} that got {\tt n=2} returns.
\end{quote}

The {\tt countdown} that got {\tt n=3} returns.

And then you're back in {\tt \_\_main\_\_} (what a trip).  So, the
total output looks like this:

\beforeverb
\begin{verbatim}
3
2
1
Blastoff!
\end{verbatim}
\afterverb
%
As a second example, look again at the functions {\tt newLine} and
{\tt threeLines}:

\beforeverb
\begin{verbatim}
def newline():
  print

def threeLines():
  newLine()
  newLine()
  newLine()
\end{verbatim}
\afterverb
%
Although these work, they would not be much help if we wanted to output 2
newlines, or 106.  A better alternative would be this:

\beforeverb
\begin{verbatim}
def nLines(n):
  if n > 0:
    print
    nLines(n-1)
\end{verbatim}
\afterverb
%
This program is similar to {\tt countdown}; as long as {\tt n} is
greater than 0, it outputs one newline and then calls itself to
output {\tt n-1} additional newlines.  Thus, the total number of
newlines is {\tt 1 + (n - 1)} which, if you do your algebra right, comes
out to {\tt n}.

The process of a function calling itself is {\bf recursion}, and
such functions are said to be recursive.

\index{recursion}
\index{function!recursive}


\section{Stack diagrams for recursive functions}
\index{stack diagram}
\index{function frame}
\index{frame}

In Section~\ref{stackdiagram}, we used a stack diagram to represent
the state of a program during a function call.  The same kind of
diagram can help interpret a recursive function.

Every time a function gets called, Python creates a new function
frame, which contains the function's local variables and parameters.
For a recursive function, there might be more than one frame on the
stack at the same time.

This figure shows a stack diagram for {\tt countdown} called with
{\tt n = 3}:

\beforefig
\centerline{\psfig{figure=illustrations/stack2.eps}}
\afterfig

As usual, the top of the stack is the frame for {\tt \_\_main\_\_}.
It is empty because we did not create any variables in {\tt
\_\_main\_\_} or pass any arguments to it.

The four {\tt countdown} frames have different values for the
parameter {\tt n}.  The bottom of the stack, where {\tt n=0}, is
called the {\bf base case}.  It does not make a recursive call, so
there are no more frames.

\adjustpage{2}
\begin{quote}
{\em As an exercise, draw a stack diagram for {\tt nLines} called with
{\tt n=4}.}
\end{quote}

\index{base case}
\index{recursion!base case}


\section{Infinite recursion}
\index{infinite recursion}
\index{recursion!infinite}
\index{runtime error}
\index{error!runtime}
\index{traceback}

If a recursion never reaches a base case, it goes on making
recursive calls forever, and the program never terminates.  This is
known as {\bf infinite recursion}, and it is generally not considered
a good idea.  Here is a minimal program with an infinite recursion:

\beforeverb
\begin{verbatim}
def recurse():
  recurse()
\end{verbatim}
\afterverb
%
In most programming environments, a program with infinite recursion
does not really run forever.  Python reports an error
message when the maximum recursion depth is reached:

\beforeverb
\begin{verbatim}
  File "<stdin>", line 2, in recurse
  (98 repetitions omitted)
  File "<stdin>", line 2, in recurse
RuntimeError: Maximum recursion depth exceeded
\end{verbatim}
\afterverb
%
This traceback is a little bigger than the one we saw in the
previous chapter.  When the error occurs, there are 100
{\tt recurse} frames on the stack!

\begin{quote}
{\em As an exercise, write a function with infinite recursion and run
it in the Python interpreter.}
\end{quote}


\section{Keyboard input}

The programs we have written so far are a bit rude in the sense that
they accept no input from the user.  They just do the same thing every
time.

Python provides built-in functions that get input from the keyboard.
The simplest is called {\tt raw\_input}.  When this function is
called, the program stops and waits for the user to type something.
When the user presses Return or the Enter key, the program resumes and
{\tt raw\_input} returns what the user typed as a {\bf string}:

\beforeverb
\begin{verbatim}
>>> input = raw_input ()
What are you waiting for?
>>> print input
What are you waiting for?
\end{verbatim}
\afterverb
%
Before calling {\tt raw\_input}, it is a good idea to print a message
telling the user what to input.  This message is called a {\bf
prompt}.  We can supply a prompt as an argument to {\tt raw\_input}:

\index{prompt}

\beforeverb
\begin{verbatim}
>>> name = raw_input ("What...is your name? ")
What...is your name? Arthur, King of the Britons!
>>> print name
Arthur, King of the Britons!
\end{verbatim}
\afterverb
%
If we expect the response to be an integer, we can use the
{\tt input} function:

\beforeverb
\begin{verbatim}
prompt = "What...is the airspeed velocity of an unladen swallow?\n"
speed = input(prompt)
\end{verbatim}
\afterverb
%
The sequence \verb+\n+ at the end of the string represents a newline,
so the user's input appears below the prompt.

If the user types a string of digits, it is converted to an
integer and assigned to {\tt speed}.  Unfortunately, if the user
types a character that is not a digit, the program crashes:

\beforeverb
\begin{verbatim}
>>> speed = input (prompt)
What...is the airspeed velocity of an unladen swallow?
What do you mean, an African or a European swallow?
SyntaxError: invalid syntax
\end{verbatim}
\afterverb
%
To avoid this kind of error, it is generally a good idea to
use {\tt raw\_input} to get a string and then use conversion
functions to convert to other types.


\section{Glossary}

\begin{description}

\item[modulus operator:]  An operator, denoted with a percent sign
({\tt \%}), that works on integers and yields the remainder when one
number is divided by another.

\item[boolean expression:]  An expression that is either true or false.

\item[comparison operator:] One of the operators that compares two
values: {\tt ==}, {\tt !=}, {\tt >}, {\tt <}, {\tt >=}, and {\tt <=}.

\item[logical operator:] One of the operators that combines boolean
expressions: {\tt and}, {\tt or}, and {\tt not}.

\item[conditional statement:]  A statement that controls the flow of
execution depending on some condition.

\item[condition:] The boolean expression in a conditional statement
that determines which branch is executed.

\item[compound statement:]  A statement that consists of a header
and a body.  The header ends with a colon (:).  The body is indented
relative to the header.

\item[block:] A group of consecutive statements with the same
indentation.

\item[body:] The block in a compound statement that follows the
header.

\item[nesting:]  One program structure within another, such as a
conditional statement inside a branch of another conditional
statement.

\item[recursion:]  The process of calling the function that is
currently executing.

\item[base case:]  A branch of the conditional statement in a
recursive function that does not result in a recursive call.

\item[infinite recursion:]  A function that calls itself recursively
without ever reaching the base case.  Eventually, an infinite recursion
causes a runtime error.

\item[prompt:]  A visual cue that tells the user to input data.

\index{modulus operator}
\index{boolean expression}
\index{expression!boolean}
\index{conditional statement}
\index{statement!conditional}
\index{condition}
\index{compound statement}
\index{branch}
\index{body}
\index{block}
\index{nesting}
\index{recursion}
\index{base case}
\index{infinite recursion}
\index{prompt}

\end{description}

\clearemptydoublepage
% LaTeX source for textbook ``How to think like a computer scientist''
% Copyright (c)  2001  Allen B. Downey, Jeffrey Elkner, and Chris Meyers.

% Permission is granted to copy, distribute and/or modify this
% document under the terms of the GNU Free Documentation License,
% Version 1.1  or any later version published by the Free Software
% Foundation; with the Invariant Sections being "Contributor List",
% with no Front-Cover Texts, and with no Back-Cover Texts. A copy of
% the license is included in the section entitled "GNU Free
% Documentation License".

% This distribution includes a file named fdl.tex that contains the text
% of the GNU Free Documentation License.  If it is missing, you can obtain
% it from www.gnu.org or by writing to the Free Software Foundation,
% Inc., 59 Temple Place - Suite 330, Boston, MA 02111-1307, USA.
\chapter{Fruitful functions}

\section{Return values}
\index{return value}

Some of the built-in functions we have used, such as the math
functions, have produced results.  Calling the function generates a
new value, which we usually assign to a variable or use as part of an
expression.

\beforeverb
\begin{verbatim}
e = math.exp(1.0)
height = radius * math.sin(angle)
\end{verbatim}
\afterverb
%
But so far, none of the functions we have written has returned a
value.

In this chapter, we are going to write functions that return values,
which we will call {\bf fruitful functions}, for want of a better
name.  The first example is {\tt area}, which returns the area of a
circle with the given radius:

\beforeverb
\begin{verbatim}
import math

def area(radius):
  temp = math.pi * radius**2
  return temp
\end{verbatim}
\afterverb
%
We have seen the {\tt return} statement before, but in a fruitful
function the {\tt return} statement includes
a {\bf return value}.  This statement means: ``Return immediately from
this function and use the following expression as a return value.''
The expression provided can be arbitrarily complicated, so we could
have written this function more concisely:

\beforeverb
\begin{verbatim}
def area(radius):
  return math.pi * radius**2
\end{verbatim}
\afterverb
%
On the other hand, {\bf temporary variables} like {\tt temp} often make
debugging easier.

\index{temporary variable}
\index{variable!temporary}

Sometimes it is useful to have multiple return statements, one in each
branch of a conditional:

\beforeverb
\begin{verbatim}
def absoluteValue(x):
  if x < 0:
    return -x
  else:
    return x
\end{verbatim}
\afterverb
%
Since these {\tt return} statements are in an alternative conditional,
only one will be executed.  As soon as one is executed, the function
terminates without executing any subsequent statements.

Code that appears after a {\tt return} statement, or any other place
the flow of execution can never reach, is called {\bf dead code}.

\index{dead code}

In a fruitful function, it is a good idea to ensure
that every possible path through the program hits a
{\tt return} statement.  For example:

\beforeverb
\begin{verbatim}
def absoluteValue(x):
  if x < 0:
    return -x
  elif x > 0:
    return x
\end{verbatim}
\afterverb
%
This program is not correct because if {\tt x} happens to be 0,
neither condition is true, and the function ends without hitting a
{\tt return} statement.
In this case, the return value is a special value called
{\tt None}:

\index{None}

\beforeverb
\begin{verbatim}
>>> print absoluteValue(0)
None
\end{verbatim}
\afterverb
%
\begin{quote}
\begin{quote}
{\em As an exercise, write a {\tt compare} function
that returns {\tt 1} if {\tt x > y},
{\tt 0} if {\tt x == y}, and {\tt -1} if {\tt x < y}.}
\end{quote}
\end{quote}


\section{Program development}
\label{program development}
\index{scaffolding}

At this point, you should be able to look at complete functions
and tell what they do.  Also, if you have been doing the exercises,
you have written some small functions.  As you write
larger functions, you might start to have more difficulty,
especially with runtime and semantic errors.

To deal with increasingly complex programs,
we are going to suggest a technique called
{\bf incremental development}.  The goal of incremental development
is to avoid long debugging sessions by adding and testing only
a small amount of code at a time.

\index{incremental development}
\index{development!incremental}

As an example, suppose you want to find the distance between two
points, given by the coordinates $(x_1, y_1)$ and $(x_2, y_2)$.
By the Pythagorean theorem, the distance is:

\begin{displaymath}
\mathrm{distance} = \sqrt{(x_2 - x_1)^2 + (y_2 - y_1)^2}
\end{displaymath}
%
The first step is to consider what a {\tt distance} function should
look like in Python. In other words, what are the inputs (parameters)
and what is the output (return value)?

In this case, the two points are the inputs, which we can represent
using four parameters.  The return value is the distance, which is
a floating-point value.

Already we can write an outline of the function:

\beforeverb
\begin{verbatim}
def distance(x1, y1, x2, y2):
  return 0.0
\end{verbatim}
\afterverb
%
Obviously, this version of the function doesn't compute distances; it
always returns zero.  But it is syntactically correct, and it will
run, which means that we can test it before we make it more
complicated.

To test the new function, we call it with sample values:

\beforeverb
\begin{verbatim}
>>> distance(1, 2, 4, 6)
0.0
\end{verbatim}
\afterverb
%
We chose these values so that the horizontal distance equals 3 and the
vertical distance equals 4; that way, the result is 5
(the hypotenuse of a 3-4-5 triangle). When testing a function, it is
useful to know the right answer.

At this point we have confirmed that the function is syntactically
correct, and we can start adding lines of code.  After each
incremental change, we test the function again.  If an error occurs at
any point, we know where it must be---in the last line
we added.

A logical first step in the computation is to find the differences
$x_2 - x_1$ and $y_2 - y_1$.  We will store those values in
temporary variables named {\tt dx} and {\tt dy} and print them.

\beforeverb
\begin{verbatim}
def distance(x1, y1, x2, y2):
  dx = x2 - x1
  dy = y2 - y1
  print "dx is", dx
  print "dy is", dy
  return 0.0
\end{verbatim}
\afterverb
%
If the function is working, the outputs should be 3 and 4.  If so,
we know that the function is getting the right arguments and performing
the first computation correctly.  If not, there are only a few lines
to check.

Next we compute the sum of squares of {\tt dx} and {\tt dy}:

\beforeverb
\begin{verbatim}
def distance(x1, y1, x2, y2):
  dx = x2 - x1
  dy = y2 - y1
  dsquared = dx**2 + dy**2
  print "dsquared is: ", dsquared
  return 0.0
\end{verbatim}
\afterverb
%
Notice that we removed the {\tt print} statements we wrote in the previous
step.  Code like that is called {\bf scaffolding} because it is
helpful for building the program but is not part of the final product.

Again, we would run the program at this stage and check the output
(which should be 25).

Finally, if we have imported the math module, we can use the
{\tt sqrt} function to compute and return the result:

\beforeverb
\begin{verbatim}
def distance(x1, y1, x2, y2):
  dx = x2 - x1
  dy = y2 - y1
  dsquared = dx**2 + dy**2
  result = math.sqrt(dsquared)
  return result
\end{verbatim}
\afterverb
%
If that works correctly, you are done.  Otherwise, you might
want to print the value of {\tt result} before the return
statement.

When you start out, you should add only a line or two of code
at a time.
As you gain more experience, you might find yourself
writing and debugging bigger chunks.  Either way,
the incremental development process can save you a lot of debugging
time.

The key aspects of the process are:

\begin{enumerate}

\item Start with a working program and make small incremental changes. 
At any point, if there is an error, you will know exactly where it is.

\item Use temporary variables to hold intermediate values so you can
output and check them.

\item Once the program is working, you might want to remove some of
the scaffolding or consolidate multiple statements into compound
expressions, but only if it does not make the program difficult to
read.

\end{enumerate}

\begin{quote}
{\em As an exercise, use incremental development to write a function
called {\tt hypotenuse} that returns the length of the hypotenuse of a
right triangle given the lengths of the two legs as arguments.
Record each stage of the incremental development process as you go.}
\end{quote}


\section{Composition}
\index{composition}
\index{function!composition}

As you should expect by now, you can call one function from
within another.  This ability is called {\bf composition}.

As an example, we'll write a function that takes two points,
the center of the circle and a point on the perimeter, and computes
the area of the circle.

Assume that the center point is stored in the variables {\tt xc} and
{\tt yc}, and the perimeter point is in {\tt xp} and {\tt yp}. The
first step is to find the radius of the circle, which is the distance
between the two points.  Fortunately, there is a function, {\tt
distance}, that does that:

\beforeverb
\begin{verbatim}
radius = distance(xc, yc, xp, yp)
\end{verbatim}
\afterverb
%
The second step is to find the area of a circle with that radius and return
it:

\beforeverb
\begin{verbatim}
result = area(radius)
return result
\end{verbatim}
\afterverb
%
Wrapping that up in a function, we get:

\beforeverb
\begin{verbatim}
def area2(xc, yc, xp, yp):
  radius = distance(xc, yc, xp, yp)
  result = area(radius)
  return result
\end{verbatim}
\afterverb
%
We called this function {\tt area2} to distinguish it from the {\tt
area} function defined earlier.  There can only be one function with a
given name within a given module.

The temporary variables {\tt radius} and {\tt result} are useful for
development and debugging, but once the program is working, we can
make it more concise by composing the function calls:

\beforeverb
\begin{verbatim}
def area2(xc, yc, xp, yp):
  return area(distance(xc, yc, xp, yp))
\end{verbatim}
\afterverb
%
\begin{quote}
{\em As an exercise, write a function {\tt slope(x1, y1, x2, y2)}
that returns the slope of the line through the points $(x1, y1)$ and
$(x2, y2)$.  Then use this function in a function called
{\tt intercept(x1, y1, x2, y2)} that returns the y-intercept of the
line through the points {\tt (x1, y1)} and {\tt (x2, y2)}.}
\end{quote}


\section{Boolean functions}
\label{boolean}
\index{boolean function}
\index{function!boolean}

Functions can return boolean values, which is often convenient for hiding
complicated tests inside functions.  For example:

\beforeverb
\begin{verbatim}
def isDivisible(x, y):
  if x % y == 0:
    return True
  else:
    return False
\end{verbatim}
\afterverb
%
The name of this function is {\tt isDivisible}.  It is common to give
boolean functions names that sound like yes/no questions.  {\tt
isDivisible} returns either {\tt True} or {\tt False} to indicate whether the
{\tt x} is or is not divisible by {\tt y}.

We can make the function more concise by taking advantage of the fact
that the condition of the {\tt if} statement is itself a boolean
expression.  We can return it directly, avoiding the {\tt if}
statement altogether:

\beforeverb
\begin{verbatim}
def isDivisible(x, y):
  return x % y == 0
\end{verbatim}
\afterverb
%
This session shows the new function in action:

\beforeverb
\begin{verbatim}
>>>   isDivisible(6, 4)
False
>>>   isDivisible(6, 3)
True
\end{verbatim}
\afterverb
%
Boolean functions are often used in conditional statements:

\beforeverb
\begin{verbatim}
if isDivisible(x, y):
  print "x is divisible by y"
else:
  print "x is not divisible by y"
\end{verbatim}
\afterverb
%
It might be tempting to write something like:

\beforeverb
\begin{verbatim}
if isDivisible(x, y) == True:
\end{verbatim}
\afterverb
%
But the extra comparison is unnecessary.

\begin{quote}
{\em As an exercise, write a function {\tt isBetween(x, y, z)} that
returns {\tt True} if $y \le x \le z$ or {\tt False} otherwise.}
\end{quote}


\section{More recursion}
\index{recursion}
\index{complete language}
\index{language!complete}
\index{Turing, Alan}
\index{Turing Thesis}

So far, you have only learned a small subset of Python, but you might
be interested to know that this subset is a {\em complete}
programming language, which means that anything that can be
computed can be expressed in this language.  Any program ever written
could be rewritten using only the language features you have learned
so far (actually, you would need a few commands to control devices
like the keyboard, mouse, disks, etc., but that's all).

Proving that claim is a nontrivial exercise first accomplished by Alan
Turing, one of the first computer scientists (some would argue that he
was a mathematician, but a lot of early computer scientists started as
mathematicians).  Accordingly, it is known as the Turing Thesis.  If
you take a course on the Theory of Computation, you will have a chance
to see the proof.

\adjustpage{2}

To give you an idea of what you can do with the tools you have learned
so far, we'll evaluate a few recursively defined mathematical
functions.  A recursive definition is similar to a circular
definition, in the sense that the definition contains a reference to
the thing being defined.  A truly circular definition is not very
useful:

\begin{description}

\item[frabjuous:] An adjective used to describe something that is frabjuous.

\end{description}

\index{frabjuous}
\index{circular definition}
\index{definition!circular}

If you saw that definition in the dictionary, you might be annoyed. On
the other hand, if you looked up the definition of the mathematical
function factorial, you might get something like this:

\vspace{-0.35in}
\begin{eqnarray*}
&&  0! = 1 \\
&&  n! = n (n-1)!
\end{eqnarray*}
\vspace{-0.25in}

This definition says that the factorial of 0 is 1, and the factorial
of any other value, $n$, is $n$ multiplied by the factorial of $n-1$.

So $3!$ is 3 times $2!$, which is 2 times $1!$, which is 1 times
$0!$. Putting it all together, $3!$ equals 3 times 2 times 1 times 1,
which is 6.

\index{factorial function}
\index{function!factorial}

If you can write a recursive definition of something, you can usually
write a Python program to evaluate it. The first step is to decide
what the parameters are for this function.  With little effort, you
should conclude that {\tt factorial} has a single parameter:

\beforeverb
\begin{verbatim}
def factorial(n):
\end{verbatim}
\afterverb
%
If the argument happens to be 0, all we have to do is return 1:

\beforeverb
\begin{verbatim}
def factorial(n):
  if n == 0:
    return 1
\end{verbatim}
\afterverb
%
Otherwise, and this is the interesting part, we have to make a
recursive call to find the factorial of $n-1$ and then multiply it by
$n$:

\beforeverb
\begin{verbatim}
def factorial(n):
  if n == 0:
    return 1
  else:
    recurse = factorial(n-1)
    result = n * recurse
    return result
\end{verbatim}
\afterverb
%
The flow of execution for this program is similar to the flow of {\tt
countdown} in Section~\ref{recursion}.  If we call {\tt factorial} with the
value 3:

\adjustpage{1}

Since 3 is not 0, we take the second branch and calculate the factorial
of {\tt n-1}...

\begin{quote}
Since 2 is not 0, we take the second branch and calculate the factorial of
{\tt n-1}...


  \begin{quote}
  Since 1 is not 0, we take the second branch and calculate the factorial
  of {\tt n-1}...


    \begin{quote}
    Since 0 {\em is} 0, we take the first branch and return 1
    without making any more recursive calls.
    \end{quote}


  The return value (1) is multiplied by $n$, which is 1, and the
  result is returned.
  \end{quote}


The return value (1) is multiplied by $n$, which is 2, and the
result is returned.
\end{quote}


The return value (2) is multiplied by $n$, which is 3, and the result, 6,
becomes the return value of the function call that started the whole
process.

Here is what the stack diagram looks like for this sequence of function
calls:

\vspace{0.1in}
\beforefig
\centerline{\psfig{figure=illustrations/stack3.eps}}
\afterfig
\vspace{0.1in}

The return values are shown being passed back up the stack.
In each frame, the return value is
the value of {\tt result}, which is the product of {\tt n}
and {\tt recurse}.

Notice that in the last frame, the local
variables {\tt recurse} and {\tt result} do not exist, because
the branch that creates them did not execute.



\section{Leap of faith}
\index{recursion}
\index{leap of faith}

Following the flow of execution is one way to read programs, but
it can quickly become labyrinthine.  An
alternative is what we call the ``leap of faith.'' When you come to a
function call, instead of following the flow of execution, you {\em
assume} that the function works correctly and returns the appropriate
value.

In fact, you are already practicing this leap of faith when you use
built-in functions.  When you call {\tt math.cos} or {\tt math.exp},
you don't examine the implementations of those functions.  You just
assume that they work because the people who wrote the built-in
functions were good programmers.

The same is true when you call one of your own functions.  For example,
in Section~\ref{boolean}, we wrote a function called {\tt isDivisible}
that determines whether one number is divisible by another.  Once we
have convinced ourselves that this function is correct---by testing
and examining the code---we can use the function without looking
at the code again.

\adjustpage{-1}

The same is true of recursive programs.  When you get to the recursive
call, instead of following the flow of execution, you should assume
that the recursive call works (yields the correct result) and then ask
yourself, ``Assuming that I can find the factorial of $n-1$, can I
compute the factorial of $n$?''  In this case, it is clear that you
can, by multiplying by $n$.

Of course, it's a bit strange to assume that the function works
correctly when you haven't finished writing it, but that's why
it's called a leap of faith!


\section{One more example}
\label{one more example}

In the previous example, we used temporary variables to spell out
the steps and to make the code easier to debug, but we could have
saved a few lines:

\beforeverb
\begin{verbatim}
def factorial(n):
  if n == 0:
    return 1
  else:
    return n * factorial(n-1)
\end{verbatim}
\afterverb
%
From now on, we will tend to use the more concise form, but we
recommend that you use the more explicit version while you are developing
code.  When you have it working, you can tighten it up if you are
feeling inspired.

\index{Fibonacci function}

After {\tt factorial}, the most common example of a recursively defined
mathematical function is {\tt fibonacci}, which has the following definition:

\vspace{-0.25in}
\begin{eqnarray*}
&& \mathrm{fibonacci}(0) = 1 \\
&& \mathrm{fibonacci}(1) = 1 \\
&& \mathrm{fibonacci}(n) = \mathrm{fibonacci}(n-1) + \mathrm{fibonacci}(n-2);
\end{eqnarray*}
%
Translated into Python, it looks like this:

\beforeverb
\begin{verbatim}
def fibonacci (n):
  if n == 0 or n == 1:
    return 1
  else:
    return fibonacci(n-1) + fibonacci(n-2)
\end{verbatim}
\afterverb
%
If you try to follow the flow of execution here, even for fairly
small values of $n$, your head explodes.  But according to the
leap of faith, if you assume that the two recursive calls
work correctly, then it is clear that you get
the right result by adding them together.

\adjustpage{-1}

\section{Checking types}
\index{type checking}
\index{error checking}
\index{factorial function}

What happens if we call {\tt factorial} and give it 1.5 as an argument?

\beforeverb
\begin{verbatim}
>>> factorial (1.5)
RuntimeError: Maximum recursion depth exceeded
\end{verbatim}
\afterverb
%
It looks like an infinite recursion.  But how can that be?  There is a
base case---when {\tt n == 0}.  The problem is that the values
of {\tt n} {\em miss} the base case.

\index{infinite recursion}
\index{recursion!infinite}

In the first recursive call, the value of {\tt n} is 0.5.
In the next, it is -0.5.  From there, it gets smaller and
smaller, but it will never be 0.

We have two choices.  We can try to generalize the {\tt factorial}
function to work with floating-point numbers, or we can make
{\tt factorial} check the type of its argument.  The first option
is called the gamma function and it's a little beyond the
scope of this book.  So we'll go for the
second.

\index{gamma function}

We can use the built-in function {\tt isinstance} to verify the type of the
argument.  While we're
at it, we also make sure the argument is positive:

\beforeverb
\begin{verbatim}
def factorial (n):
  if not isinstance(n, int):
    print "Factorial is only defined for integers."
    return -1
  elif n < 0:
    print "Factorial is only defined for positive integers."
    return -1
  elif n == 0:
    return 1
  else:
    return n * factorial(n-1)
\end{verbatim}
\afterverb
%
Now we have three base cases.  The first catches
nonintegers.  The second catches negative integers.  In both cases,
the program prints an error message and returns a special value, -1, to
indicate that something went wrong:

\beforeverb
\begin{verbatim}
>>> factorial ("fred")
Factorial is only defined for integers.
-1
>>> factorial (-2)
Factorial is only defined for positive integers.
-1
\end{verbatim}
\afterverb
%
If we get past both checks, then we know that $n$ is a positive
integer, and we can prove that the recursion terminates.

This program demonstrates a pattern sometimes called a {\bf guardian}.
The first two conditionals act as guardians, protecting the
code that follows from values that might cause an error.  The guardians
make it possible to prove the correctness of the code.


\section{Glossary}

\begin{description}

\item[fruitful function:] A function that yields a return value.

\item[return value:]  The value provided as the result of a function call.

\item[temporary variable:]  A variable used to store an intermediate value in
a complex calculation.

\item[dead code:]  Part of a program that can never be executed, often because
it appears after a {\tt return} statement.

\item[{\tt None}:]  A special Python value returned by functions that
have no return statement, or a return statement without an argument.

\item[incremental development:]  A program development plan intended to
avoid debugging by adding and testing only
a small amount of code at a time.

\item[scaffolding:]  Code that is used during program development but is
not part of the final version.

\item[guardian:]  A condition that checks for and handles circumstances that
might cause an error.

\index{temporary variable}
\index{variable!temporary}
\index{return value}
\index{dead code}
\index{None}
\index{incremental development}
\index{scaffolding}
\index{guardian}

\end{description}

\clearemptydoublepage
% LaTeX source for textbook ``How to think like a computer scientist''
% Copyright (c)  2001  Allen B. Downey, Jeffrey Elkner, and Chris Meyers.

% Permission is granted to copy, distribute and/or modify this
% document under the terms of the GNU Free Documentation License,
% Version 1.1  or any later version published by the Free Software
% Foundation; with the Invariant Sections being "Contributor List",
% with no Front-Cover Texts, and with no Back-Cover Texts. A copy of
% the license is included in the section entitled "GNU Free
% Documentation License".

% This distribution includes a file named fdl.tex that contains the text
% of the GNU Free Documentation License.  If it is missing, you can obtain
% it from www.gnu.org or by writing to the Free Software Foundation,
% Inc., 59 Temple Place - Suite 330, Boston, MA 02111-1307, USA.
\chapter{Iteration}
\index{iteration}


\section{Multiple assignment}
\index{assignment}
\index{statement!assignment}
\index{multiple assignment}

As you may have discovered, it is legal to
make more than one assignment to the same variable.  A
new assignment makes an existing variable refer to a new
value (and stop referring to the old value).

\beforeverb
\begin{verbatim}
bruce = 5
print bruce,
bruce = 7
print bruce
\end{verbatim}
\afterverb
%
The output of this program is {\tt 5 7}, because the first time
{\tt bruce} is printed, his value is 5, and the second time, his
value is 7.  The
comma at the end of the first {\tt print} statement suppresses
the newline after the output, which is why both outputs
appear on the same line.

Here is what {\bf multiple assignment} looks like in a state diagram:

\beforefig
\centerline{\psfig{figure=illustrations/assign2.eps}}
\afterfig

With multiple assignment it is especially important to distinguish
between an assignment operation and a statement of equality.  Because
Python uses the equal sign ({\tt =}) for assignment, it is tempting to
interpret a statement like {\tt a = b} as a statement of equality. It
is not!

First, equality is commutative and assignment is not.  For
example, in mathematics, if $a = 7$ then $7 = a$.  But in Python, the
statement {\tt a = 7} is legal and {\tt 7 = a} is not.

Furthermore, in mathematics, a statement of equality is always true.
If $a = b$ now, then $a$ will always equal $b$.  In Python, an
assignment statement can make two variables equal, but they don't have
to stay that way:

\beforeverb
\begin{verbatim}
a = 5
b = a    # a and b are now equal
a = 3    # a and b are no longer equal
\end{verbatim}
\afterverb
%
The third line changes the value of {\tt a} but does not change the
value of {\tt b}, so they are no longer equal. (In some
programming languages, a different symbol is used for assignment,
such as {\tt <-} or {\tt :=}, to avoid confusion.)

Although multiple assignment is frequently helpful, you should use it
with caution.  If the values of variables change frequently, it can
make the code difficult to read and debug.


\section{The {\tt while} statement}
\index{while statement}
\index{statement!while}
\index{loop!while}
\index{iteration}

Computers are often used to automate repetitive tasks.  Repeating
identical or similar tasks without making errors is something that
computers do well and people do poorly.

We have seen two programs, {\tt nLines} and {\tt countdown}, that use
recursion to perform repetition, which is also called {\bf iteration}.
Because iteration is so common, Python provides several language
features to make it easier.  The first feature we are going to look
at is the {\tt while} statement.

Here is what {\tt countdown} looks like with a 
{\tt while} statement:

\beforeverb
\begin{verbatim}
def countdown(n):
  while n > 0:
    print n
    n = n-1
  print "Blastoff!"
\end{verbatim}
\afterverb
%
Since we removed the recursive call, this function is not
recursive.

You can almost read the {\tt while} statement as if it were English.
It means, ``While {\tt n} is greater than 0, continue
displaying the value of {\tt n} and then reducing the value of
{\tt n} by 1.  When you get to 0, display the word {\tt Blastoff!}''

More formally, here is the flow of execution for a {\tt while} statement:

\begin{enumerate}

\item Evaluate the condition, yielding {\tt 0} or {\tt 1}.

\item If the condition is false (0), exit the {\tt while} statement
and continue execution at the next statement.

\item If the condition is true (1), execute each of the statements in the
body and then go back to step 1.

\end{enumerate}

The body consists of all of the statements below the header
with the same indentation.

This type of flow is called a {\bf loop} because the third step
loops back around to the top.  Notice that if the condition is false
the first time through the loop, the statements inside the loop are
never executed.

\index{condition}
\index{loop}
\index{loop!body}
\index{body!loop}
\index{infinite loop}
\index{loop!infinite}

The body of the loop should change the value of one or more variables
so that eventually the condition becomes false and the loop
terminates.  Otherwise the loop will repeat forever, which is called
an {\bf infinite loop}.  An endless source of amusement for computer
scientists is the observation that the directions on shampoo,
``Lather, rinse, repeat,'' are an infinite loop.

In the case of {\tt countdown}, we can prove that the loop
terminates because we know that the value of {\tt n} is finite, and we
can see that the value of {\tt n} gets smaller each time through the
loop, so eventually we have to get to 0.  In other
cases, it is not so easy to tell:

\beforeverb
\begin{verbatim}
def sequence(n):
  while n != 1:
    print n,
    if n%2 == 0:        # n is even
      n = n/2
    else:               # n is odd
      n = n*3+1
\end{verbatim}
\afterverb
%
The condition for this loop is {\tt n != 1}, so the loop will continue until
{\tt n} is {\tt 1}, which will make the condition false.

Each time through the loop, the program outputs the value of {\tt n}
and then checks whether it is even or odd.  If it is even, the value
of {\tt n} is divided by 2.  If it is odd, the value is replaced by
{\tt n*3+1}. For example, if the starting value (the argument passed
to {\tt sequence}) is 3, the resulting sequence is 3, 10, 5, 16, 8, 4, 2, 1.

Since {\tt n} sometimes increases and sometimes decreases, there is no
obvious proof that {\tt n} will ever reach 1, or that the program
terminates.  For some particular values of {\tt n}, we can prove
termination.  For example, if the starting value is a power of two,
then the value of {\tt n} will be even each time through the loop
until it reaches 1. The previous example ends with such a sequence,
starting with 16.

Particular values aside, the interesting question is whether we can
prove that this program terminates for {\em all positive values} of {\tt n}.
So far, no one has been able to prove it {\em or} disprove it!

\begin{quote}
{\em As an exercise, rewrite the function {\tt nLines} from
Section~\ref{recursion} using iteration instead of recursion.}
\end{quote}


\section{Tables}
\label{tables}
\index{table}
\index{logarithm}

One of the things loops are good for is generating tabular data.
Before computers were readily available, people had to calculate
logarithms, sines and cosines, and other mathematical functions
by hand.  To make that easier, mathematics books contained long tables
listing the values of these functions.  Creating the tables was
slow and boring, and they tended to be full of errors.

When computers appeared on the scene, one of the initial reactions
was, ``This is great!  We can use the computers to generate the tables,
so there will be no errors.'' That turned out to be true (mostly) but
shortsighted.  Soon thereafter, computers and calculators were so
pervasive that the tables became obsolete.

Well, almost.  For some operations, computers use
tables of values to get an approximate answer and then perform
computations to improve the approximation.  In some cases, there have
been errors in the underlying tables, most famously in the table the
Intel Pentium used to perform floating-point division.

\index{Intel}
\index{Pentium}

Although a log table is not as useful as it once was, it still makes
a good example of iteration.  The following program outputs a sequence
of values in the left column and their logarithms in the right column:

\beforeverb
\begin{verbatim}
x = 1.0
while x < 10.0:
  print x, '\t', math.log(x)
  x = x + 1.0
\end{verbatim}
\afterverb
%
The string 
\verb+'\t'+ represents a {\bf tab}
character.

As characters and strings are displayed on the screen,
an invisible marker called the {\bf cursor} keeps track of
where the next character will go.  After a {\tt print} statement, the
cursor normally goes to the beginning of the next line.

The tab character shifts the cursor to the right until it
reaches one of the tab stops.  Tabs are useful for making columns of
text line up, as in the output of the previous program:

\beforeverb
\begin{verbatim}
1.0     0.0
2.0     0.69314718056
3.0     1.09861228867
4.0     1.38629436112
5.0     1.60943791243
6.0     1.79175946923
7.0     1.94591014906
8.0     2.07944154168
9.0     2.19722457734
\end{verbatim}
\afterverb
%
If these values seem odd, remember that the {\tt log} function uses
base {\tt e}. Since powers of two are so important in computer
science, we often want to find logarithms with respect to base 2.  To
do that, we can use the following formula:

\begin{displaymath}
\log_2 x = \frac {\log_e x}{\log_e 2}
\end{displaymath}

Changing the output statement to:

\beforeverb
\begin{verbatim}
   print x, '\t',  math.log(x)/math.log(2.0)
\end{verbatim}
\afterverb
%
yields:

\beforeverb
\begin{verbatim}
1.0     0.0
2.0     1.0
3.0     1.58496250072
4.0     2.0
5.0     2.32192809489
6.0     2.58496250072
7.0     2.80735492206
8.0     3.0
9.0     3.16992500144
\end{verbatim}
\afterverb
%
We can see that 1, 2, 4, and 8 are powers of two because their
logarithms base 2 are round numbers.  If we wanted to find the
logarithms of other powers of two, we could modify the program like
this:

\beforeverb
\begin{verbatim}
x = 1.0
while x < 100.0:
  print x, '\t', math.log(x)/math.log(2.0)
  x = x * 2.0
\end{verbatim}
\afterverb
%
Now instead of adding something to {\tt x} each time through the loop, which
yields an arithmetic sequence, we multiply {\tt x} by something, yielding a
geometric sequence.  The result is:

\index{arithmetic sequence}
\index{geometric sequence}

\beforeverb
\begin{verbatim}
1.0     0.0
2.0     1.0
4.0     2.0
8.0     3.0
16.0    4.0
32.0    5.0
64.0    6.0
\end{verbatim}
\afterverb
%
Because of the tab characters between the columns, the position of the
second column does not depend on the number of digits in the first
column.

Logarithm tables may not be useful any more, but for computer
scientists, knowing the powers of two is!

\begin{quote}
{\em As an exercise, modify this program so that it outputs the powers
of two up to 65,536 (that's $2^{16}$).  Print it out and memorize it.}
\end{quote}

\index{escape sequence}

The backslash character in \verb+'\t'+ indicates the
beginning of an {\bf escape sequence}.  Escape sequences
are used to represent invisible characters like
tabs and newlines.  The sequence \verb+\n+ represents a newline.

An escape sequence can appear
anywhere in a string; in the example, the tab escape
sequence is the only thing in the string.

How do you think you represent a backslash in a string?

\begin{quote}
{\em As an exercise, write a single string that

\beforeverb
\begin{verbatim}
produces
        this
                output.
\end{verbatim}
\afterverb

}
\end{quote}


\section{Two-dimensional tables}
\index{table!two-dimensional}

A two-dimensional table is a table where you
read the value at the intersection of a row and a column.  A
multiplication table is a good example.
Let's say you want to print a multiplication table for the values
from 1 to 6.

A good way to start is to write a loop that prints the multiples of
2, all on one line:

\beforeverb
\begin{verbatim}
i = 1
while i <= 6:
  print 2*i, '   ',
  i = i + 1
print
\end{verbatim}
\afterverb
%
The first line initializes a variable named {\tt i}, which acts as a
counter or {\bf loop variable}.  As the loop executes, the value of
{\tt i} increases from 1 to 6.  When {\tt i} is 7, the loop
terminates.  Each time through the loop, it displays the value of {\tt
2*i}, followed by three spaces.  

Again, the comma in the {\tt print} statement suppresses the newline.
After the loop completes, the second {\tt print} statement starts a
new line.

The output of the program is:

\beforeverb
\begin{verbatim}
2      4      6      8      10     12
\end{verbatim}
\afterverb
%
So far, so good. The next step is to {\bf encapsulate} and {\bf generalize}.


\section{Encapsulation and generalization}
\label{encapsulation}
\index{encapsulation}
\index{generalization}
\index{program development!encapsulation}
\index{program development!generalization}

Encapsulation is the process of wrapping a piece of code in a
function, allowing you to take advantage of all the things functions
are good for.  You have seen two examples of encapsulation:
{\tt printParity} in Section~\ref{alternative execution}; and
{\tt isDivisible} in Section~\ref{boolean}.

Generalization means taking something specific, such as printing the
multiples of 2, and making it more general, such as printing the
multiples of any integer.

This function encapsulates the previous loop and
generalizes it to print multiples of {\tt n}:

\beforeverb
\begin{verbatim}
def printMultiples(n):
  i = 1
  while i <= 6:
    print n*i, '\t',
    i = i + 1
  print
\end{verbatim}
\afterverb
%
To encapsulate, all we had to do was add the first line, which
declares the name of the function and the parameter list.  To
generalize, all we had to do was replace the value 2 with the
parameter {\tt n}.

If we call this function with the argument 2, we get the same output as
before.  With the argument 3, the output is:

\beforeverb
\begin{verbatim}
3      6      9      12     15     18
\end{verbatim}
\afterverb
%
With the argument 4, the output is:

\beforeverb
\begin{verbatim}
4      8      12     16     20     24
\end{verbatim}
\afterverb
%
By now you can probably guess how to print a multiplication table---by
calling {\tt printMultiples} repeatedly with different arguments.  In
fact, we can use another loop:

\beforeverb
\begin{verbatim}
i = 1
while i <= 6:
  printMultiples(i)
  i = i + 1
\end{verbatim}
\afterverb
%
Notice how similar this loop is to the one inside
{\tt printMultiples}.  All we did was replace the {\tt print} statement with
a function call.

The output of this program is a multiplication table:

\beforeverb
\begin{verbatim}
1      2      3      4      5      6
2      4      6      8      10     12
3      6      9      12     15     18
4      8      12     16     20     24
5      10     15     20     25     30
6      12     18     24     30     36
\end{verbatim}
\afterverb
%


\section{More encapsulation}

To demonstrate encapsulation again, let's take the code from the end of
Section~\ref{encapsulation} and wrap it up in a function:

\beforeverb
\begin{verbatim}
def printMultTable():
  i = 1
  while i <= 6:
    printMultiples(i)
    i = i + 1
\end{verbatim}
\afterverb
%
This process is a common {\bf development plan}.  We develop code by
writing lines of code outside any function, or typing them in to the
interpreter.  When we get the code working, we extract it and wrap it
up in a function.

This development plan is particularly useful if you don't know, when
you start writing, how to divide the program into functions.  This
approach lets you design as you go along.

\adjustpage{-2}
\pagebreak

\section{Local variables}
\index{variable!local}
\index{local variable}

You might be wondering how we can use the same variable, {\tt i}, in
both {\tt printMultiples} and {\tt printMultTable}.  Doesn't it cause
problems when one of the functions changes the value of the variable?

The answer is no, because the {\tt i} in {\tt printMultiples} and the
{\tt i} in {\tt printMultTable} are {\em not} the same variable.

Variables created inside a function definition are local; you can't
access a local variable from outside its ``home'' function.  That
means you are free to have multiple variables with the same name as
long as they are not in the same function.

The stack diagram for this program shows that the two
variables named {\tt i} are not the same variable.  They can refer to
different values, and changing one does not affect the other.

\beforefig
\centerline{\psfig{figure=illustrations/stack4.eps}}
\afterfig

The value of {\tt i} in {\tt printMultTable} goes from 1 to 6.  In the
diagram it happens to be 3.  The next time through the loop it will
be 4.  Each time through the loop, {\tt printMultTable} calls
{\tt printMultiples} with the current value of {\tt i} as an
argument.  That value gets assigned to the parameter {\tt n}.

Inside {\tt printMultiples}, the value of {\tt i} goes from
1 to 6.  In the diagram, it happens to be 2.  Changing this variable
has no effect on the value of {\tt i} in {\tt printMultTable}.

It is common and perfectly legal to have different local variables
with the same name.  In particular, names like {\tt i} and {\tt j} are
used frequently as loop variables.  If you avoid
using them in one function just because you used them somewhere else,
you will probably make the program harder to read.


\adjustpage{-2}
\pagebreak

\section{More generalization}

As another example of generalization, imagine you wanted a program
that would print a multiplication table of any size, not just the
six-by-six table. You could add a parameter to {\tt printMultTable}:

\beforeverb
\begin{verbatim}
def printMultTable(high):
  i = 1
  while i <= high:
    printMultiples(i)
    i = i + 1
\end{verbatim}
\afterverb
%
We replaced the value 6 with the parameter {\tt high}.  If we call
{\tt printMultTable} with the argument 7, it displays:

\beforeverb
\begin{verbatim}
1      2      3      4      5      6
2      4      6      8      10     12
3      6      9      12     15     18
4      8      12     16     20     24
5      10     15     20     25     30
6      12     18     24     30     36
7      14     21     28     35     42
\end{verbatim}
\afterverb
%
This is fine, except that we probably want the table to be
square---with the same number of rows and columns.  To do that, we
add another parameter to {\tt printMultiples} to specify how many
columns the table should have.

Just to be annoying, we call this parameter {\tt high}, demonstrating
that different functions can have parameters with the same name (just like
local variables).  Here's the whole program:

\beforeverb
\begin{verbatim}
def printMultiples(n, high):
  i = 1
  while i <= high:
    print n*i, '\t',
    i = i + 1
  print

def printMultTable(high):
  i = 1
  while i <= high:
    printMultiples(i, high)
    i = i + 1
\end{verbatim}
\afterverb
%
Notice that when we added a new parameter, we had to change the first line
of the function (the function heading), and we also had to change the place
where the function is called in {\tt printMultTable}.

As expected, this program generates a square seven-by-seven table:

\beforeverb
\begin{verbatim}
1      2      3      4      5      6      7
2      4      6      8      10     12     14
3      6      9      12     15     18     21
4      8      12     16     20     24     28
5      10     15     20     25     30     35
6      12     18     24     30     36     42
7      14     21     28     35     42     49
\end{verbatim}
\afterverb
%
When you generalize a function appropriately, you often get
a program with capabilities you didn't plan.  For
example, you might
notice that, because $ab = ba$,
all the entries in the table appear twice.  You could save ink by printing
only half the table.  To do that, you only have to change one line of
{\tt printMultTable}.  Change

\beforeverb
\begin{verbatim}
    printMultiples(i, high)
\end{verbatim}
\afterverb
%
to

\beforeverb
\begin{verbatim}
    printMultiples(i, i)
\end{verbatim}
\afterverb
%
and you get

\beforeverb
\begin{verbatim}
1
2      4
3      6      9
4      8      12     16
5      10     15     20     25
6      12     18     24     30     36
7      14     21     28     35     42     49
\end{verbatim}
\afterverb
%
\begin{quote}
{\em As an exercise, trace the execution of this version of
{\tt printMultTable} and figure out how it works.}
\end{quote}


\section{Functions}
\index{function}

A few times now, we have mentioned ``all the things functions are good
for.''  By now, you might be wondering what exactly those things are.
Here are some of them:

\begin{itemize}

\item Giving a name to a sequence of statements makes your program
easier to read and debug.

\item Dividing a long program into functions allows you to separate parts of
the program, debug them in isolation, and then compose them into a whole.

\item Functions facilitate both recursion and iteration.

\item Well-designed functions are often useful for many programs.  Once you
write and debug one, you can reuse it.

\end{itemize}


\section{Glossary}

\begin{description}

\item[multiple assignment:] Making more than one assignment to the same
variable during the execution of a program.

\item[iteration:] Repeated execution of a set of statements using
either a recursive function call or a loop.

\item[loop:] A statement or group of statements that execute repeatedly until
a terminating condition is satisfied.

\item[infinite loop:] A loop in which the terminating condition is
never satisfied.

\item[body:] The statements inside a loop.

\item[loop variable:] A variable used as part of the terminating
condition of a loop.

\item[tab:] A special character that causes the cursor to move to
the next tab stop on the current line.

\item[newline:] A special character that causes the cursor to move to the
beginning of the next line.

\item[cursor:] An invisible marker that keeps track of where the next
character will be printed.

\item[escape sequence:] An escape character ($\backslash$) followed by one or
more printable characters used to designate a nonprintable character.

\item[encapsulate:] To divide a large complex program into components
(like functions) and isolate the components from each other (by
using local variables, for example).

\item[generalize:] To replace something unnecessarily specific (like a constant
value) with something appropriately general (like a variable or parameter).
Generalization makes code more versatile, more likely to be reused, and
sometimes even easier to write.

\item[development plan:] A process for developing a program. In this chapter,
we demonstrated a style of development based on developing code to do
simple, specific things and then encapsulating and generalizing.

\index{multiple assignment}
\index{assignment!multiple }
\index{iteration}
\index{loop!body}
\index{loop}
\index{infinite loop}
\index{escape sequence}
\index{cursor}
\index{tab}
\index{newline}
\index{loop variable}
\index{encapsulate}
\index{generalize}
\index{development plan}
\index{program!development}

\end{description}

\clearemptydoublepage
% LaTeX source for textbook ``How to think like a computer scientist''
% Copyright (c)  2001  Allen B. Downey, Jeffrey Elkner, and Chris Meyers.

% Permission is granted to copy, distribute and/or modify this
% document under the terms of the GNU Free Documentation License,
% Version 1.1  or any later version published by the Free Software
% Foundation; with the Invariant Sections being "Contributor List",
% with no Front-Cover Texts, and with no Back-Cover Texts. A copy of
% the license is included in the section entitled "GNU Free
% Documentation License".

% This distribution includes a file named fdl.tex that contains the text
% of the GNU Free Documentation License.  If it is missing, you can obtain
% it from www.gnu.org or by writing to the Free Software Foundation,
% Inc., 59 Temple Place - Suite 330, Boston, MA 02111-1307, USA.
\chapter{Strings}
\label{strings}


\section{A compound data type}
\index{compound data type}
\index{data type!compound}

So far we have seen three types: {\tt int}, {\tt float}, and {\tt
string}.  Strings are qualitatively different from the
other two because they are made up of smaller pieces---characters.

\index{character}

Types that comprise smaller pieces are called {\bf compound data
types}.  Depending on what we are doing, we may want to treat a
compound data type as a single thing, or we may want to access its
parts.  This ambiguity is useful.

\index{bracket operator}
\index{operator!bracket}

The bracket operator selects a single character from a string.

\beforeverb
\begin{verbatim}
>>> fruit = "banana"
>>> letter = fruit[1]
>>> print letter
\end{verbatim}
\afterverb
%
The expression {\tt fruit[1]} selects character number 1 from {\tt
fruit}.  The variable {\tt letter} refers to the result.  When we
display {\tt letter}, we get a surprise:

\beforeverb
\begin{verbatim}
a
\end{verbatim}
\afterverb
%
The first letter of {\tt "banana"} is not {\tt a}.  Unless you are a
computer scientist.  In that case you should think of the expression in
brackets as an offset from the beginning of the string, and the offset
of the first letter is zero.  So {\tt b} is the 0th letter
(``zero-eth'') of {\tt "banana"}, {\tt a} is the 1th letter
(``one-eth''), and {\tt n} is the 2th (``two-eth'') letter.

To get the first letter of a string, you just put 0, or
any expression with the value 0, in the brackets:

\beforeverb
\begin{verbatim}
>>> letter = fruit[0]
>>> print letter
b
\end{verbatim}
\afterverb
%
The expression in brackets is called an {\bf index}.  An index
specifies a member of an ordered set, in this case the set of
characters in the string.  The index {\em indicates} which one you
want, hence the name.  It can be any integer expression.

\index{index}


\section{Length}
\index{string!length}
\index{runtime error}

The {\tt len} function returns the number of characters in a string:

\beforeverb
\begin{verbatim}
>>> fruit = "banana"
>>> len(fruit)
6
\end{verbatim}
\afterverb
%
To get the last letter of a string, you might be tempted to try something
like this:

\beforeverb
\begin{verbatim}
length = len(fruit)
last = fruit[length]       # ERROR!
\end{verbatim}
\afterverb
%
That won't work. It causes the runtime error {\tt IndexError: string
index out of range}.  The reason is that there is no 6th letter in
{\tt "banana"}.  Since we started counting at zero, the six letters
are numbered 0 to 5.  To get the last character, we have to subtract
1 from {\tt length}:

\index{runtime error}

\beforeverb
\begin{verbatim}
length = len(fruit)
last = fruit[length-1]
\end{verbatim}
\afterverb
%
Alternatively, we can use negative indices, which count backward from the end
of the string.  The expression {\tt fruit[-1]} yields the last letter,
{\tt fruit[-2]} yields the second to last, and so on.

\index{index!negative}


\section{Traversal and the {\tt for} loop}
\label{for}
\index{traversal}
\index{loop!traversal}
\index{for loop}
\index{loop!for loop}

A lot of computations involve processing a string one character at a
time.  Often they start at the beginning, select each character in
turn, do something to it, and continue until the end.  This pattern of
processing is called a {\bf traversal}.  One way to encode a traversal
is with a {\tt while} statement:

\adjustpage{2}
\beforeverb
\begin{verbatim}
index = 0
while index < len(fruit):
  letter = fruit[index]
  print letter
  index = index + 1
\end{verbatim}
\afterverb
%
This loop traverses the string and displays each letter on a line by
itself.  The loop condition is {\tt index < len(fruit)}, so
when {\tt index} is equal to the length of the string, the
condition is false, and the body of the loop is not executed.  The
last character accessed is the one with the index {\tt len(fruit)-1},
which is the last character in the string.

\begin{quote}
{\em As an exercise, write a function that takes a string as an argument
and outputs the letters backward, one per line.}
\end{quote}

Using an index to
traverse a set of values is so common that
Python provides an alternative, simpler syntax---the {\tt for} loop:

\beforeverb
\begin{verbatim}
for char in fruit:
  print char
\end{verbatim}
\afterverb
%
Each time through the loop, the next character in the string is assigned
to the variable {\tt char}.  The loop continues until no characters are
left.

\index{concatenation}
\index{abecedarian}
\index{McCloskey, Robert}
\index{{\em Make Way for Ducklings}}

The following example shows how to use concatenation and a {\tt
for} loop to generate an abecedarian series.  ``Abecedarian'' refers
to a series or list in which the elements appear in alphabetical
order.  For example, in Robert McCloskey's book {\em Make Way for
Ducklings}, the names of the ducklings are Jack, Kack, Lack, Mack,
Nack, Ouack, Pack, and Quack.  This loop outputs these names in order:

\beforeverb
\begin{verbatim}
prefixes = "JKLMNOPQ"
suffix = "ack"

for letter in prefixes:
  print letter + suffix
\end{verbatim}
\afterverb
%
The output of this program is:

\beforeverb
\begin{verbatim}
Jack
Kack
Lack
Mack
Nack
Oack
Pack
Qack
\end{verbatim}
\afterverb
%
Of course, that's not quite right because ``Ouack'' and
``Quack'' are misspelled.

\begin{quote}
{\em As an exercise, modify the program to fix this error.}
\end{quote}


\section{String slices}
\label{slice}
\index{slice}
\index{string!slice}

A segment of a string is called a 
{\bf slice}.  Selecting a slice is similar to
selecting a character:

\beforeverb
\begin{verbatim}
>>> s = "Peter, Paul, and Mary"
>>> print s[0:5]
Peter
>>> print s[7:11]
Paul
>>> print s[17:21]
Mary
\end{verbatim}
\afterverb
%
The operator {\tt [n:m]} returns the part of the string from the 
``n-eth'' character to the ``m-eth'' character, including the first but
excluding the last.  This behavior is counterintuitive; it makes
more sense if you imagine the indices pointing {\em between} the
characters, as in the following diagram:

\beforefig
\centerline{\psfig{figure=illustrations/banana.eps}}
\afterfig

If you omit the first index (before the colon), the slice starts at the
beginning of the string.  If you omit the second index, the slice goes to the
end of the string.  Thus:

\beforeverb
\begin{verbatim}
>>> fruit = "banana"
>>> fruit[:3]
'ban'
>>> fruit[3:]
'ana'
\end{verbatim}
\afterverb
%
What do you think {\tt s[:]} means?


\section{String comparison}
\index{string comparison}
\index{comparison!string}

The comparison operators work on
strings.  To see if two strings are equal:

\beforeverb
\begin{verbatim}
if word == "banana":
  print  "Yes, we have no bananas!"
\end{verbatim}
\afterverb
%
\adjustpage{-2}
\pagebreak

Other comparison operations are useful for putting words in alphabetical
order:

\beforeverb
\begin{verbatim}
if word < "banana":
  print "Your word," + word + ", comes before banana."
elif word > "banana":
  print "Your word," + word + ", comes after banana."
else:
  print "Yes, we have no bananas!"
\end{verbatim}
\afterverb
%
You should be aware, though, that Python does not handle upper-
and lowercase letters the same way that people do.  All the uppercase
letters come before all the lowercase letters.  As a result:

\beforeverb
\begin{verbatim}
Your word, Zebra, comes before banana.
\end{verbatim}
\afterverb
%
A common way to address this problem is to convert strings to a standard
format, such as all lowercase, before performing the comparison.  A more
difficult problem is making the program realize that zebras are not fruit.


\section{Strings are immutable}
\index{mutable}
\index{immutable string}
\index{string!immutable}

It is tempting to use the {\tt []} operator on the left side of an
assignment, with the intention of changing a character in a string.
For example:

\beforeverb
\begin{verbatim}
greeting = "Hello, world!"
greeting[0] = 'J'            # ERROR!
print greeting
\end{verbatim}
\afterverb
%
Instead of producing the output {\tt Jello, world!}, this code
produces the runtime error {\tt TypeError: object doesn't support item
assignment}.

\index{runtime error}

Strings are {\bf immutable}, which means you can't change an
existing string.  The best you can do is create a new string
that is a variation on the original:

\beforeverb
\begin{verbatim}
greeting = "Hello, world!"
newGreeting = 'J' + greeting[1:]
print newGreeting
\end{verbatim}
\afterverb
%
The solution here is to concatenate a new first letter onto
a slice of {\tt greeting}.  This operation has no effect on
the original string.

\index{concatenation}

\adjustpage{-2}
\pagebreak

\section{A {\tt find} function}
\label{find}
\index{traversal}
\index{eureka traversal}
\index{pattern}
\index{computational pattern}

What does the following function do?

\beforeverb
\begin{verbatim}
def find(str, ch):
  index = 0
  while index < len(str):
    if str[index] == ch:
      return index
    index = index + 1
  return -1
\end{verbatim}
\afterverb
%
In a sense, {\tt find} is the opposite of the {\tt []} operator.
Instead of taking an index and extracting the corresponding character,
it takes a character and finds the index where that character
appears.  If the character is not found, the function returns {\tt
-1}.

This is the first example we have seen of a {\tt return} statement
inside a loop.
If {\tt str[index] == ch}, the function returns
immediately, breaking out of the loop prematurely.

If the character doesn't appear in the string, then the program
exits the loop normally and 
returns {\tt -1}.

This pattern of computation is sometimes called a ``eureka'' traversal
because as soon as we find what we are looking for, we can cry
``Eureka!'' and stop looking.

\begin{quote}
{\em As an exercise, modify the {\tt find} function so that it has a
third parameter, the index in the string where it should start
looking.}
\end{quote}


\section{Looping and counting}
\label{counter}
\index{counter}
\index{pattern}

The following program counts the number of times the letter {\tt a}
appears in a string:

\beforeverb
\begin{verbatim}
fruit = "banana"
count = 0
for char in fruit:
  if char == 'a':
    count = count + 1
print count
\end{verbatim}
\afterverb
%
This program demonstrates another pattern of computation called a {\bf
counter}.  The variable {\tt count} is initialized to 0 and then
incremented each time an {\tt a} is found.  (To {\bf increment} is to
increase by one; it is the opposite of {\bf decrement}, and unrelated
to ``excrement,'' which is a noun.)  When the loop exits, {\tt count}
contains the result---the total number of {\tt a}'s.

\begin{quote}
{\em As an exercise, encapsulate this code in a function named {\tt
countLetters}, and generalize it so that it accepts the string and the
letter as arguments.}
\end{quote}

\begin{quote}
{\em As a second exercise, rewrite this function so that instead of
traversing the string, it uses the three-parameter version of {\tt
find} from the previous.}
\end{quote}


\section{The {\tt string} module}
\index{module}
\index{string module}

The {\tt string} module contains useful functions that
manipulate strings.  As usual, we have to import the module before
we can use it:

\beforeverb
\begin{verbatim}
>>> import string
\end{verbatim}
\afterverb
%
The {\tt string} module includes a function named {\tt find} that does
the same thing as the function we wrote.  To call it we have to
specify the name of the module and the name of the function using dot
notation.

\beforeverb
\begin{verbatim}
>>> fruit = "banana"
>>> index = string.find(fruit, "a")
>>> print index
1
\end{verbatim}
\afterverb
%
This example demonstrates one of the benefits of modules---they help
avoid collisions between the names of built-in functions and
user-defined functions.  By using dot notation we can specify which
version of {\tt find} we want.

Actually, {\tt string.find} is more general than our version.  First,
it can find substrings, not just characters:

\beforeverb
\begin{verbatim}
>>> string.find("banana", "na")
2
\end{verbatim}
\afterverb
%
Also, it takes an additional argument that specifies the index it
should start at:

\beforeverb
\begin{verbatim}
>>> string.find("banana", "na", 3)
4
\end{verbatim}
\afterverb
%
Or it can take two additional arguments that specify a range
of indices:

\beforeverb
\begin{verbatim}
>>> string.find("bob", "b", 1, 2)
-1
\end{verbatim}
\afterverb
%
In this example, the search fails because the letter {\em b} does not
appear in the index range from {\tt 1} to {\tt 2} (not including {\tt
2}).


\section{Character classification}
\label{in}
\index{character classification}
\index{classification!character}
\index{uppercase}
\index{lowercase}
\index{whitespace}

It is often helpful to examine a character and test whether it is
upper- or lowercase, or whether it is a character or a digit.  The
{\tt string} module provides several constants that are
useful for these purposes.

The string {\tt string.lowercase} contains all of the letters that the
system considers to be lowercase.  Similarly, {\tt string.uppercase}
contains all of the uppercase letters.  Try the following and see what
you get:

\beforeverb
\begin{verbatim}
>>> print string.lowercase
>>> print string.uppercase
>>> print string.digits
\end{verbatim}
\afterverb
%
We can use these constants and {\tt find} to classify characters. For
example, if {\tt find(lowercase, ch)} returns a value other than {\tt
-1}, then {\tt ch} must be lowercase:

\beforeverb
\begin{verbatim}
def isLower(ch):
  return string.find(string.lowercase, ch) != -1
\end{verbatim}
\afterverb
%
Alternatively, we can take advantage of the {\tt in} operator, which
determines whether a character appears in a string:

\beforeverb
\begin{verbatim}
def isLower(ch):
  return ch in string.lowercase
\end{verbatim}
\afterverb
%
As yet another alternative, we can use the comparison operator:

\beforeverb
\begin{verbatim}
def isLower(ch):
  return 'a' <= ch <= 'z'
\end{verbatim}
\afterverb
%
If {\tt ch} is between {\em a} and {\em z}, it must be a lowercase
letter.

\begin{quote}
{\em As an exercise, discuss which version of {\tt isLower} you think
will be fastest.  Can you think of other reasons besides speed to
prefer one or the other?}
\end{quote}

Another constant defined in the {\tt string} module may
surprise you when you print it:

\beforeverb
\begin{verbatim}
>>> print string.whitespace
\end{verbatim}
\afterverb
%
{\bf Whitespace} characters move the cursor without printing
anything.  They create the white space between visible
characters (at least on white paper).  The constant
{\tt string.whitespace} contains all the
whitespace characters, including
space, tab (\verb+\t+), and newline
(\verb+\n+).

\index{string module}
\index{module!string}

There are other useful functions in the {\tt string} module, but this
book isn't intended to be a reference manual.  On the other hand, the
{\em Python Library Reference} is.  Along with a wealth of other
documentation, it's available from the Python website, {\tt
www.python.org}.

\index{{\em Python Library Reference}}

\section{Glossary}

\begin{description}

\item[compound data type:] A data type in which the values are made
up of components, or elements, that are themselves values.

\item[traverse:] To iterate through the elements of a set,
performing a similar operation on each.

\item[index:] A variable or value used to select a member of an
ordered set, such as a character from a string.

\item[slice:] A part of a string specified by a range of indices.

\item[mutable:] A compound data types whose elements can be assigned
new values.

\item[counter:] A variable used to count something, usually initialized
to zero and then incremented.

\item[increment:] To increase the value of a variable by one.

\item[decrement:] To decrease the value of a variable by one.

\item[whitespace:] Any of the characters that move the cursor without
printing visible characters.  The constant
{\tt string.whitespace}
contains all the white\-space characters.

\index{compound data type}
\index{traverse}
\index{index}
\index{slice}
\index{mutable}
\index{counter}
\index{increment}
\index{decrement}
\index{whitespace}

\end{description}

\clearemptydoublepage
% LaTeX source for textbook ``How to think like a computer scientist''
% Copyright (c)  2001  Allen B. Downey, Jeffrey Elkner, and Chris Meyers.

% Permission is granted to copy, distribute and/or modify this
% document under the terms of the GNU Free Documentation License,
% Version 1.1  or any later version published by the Free Software
% Foundation; with the Invariant Sections being "Contributor List",
% with no Front-Cover Texts, and with no Back-Cover Texts. A copy of
% the license is included in the section entitled "GNU Free
% Documentation License".

% This distribution includes a file named fdl.tex that contains the text
% of the GNU Free Documentation License.  If it is missing, you can obtain
% it from www.gnu.org or by writing to the Free Software Foundation,
% Inc., 59 Temple Place - Suite 330, Boston, MA 02111-1307, USA.

\chapter{Lists}
\index{list}
\index{type!list}
\index{element}
\index{sequence}

A {\bf list} is an ordered set of values, where each value is
identified by an index.  The values that make up a list are
called its {\bf elements}.  Lists are similar to strings, which are
ordered sets of characters, except that the elements of a list can
have any type.  Lists and strings---and other things that behave like
ordered sets---are called {\bf sequences}.


\section{List values}

There are several ways to create a new list; the simplest is to
enclose the elements in square brackets (\verb+[+ and \verb+]+):

\beforeverb
\begin{verbatim}
[10, 20, 30, 40]
["spam", "bungee", "swallow"]
\end{verbatim}
\afterverb
%
The first example is a list of four integers.  The second is a list of
three strings.  The elements of a list don't have to be the same type.
The following list contains a string, a float, an integer, and
(mirabile dictu) another list:

\beforeverb
\begin{verbatim}
["hello", 2.0, 5, [10, 20]]
\end{verbatim}
\afterverb
%
A list within another list is said to be {\bf nested}.

\index{list!nested}

Lists that contain consecutive integers are common, so Python provides a
simple way to create them:

\beforeverb
\begin{verbatim}
>>> range(1,5)
[1, 2, 3, 4]
\end{verbatim}
\afterverb
%
The {\tt range} function takes two arguments and returns a list that
contains all the integers from the first to the second, including the
first but not including the second!

There are two other forms of {\tt range}.  With a single argument, it
creates a list that starts at 0:

\beforeverb
\begin{verbatim}
>>> range(10)
[0, 1, 2, 3, 4, 5, 6, 7, 8, 9]
\end{verbatim}
\afterverb
%
If there is a third argument, it specifies the space between
successive values, which is called the {\bf step size}.  This example
counts from 1 to 10 by steps of 2:

\beforeverb
\begin{verbatim}
>>> range(1, 10, 2)
[1, 3, 5, 7, 9]
\end{verbatim}
\afterverb
%
Finally, there is a special list that contains no elements.  It is
called the empty list, and it is denoted {\tt []}.

With all these ways to create lists, it would be disappointing if we
couldn't assign list values to variables or pass lists as arguments
to functions.  We can.

\beforeverb
\begin{verbatim}
vocabulary = ["ameliorate", "castigate", "defenestrate"]
numbers = [17, 123]
empty = []
print vocabulary, numbers, empty
['ameliorate', 'castigate', 'defenestrate'] [17, 123] []
\end{verbatim}
\afterverb
%


\section{Accessing elements}
\index{list!element}
\index{access}

The syntax for accessing the elements of a list is the same as the
syntax for accessing the characters of a string---the bracket
operator ({\tt []}).  The expression inside the brackets
specifies the index.  Remember that the indices start at 0:

\beforeverb
\begin{verbatim}
print numbers[0]
numbers[1] = 5
\end{verbatim}
\afterverb
%
The bracket operator can appear anywhere in an expression.  When it
appears on the left side of an assignment, it changes one of the
elements in the list, so the one-eth element of {\tt numbers}, which
used to be 123, is now 5.

Any integer expression can be used as an index:

\beforeverb
\begin{verbatim}
>>> numbers[3-2]
5
>>> numbers[1.0]
TypeError: sequence index must be integer
\end{verbatim}
\afterverb
%
If you try to read or write an element that does not exist, you
get a runtime error:

\index{runtime error}

\beforeverb
\begin{verbatim}
>>> numbers[2] = 5
IndexError: list assignment index out of range
\end{verbatim}
\afterverb
%
If an index has a negative value, it counts backward from the
end of the list:

\beforeverb
\begin{verbatim}
>>> numbers[-1]
5
>>> numbers[-2]
17
>>> numbers[-3]
IndexError: list index out of range
\end{verbatim}
\afterverb
%
{\tt numbers[-1]} is the last element of the list, {\tt numbers[-2]}
is the second to last, and {\tt numbers[-3]} doesn't exist.

It is common to use a loop variable as a list index.

\beforeverb
\begin{verbatim}
horsemen = ["war", "famine", "pestilence", "death"]

i = 0
while i < 4:
  print horsemen[i]
  i = i + 1
\end{verbatim}
\afterverb
%
This {\tt while} loop counts from 0 to 4.  When the loop variable
{\tt i} is 4, the condition fails and the loop terminates.  So the
body of the loop is only executed when {\tt i} is 0, 1, 2, and 3.

Each time through the loop, the variable {\tt i} is used as an index
into the list, printing the {\tt i}-eth element.  This pattern of
computation is called a {\bf list traversal}.

\index{list!traversal}
\index{traversal!list}


\section{List length}
\index{length}
\index{list!length}

The function {\tt len} returns the length of a list.  It is a good
idea to use this value as the upper bound of a loop instead of a
constant.  That way, if the size of the list changes, you won't have
to go through the program changing all the loops; they will work
correctly for any size list:

\adjustpage{-2}
\pagebreak

\beforeverb
\begin{verbatim}
horsemen = ["war", "famine", "pestilence", "death"]

i = 0
while i < len(horsemen):
  print horsemen[i]
  i = i + 1
\end{verbatim}
\afterverb
%
The last time the body of the loop is executed, {\tt i} is {\tt
len(horsemen) - 1}, which is the index of the last element.  When {\tt
i} is equal to {\tt len(horsemen)}, the condition fails and the body
is not executed, which is a good thing, because {\tt len(horsemen)} is
not a legal index.

Although a list can contain another list, the nested
list still counts as a single element.  The length of this list is
four:

\beforeverb
\begin{verbatim}
['spam!', 1, ['Brie', 'Roquefort', 'Pol le Veq'], [1, 2, 3]]
\end{verbatim}
\afterverb
%
\begin{quote}
{\em As an exercise, write a loop that traverses the previous
list and prints the length of each element.  What happens if
you send an integer to {\tt len}?}
\end{quote}


\section{List membership}
\index{list!membership}
\index{in operator}
\index{operator!in}

{\tt in} is a boolean operator that tests membership in a sequence.
We used it in Section~\ref{in} with strings, but it also works with
lists and other sequences:

\beforeverb
\begin{verbatim}
>>> horsemen = ['war', 'famine', 'pestilence', 'death']
>>> 'pestilence' in horsemen
True
>>> 'debauchery' in horsemen
False
\end{verbatim}
\afterverb

Since ``pestilence'' is a member of the {\tt horsemen} list, the {\tt in}
operator returns true.  Since ``debauchery'' is not in the list, {\tt
in} returns false.

We can use the {\tt not} in combination
with {\tt in} to test whether an element is not a member of a list:

\beforeverb
\begin{verbatim}
>>> 'debauchery' not in horsemen
True
\end{verbatim}
\afterverb


\section{Lists and {\tt for} loops}
\index{for loop}
\index{list!for loop}
\index{traversal}

The {\tt for} loop we saw in Section~\ref{for} also works with
lists.
The generalized syntax of a {\tt for} loop is:

\beforeverb
\begin{verbatim}
for VARIABLE in LIST:
  BODY
\end{verbatim}
\afterverb
%
This statement is equivalent to:

\beforeverb
\begin{verbatim}
i = 0
while i < len(LIST):
  VARIABLE = LIST[i]
  BODY
  i = i + 1
\end{verbatim}
\afterverb
%
The {\tt for} loop is more concise because we can 
eliminate the loop variable, {\tt i}.
Here is the previous loop written with a {\tt for} loop.

\beforeverb
\begin{verbatim}
for horseman in horsemen:
  print horseman
\end{verbatim}
\afterverb
%
It almost reads like English: ``For (every) horseman
in (the list of) horsemen, print (the name of the) horseman.''

Any list expression can be used in a {\tt for} loop:

\beforeverb
\begin{verbatim}
for number in range(20):
  if number % 2 == 0:
    print  number

for fruit in ["banana", "apple", "quince"]:
  print "I like to eat " + fruit + "s!"
\end{verbatim}
\afterverb
%
The first
example prints all the even numbers between zero and nineteen.
The second example expresses enthusiasm for various fruits.



\section{List operations}
\index{list operation}
\index{operation!list}

The {\tt +} operator concatenates lists:

\index{concatenation!list}

\beforeverb
\begin{verbatim}
>>> a = [1, 2, 3]
>>> b = [4, 5, 6]
>>> c = a + b
>>> print c
[1, 2, 3, 4, 5, 6]
\end{verbatim}
\afterverb
%
Similarly, the {\tt *} operator repeats a list a given number of times:

\index{repetition!list}

\beforeverb
\begin{verbatim}
>>> [0] * 4
[0, 0, 0, 0]
>>> [1, 2, 3] * 3
[1, 2, 3, 1, 2, 3, 1, 2, 3]
\end{verbatim}
\afterverb
%
The first example repeats {\tt [0]} four times.  The second example
repeats the list {\tt [1, 2, 3]} three times.


\section{List slices}
\index{slice}
\index{list!slice}

The slice operations we saw in Section~\ref{slice}
also work on lists:

\beforeverb
\begin{verbatim}
>>> list = ['a', 'b', 'c', 'd', 'e', 'f']
>>> list[1:3]
['b', 'c']
>>> list[:4]
['a', 'b', 'c', 'd']
>>> list[3:]
['d', 'e', 'f']
\end{verbatim}
\afterverb
%
If you omit the first index, the slice starts at the beginning.
If you omit the second, the slice goes to the end.  So if you
omit both, the slice is really a copy of the whole list.

\beforeverb
\begin{verbatim}
>>> list[:]
['a', 'b', 'c', 'd', 'e', 'f']
\end{verbatim}
\afterverb
%


\section{Lists are mutable}
\index{mutable!list}
\index{list!mutable}

Unlike strings, lists are mutable, which means we can change
their elements.  Using the bracket operator on the left side
of an assignment, we can update one of the elements:

\beforeverb
\begin{verbatim}
>>> fruit = ["banana", "apple", "quince"]
>>> fruit[0] = "pear"
>>> fruit[-1] = "orange"
>>> print fruit
['pear', 'apple', 'orange']
\end{verbatim}
\afterverb
%
With the slice operator we can update several elements at once:

\beforeverb
\begin{verbatim}
>>> list = ['a', 'b', 'c', 'd', 'e', 'f']
>>> list[1:3] = ['x', 'y']
>>> print list
['a', 'x', 'y', 'd', 'e', 'f']
\end{verbatim}
\afterverb
%
We can also remove elements from a list by assigning the empty list to
them:

\beforeverb
\begin{verbatim}
>>> list = ['a', 'b', 'c', 'd', 'e', 'f']
>>> list[1:3] = []
>>> print list
['a', 'd', 'e', 'f']
\end{verbatim}
\afterverb
%
And we can add elements to a list by squeezing them into an empty
slice at the desired location:

\beforeverb
\begin{verbatim}
>>> list = ['a', 'd', 'f']
>>> list[1:1] = ['b', 'c']
>>> print list
['a', 'b', 'c', 'd', 'f']
>>> list[4:4] = ['e']
>>> print list
['a', 'b', 'c', 'd', 'e', 'f']
\end{verbatim}
\afterverb
%

\section{List deletion}
\index{deletion!list}
\index{list deletion}

Using slices to delete list elements can be
awkward, and therefore error-prone.  Python provides an alternative
that is more readable.

{\tt del} removes an element from a list:

\beforeverb
\begin{verbatim}
>>> a = ['one', 'two', 'three']
>>> del a[1]
>>> a
['one', 'three']
\end{verbatim}
\afterverb
%
As you might expect, {\tt del} handles negative indices
and causes a runtime error if the index is
out of range.

You can use a slice as an index for {\tt del}:

\beforeverb
\begin{verbatim}
>>> list = ['a', 'b', 'c', 'd', 'e', 'f']
>>> del list[1:5]
>>> print list
['a', 'f']
\end{verbatim}
\afterverb
%
As usual, slices select all the elements up to, but not
including, the second index.

% Had to remove this because append is a method, not a function.
% The {\tt append} function adds an element (or a list) to
% the end of an existing list.

% \beforeverb
% \begin{verbatim}
% >>> a = ['one', 'two']
% >>> append (a, 'three')
% >>> print a
% ['one', 'two', 'three']
% \end{verbatim}
% \afterverb


\adjustpage{-2}
\pagebreak

\section{Objects and values}
\index{object}
\index{value}

If we execute these assignment statements,

\beforeverb
\begin{verbatim}
a = "banana"
b = "banana"
\end{verbatim}
\afterverb
%
we know that {\tt a} and {\tt b} will refer to a
string with the letters {\tt "banana"}.  But we can't
tell whether they point to the {\em same} string.

There are two possible states:

\beforefig
\centerline{\psfig{figure=illustrations/list1.eps}}
\afterfig

In one case, {\tt a} and {\tt b} refer to two different things that
have the same value.  In the second case, they refer to the same
thing.  These ``things'' have names---they are called {\bf objects}.
An object is something a variable can refer to.

Every object has a unique {\bf identifier}, which we can obtain with
the {\tt id} function.  By printing the identifier of {\tt a}
and {\tt b}, we can tell whether they refer to the same object.

\beforeverb
\begin{verbatim}
>>> id(a)
135044008
>>> id(b)
135044008
\end{verbatim}
\afterverb
%
In fact, we get the same identifier twice, which means that
Python only created one string,
and both {\tt a} and {\tt b} refer to it.

Interestingly, lists behave differently.
When we create two lists, we get two objects:

\beforeverb
\begin{verbatim}
>>> a = [1, 2, 3]
>>> b = [1, 2, 3]
>>> id(a)
135045528
>>> id(b)
135041704
\end{verbatim}
\afterverb
%
\adjustpage{1}

So the state diagram looks like this:

\beforefig
\centerline{\psfig{figure=illustrations/list2.eps}}
\afterfig

{\tt a} and {\tt b} have the same value but do not
refer to the same object.


\section{Aliasing}
\index{aliasing}
\index{reference!aliasing}

Since variables refer to objects, if we assign one
variable to another, both variables refer to the same object:

\beforeverb
\begin{verbatim}
>>> a = [1, 2, 3]
>>> b = a
\end{verbatim}
\afterverb
%
In this case, the state diagram looks like this:

\beforefig
\centerline{\psfig{figure=illustrations/list3.eps}}
\afterfig

Because the same list has two different names, {\tt a} and {\tt b}, we
say that it is {\bf aliased}.  Changes made with one alias affect
the other:

\beforeverb
\begin{verbatim}
>>> b[0] = 5
>>> print a
[5, 2, 3]
\end{verbatim}
\afterverb
%
Although this behavior can be useful, it is sometimes unexpected or
undesirable.  In general, it is safer to avoid aliasing when you
are working with mutable objects.  Of course, for immutable
objects, there's no problem.  That's why Python is free to
alias strings when it sees an opportunity to economize.


\section{Cloning lists}
\index{list!cloning}
\index{cloning}

If we want to modify a list and also keep a copy of the original, we
need to be able to make a copy of the list itself, not just the
reference.  This process is sometimes called {\bf cloning}, to avoid the
ambiguity of the word ``copy.''

The easiest way to clone a list is to use the slice operator:

\beforeverb
\begin{verbatim}
>>> a = [1, 2, 3]
>>> b = a[:]
>>> print b
[1, 2, 3]
\end{verbatim}
\afterverb
%
Taking any slice of {\tt a} creates a new list.  
In this case the slice happens to consist of the whole list.

Now we are free to make changes to {\tt b} without worrying about {\tt a}:

\beforeverb
\begin{verbatim}
>>> b[0] = 5
>>> print a
[1, 2, 3]
\end{verbatim}
\afterverb
%
\begin{quote}
{\em As an exercise, draw a state diagram for {\tt a} and {\tt b}
before and after this change.}
\end{quote}



\section{List parameters}
\index{list!as parameter}
\index{parameter}
\index{parameter!list}

Passing a list as an argument actually passes a reference to the
list, not a copy of the list.
For example, the function {\tt head} takes a list as an argument
and returns the first element:

\beforeverb
\begin{verbatim}
def head(list):
  return list[0]
\end{verbatim}
\afterverb
%
Here's how it is used:

\beforeverb
\begin{verbatim}
>>> numbers = [1, 2, 3]
>>> head(numbers)
1
\end{verbatim}
\afterverb
%
The parameter {\tt list} and the variable {\tt numbers} are
aliases for the same object.  The state diagram looks like
this:

\beforefig
\centerline{\psfig{figure=illustrations/stack5.eps}}
\afterfig

Since the list object is shared by two frames, we drew
it between them.

If a function modifies a list parameter, the caller sees the change.
For example, {\tt deleteHead} removes the first element from a list:

\beforeverb
\begin{verbatim}
def deleteHead(list):
  del list[0]
\end{verbatim}
\afterverb
%
Here's how {\tt deleteHead} is used:

\beforeverb
\begin{verbatim}
>>> numbers = [1, 2, 3]
>>> deleteHead(numbers)
>>> print numbers
[2, 3]
\end{verbatim}
\afterverb
%
If a function returns a list, it returns a reference to the list.  For
example, {\tt tail} returns a list that contains all but the first
element of the given list:

\beforeverb
\begin{verbatim}
def tail(list):
  return list[1:]
\end{verbatim}
\afterverb
%
Here's how {\tt tail} is used:

\beforeverb
\begin{verbatim}
>>> numbers = [1, 2, 3]
>>> rest = tail(numbers)
>>> print rest
[2, 3]
\end{verbatim}
\afterverb
%
Because the return value was created with the slice operator, it
is a new list.  Creating {\tt rest}, and any subsequent changes
to {\tt rest}, have no effect on {\tt numbers}.



\section{Nested lists}
\label{nested lists}
\index{nested list}
\index{list!nested}

A nested list is a list that appears as an element in another
list.  In this list, the three-eth element is a nested list:

\beforeverb
\begin{verbatim}
>>> list = ["hello", 2.0, 5, [10, 20]]
\end{verbatim}
\afterverb
%
If we print {\tt list[3]}, we get {\tt [10, 20]}.  To extract an
element from the nested list, we can proceed in two steps:

\beforeverb
\begin{verbatim}
>>> elt = list[3]
>>> elt[0]
10
\end{verbatim}
\afterverb
%
Or we can combine them:

\beforeverb
\begin{verbatim}
>>> list[3][1]
20
\end{verbatim}
\afterverb
%
Bracket operators evaluate from left to right, so this expression
gets the three-eth element of {\tt list} and extracts the one-eth
element from it.

\section{Matrices}
\index{matrix}
\index{list!nested}

Nested lists are often used to represent matrices.  For example,
the matrix:

\beforefig
\centerline{\psfig{figure=illustrations/matrix.eps}}
\afterfig

might be represented as:

\beforeverb
\begin{verbatim}
>>> matrix = [[1, 2, 3], [4, 5, 6], [7, 8, 9]]
\end{verbatim}
\afterverb
%
{\tt matrix} is a list with three elements, where each
element is a row of the matrix.  We can select an entire row from the
matrix in the usual way:

\beforeverb
\begin{verbatim}
>>> matrix[1]
[4, 5, 6]
\end{verbatim}
\afterverb
%
Or we can extract a single element from the matrix using the
double-index form:

\beforeverb
\begin{verbatim}
>>> matrix[1][1]
5
\end{verbatim}
\afterverb
%
The first index selects the row, and the second index selects the
column.  Although this way of representing matrices is common, it is
not the only possibility.  A small variation is to use a list of
columns instead of a list of rows.  Later we will see a more
radical alternative using a dictionary.

\index{dictionary}
\index{row}
\index{column}


\section{Strings and lists}
\index{split function}
\index{join function}

Two of the most useful functions in the {\tt string} module involve
lists of strings.  The {\tt split} function breaks a string into a
list of words.  By default, any number of whitespace characters is
considered a word boundary:

\beforeverb
\begin{verbatim}
>>> import string
>>> song = "The rain in Spain..."
>>> string.split(song)
['The', 'rain', 'in', 'Spain...']
\end{verbatim}
\afterverb
%
An optional argument called a {\bf delimiter} can be used to specify which
characters to use as word boundaries.
The following example
uses the string {\tt ai} as the delimiter:

\beforeverb
\begin{verbatim}
>>> string.split(song, 'ai')
['The r', 'n in Sp', 'n...']
\end{verbatim}
\afterverb
%
Notice that the delimiter doesn't appear in the list.

The {\tt join} function is the inverse of {\tt split}.  It
takes a list of strings and
concatenates the elements with a space between each pair:

\beforeverb
\begin{verbatim}
>>> list = ['The', 'rain', 'in', 'Spain...']
>>> string.join(list)
'The rain in Spain...'
\end{verbatim}
\afterverb
%
Like {\tt split}, {\tt join} takes an optional delimiter
that is inserted between elements:

\beforeverb
\begin{verbatim}
>>> string.join(list, '_')
'The_rain_in_Spain...'
\end{verbatim}
\afterverb

\begin{quote}
\begin{quote}
{\em As an exercise, describe the relationship between {\tt
string.join(string.split(song))} and {\tt song}.  Are they the same
for all strings?  When would they be different?}
\end{quote}
\end{quote}


\section{Glossary}

\begin{description}

\item[list:] A named collection of objects, where each object is
identified by an index.

\item[index:] An integer variable or value that indicates an element of a
list.

\item[element:] One of the values in a list (or other sequence).  The
bracket operator selects elements of a list.

\item[sequence:] Any of the data types that consist of an ordered set of
elements, with each element identified by an index.

\item[nested list:] A list that is an element of another list.

\item[list traversal:] The sequential accessing of each element in a list.

\item[object:] A thing to which a variable can refer.

\item[aliases:] Multiple variables that contain references to the same object.

\item[clone:] To create a new object that has the same value as an
existing object.  Copying a reference to an object creates an alias
but doesn't clone the object.

\item[delimiter:] A character or string used to indicate where a
string should be split.

\index{list}
\index{index}
\index{sequence}
\index{element}
\index{nested list}
\index{list traversal}
\index{object}
\index{aliasing}
\index{clone}
\index{delimiter}

\end{description}

\clearemptydoublepage
% LaTeX source for textbook ``How to think like a computer scientist'
% Copyright (c)  2001  Allen B. Downey, Jeffrey Elkner, and Chris Meyers.

% Permission is granted to copy, distribute and/or modify this
% document under the terms of the GNU Free Documentation License,
% Version 1.1  or any later version published by the Free Software
% Foundation; with the Invariant Sections being "Contributor List",
% with no Front-Cover Texts, and with no Back-Cover Texts. A copy of
% the license is included in the section entitled "GNU Free
% Documentation License".

% This distribution includes a file named fdl.tex that contains the text
% of the GNU Free Documentation License.  If it is missing, you can obtain
% it from www.gnu.org or by writing to the Free Software Foundation,
% Inc., 59 Temple Place - Suite 330, Boston, MA 02111-1307, USA.



\chapter{Tuples}
\label{tuplechap}
\index{tuple}

\section{Mutability and tuples}
\index{tuple}
\index{data type!tuple}
\index{type!tuple}
\index{data type!immutable}

So far, you have seen two compound types: strings, which are made up
of characters; and lists, which are made up of elements of any type.
One of the differences we noted is that the elements of a list can be
modified, but the characters in a string cannot.  In other words, strings
are {\bf immutable} and lists are {\bf mutable}.

\index{mutable}
\index{immutable}

There is another type in Python called a {\bf tuple} that is similar
to a list except that it is immutable.  Syntactically, a tuple is a
comma-separated list of values:

\beforeverb
\begin{verbatim}
>>> tuple = 'a', 'b', 'c', 'd', 'e'
\end{verbatim}
\afterverb
%
Although it is not necessary, it is conventional to enclose tuples in
parentheses:

\beforeverb
\begin{verbatim}
>>> tuple = ('a', 'b', 'c', 'd', 'e')
\end{verbatim}
\afterverb
%
To create a tuple with a single element, we have to include the final
comma:

\beforeverb
\begin{verbatim}
>>> t1 = ('a',)
>>> type(t1)
<type 'tuple'>
\end{verbatim}
\afterverb
%
Without the comma, Python treats {\tt ('a')} as a string in
parentheses:

\beforeverb
\begin{verbatim}
>>> t2 = ('a')
>>> type(t2)
<type 'str'>
\end{verbatim}
\afterverb
%
Syntax issues aside, the operations on tuples are the same as the
operations on lists.  The index operator selects an element from
a tuple.

\beforeverb
\begin{verbatim}
>>> tuple = ('a', 'b', 'c', 'd', 'e')
>>> tuple[0]
'a'
\end{verbatim}
\afterverb
%
And the slice operator selects a range of elements.

\beforeverb
\begin{verbatim}
>>> tuple[1:3]
('b', 'c')
\end{verbatim}
\afterverb
%
But if we try to modify one of the elements of the tuple, we get
an error:

\index{runtime error}

\beforeverb
\begin{verbatim}
>>> tuple[0] = 'A'
TypeError: object doesn't support item assignment
\end{verbatim}
\afterverb
%
Of course, even if we can't modify the elements of a tuple, we can
replace it with a different tuple:

\beforeverb
\begin{verbatim}
>>> tuple = ('A',) + tuple[1:]
>>> tuple
('A', 'b', 'c', 'd', 'e')
\end{verbatim}
\afterverb
%

\section{Tuple assignment}
\label{tuple assignment}
\index{tuple assignment}
\index{assignment!tuple}

Once in a while, it is useful to swap the values of two variables.
With conventional assignment statements, we have to use a temporary
variable.  For example, to swap {\tt a} and {\tt b}:

\beforeverb
\begin{verbatim}
>>> temp = a
>>> a = b
>>> b = temp
\end{verbatim}
\afterverb
%
If we have to do this often, this approach becomes cumbersome.  Python
provides a form of {\bf tuple assignment} that solves this problem neatly:

\beforeverb
\begin{verbatim}
>>> a, b = b, a
\end{verbatim}
\afterverb
%
The left side is a tuple of variables; the right side is a tuple of
values.  Each value is assigned to its respective variable.  
All the expressions on the right side are evaluated before any
of the assignments.
This feature makes tuple
assignment quite versatile.

Naturally, the number of variables on the left and the number of
values on the right have to be the same:

\beforeverb
\begin{verbatim}
>>> a, b, c, d = 1, 2, 3
ValueError: unpack tuple of wrong size
\end{verbatim}
\afterverb
%

\section{Tuples as return values}
\index{tuple}
\index{value!tuple}
\index{return value!tuple}
\index{function!tuple as return value}

Functions can return tuples as return values.  For example, we could
write a function that swaps two parameters:

\beforeverb
\begin{verbatim}
def swap(x, y):
  return y, x
\end{verbatim}
\afterverb
%
Then we can assign the return value to a
tuple with two variables:

\beforeverb
\begin{verbatim}
a, b = swap(a, b)
\end{verbatim}
\afterverb
%
In this case, there is no great advantage in making {\tt swap} a
function.  In fact, there is a danger in trying to encapsulate {\tt
swap}, which is the following tempting mistake:

\beforeverb
\begin{verbatim}
def swap(x, y):      # incorrect version
  x, y = y, x
\end{verbatim}
\afterverb
%
If we call this function like this:

\beforeverb
\begin{verbatim}
swap(a, b)
\end{verbatim}
\afterverb
%
then {\tt a} and {\tt x} are aliases for the same value.  Changing {\tt x}
inside {\tt swap} makes {\tt x} refer to a different value, but it has no
effect on {\tt a} in {\tt \_\_main\_\_}.  Similarly, changing {\tt y} has no
effect on {\tt b}.

This function runs without producing an error message, but it
doesn't do what we intended.  This is an example of a semantic
error.

\index{semantic error}

\begin{quote}
{\em As an exercise, draw a state diagram for this function so that
you can see why it doesn't work.}
\end{quote}


\section{Random numbers}
\index{random number}
\index{number!random}

Most computer programs do the same thing every time they execute,
so they are said to be {\bf deterministic}.  Determinism is usually a
good thing, since we expect the same calculation to yield the same
result.  For some applications, though, we want the computer to
be unpredictable.  Games are an obvious example, but there are
more.

Making a program truly nondeterministic turns out to be not so easy,
but there are ways to make it at least seem nondeterministic.  One of
them is to generate random numbers and use them to determine the
outcome of the program.  Python provides a built-in function that
generates {\bf pseudorandom} numbers, which are not truly random in
the mathematical sense, but for our purposes they will do.

The {\tt random} module contains a function called {\tt random} that
returns a floating-point number between 0.0 and 1.0.  Each time you
call {\tt random}, you get the next number in a long series.  To see a
sample, run this loop:

\beforeverb
\begin{verbatim}
import random

for i in range(10):
  x = random.random()
  print x
\end{verbatim}
\afterverb
%
To generate a random number between 0.0 and an upper bound like
{\tt high}, multiply {\tt x} by {\tt high}.

\begin{quote}
{\em As an exercise, generate a random number between {\tt low} and
{\tt high}.}
\end{quote}

\begin{quote}
{\em As an additional exercise, generate a random {\em integer}
between {\tt low} and {\tt high}, including both end points.}
\end{quote}


\section{List of random numbers}

The first step is to generate a list of random values.  {\tt
randomList} takes an integer argument and returns a list of random
numbers with the given length.  It starts with a list of {\tt n}
zeros.  Each time through the loop, it replaces one of the elements
with a random number.
The return value is a reference to the complete list:

\beforeverb
\begin{verbatim}
def randomList(n):
  s = [0] * n
  for i in range(n):
    s[i] = random.random()
  return s
\end{verbatim}
\afterverb
%
We'll test this function with a list of eight elements.  For
purposes of debugging, it is a good idea to start small.

\adjustpage{-2}
\pagebreak

\beforeverb
\begin{verbatim}
>>> randomList(8)
0.15156642489
0.498048560109
0.810894847068
0.360371157682
0.275119183077
0.328578797631
0.759199803101
0.800367163582
\end{verbatim}
\afterverb
%
The numbers generated by {\tt random} are supposed to be distributed
uniformly, which means that every value is equally likely.

If we divide the range of possible
values into equal-sized ``buckets,'' and count the number of times a
random value falls in each bucket, we should get roughly the
same number in each.

We can test this theory by writing a program to 
divide the range into
buckets and count the number of values in each.


\section{Counting}
\index{counting}

A good approach to problems like this is to divide the problem into
subproblems and look for subproblems that fit a computational pattern
you have seen before.

In this case, we want to traverse a list of numbers and count the
number of times a value falls in a given range.  That sounds familiar.
In Section~\ref{counter}, we wrote a program that traversed a string and
counted the number of times a given letter appeared.

So, we can proceed by copying the old program and adapting it
for the current problem.  The original program was:

\beforeverb
\begin{verbatim}
count = 0
for char in fruit:
  if char == 'a':
    count = count + 1
print count
\end{verbatim}
\afterverb
%
The first step is to replace {\tt fruit} with {\tt t} and
{\tt char} with {\tt num}.  That doesn't change the program;
it just makes it more readable.

The second step is to change the test.  We aren't interested
in finding letters.  We want to see if {\tt num} is between
the given values {\tt low} and {\tt high}.  

\beforeverb
\begin{verbatim}
count = 0
for num in t:
  if low < num < high:
    count = count + 1
print count
\end{verbatim}
\afterverb
%
The last step is to encapsulate this code in a function called
{\tt inBucket}.  The parameters are the list and the values
{\tt low} and {\tt high}.

\beforeverb
\begin{verbatim}
def inBucket(t, low, high):
  count = 0
  for num in t:
    if low < num < high:
      count = count + 1
  return count
\end{verbatim}
\afterverb
%
By copying and modifying an existing program, we were able
to write this function quickly and save a lot of debugging
time.  This development plan is called {\bf pattern matching}.
If you find yourself working on a problem you have solved
before, reuse the solution.


\section{Many buckets}

As the number of buckets increases, {\tt inBucket} gets
a little unwieldy.  With two buckets, it's not bad:

\beforeverb
\begin{verbatim}
low = inBucket(a, 0.0, 0.5)
high = inBucket(a, 0.5, 1)
\end{verbatim}
\afterverb
%
But with four buckets it is getting cumbersome.

\beforeverb
\begin{verbatim}
bucket1 = inBucket(a, 0.0, 0.25)
bucket2 = inBucket(a, 0.25, 0.5)
bucket3 = inBucket(a, 0.5, 0.75)
bucket4 = inBucket(a, 0.75, 1.0)
\end{verbatim}
\afterverb
%
There are two problems.  One is that we have to make up new
variable names for each result.  The other is that we have to
compute the range for each bucket.

We'll solve the second problem first.  If the number of buckets
is {\tt numBuckets}, then the width of each bucket is
{\tt 1.0 / numBuckets}.

We'll use a loop to compute the range of each bucket.
The loop variable, {\tt i},
counts from 0 to {\tt numBuckets-1}:

\beforeverb
\begin{verbatim}
bucketWidth = 1.0 / numBuckets
for i in range(numBuckets):
  low = i * bucketWidth
  high = low + bucketWidth
  print low, "to", high
\end{verbatim}
\afterverb
%
To compute the low end of each bucket, we multiply the loop variable
by the bucket width.  The high end is just a {\tt bucketWidth} away.

With {\tt numBuckets = 8}, the output is:

\beforeverb
\begin{verbatim}
0.0 to 0.125
0.125 to 0.25
0.25 to 0.375
0.375 to 0.5
0.5 to 0.625
0.625 to 0.75
0.75 to 0.875
0.875 to 1.0
\end{verbatim}
\afterverb
%
You can confirm that each bucket is the same width, that they don't
overlap, and that they cover the entire range from 0.0 to 1.0.

Now back to the first problem.
We need a way to store eight integers, using the loop variable
to indicate one at a time.  By now you should be thinking,
``List!''

We have to create the bucket list outside the loop, because we only
want to do it once.  Inside the loop, we'll call {\tt inBucket}
repeatedly and update the {\tt i}-eth element of the list:

\beforeverb
\begin{verbatim}
numBuckets = 8
buckets = [0] * numBuckets
bucketWidth = 1.0 / numBuckets
for i in range(numBuckets):
  low = i * bucketWidth
  high = low + bucketWidth
  buckets[i] = inBucket(t, low, high)
print buckets
\end{verbatim}
\afterverb
%
With a list of 1000 values, this code produces this bucket list:

\beforeverb
\begin{verbatim}
[138, 124, 128, 118, 130, 117, 114, 131]
\end{verbatim}
\afterverb
%
These numbers are fairly close to 125, which is what we expected.  At
least, they are close enough that we can believe the random number
generator is working.

\begin{quote}
{\em As an exercise, 
test this function with some longer lists, and see if the
number of values in each bucket tends to level off.}
\end{quote}


\section{A single-pass solution}
\index{histogram}

Although this program works, it is not as efficient as it could be.
Every time it calls {\tt inBucket}, it traverses the entire list.  As
the number of buckets increases, that gets to be a lot of traversals.

It would be better to make a single pass through the list and compute
for each value the index of the bucket in which it falls.  Then we can
increment the appropriate counter.

In the previous section we took an index, {\tt i}, and multiplied it
by the {\tt bucketWidth} to find the lower bound of a given
bucket.  Now we want to take a value in the range 0.0 to 1.0 and find the
index of the bucket where it falls.

Since this problem is the inverse of the previous problem, we might
guess that we should divide by {\tt bucketWidth} instead of
multiplying.  That guess is correct.

Since {\tt bucketWidth = 1.0 / numBuckets}, dividing by {\tt
bucketWidth} is the same as multiplying by {\tt numBuckets}.  If we
multiply a number in the range 0.0 to 1.0 by {\tt numBuckets}, we get
a number in the range from 0.0 to {\tt numBuckets}.  If we round that
number to the next lower integer, we get exactly what we are looking
for---a bucket index:

\beforeverb
\begin{verbatim}
numBuckets = 8
buckets = [0] * numBuckets
for i in t:
  index = int(i * numBuckets)
  buckets[index] = buckets[index] + 1
\end{verbatim}
\afterverb
%
We used the {\tt int} function to convert a floating-point
number to an integer.

Is it possible for this calculation to produce an index that is out of
range (either negative or greater than {\tt len(buckets)-1})?

A list like {\tt buckets} that contains counts of the number of values
in each range is called a {\bf histogram}.

\begin{quote}
{\em As an exercise, write a function called {\tt histogram} that
takes a list and a number of buckets as arguments and returns
a histogram with the given number of buckets.}
\end{quote}

\adjustpage{-2}
\pagebreak

\section{Glossary}

\begin{description}

\item[immutable type:] A type in which the elements cannot be
modified.  Assignments
to elements or slices of immutable types cause an error.

\item[mutable type:] A data type in which the elements can be
modified.  All mutable
types are compound types.  Lists and dictionaries are mutable data
types; strings and tuples are not.

\item[tuple:] A sequence type that is similar to a list except that it is
immutable.  Tuples can be used wherever an immutable type is required, such
as a key in a dictionary.

\item[tuple assignment:] An assignment to all of the elements in a tuple using
a single assignment statement. Tuple assignment occurs in parallel rather
than in sequence, making it useful for swapping values.

\item[deterministic:] A program that does the same thing each time it is
called.

\item[pseudorandom:] A sequence of numbers that appear to be random but that
are actually the result of a deterministic computation.

\item[histogram:] A list of integers in which each element counts the 
number of times something happens.

\item[pattern matching:] A program development plan that involves
identifying a familiar computational pattern and copying the
solution to a similar problem.

\index{mutable type}
\index{immutable type}
\index{tuple}
\index{tuple assignment}
\index{assignment!tuple}
\index{deterministic}
\index{pseudorandom}
\index{histogram}
\index{pattern matching}

\end{description}


\clearemptydoublepage
% LaTeX source for textbook ``How to think like a computer scientist''
% Copyright (c)  2001  Allen B. Downey, Jeffrey Elkner, and Chris Meyers.

% Permission is granted to copy, distribute and/or modify this
% document under the terms of the GNU Free Documentation License,
% Version 1.1  or any later version published by the Free Software
% Foundation; with the Invariant Sections being "Contributor List",
% with no Front-Cover Texts, and with no Back-Cover Texts. A copy of
% the license is included in the section entitled "GNU Free
% Documentation License".

% This distribution includes a file named fdl.tex that contains the text
% of the GNU Free Documentation License.  If it is missing, you can obtain
% it from www.gnu.org or by writing to the Free Software Foundation,
% Inc., 59 Temple Place - Suite 330, Boston, MA 02111-1307, USA.

\chapter{Dictionaries}
\index{dictionary}

\index{dictionary}
\index{data type!dictionary}
\index{type!dict}
\index{key}
\index{key-value pair}
\index{index}

The compound types you have learned about---strings, lists, and
tuples---use integers as indices.  If you try to use any other type as
an index, you get an error.

{\bf Dictionaries} are similar to other compound types except that
they can use any immutable type as an index.  As an example, we will
create a dictionary to translate English words into Spanish.  For this
dictionary, the indices are {\tt strings}.

One way to create a dictionary is to start with the empty dictionary
and add elements.  The empty dictionary is denoted {\verb+{}+}:

\beforeverb
\begin{verbatim}
>>> eng2sp = {}
>>> eng2sp['one'] = 'uno'
>>> eng2sp['two'] = 'dos'
\end{verbatim}
\afterverb
%
The first assignment creates a dictionary named {\tt eng2sp}; the
other assignments add new elements to the dictionary.  We can print
the current value of the dictionary in the usual way:

\beforeverb
\begin{verbatim}
>>> print eng2sp
{'one': 'uno', 'two': 'dos'}
\end{verbatim}
\afterverb
%
The elements of a dictionary appear in a comma-separated list.  Each
entry contains an index and a value separated by a colon.  In a
dictionary, the indices are called {\bf keys}, so the elements are
called {\bf key-value pairs}.

Another way to create a dictionary is to provide a list of key-value
pairs using the same syntax as the previous output:

\beforeverb
\begin{verbatim}
>>> eng2sp = {'one': 'uno', 'two': 'dos', 'three': 'tres'}
\end{verbatim}
\afterverb
%
If we print the value of {\tt eng2sp} again, we get a surprise:

\beforeverb
\begin{verbatim}
>>> print eng2sp
{'one': 'uno', 'three': 'tres', 'two': 'dos'}
\end{verbatim}
\afterverb
%
The key-value pairs are not in order!  Fortunately, there is no reason
to care about the order, since the elements of a dictionary are never
indexed with integer indices.  Instead, we use the keys to look up the
corresponding values:

\beforeverb
\begin{verbatim}
>>> print eng2sp['two']
'dos'
\end{verbatim}
\afterverb
%
The key {\tt 'two'} yields the value {\tt 'dos'} even though it
appears in the third key-value pair.


\section{Dictionary operations}
\index{dictionary!operation}
\index{operation!dictionary}

The {\tt del} statement removes a key-value pair from a dictionary.
For example, the following dictionary contains the names of various
fruits and the number of each fruit in stock:

\beforeverb
\begin{verbatim}
>>> inventory = {'apples': 430, 'bananas': 312, 'oranges': 525, 
'pears': 217}
>>> print inventory
{'oranges': 525, 'apples': 430, 'pears': 217, 'bananas': 312}
\end{verbatim}
\afterverb
%
If someone buys all of the pears, we can remove the entry from the
dictionary:

\beforeverb
\begin{verbatim}
>>> del inventory['pears']
>>> print inventory
{'oranges': 525, 'apples': 430, 'bananas': 312}
\end{verbatim}
\afterverb
%
Or if we're expecting more pears soon, we might just change the
value associated with pears:

\beforeverb
\begin{verbatim}
>>> inventory['pears'] = 0
>>> print inventory
{'oranges': 525, 'apples': 430, 'pears': 0, 'bananas': 312}
\end{verbatim}
\afterverb
%
The {\tt len} function also works on dictionaries; it returns the
number of key-value pairs:

\beforeverb
\begin{verbatim}
>>> len(inventory)
4
\end{verbatim}
\afterverb
%


\section{Dictionary methods}
\index{dictionary!method}
\index{method!dictionary}
\index{method}
\index{method!invocation}
\index{invoking method}

A {\bf method} is similar to a function---it takes arguments and
returns a value---but the syntax is different.  For example, the {\tt
keys} method takes a dictionary and returns a list of the keys that
appear, but instead of the function syntax {\tt keys(eng2sp)}, we use
the method syntax {\tt eng2sp.keys()}.

\index{dot notation}

\beforeverb
\begin{verbatim}
>>> eng2sp.keys()
['one', 'three', 'two']
\end{verbatim}
\afterverb
%
This form of dot notation specifies the name of the function, {\tt
keys}, and the name of the object to apply the function to, {\tt
eng2sp}.  The parentheses indicate that this method has no
parameters.

A method call is called an {\bf invocation}; in this case, we would
say that we are invoking {\tt keys} on the object {\tt eng2sp}.

The {\tt values} method is similar; it returns a list of the values in
the dictionary:

\beforeverb
\begin{verbatim}
>>> eng2sp.values()
['uno', 'tres', 'dos']
\end{verbatim}
\afterverb
%
The {\tt items} method returns both, in the form of a
list of tuples---one for each key-value
pair:

\beforeverb
\begin{verbatim}
>>> eng2sp.items()
[('one','uno'), ('three', 'tres'), ('two', 'dos')]
\end{verbatim}
\afterverb
%
The syntax provides useful type information.  The square brackets
indicate that this is a list.  The parentheses indicate that
the elements of the list are tuples.

If a method takes an argument, it uses the same syntax as a function
call.  For example, the method {\tt has\_key} takes a key
and returns
true (1) if the key appears in the dictionary:

\beforeverb
\begin{verbatim}
>>> eng2sp.has_key('one')
True
>>> eng2sp.has_key('deux')
False
\end{verbatim}
\afterverb
%
If you try to call a method without specifying an object, you get an
error.  In this case, the error message is not very helpful:

\beforeverb
\begin{verbatim}
>>> has_key('one')
NameError: has_key
\end{verbatim}
\afterverb
%

\index{runtime error}


\section{Aliasing and copying}
\index{aliasing}
\index{copying}
\index{cloning}

Because dictionaries are mutable, you need to be aware of aliasing.
Whenever two variables refer to the same object, changes to one affect
the other.

If you want to modify a dictionary and keep a copy of the original,
use the {\tt copy} method.  For example, {\tt opposites} is a
dictionary that contains pairs of opposites:

\beforeverb
\begin{verbatim}
>>> opposites = {'up': 'down', 'right': 'wrong', 'true': 'false'}
>>> alias = opposites
>>> copy = opposites.copy()
\end{verbatim}
\afterverb
%
{\tt alias} and {\tt opposites} refer to the same object; {\tt copy}
refers to a fresh copy of the same dictionary.  If we modify {\tt
alias}, {\tt opposites} is also changed:

\beforeverb
\begin{verbatim}
>>> alias['right'] = 'left'
>>> opposites['right']
'left'
\end{verbatim}
\afterverb
%
If we modify {\tt copy}, {\tt opposites} is unchanged:

\beforeverb
\begin{verbatim}
>>> copy['right'] = 'privilege'
>>> opposites['right']
'left'
\end{verbatim}
\afterverb
%


\section{Sparse matrices }
\index{matrix!sparse}
\index{nested list}
\index{list!nested}

In Section~\ref{nested lists}, we used a list of lists to represent a
matrix.  That is a good choice for a matrix with mostly nonzero
values, but consider a sparse matrix like this one:

\beforefig
\centerline{\psfig{figure=illustrations/sparse.eps}}
\afterfig

The list representation contains a lot of zeroes:

\beforeverb
\begin{verbatim}
matrix = [ [0,0,0,1,0],
           [0,0,0,0,0],
           [0,2,0,0,0],
           [0,0,0,0,0],
           [0,0,0,3,0] ]
\end{verbatim}
\afterverb
%
An alternative is to use a dictionary.
For the keys, we can use tuples that contain the row and column
numbers.  Here is the dictionary representation of the same matrix:

\beforeverb
\begin{verbatim}
matrix = {(0,3): 1, (2, 1): 2, (4, 3): 3}
\end{verbatim}
\afterverb
%
We only need three key-value pairs, one for each nonzero element of the
matrix.  Each key is a tuple, and each value is an integer.

To access an element of the matrix, we could use the {\tt []}
operator:

\beforeverb
\begin{verbatim}
matrix[0,3]
1
\end{verbatim}
\afterverb
%
Notice that the syntax for the dictionary representation is not the
same as the syntax for the nested list representation.  Instead of
two integer indices, we use one index, which is a tuple of integers.

There is one problem.
If we specify an element that is zero, we get an
error, because there is no entry in the dictionary with that key:

\index{runtime error}

\beforeverb
\begin{verbatim}
>>> matrix[1,3]
KeyError: (1, 3)
\end{verbatim}
\afterverb
%
The {\tt get} method solves this problem:

\beforeverb
\begin{verbatim}
>>> matrix.get((0,3), 0)
1
\end{verbatim}
\afterverb
%
The first argument is the key; the second argument is the value
{\tt get} should return if the key is not in the dictionary:

\beforeverb
\begin{verbatim}
>>> matrix.get((1,3), 0)
0
\end{verbatim}
\afterverb
%
{\tt get} definitely improves the semantics of accessing
a sparse matrix.  Shame about the syntax.


\section{Hints}
\index{hint}
\index{Fibonacci function}

If you played around with the {\tt fibonacci} function from
Section~\ref{one more example}, you might have noticed that the bigger
the argument you provide, the longer the function takes to run.
Furthermore, the run time increases very quickly.  On one of our
machines, {\tt fibonacci(20)} finishes instantly, {\tt fibonacci(30)}
takes about a second, and {\tt fibonacci(40)} takes roughly forever.

To understand why, consider this {\bf call graph} for
{\tt fibonacci} with {\tt n=4}:

\beforefig
\centerline{\psfig{figure=illustrations/fibonacci.eps,height=2in}}
\afterfig

A call graph shows a set function frames, with lines connecting each
frame to the frames of the functions it calls.  At the top of the
graph, {\tt fibonacci} with {\tt n=4} calls {\tt fibonacci} with {\tt
n=3} and {\tt n=2}.  In turn, {\tt fibonacci} with {\tt n=3} calls
{\tt fibonacci} with {\tt n=2} and {\tt n=1}.  And so on.

\index{function frame}
\index{frame}
\index{call graph}

Count how many times {\tt fibonacci(0)} and {\tt fibonacci(1)} are
called.  This is an inefficient solution to the problem, and it gets
far worse as the argument gets bigger.

A good solution is to keep track of values that have already been
computed by storing them in a dictionary.  A previously computed value
that is stored for later use is called a {\bf hint}.  Here is
an implementation of {\tt fibonacci} using hints:

\beforeverb
\begin{verbatim}
previous = {0:1, 1:1}

def fibonacci(n):
  if previous.has_key(n):
    return previous[n]
  else:
    newValue = fibonacci(n-1) + fibonacci(n-2)
    previous[n] = newValue
    return newValue
\end{verbatim}
\afterverb
%
The dictionary named {\tt previous} keeps track of the Fibonacci
numbers we already know.  We start with only
two pairs: 0 maps to 1; and 1 maps to 1.

Whenever {\tt fibonacci} is called, it checks the dictionary to
determine if it contains the result.
If it's there, the function can return
immediately without making any more recursive calls.  If not, it has
to compute the new value.  The new value is added to the dictionary
before the function returns.

Using this version of {\tt fibonacci}, our machines can compute
{\tt fibonacci(40)} in an eyeblink.  But when we try to compute
{\tt fibonacci(50)}, we see the following:

\beforeverb
\begin{verbatim}
>>> fibonacci(50)
20365011074L
\end{verbatim}
\afterverb
%
The {\tt L} at the end of the result indicates that the answer
+(20,365,011,074) is too big to fit into a Python integer.  Python
has automatically converted the result to a long integer.

\section{Long integers}
\index{long integer}
\index{data type!long integer}
\index{type!long}
\index{integer!long}

Python provides a type called {\tt long} that can handle any size
integer.  There are two ways to create a {\tt long} value.  One is
to write an integer with a capital {\tt L} at the end:

\beforeverb
\begin{verbatim}
>>> type(1L)
<type 'long'>
\end{verbatim}
\afterverb
%
The other is to use the {\tt long} function to convert a value to a
{\tt long}.  {\tt long} can accept any numerical type and even
strings of digits:

\index{type coercion}
\index{coercion!type}

\beforeverb
\begin{verbatim}
>>> long(1)
1L
>>> long(3.9)
3L
>>> long('57')
57L
\end{verbatim}
\afterverb
%
All of the math operations work on {\tt long}s, so in general
any code that works with integers will also work with long
integers.  Any time the result of a computation is too big
to be represented with an integer, Python detects the overflow
and returns the result as a long integer.  For example:

\beforeverb
\begin{verbatim}
>>> 1000 * 1000
1000000
>>> 100000 * 100000
10000000000L
\end{verbatim}
\afterverb
%
In the first case the result has type {\tt int}; in the
second case it is {\tt long}.


\section{Counting letters}
\index{counting}
\index{histogram}
\index{compression}

In Chapter~\ref{strings}, we wrote a function that counted the number
of occurrences of a letter in a string.  A more general version of
this problem is to form a histogram of the letters in the string, that
is, how many times each letter appears.

Such a histogram might be useful for compressing a text
file.  Because different letters appear with different frequencies, we
can compress a file by using shorter codes for common letters and
longer codes for letters that appear less frequently.

Dictionaries provide an elegant way to generate a histogram:

\beforeverb
\begin{verbatim}
>>> letterCounts = {}
>>> for letter in "Mississippi":
...   letterCounts[letter] = letterCounts.get (letter, 0) + 1
...
>>> letterCounts
{'M': 1, 's': 4, 'p': 2, 'i': 4}
\end{verbatim}
\afterverb
%
We start with an empty dictionary.  For each letter in the string, we
find the current count (possibly zero) and increment it.  At the end,
the dictionary contains pairs of letters and their frequencies.

It might be more appealing to display the histogram in alphabetical
order.  We can do that with the {\tt items} and {\tt sort} methods:

\beforeverb
\begin{verbatim}
>>> letterItems = letterCounts.items()
>>> letterItems.sort()
>>> print letterItems
[('M', 1), ('i', 4), ('p', 2), ('s', 4)]
\end{verbatim}
\afterverb
%
You have seen the {\tt items} method before, but {\tt sort} is the
first method you have encountered that applies to lists.  There are
several other list methods, including {\tt append}, {\tt extend}, and
{\tt reverse}.  Consult the Python documentation for details.

\index{method!list}
\index{list method}


\section{Glossary}

\begin{description}

\item[dictionary:] A collection of key-value pairs that maps from
keys to values.
The keys can be any immutable type, and the values can be any type.

\item[key:] A value that is used to look up an entry in a dictionary.

\item[key-value pair:] One of the items in a dictionary.

\item[method:] A kind of function that is called with a different syntax and
invoked ``on'' an object.

\item[invoke:] To call a method.

\item[hint:] Temporary storage of a precomputed value to avoid redundant
computation.

\item[overflow:] A numerical result that is too large to be represented
in a numerical format.

\index{dictionary}
\index{key}
\index{key-value pair}
\index{hint}
\index{method}
\index{invoke}

\end{description}

\clearemptydoublepage
% LaTeX source for textbook ``How to think like a computer scientist''
% Copyright (c)  2001  Allen B. Downey, Jeffrey Elkner, and Chris Meyers.

% Permission is granted to copy, distribute and/or modify this
% document under the terms of the GNU Free Documentation License,
% Version 1.1  or any later version published by the Free Software
% Foundation; with the Invariant Sections being "Contributor List",
% with no Front-Cover Texts, and with no Back-Cover Texts. A copy of
% the license is included in the section entitled "GNU Free
% Documentation License".

% This distribution includes a file named fdl.tex that contains the text
% of the GNU Free Documentation License.  If it is missing, you can obtain
% it from www.gnu.org or by writing to the Free Software Foundation,
% Inc., 59 Temple Place - Suite 330, Boston, MA 02111-1307, USA.

\chapter{Files and exceptions}
\index{file}
\index{type!file}

While a program is running, its data is in memory.  When the program
ends, or the computer shuts down, data in memory disappears.  To
store data permanently, you have to put it in a {\bf file}.
Files are usually stored on a
hard drive, floppy drive, or CD-ROM.

When there are a large number of files, they are often organized
into {\bf directories} (also called ``folders'').
Each file is identified by a unique name, or a combination of a
file name and a directory name.

By reading and writing files, programs can
exchange information with each other and generate printable formats
like PDF.

Working with files is a lot like working with books.  To use a book,
you have to open it.  When you're done, you have to close it.  While
the book is open, you can either write in it or read from it.  In
either case, you know where you are in the book.  Most of the time,
you read the whole book in its natural order, but you can also skip
around.

All of this applies to files as well.  To open a file, you specify
its name and indicate whether you want to read or write.

Opening a file creates a file object.  In this example,
the variable {\tt f} refers to the new file object.

\beforeverb
\begin{verbatim}
>>> f = open("test.dat","w")
>>> print f
<open file 'test.dat', mode 'w' at fe820>
\end{verbatim}
\afterverb
%
The open function takes two arguments.  The first is the name of the file,
and the second is the mode.  Mode {\tt "w"} means that we are opening
the file for writing.

If there is no file named {\tt test.dat}, it will be created.
If there already is one, it will be replaced by the file we are
writing.

When we print the file object, we see the name of the file, the
mode, and the location of the object.

To put data in the file we invoke the {\tt write} method on the
file object:

\beforeverb
\begin{verbatim}
>>> f.write("Now is the time")
>>> f.write("to close the file")
\end{verbatim}
\afterverb
%
Closing the file tells the system that we are done writing and
makes the file available for reading:

\beforeverb
\begin{verbatim}
>>> f.close()
\end{verbatim}
\afterverb
%
Now we can open the file again, this time for reading, and read the
contents into a string.  This time, the mode argument
is {\tt "r"} for reading:

\beforeverb
\begin{verbatim}
>>> f = open("test.dat","r")
\end{verbatim}
\afterverb
%
If we try to open a file that doesn't exist, we get an error:

\index{runtime error}

\beforeverb
\begin{verbatim}
>>> f = open("test.cat","r")
IOError: [Errno 2] No such file or directory: 'test.cat'
\end{verbatim}
\afterverb
%
Not surprisingly, the {\tt read} method reads data from the
file.  With no arguments, it reads the entire contents of the file:

\beforeverb
\begin{verbatim}
>>> text = f.read()
>>> print text
Now is the timeto close the file
\end{verbatim}
\afterverb
%
There is no space between ``time'' and ``to'' because we did
not write a space between the strings.

{\tt read} can also take an argument that indicates how many
characters to read:

\beforeverb
\begin{verbatim}
>>> f = open("test.dat","r")
>>> print f.read(5)
Now i
\end{verbatim}
\afterverb
%
If not enough characters are left in the file,
{\tt read} returns the remaining characters.
When we get to the end of the file,
{\tt read} returns the empty string:

\beforeverb
\begin{verbatim}
>>> print f.read(1000006)
s the timeto close the file
>>> print f.read()

>>>
\end{verbatim}
\afterverb
%
The following function copies a file, reading and writing
up to fifty characters at a time.  The first argument is the name of
the original file; the second is the name of the new file:

\beforeverb
\begin{verbatim}
def copyFile(oldFile, newFile):
  f1 = open(oldFile, "r")
  f2 = open(newFile, "w")
  while True:
    text = f1.read(50)
    if text == "":
      break
    f2.write(text)
  f1.close()
  f2.close()
  return
\end{verbatim}
\afterverb
%
The {\tt break} statement is new.  Executing it breaks out of the
loop; the flow of execution moves to the first statement after
the loop.

\index{break statement}
\index{statement!break}

In this example, the {\tt while} loop is infinite because the
value {\tt True} is always true.  The {\em only} way to get out
of the loop is to execute {\tt break}, which happens when
{\tt text} is the empty string, which happens when we get to
the end of the file.



\section{Text files}
\index{text file}
\index{file!text}

A {\bf text file} is a file that contains printable characters and
whitespace, organized into lines separated by newline characters.
Since Python is specifically designed to process text files, it
provides methods that make the job easy.

To demonstrate, we'll
create a text file with three lines of text separated by
newlines:

\beforeverb
\begin{verbatim}
>>> f = open("test.dat","w")
>>> f.write("line one\nline two\nline three\n")
>>> f.close()
\end{verbatim}
\afterverb
%
The {\tt readline} method reads all the characters
up to and including the next newline character:

\beforeverb
\begin{verbatim}
>>> f = open("test.dat","r")
>>> print f.readline()
line one

>>>
\end{verbatim}
\afterverb
%
{\tt readlines} returns all of the remaining
lines as a list of strings:

\beforeverb
\begin{verbatim}
>>> print f.readlines()
['line two\012', 'line three\012']
\end{verbatim}
\afterverb
%
In this case, the output is in list format, which means that the
strings appear with quotation marks and the newline character
appears as the escape sequence {\tt \\012}.

At the end of the file, {\tt readline} returns the empty string
and {\tt readlines} returns the empty list:

\beforeverb
\begin{verbatim}
>>> print f.readline()

>>> print f.readlines()
[]
\end{verbatim}
\afterverb
%
The following is an example of a line-processing program.
{\tt filterFile} makes a copy of {\tt oldFile}, omitting
any lines that begin with {\tt \#}:

\beforeverb
\begin{verbatim}
def filterFile(oldFile, newFile):
  f1 = open(oldFile, "r")
  f2 = open(newFile, "w")
  while True:
    text = f1.readline()
    if text == "":
      break
    if text[0] == '#':
      continue
    f2.write(text)
  f1.close()
  f2.close()
  return
\end{verbatim}
\afterverb
%
The {\tt continue} statement ends the current iteration of the
loop, but continues looping.  The flow of
execution moves to the top of the loop, checks the condition,
and proceeds accordingly.

\index{continue statement}
\index{statement!continue}

Thus, if {\tt text} is the empty string, the loop exits.  If
the first character of {\tt text} is a hash mark, the flow
of execution goes to
the top of the loop.  Only if both conditions fail do we copy
{\tt text} into the new file.


\section{Writing variables}
\index{format operator}
\index{format string}
\index{operator!format}

The argument of {\tt write} has to be a string, so if we want
to put other values in a file, we have to convert them to
strings first.  The easiest way to do that is with the {\tt str}
function:

\beforeverb
\begin{verbatim}
>>> x = 52
>>> f.write (str(x))
\end{verbatim}
\afterverb
%
An alternative is to use the {\bf format operator} {\tt \%}.  When
applied to integers, {\tt \%} is the modulus operator.  But
when the first operand is a string, {\tt \%} is the format operator.

The first operand is the {\bf format string}, and the second operand
is a tuple of expressions.  The result is a string that contains
the values of the expressions, formatted according to the format
string.

As a simple example, the {\bf format sequence} {\tt "\%d"} means that
the first expression in the tuple should be formatted as an
integer.  Here the letter {\em d} stands for ``decimal'':

\beforeverb
\begin{verbatim}
>>> cars = 52
>>> "%d" % cars
'52'
\end{verbatim}
\afterverb
%
The result is the string {\tt '52'}, which is not to be confused
with the integer value {\tt 52}.

A format sequence can appear anywhere in the format string,
so we can embed a value in a sentence:

\beforeverb
\begin{verbatim}
>>> cars = 52
>>> "In July we sold %d cars." % cars
'In July we sold 52 cars.'
\end{verbatim}
\afterverb
%
The format sequence {\tt "\%f"} formats the next item in
the tuple as a floating-point number, and {\tt "\%s"} formats
the next item as a string:

\beforeverb
\begin{verbatim}
>>> "In %d days we made %f million %s." % (34,6.1,'dollars')
'In 34 days we made 6.100000 million dollars.'
\end{verbatim}
\afterverb
%
By default, the floating-point format prints six decimal places.

The number of expressions in the tuple has to match the number
of format sequences in the string.  Also, the types of the
expressions have to match the format sequences:

\index{runtime error}

\beforeverb
\begin{verbatim}
>>> "%d %d %d" % (1,2)
TypeError: not enough arguments for format string
>>> "%d" % 'dollars'
TypeError: illegal argument type for built-in operation
\end{verbatim}
\afterverb
%
In the first example, there aren't enough expressions; in the
second, the expression is the wrong type.

For more control over the format of numbers, we can specify
the number of digits as part of the format sequence:

\beforeverb
\begin{verbatim}
>>> "%6d" % 62
'    62'
>>> "%12f" % 6.1
'    6.100000'
\end{verbatim}
\afterverb
%
The number after the percent sign is the minimum number of spaces
the number will take up.  If the value provided takes fewer digits,
leading spaces are added.  If the number of spaces is negative,
trailing spaces are added:

\beforeverb
\begin{verbatim}
>>> "%-6d" % 62
'62    '
\end{verbatim}
\afterverb
%
For floating-point numbers, we can also
specify the number of digits after the decimal point:

\beforeverb
\begin{verbatim}
>>> "%12.2f" % 6.1
'        6.10'
\end{verbatim}
\afterverb
%
In this example, the result takes up twelve spaces and includes two
digits after the decimal.  This format is useful for printing
dollar amounts with the decimal points aligned.

\index{dictionary}

For example, imagine a dictionary that contains
student names as keys and hourly wages as values.
Here is a function that prints the contents of the dictionary
as a formatted report:

\beforeverb
\begin{verbatim}
def report (wages) :
  students = wages.keys()
  students.sort()
  for student in students :
    print "%-20s %12.2f" % (student, wages[student])
\end{verbatim}
\afterverb
%
To test this function, we'll create a small dictionary
and print the contents:

\beforeverb
\begin{verbatim}
>>> wages = {'mary': 6.23, 'joe': 5.45, 'joshua': 4.25}
>>> report (wages)
joe                          5.45
joshua                       4.25
mary                         6.23
\end{verbatim}
\afterverb
%
By controlling the width of each value, we guarantee that the columns
will line up, as long as the names contain fewer than twenty-one
characters and the wages are less than one billion dollars an hour.


\section{Directories}
\index{directory}

When you create a new file by opening it and writing, the new
file goes in the current directory (wherever you were when
you ran the program).  Similarly, when you open a file for
reading, Python looks for it in the current directory.

If you want to open a file somewhere else, you have to specify
the {\bf path} to the file, which is the name of the directory
(or folder) where the file is located:

\beforeverb
\begin{verbatim}
>>>   f = open("/usr/share/dict/words","r")
>>>   print f.readline()
Aarhus
\end{verbatim}
\afterverb
%
This example opens a file named {\tt words} that resides in a
directory named {\tt dict}, which resides in {\tt share}, which
resides in {\tt usr}, which resides in the top-level directory
of the system, called {\tt /}.

\index{path}
\index{delimiter}

You cannot use {\tt /}
as part of a filename; it is reserved as a delimiter between
directory and filenames.

The file {\tt /usr/share/dict/words} contains a list of words
in alphabetical order, of which the first is the name of a
Danish university.


\section{Pickling}
\index{pickling}

In order to put values into a file, you have to convert them
to strings.  You have already seen how to do that with {\tt str}:

\beforeverb
\begin{verbatim}
>>> f.write (str(12.3))
>>> f.write (str([1,2,3]))
\end{verbatim}
\afterverb
%
The problem is that when you read the value back, you get a string.
The original type information has been lost.  In fact, you can't
even tell where one value ends and the next begins:

\beforeverb
\begin{verbatim}
>>>   f.readline()
'12.3[1, 2, 3]'
\end{verbatim}
\afterverb
%
The solution is {\bf pickling}, so called because it ``preserves''
data structures.  The {\tt pickle} module contains the necessary
commands.  To use it, import {\tt pickle} and then open the file in
the usual way:

\beforeverb
\begin{verbatim}
>>> import pickle
>>> f = open("test.pck","w")
\end{verbatim}
\afterverb
%
To store a data structure, use the {\tt dump} method and
then close the file in the usual way:

\beforeverb
\begin{verbatim}
>>> pickle.dump(12.3, f)
>>> pickle.dump([1,2,3], f)
>>> f.close()
\end{verbatim}
\afterverb
%
Then we can open the file for reading and load the data structures
we dumped:

\beforeverb
\begin{verbatim}
>>> f = open("test.pck","r")
>>> x = pickle.load(f)
>>> x
12.3
>>> type(x)
<type 'float'>
>>> y = pickle.load(f)
>>> y
[1, 2, 3]
>>> type(y)
<type 'list'>
\end{verbatim}
\afterverb
%
Each time we invoke {\tt load}, we get a single value from
the file, complete with its original type.


\section{Exceptions}
\index{try statement}
\index{statement!try}
\index{raise exception}
\index{handle exception}
\index{except statement}
\index{statement!except}
\index{exception}

Whenever a runtime error occurs, it creates an
{\bf exception}.  Usually, the program stops and Python
prints 
an error message.

For example, dividing by zero creates an exception:

\beforeverb
\begin{verbatim}
>>> print 55/0
ZeroDivisionError: integer division or modulo
\end{verbatim}
\afterverb
%
So does accessing a nonexistent list item:

\beforeverb
\begin{verbatim}
>>> a = []
>>> print a[5]
IndexError: list index out of range
\end{verbatim}
\afterverb
%
Or accessing a key that isn't in the dictionary:

\beforeverb
\begin{verbatim}
>>> b = {}
>>> print b['what']
KeyError: what
\end{verbatim}
\afterverb
%
Or trying to open a nonexistent file:

\beforeverb
\begin{verbatim}
>>> f = open("Idontexist", "r")
IOError: [Errno 2] No such file or directory: 'Idontexist'
\end{verbatim}
\afterverb
%
In each case, the error
message has two parts: the type of error before
the colon, and specifics about the error after the colon.
Normally Python also prints a traceback of where the program
was, but we have omitted that from the examples.

\index{traceback}

Sometimes we want to execute an operation that could cause
an exception, but we don't want the program to stop.  We can
{\bf handle} the exception using the {\tt try} and
{\tt except} statements.

For example, we might prompt the user for the name of a file
and then try to open it.  If the file doesn't exist, we don't
want the program to crash; we want to handle the exception:

\beforeverb
\begin{verbatim}
filename = raw_input('Enter a file name: ')
try:
  f = open (filename, "r")
except IOError:
  print 'There is no file named', filename
\end{verbatim}
\afterverb
%
The {\tt try} statement executes the statements in the first block.
If no exceptions occur, it ignores the {\tt except} statement.  If an
exception of type {\tt IOError} occurs, it executes the statements in
the {\tt except} branch and then continues.

We can encapsulate this capability in a function: {\tt exists} takes a
filename and returns true if the file exists, false if it doesn't:

\beforeverb
\begin{verbatim}
def exists(filename):
  try:
    f = open(filename)
    f.close()
    return True
  except IOError:
    return False
\end{verbatim}
\afterverb
%
You can use multiple {\tt except} blocks to handle different kinds of
exceptions.  The {\em Python Reference Manual} has the details.

If your program detects an error condition, you can make it
{\bf raise} an exception.  Here is an example that gets input
from the user and checks for the value 17.  
Assuming that 17 is not valid input for some reason, we raise an
exception.

\beforeverb
\begin{verbatim}
def inputNumber () :
  x = input ('Pick a number: ')
  if x == 17 :
    raise ValueError, '17 is a bad number'
  return x
\end{verbatim}
\afterverb
%
The {\tt raise} statement takes two arguments: the exception type and
specific information about the error.  {\tt ValueError} is one of the
exception types Python provides for a variety of occasions.  Other
examples include {\tt TypeError}, {\tt KeyError}, and my favorite,
{\tt NotImplementedError}.

If the function that called {\tt inputNumber} handles the error,
then the program can continue; otherwise, Python prints the
error message and exits:

\beforeverb
\begin{verbatim}
>>> inputNumber ()
Pick a number: 17
ValueError: 17 is a bad number
\end{verbatim}
\afterverb
%
The error message includes the exception type and the
additional information you provided.

\begin{quote}
{\em As an exercise, write a function that uses {\tt inputNumber}
to input a number from the keyboard and that handles the
{\tt ValueError} exception.}
\end{quote}


\section{Glossary}

\index{file}
\index{text file}
\index{break statement}
\index{statement!break}
\index{continue statement}
\index{statement!continue}
\index{format operator}
\index{format string}
\index{operator!format}
\index{directory}
\index{pickle}
\index{try}
\index{raise exception}
\index{raise exception}
\index{handle exception}
\index{except statement}
\index{exception}

\begin{description}

\item[file:] A named entity, usually stored on a hard drive, floppy disk,
or CD-ROM, that contains a stream of characters.

\item[directory:] A named collection of files, also called a folder.

\item[path:] A sequence of directory names that specifies the
exact location of a file.

\item[text file:] A file that contains printable characters organized
into lines separated by newline characters.

\item[break statement:] A statement that causes the flow of execution
to exit a loop.

\item[continue statement:] A statement that causes the current iteration
of a loop to end.  The flow of execution goes to the top of the loop,
evaluates the condition, and proceeds accordingly.

\item[format operator:] The {\tt \%} operator takes a format
string and a tuple of expressions and yields a string that includes
the expressions, formatted according to the format string.

\item[format string:] A string that contains printable characters
and format sequences that indicate how to format values.

\item[format sequence:] A sequence of characters beginning with
{\tt \%} that indicates how to format a value.

\item[pickle:] To write a data value in a file along with its
type information so that it can be reconstituted later.

\item[exception:] An error that occurs at runtime.

\item[handle:] To prevent an exception from terminating
a program using the {\tt try}
and {\tt except} statements.

\item[raise:] To signal an exception using the {\tt raise}
statement.


\end{description}


\clearemptydoublepage
% LaTeX source for textbook ``How to think like a computer scientist''
% Copyright (c)  2001  Allen B. Downey, Jeffrey Elkner, and Chris Meyers.

% Permission is granted to copy, distribute and/or modify this
% document under the terms of the GNU Free Documentation License,
% Version 1.1  or any later version published by the Free Software
% Foundation; with the Invariant Sections being "Contributor List",
% with no Front-Cover Texts, and with no Back-Cover Texts. A copy of
% the license is included in the section entitled "GNU Free
% Documentation License".

% This distribution includes a file named fdl.tex that contains the text
% of the GNU Free Documentation License.  If it is missing, you can obtain
% it from www.gnu.org or by writing to the Free Software Foundation,
% Inc., 59 Temple Place - Suite 330, Boston, MA 02111-1307, USA.

\chapter{Classes and objects}
\index{class}
\index{object}


\section{User-defined compound types}
\label{point}
\index{compound data type}
\index{data type!compound}
\index{user-defined data type}
\index{data type!user-defined}
\index{constructor}

Having used some of Python's built-in types, we are ready to create a
user-defined type: the {\tt Point}.

Consider the concept of a mathematical point.  In two dimensions, a
point is two numbers (coordinates) that are treated collectively as a
single object.  In mathematical notation, points are often written in
parentheses with a comma separating the coordinates. For example,
$(0, 0)$ represents the origin, and $(x, y)$ represents the
point $x$ units to the right and $y$ units up from the origin.

A natural way to represent a point in Python is with two
floating-point values.  The question, then, is how to group these two
values into a compound object.  The quick and dirty solution
is to use a list or tuple, and for some applications that might
be the best choice.

\index{floating-point}

An alternative is to define a new user-defined compound type, also
called a {\bf class}.  This approach involves a bit more effort, but
it has advantages that will be apparent soon.

A class definition looks like this:

\beforeverb
\begin{verbatim}
class Point:
  pass
\end{verbatim}
\afterverb
%
Class definitions can appear anywhere in a program, but they are
usually near the beginning (after the {\tt import} statements).  The
syntax rules for a class definition are the same as for other compound
statements (see Section~\ref{conditional execution}).

This definition creates a new class called {\tt Point}.  The
{\bf pass} statement has no effect; it is only
necessary because a compound statement must have something in its
body.

By creating the {\tt Point} class, we created a new type, also
called {\tt Point}.  The members of this type are called
{\bf instances} of the type or {\bf objects}.  Creating a new instance
is called {\bf instantiation}.  To instantiate a {\tt Point} object,
we call a function named (you guessed it) {\tt Point}:

\index{instance!object}
\index{object instance}
\index{instantiation}

\beforeverb
\begin{verbatim}
blank = Point()
\end{verbatim}
\afterverb
%
The variable {\tt blank} is assigned a reference to a new
{\tt Point} object.  A function like {\tt Point} that creates
new objects is called a {\bf constructor}.


\section{Attributes}
\index{attribute}

We can add new data to an instance using dot notation:

\beforeverb
\begin{verbatim}
>>> blank.x = 3.0
>>> blank.y = 4.0
\end{verbatim}
\afterverb
%
This syntax is similar to the syntax for selecting a variable from a
module, such as {\tt math.pi} or {\tt string.uppercase}.  In this case,
though, we are selecting a data item from an instance.  These
named items are called {\bf attributes}.

The following state diagram shows the result of these assignments:

\beforefig
\centerline{\psfig{figure=illustrations/point.eps}}
\afterfig

The variable {\tt blank} refers to a Point object, which
contains two attributes.  Each attribute refers to a
floating-point number.

We can read the value of an attribute using the same syntax:

\beforeverb
\begin{verbatim}
>>> print blank.y
4.0
>>> x = blank.x
>>> print x
3.0
\end{verbatim}
\afterverb
%
The expression {\tt blank.x} means, ``Go to the object {\tt blank}
refers to and get the value of {\tt x}.'' In this case, we assign that
value to a variable named {\tt x}.  There is no conflict between
the variable {\tt x} and the attribute {\tt x}.  The
purpose of dot notation is to identify which variable you are
referring to unambiguously.

You can use dot notation as part of any expression, so the following
statements are legal:

\beforeverb
\begin{verbatim}
print '(' + str(blank.x) + ', ' + str(blank.y) + ')'
distanceSquared = blank.x * blank.x + blank.y * blank.y
\end{verbatim}
\afterverb
%
The first line outputs {\tt (3.0, 4.0)}; the second line calculates
the value 25.0.

You might be tempted to print the value of {\tt blank} itself:

\beforeverb
\begin{verbatim}
>>> print blank
<__main__.Point instance at 80f8e70>
\end{verbatim}
\afterverb
%
The result indicates that {\tt blank} is an instance of the {\tt
Point} class and it was defined in {\tt \_\_main\_\_}.  {\tt 80f8e70}
is the unique identifier for this object, written in hexadecimal (base
16).  This is probably not the most informative way to display a {\tt
Point} object.  You will see how to change it shortly.

\index{printing!object}

\begin{quote}
{\em As an exercise, create and print a {\tt Point} object, and then
use {\tt id} to print the object's unique identifier.
Translate the hexadecimal form into decimal and confirm that they
match.}
\end{quote}


\section{Instances as arguments}
\index{instance}
\index{parameter}

You can pass an instance as an argument in the usual way.
For example:

\beforeverb
\begin{verbatim}
def printPoint(p):
  print '(' + str(p.x) + ', ' + str(p.y) + ')'
\end{verbatim}
\afterverb
%
{\tt printPoint} takes a point as an argument and displays it in
the standard format.  If you call {\tt printPoint(blank)}, the
output is {\tt (3.0, 4.0)}.

\begin{quote}
{\em As an exercise, rewrite the {\tt distance} function from
Section~\ref{program development} so that it takes two {\tt Point}s as
arguments instead of four numbers.}
\end{quote}


\section{Sameness}
\index{sameness}

The meaning of the word ``same'' seems perfectly clear until you give
it some thought, and then you realize there is more to it than you
expected.

\index{ambiguity}
\index{natural language}
\index{language!}

For example, if you say, ``Chris and I have the same car,'' you mean
that his car and yours are the same make and model, but that they are
two different cars.  If you say, ``Chris and I have the same mother,''
you mean that his mother and yours are the same person.\footnote{Not all
languages have the same problem.  For example, German has different
words for different kinds of sameness.  ``Same car'' in this context
would be ``gleiche Auto,'' and ``same mother'' would be ``selbe
Mutter.''}  So the idea of ``sameness'' is different depending on the
context.

When you talk about objects, there is a similar ambiguity.  For
example, if two {\tt Point}s are the same, does that mean they
contain the same data (coordinates) or that they are actually
the same object?

To find out if two references refer to the same object, use
the {\tt is} operator.  For example:

\beforeverb
\begin{verbatim}
>>> p1 = Point()
>>> p1.x = 3
>>> p1.y = 4
>>> p2 = Point()
>>> p2.x = 3
>>> p2.y = 4
>>> p1 is p2
False
\end{verbatim}
\afterverb
%
Even though {\tt p1} and {\tt p2} contain the same coordinates,
they are not the same object.  If we assign {\tt p1} to
{\tt p2}, then the two variables are aliases of the same
object:

\beforeverb
\begin{verbatim}
>>> p2 = p1
>>> p1 is p2
True
\end{verbatim}
\afterverb
%
This type of equality is called {\bf shallow equality} because
it compares only the references, not the contents of the objects.

\index{equality}
\index{identity}
\index{shallow equality}
\index{deep equality}

To compare the contents of the objects---{\bf deep equality}---we
can write a function called {\tt samePoint}:

\beforeverb
\begin{verbatim}
def samePoint(p1, p2) :
  return (p1.x == p2.x) and (p1.y == p2.y)
\end{verbatim}
\afterverb
%
Now if we create two different objects that contain the same
data, we can use {\tt samePoint} to find out if they represent the
same point.

\adjustpage{-2}
\pagebreak

\beforeverb
\begin{verbatim}
>>> p1 = Point()
>>> p1.x = 3
>>> p1.y = 4
>>> p2 = Point()
>>> p2.x = 3
>>> p2.y = 4
>>> samePoint(p1, p2)
True
\end{verbatim}
\afterverb
%
Of course, if the two variables refer to the same object,
they have both shallow and deep equality.


\section{Rectangles}
\index{rectangle}

Let's say that we want a class to represent a rectangle.  The question
is, what information do we have to provide in order to specify a
rectangle? To keep things simple, assume that the rectangle is
oriented either vertically or horizontally, never at an angle.

There are a few possibilities: we could specify the center of the
rectangle (two coordinates) and its size (width and height); or we
could specify one of the corners and the size; or we could specify two
opposing corners.  A conventional choice is to specify the upper-left
corner of the rectangle and the size.

Again, we'll define a new class:

\beforeverb
\begin{verbatim}
class Rectangle:
  pass
\end{verbatim}
\afterverb
%
And instantiate it:

\beforeverb
\begin{verbatim}
box = Rectangle()
box.width = 100.0
box.height = 200.0
\end{verbatim}
\afterverb
%
This code creates a new {\tt Rectangle} object with two floating-point
attributes.  To specify the upper-left corner, we can embed an
object within an object!

\beforeverb
\begin{verbatim}
box.corner = Point()
box.corner.x = 0.0
box.corner.y = 0.0
\end{verbatim}
\afterverb
%
The dot operator composes.  The expression {\tt box.corner.x} means,
``Go to the object {\tt box} refers to and select the attribute named
{\tt corner}; then go to that object and select the attribute named
{\tt x}.''

The figure shows the state of this object:

\beforefig
\centerline{\psfig{figure=illustrations/rectangle.eps}}
\afterfig


\section{Instances as return values}
\index{instance}
\index{return value}

Functions can return instances.  For example, {\tt findCenter}
takes a {\tt Rectangle} as an argument and returns a {\tt Point}
that contains the coordinates of the center of the {\tt Rectangle}:

\beforeverb
\begin{verbatim}
def findCenter(box):
  p = Point()
  p.x = box.corner.x + box.width/2.0
  p.y = box.corner.y - box.height/2.0
  return p
\end{verbatim}
\afterverb
%
To call this function, pass {\tt box} as an argument and assign
the result to a variable:

\beforeverb
\begin{verbatim}
>>> center = findCenter(box)
>>> printPoint(center)
(50.0, -100.0)
\end{verbatim}
\afterverb
%

\section{Objects are mutable}
\index{object!mutable}
\index{mutable!object}

We can change the state of an object by making an assignment
to one of its attributes.  For example, to change the size
of a rectangle without changing its position, we could
modify the values of {\tt width} and {\tt height}:

\beforeverb
\begin{verbatim}
box.width = box.width + 50
box.height = box.height + 100
\end{verbatim}
\afterverb
%
We could encapsulate this code in a method and
generalize it to grow the rectangle by any amount:

\index{encapsulation}
\index{generalization}

\beforeverb
\begin{verbatim}
def growRect(box, dwidth, dheight) :
  box.width = box.width + dwidth
  box.height = box.height + dheight
\end{verbatim}
\afterverb
%
The variables {\tt dwidth} and {\tt dheight} indicate how much the
rectangle should grow in each direction.  Invoking this method has the
effect of modifying the {\tt Rectangle} that is passed as an argument.

For example, we could create a new {\tt Rectangle} named {\tt bob}
and pass it to {\tt growRect}:

\beforeverb
\begin{verbatim}
>>> bob = Rectangle()
>>> bob.width = 100.0
>>> bob.height = 200.0
>>> bob.corner = Point()
>>> bob.corner.x = 0.0
>>> bob.corner.y = 0.0
>>> growRect(bob, 50, 100)
\end{verbatim}
\afterverb
%
While {\tt growRect} is running, the parameter {\tt box} is an
alias for {\tt bob}.  Any changes made to {\tt box} also
affect {\tt bob}.

\begin{quote}
{\em As an exercise, write a function named {\tt moveRect} that takes
a {\tt Rectangle} and two parameters named {\tt dx} and {\tt dy}.  It
should change the location of the rectangle by adding {\tt dx}
to the {\tt x} coordinate of {\tt corner} and adding {\tt dy}
to the {\tt y} coordinate of {\tt corner}.}
\end{quote}


\section{Copying}
\label{embedded}
\index{aliasing}
\index{copying}
\index{copy module}
\index{module!copy}

Aliasing can make a program difficult to read because changes
made in one place might have unexpected effects in another place.
It is hard to keep track of all the variables that might refer
to a given object.

Copying an object is often an alternative to aliasing.
The {\tt copy} module contains a function called {\tt copy} that
can duplicate any object:

\beforeverb
\begin{verbatim}
>>> import copy
>>> p1 = Point()
>>> p1.x = 3
>>> p1.y = 4
>>> p2 = copy.copy(p1)
>>> p1 == p2
False
>>> samePoint(p1, p2)
True
\end{verbatim}
\afterverb
%
Once we import the {\tt copy} module, we can use the {\tt copy} method
to make a new {\tt Point}.  {\tt p1} and {\tt p2} are not the
same point, but they contain the same data.

To copy a simple object like a {\tt Point}, which doesn't
contain any embedded objects, {\tt copy} is sufficient.  This is called 
{\bf shallow copying}.

For something like a {\tt Rectangle}, which contains a reference
to a {\tt Point}, {\tt copy} doesn't do quite the right thing.  It
copies the reference to the {\tt Point} object, so both the old
{\tt Rectangle} and the new one refer to a single {\tt Point}.

\index{embedded reference}
\index{reference!embedded}

If we create a box, {\tt b1}, in the usual way and then make
a copy, {\tt b2}, using {\tt copy}, the resulting
state diagram looks like this:

\vspace{0.1in}
\beforefig
\centerline{\psfig{figure=illustrations/rectangle2.eps}}
\afterfig
\vspace{0.1in}

This is almost certainly not what we want.  In this case, invoking
{\tt growRect} on one of the {\tt Rectangles} would not affect
the other, but invoking {\tt moveRect} on either would affect both!
This behavior is confusing and error-prone.

Fortunately, the {\tt copy} module contains a method named {\tt
deepcopy} that copies not only the object but also any embedded
objects.  You will not be surprised to learn that this operation is
called a {\bf deep copy}.

\beforeverb
\begin{verbatim}
>>> b2 = copy.deepcopy(b1)
\end{verbatim}
\afterverb
%
Now {\tt b1} and {\tt b2} are completely separate objects.

We can use {\tt deepcopy} to rewrite {\tt growRect} so that
instead of modifying an existing {\tt Rectangle}, it creates a new
{\tt Rectangle} that has the same location as the old one but new
dimensions:

\beforeverb
\begin{verbatim}
def growRect(box, dwidth, dheight) :
  import copy
  newBox = copy.deepcopy(box)
  newBox.width = newBox.width + dwidth
  newBox.height = newBox.height + dheight
  return newBox
\end{verbatim}
\afterverb
%

\begin{quote}
{\em An an exercise, rewrite {\tt moveRect} so that it creates and
returns a new {\tt Rectangle} instead of modifying the old one.}
\end{quote}


\adjustpage{-2}
\pagebreak

\section{Glossary}

\begin{description}

\item[class:] A user-defined compound type.
A class can also be thought of
as a template for the objects that are instances of it.

\item[instantiate:] To create an instance of a class.

\item[instance:] An object that belongs to a class.

\item[object:] A compound data type that is often used to
model a thing or concept in the real world.

\item[constructor:] A method used to create new objects.

\item[attribute:] One of the named data items that makes up
an instance.

\item[shallow equality:]  Equality of references, or two
references that point to the same object.

\item[deep equality:]  Equality of values, or two references
that point to objects that have the same value.

\item[shallow copy:] To copy the contents of an object, including
any references to embedded objects; implemented by the {\tt copy}
function in the {\tt copy} module.

\item[deep copy:] To copy the contents of an object as well as any
embedded objects, and any objects embedded in them, and so on;
implemented by the {\tt deepcopy} function in the {\tt copy} module.

\index{class}
\index{instantiate}
\index{instance}
\index{object}
\index{constructor}
\index{attribute}
\index{shallow equality}
\index{deep equality}
\index{shallow copy}
\index{deep copy}

\end{description}

\clearemptydoublepage
% LaTeX source for textbook ``How to think like a computer scientist''
% Copyright (c)  2001  Allen B. Downey, Jeffrey Elkner, and Chris Meyers.

% Permission is granted to copy, distribute and/or modify this
% document under the terms of the GNU Free Documentation License,
% Version 1.1  or any later version published by the Free Software
% Foundation; with the Invariant Sections being "Contributor List",
% with no Front-Cover Texts, and with no Back-Cover Texts. A copy of
% the license is included in the section entitled "GNU Free
% Documentation License".

% This distribution includes a file named fdl.tex that contains the text
% of the GNU Free Documentation License.  If it is missing, you can obtain
% it from www.gnu.org or by writing to the Free Software Foundation,
% Inc., 59 Temple Place - Suite 330, Boston, MA 02111-1307, USA.

\chapter{Classes and functions}
\label{time}
\index{function}
\index{method}


\section{Time}

As another example of a user-defined type, we'll define a class called
{\tt Time} that records the time of day.  The class definition looks
like this:

\beforeverb
\begin{verbatim}
class Time:
  pass
\end{verbatim}
\afterverb
%
We can create a new {\tt Time} object and assign
attributes for hours, minutes, and seconds:

\beforeverb
\begin{verbatim}
time = Time()
time.hours = 11
time.minutes = 59
time.seconds = 30
\end{verbatim}
\afterverb
%
The state diagram for the {\tt Time} object looks like this:

\beforefig
\centerline{\psfig{figure=illustrations/time.eps}}
\afterfig

\begin{quote}
{\em As an exercise, write a function {\tt printTime} that takes a 
{\tt Time} object
as an argument and prints it in the form {\tt hours:minutes:seconds}.}
\end{quote}

\begin{quote}
{\em As a second exercise, write a boolean function {\tt after} that
takes two {\tt Time} objects, {\tt t1} and {\tt t2}, as arguments, and
returns {\tt True} if {\tt t1} follows {\tt t2} chronologically and
{\tt False} otherwise.}
\end{quote}


\section{Pure functions}
\index{pure function}
\index{function type!pure}

In the next few sections, we'll write two versions of a function
called {\tt addTime}, which calculates the sum of two {\tt Time}s.
They will demonstrate two kinds of functions: pure functions and
modifiers.

The following is a rough version of {\tt addTime}:

\beforeverb
\begin{verbatim}
def addTime(t1, t2):
  sum = Time()
  sum.hours = t1.hours + t2.hours
  sum.minutes = t1.minutes + t2.minutes
  sum.seconds = t1.seconds + t2.seconds
  return sum
\end{verbatim}
\afterverb
%
The function creates a new {\tt Time} object, initializes its
attributes, and returns a reference to the new object.  This is called
a {\bf pure function} because it does not modify any of the objects
passed to it as arguments and it has no side effects, such as
displaying a value or getting user input.

Here is an example of how to use this function.  We'll create two {\tt
Time} objects: {\tt currentTime}, which contains the current time; and
{\tt breadTime}, which contains the amount of time it takes for a
breadmaker to make bread.  Then we'll use {\tt addTime} to figure out
when the bread will be done.  If you haven't finished writing {\tt
printTime} yet, take a look ahead to Section~\ref{printTime} before
you try this:

\beforeverb
\begin{verbatim}
>>> currentTime = Time()
>>> currentTime.hours = 9
>>> currentTime.minutes = 14
>>> currentTime.seconds =  30

>>> breadTime = Time()
>>> breadTime.hours =  3
>>> breadTime.minutes =  35
>>> breadTime.seconds =  0

>>> doneTime = addTime(currentTime, breadTime)
>>> printTime(doneTime)
\end{verbatim}
\afterverb
%
The output of this program is {\tt 12:49:30}, which is correct. On the
other hand, there are cases where the result is not correct.  Can you
think of one?

The problem is that this function does not deal with cases where the
number of seconds or minutes adds up to more than sixty.  When that
happens, we have to ``carry'' the extra seconds into the minutes column
or the extra minutes into the hours column.

Here's a second corrected version of the function:

\beforeverb
\begin{verbatim}
def addTime(t1, t2):
  sum = Time()
  sum.hours = t1.hours + t2.hours
  sum.minutes = t1.minutes + t2.minutes
  sum.seconds = t1.seconds + t2.seconds

  if sum.seconds >= 60:
    sum.seconds = sum.seconds - 60
    sum.minutes = sum.minutes + 1

  if sum.minutes >= 60:
    sum.minutes = sum.minutes - 60
    sum.hours = sum.hours + 1

  return sum
\end{verbatim}
\afterverb
%
Although this function is correct, it is starting to get big.  Later
we will suggest an alternative approach that yields shorter code.


\section{Modifiers}
\label{increment}
\index{modifier}
\index{function type!modifier}

There are times when it is useful for a function to modify one or more
of the objects it gets as arguments.  Usually, the caller keeps a
reference to the objects it passes, so any changes the function makes
are visible to the caller.  Functions that work this way are called
{\bf modifiers}.

{\tt increment}, which adds a given number of seconds to a {\tt Time}
object, would be written most naturally as a
modifier.  A rough draft of the function
looks like this:

\adjustpage{-2}
\pagebreak

\beforeverb
\begin{verbatim}
def increment(time, seconds):
  time.seconds = time.seconds + seconds

  if time.seconds >= 60:
    time.seconds = time.seconds - 60
    time.minutes = time.minutes + 1

  if time.minutes >= 60:
    time.minutes = time.minutes - 60
    time.hours = time.hours + 1
\end{verbatim}
\afterverb
%
The first line performs the basic operation; the remainder deals
with the special cases we saw before.

Is this function correct?  What happens if the parameter {\tt seconds} is
much greater than sixty?  In that case, it is not enough to carry
once; we have to keep doing it until {\tt seconds} is less than sixty.
One solution is to
replace the {\tt if} statements with {\tt while}
statements:

\beforeverb
\begin{verbatim}
def increment(time, seconds):
  time.seconds = time.seconds + seconds

  while time.seconds >= 60:
    time.seconds = time.seconds - 60
    time.minutes = time.minutes + 1

  while time.minutes >= 60:
    time.minutes = time.minutes - 60
    time.hours = time.hours + 1
\end{verbatim}
\afterverb
%
This function is now correct, but it is not the most efficient
solution.

\begin{quote}
{\em As an exercise, rewrite this function so that it doesn't contain 
any loops.}
\end{quote}

\begin{quote}
{\em As a second exercise, rewrite {\tt increment} as a pure function, and
write function calls to both versions.}
\end{quote}


\section{Which is better?}
\index{functional programming style}

\adjustpage{1}

Anything that can be done with modifiers can also be done with pure
functions.  In fact, some programming languages only allow pure
functions.  There is some evidence that programs that use pure
functions are faster to develop and less error-prone than programs
that use modifiers.  Nevertheless, modifiers are convenient at times,
and in some cases, functional programs are less efficient.

In general, we recommend that you write pure functions whenever it is
reasonable to do so and resort to modifiers only if there is a
compelling advantage.  This approach might be called a {\bf functional
programming style}.


\section{Prototype development versus planning}
\label{convert}
\index{prototype development}

In this chapter, we demonstrated an approach to program
development that we call {\bf prototype development}. In each
case, we wrote a rough draft (or prototype) that performed the basic
calculation and then tested it on a few cases, correcting flaws as we
found them.

Although this approach can be effective, it can lead to code that is
unnecessarily complicated---since it deals with many special
cases---and unreliable---since it is hard to know if you have found
all the errors.

An alternative is {\bf planned development}, in which high-level insight
into the problem can make the programming much easier.  In this case,
the insight is that a {\tt Time} object is really a three-digit number
in base 60!  The {\tt second} component is the ``ones column,'' the
{\tt minute} component is the ``sixties column,'' and the {\tt hour}
component is the ``thirty-six hundreds column.''

When we wrote {\tt addTime} and {\tt increment}, we were effectively
doing addition in base 60, which is why we had to carry from one
column to the next.

This observation suggests another approach to the whole problem---we
can convert a {\tt Time} object into a single number and take
advantage of the fact that the computer knows how to do arithmetic
with numbers.  The following function converts a {\tt Time}
object into an integer:

\beforeverb
\begin{verbatim}
def convertToSeconds(t):
  minutes = t.hours * 60 + t.minutes
  seconds = minutes * 60 + t.seconds
  return seconds
\end{verbatim}
\afterverb
%
Now, all we need is a way to convert from an integer to a {\tt Time}
object:

\adjustpage{1}

\beforeverb
\begin{verbatim}
def makeTime(seconds):
  time = Time()
  time.hours = seconds // 3600
  time.minutes = (seconds%3600) // 60
  time.seconds = seconds%60
  return time
\end{verbatim}
\afterverb
%
You might have to think a bit to convince yourself that this function
is correct.  Assuming you are convinced, you can use it and
{\tt convertToSeconds}
to rewrite {\tt addTime}:

\beforeverb
\begin{verbatim}
def addTime(t1, t2):
  seconds = convertToSeconds(t1) + convertToSeconds(t2)
  return makeTime(seconds)
\end{verbatim}
\afterverb
%
This version is much shorter than the original, and it is much easier to
demonstrate that it is correct.

\begin{quote}
{\em As an exercise, rewrite {\tt increment} the same way.}
\end{quote}


\section{Generalization}
\index{generalization}

In some ways, converting from base 60 to base 10 and back is harder
than just dealing with times.  Base conversion is more abstract; our
intuition for dealing with times is better.

But if we have the insight to treat times as base 60 numbers and make
the investment of writing the conversion functions ({\tt
convertToSeconds} and {\tt makeTime}), we get a program that is
shorter, easier to read and debug, and more reliable.

It is also easier to add features later.  For example, imagine
subtracting two {\tt Time}s to find the duration between them.  The
na\"{\i}ve approach would be to implement subtraction with borrowing.
Using the conversion functions would be easier and more likely to be
correct.

Ironically, sometimes making a problem harder (or more general) makes it
easier (because there are fewer special cases and fewer opportunities
for error).


\section{Algorithms}
\index{algorithm}

When you write a general solution for a class of problems, as opposed
to a specific solution to a single problem, you have written an {\bf
algorithm}.  We mentioned this word before but did not define it
carefully.  It is not easy to define, so we will try a couple of
approaches.

First, consider something that is not an algorithm.  When you learned
to multiply single-digit numbers, you probably memorized the
multiplication table.  In effect, you memorized 100 specific solutions.
That kind of knowledge is not algorithmic.

But if you were ``lazy,'' you probably cheated by learning a few
tricks.  For example, to find the product of $n$ and 9, you can
write $n-1$ as the first digit and $10-n$ as the second
digit.  This trick is a general solution for multiplying any
single-digit number by 9.  That's an algorithm!

Similarly, the techniques you learned for addition with carrying,
subtraction with borrowing, and long division are all algorithms.  One
of the characteristics of algorithms is that they do not require any
intelligence to carry out.  They are mechanical processes in which
each step follows from the last according to a simple set of rules.

In our opinion, it is embarrassing that humans spend so much time in
school learning to execute algorithms that, quite literally, require
no intelligence.

On the other hand, the process of designing algorithms is interesting,
intellectually challenging, and a central part of what we call
programming.

Some of the things that people do naturally, without difficulty or
conscious thought, are the hardest to express algorithmically.
Understanding natural language is a good example.  We all do it, but
so far no one has been able to explain {\em how} we do it, at least
not in the form of an algorithm.


\section{Glossary}

\begin{description}

\item[pure function:] A function that does not modify any of the objects it
receives as arguments.  Most pure functions are fruitful.

\item[modifier:] A function that changes one or more of the objects it
receives as arguments.  Most modifiers are fruitless.

\item[functional programming style:] A style of program design in which the
majority of functions are pure.

\item[prototype development:] A way of developing programs starting with a
prototype and gradually testing and improving it.

\item[planned development:] A way of developing programs that involves
high-level insight into the problem and more planning than incremental
development or prototype development.

\item[algorithm:] A set of instructions for solving a class of problems by a
mechanical, unintelligent process.

\index{pure function}
\index{modifier}
\index{functional programming style}
\index{incremental development}
\index{development!incremental}
\index{planned development}
\index{development!planned}
\index{algorithm}

\end{description}

\clearemptydoublepage
% LaTeX source for textbook ``How to think like a computer scientist''
% Copyright (c)  2001  Allen B. Downey, Jeffrey Elkner, and Chris Meyers.

% Permission is granted to copy, distribute and/or modify this
% document under the terms of the GNU Free Documentation License,
% Version 1.1  or any later version published by the Free Software
% Foundation; with the Invariant Sections being "Contributor List",
% with no Front-Cover Texts, and with no Back-Cover Texts. A copy of
% the license is included in the section entitled "GNU Free
% Documentation License".

% This distribution includes a file named fdl.tex that contains the text
% of the GNU Free Documentation License.  If it is missing, you can obtain
% it from www.gnu.org or by writing to the Free Software Foundation,
% Inc., 59 Temple Place - Suite 330, Boston, MA 02111-1307, USA.

\chapter{Classes and methods}


\section{Object-oriented features}
\index{object-oriented programming language}
\index{object-oriented programming}

Python is an {\bf object-oriented programming language}, which means
that it provides features that support {\bf object-oriented
programming}.

It is not easy to define object-oriented programming, but we have
already seen some of its characteristics:

\begin{itemize}

\item Programs are made up of object definitions and function
definitions, and most of the computation is expressed in terms
of operations on objects.

\item Each object definition corresponds to some object or concept
in the real world, and the functions that operate on that object
correspond to the ways real-world objects interact.

\end{itemize}

For example, the {\tt Time} class defined in Chapter~\ref{time}
corresponds to the way people record the time of day, and the
functions we defined correspond to the kinds of things people do with
times.  Similarly, the {\tt Point} and {\tt Rectangle} classes
correspond to the mathematical concepts of a point and a rectangle.

So far, we have not taken advantage of the features Python provides to
support object-oriented programming.  Strictly speaking, these features are
not necessary.  For the most part, they provide an alternative syntax
for things we have already done, but in many cases, the
alternative is more concise and more accurately conveys the structure of
the program.

For example, in the {\tt Time} program, there is no obvious
connection between the class definition and the function definitions
that follow.  With some examination, it is apparent that every function
takes at least one {\tt Time} object as an argument.

This observation is the motivation for {\bf methods}.  We have already
seen some methods, such as {\tt keys} and {\tt values}, which were
invoked on dictionaries.  Each method is associated with a class and is
intended to be invoked on instances of that class.

\index{method}
\index{function}
\index{instance!object}
\index{object instance}

Methods are just like functions, with
two differences:

\begin{itemize}

\item Methods are defined inside a class definition in order
to make the relationship between the class and the method explicit.

\item The syntax for invoking a method is different from the
syntax for calling a function.

\end{itemize}

In the next few sections, we will take the functions from the previous
two chapters and transform them into methods.  This transformation is
purely mechanical; you can do it simply by following a sequence of
steps.  If you are comfortable converting from one form to another,
you will be able to choose the best form for whatever you are doing.


\section{{\tt printTime}}
\label{printTime}
\index{printing!object}

In Chapter~\ref{time}, we defined a class named
{\tt Time} and you wrote a function named {\tt printTime}, which
should have looked something like this:

\beforeverb
\begin{verbatim}
class Time:
  pass

def printTime(time):
  print str(time.hours) + ":" + \
        str(time.minutes) + ":" + \
        str(time.seconds)
\end{verbatim}
\afterverb
%
To call this function, we passed a {\tt Time} object as an argument:

\beforeverb
\begin{verbatim}
>>> currentTime = Time()
>>> currentTime.hours = 9
>>> currentTime.minutes = 14
>>> currentTime.seconds = 30
>>> printTime(currentTime)
\end{verbatim}
\afterverb
%
To make {\tt printTime} a method, all we have to do is
move the function definition inside the class definition.  Notice
the change in indentation.

\beforeverb
\begin{verbatim}
class Time:
  def printTime(time):
    print str(time.hours) + ":" +  \
          str(time.minutes) + ":" +  \
          str(time.seconds)
\end{verbatim}
\afterverb
%
Now we can invoke {\tt printTime} using dot notation.

\index{dot notation}

\beforeverb
\begin{verbatim}
>>> currentTime.printTime()
\end{verbatim}
\afterverb
%
As usual, the object on which the method is invoked appears
before the dot and the
name of the method appears after the dot.

The object on which the method is invoked is assigned to the
first parameter, so in this case {\tt currentTime} is assigned
to the parameter {\tt time}.

By convention, the first parameter of a method is
called {\tt self}.  The reason for this is a little
convoluted, but it is based on a useful metaphor.

The syntax for a function call, {\tt printTime(currentTime)},
suggests that the function is the active agent.  It says
something like, ``Hey {\tt printTime}!  Here's an object for
you to print.''

In object-oriented programming, the objects are the active
agents.  An invocation like {\tt currentTime.printTime()}
says ``Hey {\tt currentTime}!  Please print yourself!''

This change in perspective might be more polite, but
it is not obvious that it is useful.  In the examples we
have seen so far, it may not be.  But sometimes shifting
responsibility from the functions onto the objects
makes it possible to write more versatile functions,
and makes it easier to maintain and reuse code.


\section{Another example}

Let's convert {\tt increment} (from Section~\ref{increment}) to a
method.  To save space, we will leave out previously defined methods,
but you should keep them in your version:

\adjustpage{-1}

\beforeverb
\begin{verbatim}
class Time:
  #previous method definitions here...

  def increment(self, seconds):
    self.seconds = seconds + self.seconds

    while self.seconds >= 60:
      self.seconds = self.seconds - 60
      self.minutes = self.minutes + 1

    while self.minutes >= 60:
      self.minutes = self.minutes - 60
      self.hours = self.hours + 1
\end{verbatim}
\afterverb
%
The transformation is purely mechanical---we move the method
definition into the class definition and change the name of the first
parameter.

Now we can invoke {\tt increment} as a method.

\beforeverb
\begin{verbatim}
currentTime.increment(500)
\end{verbatim}
\afterverb
%
Again, the object on which the method is invoked gets assigned
to the first parameter, {\tt self}.  The second parameter,
{\tt seconds} gets the value {\tt 500}.

\begin{quote}
{\em As an exercise, convert {\tt convertToSeconds} 
(from Section~\ref{convert}) to a method in the
{\tt Time} class.}
\end{quote}


\section{A more complicated example}

The {\tt after} function is slightly more complicated because it
operates on two {\tt Time} objects, not just one.  We can only convert
one of the parameters to {\tt self}; the other stays the same:

\beforeverb
\begin{verbatim}
class Time:
  #previous method definitions here...

  def after(self, time2):
    if self.hour > time2.hour:
      return 1
    if self.hour < time2.hour:
      return 0

    if self.minute > time2.minute:
      return 1
    if self.minute < time2.minute:
      return 0

    if self.second > time2.second:
      return 1
    return 0
\end{verbatim}
\afterverb
%
We invoke this method on one object and pass the other as an argument:

\beforeverb
\begin{verbatim}
if doneTime.after(currentTime):
  print "The bread is not done yet."
\end{verbatim}
\afterverb
%
You can almost read the invocation like English: ``If the done-time is
after the current-time, then...''


\section{Optional arguments}

We have seen built-in functions that take a variable number of
arguments.  For example, {\tt string.find} can take two, three, or
four arguments.

It is possible to write user-defined functions with optional argument
lists.  For example, we can upgrade our own version of {\tt find} to do
the same thing as {\tt string.find}.

This is the original version from Section~\ref{find}:

\beforeverb
\begin{verbatim}
def find(str, ch):
  index = 0
  while index < len(str):
    if str[index] == ch:
      return index
    index = index + 1
  return -1
\end{verbatim}
\afterverb
%
This is the new and improved version:

\beforeverb
\begin{verbatim}
def find(str, ch, start=0):
  index = start
  while index < len(str):
    if str[index] == ch:
      return index
    index = index + 1
  return -1
\end{verbatim}
\afterverb
%
The third parameter, {\tt start}, is optional because a default value,
{\tt 0}, is provided.  If we invoke {\tt find} with only two
arguments, it uses the default value and starts from the beginning of
the string:

\beforeverb
\begin{verbatim}
>>> find("apple", "p")
1
\end{verbatim}
\afterverb
%
If we provide a third argument, it {\bf overrides} the default:

\beforeverb
\begin{verbatim}
>>> find("apple", "p", 2)
2
>>> find("apple", "p", 3)
-1
\end{verbatim}
\afterverb
%
\begin{quote}
{\em As an exercise, add a fourth parameter, {\tt end}, that specifies where
to stop looking.

Warning: This exercise is a bit tricky.  The default value of
{\tt end} should be {\tt len(str)}, but that doesn't work.  The
default values are evaluated when the function is defined, not when it
is called.  When {\tt find} is defined, {\tt str} doesn't exist yet,
so you can't find its length.}
\end{quote}


\section{The initialization method}
\index{initialization method}
\index{method!initialization}

The {\bf initialization method} is
a special method that is invoked when an object is created.  The name
of this method is {\tt \_\_init\_\_} (two underscore characters,
followed by {\tt init}, and then two more underscores).  An
initialization method for the {\tt Time} class looks like this:

\beforeverb
\begin{verbatim}
class Time:
  def __init__(self, hours=0, minutes=0, seconds=0):
    self.hours = hours
    self.minutes = minutes
    self.seconds = seconds
\end{verbatim}
\afterverb
%
There is no conflict between the attribute {\tt self.hours}
and the parameter {\tt hours}.  Dot notation specifies which
variable we are referring to.

\index{dot notation}

When we invoke the {\tt Time} constructor, the arguments we provide
are passed along to {\tt init}:

\beforeverb
\begin{verbatim}
>>> currentTime = Time(9, 14, 30)
>>> currentTime.printTime()
9:14:30
\end{verbatim}
\afterverb
%
Because the arguments are optional, we can omit them:

\beforeverb
\begin{verbatim}
>>> currentTime = Time()
>>> currentTime.printTime()
0:0:0
\end{verbatim}
\afterverb
%
Or provide only the first:

\beforeverb
\begin{verbatim}
>>> currentTime = Time (9)
>>> currentTime.printTime()
9:0:0
\end{verbatim}
\afterverb
%
Or the first two:

\beforeverb
\begin{verbatim}
>>> currentTime = Time (9, 14)
>>> currentTime.printTime()
9:14:0
\end{verbatim}
\afterverb
%
Finally, we can make assignments to a subset of the
parameters by naming them explicitly:

\beforeverb
\begin{verbatim}
>>> currentTime = Time(seconds = 30, hours = 9)
>>> currentTime.printTime()
9:0:30
\end{verbatim}
\afterverb
%

\section{Points revisited}
\index{Point class}
\index{class!Point}

Let's rewrite the {\tt Point} class from
Section~\ref{point} in a more object-oriented style:

\beforeverb
\begin{verbatim}
class Point:
  def __init__(self, x=0, y=0):
    self.x = x
    self.y = y

  def __str__(self):
    return '(' + str(self.x) + ', ' + str(self.y) + ')'
\end{verbatim}
\afterverb
%
The initialization method
takes $x$ and $y$ values as optional parameters;
the default for either parameter is 0.

The next method, {\tt \_\_str\_\_}, returns a string representation
of a {\tt Point} object.
If a class provides a method named {\tt \_\_str\_\_}, it
overrides the default behavior of the Python built-in {\tt str} function.

\beforeverb
\begin{verbatim}
>>> p = Point(3, 4)
>>> str(p)
'(3, 4)'
\end{verbatim}
\afterverb
%
Printing a {\tt Point} object implicitly invokes {\tt \_\_str\_\_} on
the object, so defining {\tt \_\_str\_\_} also changes the behavior of
{\tt print}:

\beforeverb
\begin{verbatim}
>>> p = Point(3, 4)
>>> print p
(3, 4)
\end{verbatim}
\afterverb
%
When we write a new class, we almost always start by writing {\tt
\_\_init\_\_}, which makes it easier to instantiate objects, and {\tt
\_\_str\_\_}, which is almost always useful for debugging.


\section{Operator overloading}
\label{operator overloading}
\index{operator overloading}
\index{operator!overloading}
\index{dot product}
\index{scalar multiplication}

Some languages make it possible to change the definition of the
built-in operators when they are applied to user-defined types.  This
feature is called {\bf operator overloading}.  It is especially useful when
defining new mathematical types.

For example, to override the addition operator {\tt +}, we
provide a method named {\tt \_\_add\_\_}:

\beforeverb
\begin{verbatim}
class Point:
  # previously defined methods here...

  def __add__(self, other):
    return Point(self.x + other.x, self.y + other.y)
\end{verbatim}
\afterverb
%
As usual, the first parameter is the object on which the method is
invoked.  The second parameter is conveniently named {\tt other}
to distinguish it from {\tt self}.  To add two {\tt Point}s, we create
and return a new {\tt Point} that contains  the sum of the
$x$ coordinates and the sum of the $y$ coordinates.

Now, when we apply the {\tt +} operator to {\tt Point} objects, Python
invokes {\tt \_\_add\_\_}:

\beforeverb
\begin{verbatim}
>>>   p1 = Point(3, 4)
>>>   p2 = Point(5, 7)
>>>   p3 = p1 + p2
>>>   print p3
(8, 11)
\end{verbatim}
\afterverb
%
The expression {\tt p1 + p2} is equivalent to
{\tt p1.\_\_add\_\_(p2)}, but obviously more elegant.

\begin{quote}
{\em As an exercise, add a method {\tt \_\_sub\_\_(self, other)} that
overloads the subtraction operator, and try it out.}
\end{quote}

There are several ways to override the behavior of the
multiplication operator: by defining a method named
{\tt \_\_mul\_\_}, or {\tt \_\_rmul\_\_}, or both.

If the left operand of {\tt *} is a {\tt Point}, Python invokes
{\tt \_\_mul\_\_}, which assumes that the other operand is also
a {\tt Point}.  It computes the {\bf dot product} of the two
points, defined according to the rules of linear algebra:

\beforeverb
\begin{verbatim}
def __mul__(self, other):
  return self.x * other.x + self.y * other.y
\end{verbatim}
\afterverb
%
If the left operand of {\tt *} is a primitive type and the right
operand is a {\tt Point}, Python invokes {\tt \_\_rmul\_\_}, which
performs {\bf scalar multiplication}:

\beforeverb
\begin{verbatim}
def __rmul__(self, other):
  return Point(other * self.x,  other * self.y)
\end{verbatim}
\afterverb
%
The result is a new {\tt Point} whose coordinates are a multiple
of the original coordinates.  If {\tt other} is a type that cannot
be multiplied by a floating-point number, then
{\tt \_\_rmul\_\_} will yield an error.

This example demonstrates both kinds of multiplication:

\beforeverb
\begin{verbatim}
>>> p1 = Point(3, 4)
>>> p2 = Point(5, 7)
>>> print p1 * p2
43
>>> print 2 * p2
(10, 14)
\end{verbatim}
\afterverb
%
What happens if we try to evaluate {\tt p2 * 2}?  Since
the first operand is a {\tt Point}, Python invokes
{\tt \_\_mul\_\_} with {\tt 2} as the second argument.
Inside {\tt \_\_mul\_\_}, the program tries to access the {\tt x}
coordinate of {\tt other}, which fails because
an integer has no attributes:

\beforeverb
\begin{verbatim}
>>> print p2 * 2
AttributeError: 'int' object has no attribute 'x'
\end{verbatim}
\afterverb
%
Unfortunately, the error message is a bit opaque.  This example
demonstrates some of the difficulties of object-oriented programming.
Sometimes it is hard enough just to figure out what code is running.

For a more complete example of operator overloading, see
Appendix~\ref{overloading}.


\section{Polymorphism}
\index{polymorphism}

Most of the methods we have written only work for a specific
type.  When you create a new object, you write methods that operate
on that type.

But there are certain operations that you will want to apply to many
types, such as the arithmetic operations in the previous sections.
If many types support the same set of operations, you
can write functions that work on any of those types.

For example, the {\tt multadd} operation (which is common in
linear algebra) takes three arguments; it multiplies the first
two and then adds the third.  We can write it in Python like
this:

\beforeverb
\begin{verbatim}
def multadd (x, y, z):
  return x * y + z
\end{verbatim}
\afterverb
%
This method will work for any values of {\tt x} and {\tt y}
that can be multiplied and for any value of {\tt z} that can be
added to the product.

We can invoke it with numeric values:

\beforeverb
\begin{verbatim}
>>> multadd (3, 2, 1)
7
\end{verbatim}
\afterverb
%
Or with {\tt Point}s:

\beforeverb
\begin{verbatim}
>>> p1 = Point(3, 4)
>>> p2 = Point(5, 7)
>>> print multadd (2, p1, p2)
(11, 15)
>>> print multadd (p1, p2, 1)
44
\end{verbatim}
\afterverb
%
In the first case, the {\tt Point} is multiplied by a scalar
and then added to another {\tt Point}.
In the second case, the dot product yields a numeric
value, so the third argument also has to be a numeric value.

A function like this that can take arguments with different
types is called {\bf polymorphic}.

As another example, consider the method {\tt frontAndBack},
which prints a list twice, forward and backward:

\beforeverb
\begin{verbatim}
def frontAndBack(front):
  import copy
  back = copy.copy(front)
  back.reverse()
  print str(front) + str(back)
\end{verbatim}
\afterverb
%
Because the {\tt reverse} method is a modifier, we make a copy
of the list before reversing it.  That way, this method doesn't
modify the list it gets as an argument.

Here's an example that applies {\tt frontAndBack} to a list:

\beforeverb
\begin{verbatim}
>>>   myList = [1, 2, 3, 4]
>>>   frontAndBack(myList)
[1, 2, 3, 4][4, 3, 2, 1]
\end{verbatim}
\afterverb
%
Of course, we intended to apply this function to lists, so
it is not surprising that it works.
What would be surprising is if we could apply it to a {\tt Point}.

To determine whether a function can be applied to a new type,
we apply the fundamental rule of polymorphism:

\begin{quote}
{\bf If all of the operations inside the function can be applied
to the type, the function can be applied to the type.}
\end{quote}

The operations in the method include {\tt copy}, {\tt reverse}, and
{\tt print}.

{\tt copy} works on any object, and we have already written
a {\tt \_\_str\_\_} method for {\tt Point}s, so all we need
is a {\tt reverse} method in the {\tt Point} class:

\beforeverb
\begin{verbatim}
def reverse(self):
  self.x , self.y = self.y, self.x
\end{verbatim}
\afterverb
%
Then we can pass {\tt Point}s to {\tt frontAndBack}:

\beforeverb
\begin{verbatim}
>>>   p = Point(3, 4)
>>>   frontAndBack(p)
(3, 4)(4, 3)
\end{verbatim}
\afterverb
%
The best kind of polymorphism is the unintentional kind, where
you discover that a function you have already written can be
applied to a type for which you never planned.



\section{Glossary}

\begin{description}

\item[object-oriented language:] A language that provides
features, such as user-defined classes and inheritance, that facilitate
object-oriented programming.

\item[object-oriented programming:] A style of programming in which
data and the operations that manipulate it are organized into classes
and methods.

\item[method:] A function that is defined inside a class definition and
is invoked on instances of that class.

\item[override:] To replace a default.  Examples include replacing a default
value with a particular argument and replacing a default method
by providing a new method with the same name.

\item[initialization method:] A special method that is invoked automatically
when a new object is created and that initializes the object's attributes.

\item[operator overloading:] Extending built-in operators
({\tt +}, {\tt -}, {\tt *}, {\tt >}, {\tt <}, etc.) so that they work
with user-defined types.

\item[dot product:] An operation defined in linear algebra that
multiplies two {\tt Point}s and yields a numeric value.

\item[scalar multiplication:] An operation defined in linear algebra that
multiplies each of the coordinates of a {\tt Point} by a numeric
value.

\item[polymorphic:] A function that can operate on more than one
type.  If all the operations in a function can be
applied to a type, then the function can be applied to a type.


\index{object-oriented programming language}
\index{method}
\index{initialization method}
\index{override}
\index{overloading}
\index{operator overloading}
\index{dot product}
\index{scalar multiplication}
\index{polymorphic}

\end{description}

\clearemptydoublepage
% LaTeX source for textbook ``How to think like a computer scientist''
% Copyright (c)  2001  Allen B. Downey, Jeffrey Elkner, and Chris Meyers.

% Permission is granted to copy, distribute and/or modify this
% document under the terms of the GNU Free Documentation License,
% Version 1.1  or any later version published by the Free Software
% Foundation; with the Invariant Sections being "Contributor List",
% with no Front-Cover Texts, and with no Back-Cover Texts. A copy of
% the license is included in the section entitled "GNU Free
% Documentation License".

% This distribution includes a file named fdl.tex that contains the text
% of the GNU Free Documentation License.  If it is missing, you can obtain
% it from www.gnu.org or by writing to the Free Software Foundation,
% Inc., 59 Temple Place - Suite 330, Boston, MA 02111-1307, USA.

\chapter{Sets of objects}

\section{Composition}
\index{composition}
\index{nested structure}

By now, you have seen several examples of composition.
One of the
first examples was using a method invocation as part of an
expression.  Another example is the nested structure of statements;
you can put an {\tt if} statement within a {\tt while} loop, within
another {\tt if} statement, and so on.

Having seen this pattern, and having learned about lists and objects,
you should not be surprised to learn that you can create lists of
objects.  You can also create objects that contain lists (as
attributes); you can create lists that contain lists; you can
create objects that contain objects; and so on.

In this chapter and the next, we will look at some examples of these
combinations, using {\tt Card} objects as an example.


\section{{\tt Card} objects}
\index{Card}
\index{class!Card}

If you are not familiar with common playing cards, now would be a good
time to get a deck, or else this chapter might not make much sense.
There are fifty-two cards in a deck, each of which belongs to one of four
suits and one of thirteen ranks.  The suits are Spades, Hearts, Diamonds, and
Clubs (in descending order in bridge).  The ranks are Ace, 2, 3, 4, 5,
6, 7, 8, 9, 10, Jack, Queen, and King.  Depending on the game that you are
playing, the rank of Ace may be higher than King or lower than 2.

\index{rank}
\index{suit}

If we want to define a new object to represent a playing card, it is
obvious what the attributes should be: {\tt rank} and
{\tt suit}.  It is not as obvious what type the attributes
should be.  One possibility is to use strings containing words like
{\tt "Spade"} for suits and {\tt "Queen"} for ranks.  One problem with
this implementation is that it would not be easy to compare cards to
see which had a higher rank or suit.

\index{encode}
\index{encrypt}
\index{map to}

An alternative is to use integers to {\bf encode} the ranks and suits.
By ``encode,'' we do not mean what some people think, which is to
encrypt or translate into a secret code.  What a computer scientist
means by ``encode'' is ``to define a mapping between a
sequence of numbers and the items I want to represent.'' For example:

\beforefig
\begin{tabular}{l c l}
Spades & $\mapsto$ & 3 \\
Hearts & $\mapsto$ & 2 \\
Diamonds & $\mapsto$ & 1 \\
Clubs & $\mapsto$ & 0
\end{tabular}
\afterfig

An obvious feature of this mapping is that the suits map to integers in
order, so we can compare suits by comparing integers.  The mapping for
ranks is fairly obvious; each of the numerical ranks maps to the
corresponding integer, and for face cards:

\beforefig
\begin{tabular}{l c l}
Jack & $\mapsto$ & 11 \\
Queen & $\mapsto$ & 12 \\
King & $\mapsto$ & 13 \\
\end{tabular}
\afterfig

The reason we are using mathematical notation for these mappings is
that they are not part of the Python program.  They are part of the
program design, but they never appear explicitly in the code.  The
class definition for the {\tt Card} type looks like this:

\beforeverb
\begin{verbatim}
class Card:
  def __init__(self, suit=0, rank=2):
    self.suit = suit
    self.rank = rank
\end{verbatim}
\afterverb
%
As usual, we provide an initialization method that takes an optional
parameter for each attribute.  The default value of {\tt suit} is
0, which represents Clubs.

\index{constructor}

To create a Card, we invoke the Card constructor with the
suit and rank of the card we want.

\beforeverb
\begin{verbatim}
threeOfClubs = Card(3, 1)
\end{verbatim}
\afterverb
%
In the next section we'll figure out which card we just made.


\section{Class attributes and the {\tt \_\_str\_\_} method}
\index{class attribute}
\index{attribute!class}

In order to print {\tt Card} objects in a way that people can easily
read, we want to map the integer codes onto words.  A natural way to
do that is with lists of strings.  We assign these lists to {\bf class
attributes} at the top of the class definition:

\beforeverb
\begin{verbatim}
class Card:
  suitList = ["Clubs", "Diamonds", "Hearts", "Spades"]
  rankList = ["narf", "Ace", "2", "3", "4", "5", "6", "7", 
              "8", "9", "10", "Jack", "Queen", "King"]

  #init method omitted

  def __str__(self):
    return (self.rankList[self.rank] + " of " + 
            self.suitList[self.suit])
\end{verbatim}
\afterverb
%
A class attribute is defined outside of any method, and it can be
accessed from any of the methods in the class.

Inside {\tt \_\_str\_\_}, we can use {\tt suitList} and {\tt rankList}
to map the numerical values of {\tt suit} and {\tt rank} to strings.
For example, the expression \verb+self.suitList[self.suit]+ means
``use the attribute {\tt suit} from the object {\tt self} as an index
into the class attribute named {\tt suitList}, and select the
appropriate string.''

The reason for the {\tt "narf"} in the first element in {\tt
rankList} is to act as a place keeper for the zero-eth element of the
list, which should never be used.  The only valid ranks are 1 to 13.  This
wasted item is not entirely necessary.  We could have started at 0,
as usual, but it is less confusing to encode 2 as 2, 3 as 3, and so on.

With the methods we have so far, we can create and print cards:

\beforeverb
\begin{verbatim}
>>> card1 = Card(1, 11)
>>> print card1
Jack of Diamonds
\end{verbatim}
\afterverb
%
Class attributes like {\tt suitList} are shared by all {\tt Card}
objects.  The advantage of this is that we can use any {\tt Card}
object to access the class attributes:

\beforeverb
\begin{verbatim}
>>> card2 = Card(1, 3)
>>> print card2
3 of Diamonds
>>> print card2.suitList[1]
Diamonds
\end{verbatim}
\afterverb
%
The disadvantage is that if we modify a class attribute, it
affects every instance of the class.  For example, if we decide
that ``Jack of Diamonds'' should really be called
``Jack of Swirly Whales,'' we could do this:

\index{instance!object}
\index{object instance}

\beforeverb
\begin{verbatim}
>>> card1.suitList[1] = "Swirly Whales"
>>> print card1
Jack of Swirly Whales
\end{verbatim}
\afterverb
%
The problem is that {\em all} of the Diamonds just became
Swirly Whales:

\beforeverb
\begin{verbatim}
>>> print card2
3 of Swirly Whales
\end{verbatim}
\afterverb
%
It is usually not a good idea to modify class attributes.



\section{Comparing cards}
\label{comparecard}
\index{operator!conditional}
\index{conditional operator}

For primitive types, there are conditional operators
({\tt <}, {\tt >}, {\tt ==}, etc.)
that compare
values and determine when one is greater than, less than, or equal to
another.  For user-defined types, we can override the behavior of
the built-in operators by providing a method named
{\tt \_\_cmp\_\_}.  By convention, {\tt \_\_cmp\_\_}
has two parameters, {\tt self} and {\tt other}, and returns
1 if the first object is greater, -1 if the
second object is greater, and 0 if they are equal to each other.

\index{override}
\index{operator overloading}
\index{ordering}
\index{complete ordering}
\index{partial ordering}

Some types are completely ordered, which means that you can compare
any two elements and tell which is bigger.  For example, the integers
and the floating-point numbers are completely ordered.  Some sets are
unordered, which means that there is no meaningful way to say that one
element is bigger than another.  For example, the fruits are
unordered, which is why you cannot compare apples and oranges.

The set of playing cards is partially ordered, which means that
sometimes you can compare cards and sometimes not.  For example, you
know that the 3 of Clubs is higher than the 2 of Clubs, and the 3 of
Diamonds is higher than the 3 of Clubs.  But which is better, the 3 of
Clubs or the 2 of Diamonds?  One has a higher rank, but the other has
a higher suit.

\index{comparable}

In order to make cards comparable, you have to decide which is more
important, rank or suit.  To be honest, the choice is
arbitrary.  For the sake of choosing, we will say that suit is more
important, because a new deck of cards comes sorted
with all the Clubs together, followed by all the Diamonds, and so on.

With that decided, we can write {\tt \_\_cmp\_\_}:

\beforeverb
\begin{verbatim}
def __cmp__(self, other):
  # check the suits
  if self.suit > other.suit: return 1
  if self.suit < other.suit: return -1
  # suits are the same... check ranks
  if self.rank > other.rank: return 1
  if self.rank < other.rank: return -1
  # ranks are the same... it's a tie
  return 0
\end{verbatim}
\afterverb
%
In this ordering, Aces appear lower than Deuces (2s).

\begin{quote}
{\em As an exercise, modify {\tt \_\_cmp\_\_} so that Aces are
ranked higher than Kings.}
\end{quote}


\section{Decks}
\index{list!of objects}
\index{object!list of}
\index{deck}

Now that we have objects to represent {\tt Card}s, the next logical
step is to define a class to represent a {\tt Deck}.  Of course, a
deck is made up of cards, so each {\tt Deck} object will contain a
list of cards as an attribute.

\index{initialization method}
\index{method!initialization}

The following is a class definition for the {\tt Deck} class.  The
initialization method creates the attribute {\tt cards} and generates
the standard set of fifty-two cards:

\index{composition}
\index{loop!nested}

\beforeverb
\begin{verbatim}
class Deck:
  def __init__(self):
    self.cards = []
    for suit in range(4):
      for rank in range(1, 14):
        self.cards.append(Card(suit, rank))
\end{verbatim}
\afterverb
%
The easiest way to populate the deck is with a nested loop.  The outer
loop enumerates the suits from 0 to 3.  The inner loop enumerates the
ranks from 1 to 13.  Since the outer loop iterates four times, and the
inner loop iterates thirteen times, the total number of times the body
is executed is fifty-two (thirteen times four).  Each iteration
creates a new instance of {\tt Card} with the current suit and rank,
and appends that card to the {\tt cards} list.

The {\tt append} method works on lists but not, of course, tuples.

\index{append method}
\index{list method}
\index{method!list}

\adjustpage{1}

\section{Printing the deck}
\label{printdeck}
\index{printing!deck object}


As usual, when we define a new type of object we want a method
that prints the contents of an object.
To print a {\tt Deck}, we traverse the list and print each {\tt Card}:

\beforeverb
\begin{verbatim}
class Deck:
  ...
  def printDeck(self):
    for card in self.cards:
      print card
\end{verbatim}
\afterverb
%
Here, and from now on, the ellipsis ({\tt ...}) indicates that we have
omitted the other methods in the class.

As an alternative to {\tt printDeck}, we could
write a {\tt \_\_str\_\_} method for the {\tt Deck} class.  The
advantage of {\tt \_\_str\_\_} is that it is more flexible.  Rather
than just printing the contents of the object, it generates a string
representation that other parts of the program can manipulate
before printing, or store for later use.

Here is a version of {\tt \_\_str\_\_} that returns a string
representation of a {\tt Deck}.
To add a bit of pizzazz, it arranges the cards in a cascade
where each card is indented one space more than the previous card:

\beforeverb
\begin{verbatim}
class Deck:
  ...
  def __str__(self):
    s = ""
    for i in range(len(self.cards)):
      s = s + " "*i + str(self.cards[i]) + "\n"
    return s
\end{verbatim}
\afterverb
%
This example demonstrates several features.  First, instead of
traversing {\tt self.cards} and assigning each card to a variable,
we are using {\tt i} as a loop
variable and an index into the list of cards.

Second, we are using the string multiplication operator to indent
each card by one more space than the last.  The expression
{\tt " "*i} yields a number of spaces equal to the current value
of {\tt i}.

Third, instead of using the {\tt print} command to print the cards,
we use the {\tt str} function.  Passing an object as an argument to
{\tt str} is equivalent to invoking the {\tt \_\_str\_\_} method on
the object.

\index{accumulator}

Finally, we are using the variable {\tt s} as an {\bf accumulator}.
Initially, {\tt s} is the empty string.  Each time through the loop, a
new string is generated and concatenated with the old value of {\tt s}
to get the new value.  When the loop ends, {\tt s} contains the
complete string representation of the {\tt Deck}, which looks like
this:

\adjustpage{-2}
\pagebreak

\beforeverb
\begin{verbatim}
>>> deck = Deck()
>>> print deck
Ace of Clubs
 2 of Clubs
  3 of Clubs
   4 of Clubs
    5 of Clubs
     6 of Clubs
      7 of Clubs
       8 of Clubs
        9 of Clubs
         10 of Clubs
          Jack of Clubs
           Queen of Clubs
            King of Clubs
             Ace of Diamonds
\end{verbatim}
\afterverb
%
And so on.  Even though the result appears on 52 lines, it is
one long string that contains newlines.


\section{Shuffling the deck}
\index{shuffle}

If a deck is perfectly shuffled, then any card is equally likely
to appear anywhere in the deck, and any location in the deck is
equally likely to contain any card.

\index{random}
\index{randrange}

To shuffle the deck, we will use the {\tt randrange} function
from the {\tt random} module.  With two integer arguments,
{\tt a} and {\tt b}, {\tt randrange} chooses a random integer in
the range {\tt a <= x < b}.  Since the upper bound is strictly
less than {\tt b}, we can use the length of a list as the
second argument, and we are guaranteed to get a legal index.
For example, this expression chooses the index of a random card in a deck:

\beforeverb
\begin{verbatim}
random.randrange(0, len(self.cards))
\end{verbatim}
\afterverb
%
An easy way to shuffle the deck is by traversing the cards and
swapping each card with a randomly chosen one.  It is possible that
the card will be swapped with itself, but that is fine.  In fact, if
we precluded that possibility, the order of the cards would be less
than entirely random:

\adjustpage{-2}
\pagebreak

\beforeverb
\begin{verbatim}
class Deck:
  ...
  def shuffle(self):
    import random
    nCards = len(self.cards)
    for i in range(nCards):
      j = random.randrange(i, nCards)
      self.cards[i], self.cards[j] = self.cards[j], self.cards[i]
\end{verbatim}
\afterverb
%
Rather than assume that there are fifty-two cards in the deck, we get
the actual length of the list and store it in {\tt nCards}.

\index{swap}
\index{tuple assignment}
\index{assignment!tuple}

For each card in the deck, we choose a random card from among the
cards that haven't been shuffled yet.  Then we swap the current
card ({\tt i}) with the selected card ({\tt j}).  To swap the
cards we use a tuple assignment, as in Section~\ref{tuple assignment}:

\beforeverb
\begin{verbatim}
self.cards[i], self.cards[j] = self.cards[j], self.cards[i]
\end{verbatim}
\afterverb
%
\begin{quote}
{\em As an exercise, rewrite this line of code
without using a sequence assignment.}
\end{quote}


\section{Removing and dealing cards}
\index{removing cards}

Another method that would be useful for the {\tt Deck} class is {\tt
removeCard}, which takes a card as an argument, removes it, and
returns {\tt True} if the card was in the deck and {\tt False}
otherwise:

\beforeverb
\begin{verbatim}
class Deck:
  ...
  def removeCard(self, card):
    if card in self.cards:
      self.cards.remove(card)
      return True
    else: 
      return False
\end{verbatim}
\afterverb
%
The {\tt in} operator returns true if the first operand is in the
second, which must be a list or a tuple.  If the first operand is an
object, Python uses the object's {\tt \_\_cmp\_\_} method to determine
equality with items in the list.  Since the {\tt \_\_cmp\_\_} in the
{\tt Card} class checks for deep equality, the {\tt removeCard} method
checks for deep equality.

\index{in operator}
\index{operator!in}

To deal cards, we want to remove and return the top card.
The list method {\tt pop} provides a convenient way to do that:

\beforeverb
\begin{verbatim}
class Deck:
  ...
  def popCard(self):
    return self.cards.pop()
\end{verbatim}
\afterverb
%
Actually, {\tt pop} removes the {\em last} card in the list, so we are in
effect dealing from the bottom of the deck.

\index{boolean function}
\index{function!boolean}

One more operation that we are likely to want is the boolean function
{\tt isEmpty}, which returns true if the deck contains no cards:

\beforeverb
\begin{verbatim}
class Deck:
  ...
  def isEmpty(self):
    return (len(self.cards) == 0)
\end{verbatim}
\afterverb


\section{Glossary}

\begin{description}

\item[encode:]  To represent one set of values using another
set of values by constructing a mapping between them.

\item[class attribute:] A variable that is defined inside
a class definition but outside any method.  Class attributes
are accessible from any method in the class and are shared
by all instances of the class.

\item[accumulator:] A variable used in a loop to accumulate
a series of values, such as by concatenating them onto
a string or adding them to a running sum.

\index{encode}
\index{class attribute}
\index{attribute!class}
\index{accumulator}

\end{description}

\clearemptydoublepage
% LaTeX source for textbook ``How to think like a computer scientist''
% Copyright (c)  2001  Allen B. Downey, Jeffrey Elkner, and Chris Meyers.

% Permission is granted to copy, distribute and/or modify this
% document under the terms of the GNU Free Documentation License,
% Version 1.1  or any later version published by the Free Software
% Foundation; with the Invariant Sections being "Contributor List",
% with no Front-Cover Texts, and with no Back-Cover Texts. A copy of
% the license is included in the section entitled "GNU Free
% Documentation License".

% This distribution includes a file named fdl.tex that contains the text
% of the GNU Free Documentation License.  If it is missing, you can obtain
% it from www.gnu.org or by writing to the Free Software Foundation,
% Inc., 59 Temple Place - Suite 330, Boston, MA 02111-1307, USA.

\chapter{Inheritance}

\section{Inheritance}
\index{inheritance}
\index{object-oriented programming}
\index{parent class}
\index{child class}
\index{subclass}

The language feature most often associated with object-oriented
programming is {\bf inheritance}.  Inheritance is the ability to
define a new class that is a modified version of an existing
class.

The primary advantage of this feature is that you can add new methods
to a class without modifying the existing class.  It is
called ``inheritance'' because the new class inherits all of the
methods of the existing class.  Extending this metaphor, the existing
class is sometimes called the {\bf parent} class.  The new class may
be called the {\bf child} class or sometimes ``subclass.''

\index{object-oriented design}

Inheritance is a powerful feature.  Some programs that would be
complicated without inheritance can be written concisely and simply
with it.  Also, inheritance can facilitate code reuse, since you can
customize the behavior of parent classes without having to modify
them.  In some cases, the inheritance structure reflects the natural
structure of the problem, which makes the program easier to
understand.

On the other hand, inheritance can make programs difficult to read.
When a method is invoked, it is sometimes not clear where to find its
definition.  The relevant code may be scattered among several modules.
Also, many of the things that can be done using inheritance can be
done as elegantly (or more so) without it.  If the natural
structure of the problem does not lend itself to inheritance, this
style of programming can do more harm than good.

In this chapter we will demonstrate the use of inheritance as part of
a program that plays the card game Old Maid.  One of our goals is to
write code that could be reused to implement other card games.


\section{A hand of cards}

For almost any card game, we need to represent a hand of cards.
A hand is similar to a deck, of course.  Both are made up of
a set of cards, and both require operations like adding and
removing cards.  Also, we might like the ability to shuffle
both decks and hands.

A hand is also different from a deck.  Depending on the game being
played, we might want to perform some operations on hands that
don't make sense for a deck.  For example, in poker we might classify
a hand (straight, flush, etc.) or compare it with another hand.  In
bridge, we might want to compute a score for a hand in order to make
a bid.

This situation suggests the use of inheritance.  If {\tt Hand} is a
subclass of {\tt Deck}, it will have all the methods
of {\tt Deck}, and new methods can be added.

\index{parent class}
\index{class!parent}

In the class definition, the name of the parent class appears
in parentheses:

\beforeverb
\begin{verbatim}
class Hand(Deck):
  pass
\end{verbatim}
\afterverb
%
This statement indicates that the new {\tt Hand} class inherits from
the existing {\tt Deck} class.

The {\tt Hand} constructor initializes the attributes
for the hand, which are {\tt name} and {\tt cards}.  The string {\tt name}
identifies this hand, probably by the name of
the player that holds it.  The name is an optional parameter with
the empty string as a default value.
{\tt cards} is the list of cards in
the hand, initialized to the empty list:

\beforeverb
\begin{verbatim}
class Hand(Deck):
  def __init__(self, name=""):
    self.cards = []
    self.name = name
\end{verbatim}
\afterverb
%
For just about any card game, it is necessary to add and
remove cards from the deck.  Removing cards is already taken
care of, since {\tt Hand} inherits {\tt removeCard} from {\tt Deck}.
But we have to write {\tt addCard}:

\beforeverb
\begin{verbatim}
class Hand(Deck):
  ...
  def addCard(self,card) :
    self.cards.append(card)
\end{verbatim}
\afterverb
%
Again, the ellipsis indicates that we have omitted other methods.
The list {\tt append} method adds the new card to
the end of the list of cards.


\section{Dealing cards}
\index{dealing cards}

Now that we have a {\tt Hand} class, we want to deal cards from the
{\tt Deck} into hands.  It is not immediately obvious whether this
method should go in the {\tt Hand} class or in the {\tt Deck} class,
but since it operates on a single deck and (possibly) several hands,
it is more natural to put it in {\tt Deck}.

{\tt deal} should be fairly general,
since different games will have different requirements.  We may want
to deal out the entire deck at once or add one card to each hand.

{\tt deal} takes three parameters: the deck, a list (or tuple) of
hands, and the total number of cards to deal.  If there are not enough
cards in the deck, the method deals out all of the cards and stops:

\beforeverb
\begin{verbatim}
class Deck :
  ...
  def deal(self, hands, nCards=999):
    nHands = len(hands)
    for i in range(nCards):
      if self.isEmpty(): break    # break if out of cards
      card = self.popCard()       # take the top card
      hand = hands[i % nHands]    # whose turn is next?
      hand.addCard(card)          # add the card to the hand
\end{verbatim}
\afterverb
%
The last parameter, {\tt nCards}, is optional; the default is a large
number, which effectively means that all of the cards in the deck
will get dealt.

\index{loop variable}
\index{variable!loop}

The loop variable {\tt i} goes from 0 to {\tt nCards-1}.  Each
time through the loop, a card is removed from the deck using the
list method {\tt pop}, which removes and returns the last item
in the list.

\index{modulus operator}
\index{operator!modulus}

The modulus operator ({\tt \%}) allows us to deal cards in a
round robin (one card at a time to each hand).  When {\tt i} is
equal to the number of hands in the list, the expression
{\tt i \% nHands} wraps around to the beginning of the list
(index 0).



\section {Printing a Hand}
\index{printing!hand of cards}

To print the contents of a hand, we can take advantage of
the {\tt printDeck} and {\tt \_\_str\_\_} methods inherited
from {\tt Deck}.  For example:

\adjustpage{-2}
\pagebreak

\beforeverb
\begin{verbatim}
>>> deck = Deck()
>>> deck.shuffle()
>>> hand = Hand("frank")
>>> deck.deal([hand], 5)
>>> print hand
Hand frank contains
2 of Spades
 3 of Spades
  4 of Spades
   Ace of Hearts
    9 of Clubs
\end{verbatim}
\afterverb
%
It's not a great hand, but it has the makings
of a straight flush.

\index{straight flush}

Although it is convenient to inherit the existing methods,
there is additional information in a {\tt Hand}
object we might want to include when we print one.  To do that,
we can provide a {\tt \_\_str\_\_} method in the {\tt Hand} class
that overrides the one in the {\tt Deck} class:

\beforeverb
\begin{verbatim}
class Hand(Deck)
  ...
  def __str__(self):
    s = "Hand " + self.name
    if self.isEmpty():
      return s + " is empty\n"
    else:
      return s + " contains\n" + Deck.__str__(self)
\end{verbatim}
\afterverb
%
Initially, {\tt s} is a string that identifies the hand.  If the hand
is empty, the program appends the words {\tt is empty} and returns the
result.

Otherwise, the program appends the word {\tt contains} and the string
representation of the {\tt Deck}, computed by invoking the {\tt
\_\_str\_\_} method in the {\tt Deck} class on {\tt self}.

It may seem odd to send {\tt self}, which refers to the current {\tt
Hand}, to a {\tt Deck} method, until you remember that a {\tt Hand} is
a kind of {\tt Deck}.  {\tt Hand} objects can do everything {\tt Deck}
objects can, so it is legal to send a {\tt Hand} to a {\tt Deck}
method.

\index{subclass}
\index{parent class}
\index{class!parent}

In general, it is always legal to use an instance of a subclass
in place of an instance of a parent class.


\section {The {\tt CardGame} class}

The {\tt CardGame} class takes care
of some basic chores common to all games, such as creating the
deck and shuffling it:

\beforeverb
\begin{verbatim}
class CardGame:
  def __init__(self):
    self.deck = Deck()
    self.deck.shuffle()
\end{verbatim}
\afterverb
%
This is the first case we have seen where the initialization
method performs a significant computation, beyond initializing
attributes.

To implement specific games, we can inherit from {\tt CardGame}
and add features for the new game.
As an example, we'll write
a simulation of Old Maid.

The object of Old Maid is to get rid of cards in your hand.  You do
this by matching cards by rank and color.  For example, the 4 of Clubs
matches the 4 of Spades since both suits are black.  The Jack of Hearts
matches the Jack of Diamonds since both are red.

To begin the game, the Queen of Clubs is removed from the deck so that
the Queen of Spades has no match.  The fifty-one remaining cards are
dealt to the players in a round robin.  After the deal, all players
match and discard as many cards as possible.

When no more matches can be made, play begins.  In turn, each player
picks a card (without looking) from the closest neighbor to the left
who still has cards.  If the chosen card matches a card in the
player's hand, the pair is removed.  Otherwise, the card is added to
the player's hand.  Eventually all possible matches are made, leaving
only the Queen of Spades in the loser's hand.

In our computer simulation of the game, the computer plays
all hands.  Unfortunately, some nuances of the real game are lost.
In a real game, the player with the Old Maid
goes to some effort to get their neighbor to pick that card,
by displaying it a little more prominently, or perhaps failing
to display it more prominently, or even failing to fail to display
that card more prominently.  The computer simply picks a neighbor's
card at random.


\section {{\tt OldMaidHand} class}
\index{class!OldMaidHand}

A hand for playing Old Maid requires some abilities beyond the
general abilities of a {\tt Hand}.  We will define a new class, {\tt
OldMaidHand}, that inherits from {\tt Hand} and provides an additional
method called {\tt removeMatches}:


\beforeverb
\begin{verbatim}
class OldMaidHand(Hand):
  def removeMatches(self):
    count = 0
    originalCards = self.cards[:]
    for card in originalCards:
      match = Card(3 - card.suit, card.rank)
      if match in self.cards:
        self.cards.remove(card)
        self.cards.remove(match)
        print "Hand %s: %s matches %s" % (self.name,card,match)
        count = count + 1
    return count
\end{verbatim}
\afterverb
%
We start by making a copy of the list of cards, so that we can
traverse the copy while removing cards from the original.
Since {\tt self.cards} is modified in the
loop, we don't want to use it to control the traversal.  Python can get
quite confused if it is traversing a list that is changing!

\index{traversal}

\adjustpage{1}

For each card in the hand, we figure out what the matching card is and
go looking for it.  The match card has the same rank and the other
suit of the same color.  The expression {\tt 3 - card.suit} turns a
Club (suit 0) into a Spade (suit 3) and a Diamond (suit 1) into a
Heart (suit 2).  You should satisfy yourself that the opposite
operations also work.  If the match card is also in the hand, both
cards are removed.

The following example demonstrates how to use {\tt removeMatches}:

\beforeverb
\begin{verbatim}
>>> game = CardGame()
>>> hand = OldMaidHand("frank")
>>> game.deck.deal([hand], 13)
>>> print hand
Hand frank contains
Ace of Spades
 2 of Diamonds
  7 of Spades
   8 of Clubs
    6 of Hearts
     8 of Spades
      7 of Clubs
       Queen of Clubs
        7 of Diamonds
         5 of Clubs
          Jack of Diamonds
           10 of Diamonds
            10 of Hearts

>>> hand.removeMatches()
Hand frank: 7 of Spades matches 7 of Clubs
Hand frank: 8 of Spades matches 8 of Clubs
Hand frank: 10 of Diamonds matches 10 of Hearts
>>> print hand
Hand frank contains
Ace of Spades
 2 of Diamonds
  6 of Hearts
   Queen of Clubs
    7 of Diamonds
     5 of Clubs
      Jack of Diamonds
\end{verbatim}
\afterverb
%
Notice that there is no {\tt \_\_init\_\_} method for the
{\tt OldMaidHand} class.  We inherit it from {\tt Hand}.


\section {{\tt OldMaidGame} class}
\index{class!OldMaidGame}

Now we can turn our attention to the game itself.
{\tt OldMaidGame} is a subclass of {\tt CardGame} with a new
method called {\tt play} that takes a list of players as an argument.

Since {\tt \_\_init\_\_} is inherited from {\tt CardGame},
a new {\tt OldMaidGame} object contains a new shuffled deck:

\adjustpage{-1}

\beforeverb
\begin{verbatim}
class OldMaidGame(CardGame):
  def play(self, names):
    # remove Queen of Clubs
    self.deck.removeCard(Card(0,12))

    # make a hand for each player
    self.hands = []
    for name in names :
      self.hands.append(OldMaidHand(name))

    # deal the cards
    self.deck.deal(self.hands)
    print "---------- Cards have been dealt"
    self.printHands()

    # remove initial matches
    matches = self.removeAllMatches()
    print "---------- Matches discarded, play begins"
    self.printHands()

    # play until all 50 cards are matched
    turn = 0
    numHands = len(self.hands)
    while matches < 25:
      matches = matches + self.playOneTurn(turn)
      turn = (turn + 1) % numHands

    print "---------- Game is Over"
    self.printHands()
\end{verbatim}
\afterverb
%
Some of the steps of the game have been separated into methods.
{\tt removeAllMatches} traverses the list of hands and
invokes {\tt removeMatches} on each:

\beforeverb
\begin{verbatim}
class OldMaidGame(CardGame):
  ...
  def removeAllMatches(self):
    count = 0
    for hand in self.hands:
      count = count + hand.removeMatches()
    return count
\end{verbatim}
\afterverb
%
\begin{quote}
{\em As an exercise, write {\tt printHands} which traverses
{\tt self.hands} and prints each hand.}
\end{quote}

{\tt count} is
an accumulator that adds up the number of matches in each
hand and returns the total.

\index{accumulator}

When the total number of matches reaches twenty-five,
fifty cards have been removed from the hands, which means that
only one card is left and the game is over.

The variable {\tt turn} keeps track of which player's turn
it is.  It starts at 0 and increases by one each time;
when it reaches {\tt numHands}, the modulus operator
wraps it back around to 0.

The method {\tt playOneTurn} takes an argument that indicates
whose turn it is.  The return value is the number of matches
made during this turn:

\adjustpage{-2}
\pagebreak

\beforeverb
\begin{verbatim}
class OldMaidGame(CardGame):
  ...
  def playOneTurn(self, i):
    if self.hands[i].isEmpty():
      return 0
    neighbor = self.findNeighbor(i)
    pickedCard = self.hands[neighbor].popCard()
    self.hands[i].addCard(pickedCard)
    print "Hand", self.hands[i].name, "picked", pickedCard
    count = self.hands[i].removeMatches()
    self.hands[i].shuffle()
    return count
\end{verbatim}
\afterverb
%
If a player's hand is empty, that player is out of the game, so he or she
does nothing and returns 0.

Otherwise, a turn consists of finding the first player on the left
that has cards, taking one card from the neighbor, and checking
for matches.  Before returning, the cards in the hand are shuffled
so that the next player's choice is random.

The method {\tt findNeighbor} starts with the player to the
immediate left and continues around the circle until it finds
a player that still has cards:

\beforeverb
\begin{verbatim}
class OldMaidGame(CardGame):
  ...
  def findNeighbor(self, i):
    numHands = len(self.hands)
    for next in range(1,numHands):
      neighbor = (i + next) % numHands
      if not self.hands[neighbor].isEmpty():
        return neighbor
\end{verbatim}
\afterverb
%
If {\tt findNeighbor} ever went all the way around the circle without
finding cards, it would return {\tt None} and cause an error
elsewhere in the program.  Fortunately, we can prove that that will
never happen (as long as the end of the game is detected correctly).

We have omitted the {\tt printHands} method.  You
can write that one yourself.

The following output is from a truncated form of the game where only
the top fifteen cards (tens and higher) were dealt to three players.
With this small deck, play stops after seven matches instead of
twenty-five.

\beforeverb
\begin{verbatim}
>>> import cards
>>> game = cards.OldMaidGame()
>>> game.play(["Allen","Jeff","Chris"])
---------- Cards have been dealt
Hand Allen contains
King of Hearts
 Jack of Clubs
  Queen of Spades
   King of Spades
    10 of Diamonds

Hand Jeff contains
Queen of Hearts
 Jack of Spades
  Jack of Hearts
   King of Diamonds
    Queen of Diamonds

Hand Chris contains
Jack of Diamonds
 King of Clubs
  10 of Spades
   10 of Hearts
    10 of Clubs

Hand Jeff: Queen of Hearts matches Queen of Diamonds
Hand Chris: 10 of Spades matches 10 of Clubs
---------- Matches discarded, play begins
Hand Allen contains
King of Hearts
 Jack of Clubs
  Queen of Spades
   King of Spades
    10 of Diamonds

Hand Jeff contains
Jack of Spades
 Jack of Hearts
  King of Diamonds

Hand Chris contains
Jack of Diamonds
 King of Clubs
  10 of Hearts

Hand Allen picked King of Diamonds
Hand Allen: King of Hearts matches King of Diamonds
Hand Jeff picked 10 of Hearts
Hand Chris picked Jack of Clubs
Hand Allen picked Jack of Hearts
Hand Jeff picked Jack of Diamonds
Hand Chris picked Queen of Spades
Hand Allen picked Jack of Diamonds
Hand Allen: Jack of Hearts matches Jack of Diamonds
Hand Jeff picked King of Clubs
Hand Chris picked King of Spades
Hand Allen picked 10 of Hearts
Hand Allen: 10 of Diamonds matches 10 of Hearts
Hand Jeff picked Queen of Spades
Hand Chris picked Jack of Spades
Hand Chris: Jack of Clubs matches Jack of Spades
Hand Jeff picked King of Spades
Hand Jeff: King of Clubs matches King of Spades
---------- Game is Over
Hand Allen is empty

Hand Jeff contains
Queen of Spades

Hand Chris is empty

\end{verbatim}
\afterverb
%
So Jeff loses.



\section{Glossary}

\begin{description}

\item[inheritance:] The ability to define a new class that is a
modified version of a previously defined class.

\item[parent class:] The class from which a child class inherits.

\item[child class:] A new class created by inheriting from an
existing class; also called a ``subclass.''

\index{inheritance}
\index{parent class}
\index{child class}
\index{subclass}

\end{description}

\clearemptydoublepage
% LaTeX source for textbook ``How to think like a computer scientist''
% Copyright (C) 1999  Allen B. Downey

% This LaTeX source is free software; you can redistribute it and/or
% modify it under the terms of the GNU General Public License as
% published by the Free Software Foundation (version 2).

% This LaTeX source is distributed in the hope that it will be useful,
% but WITHOUT ANY WARRANTY; without even the implied warranty of
% MERCHANTABILITY or FITNESS FOR A PARTICULAR PURPOSE.  See the GNU
% General Public License for more details.

% Compiling this LaTeX source has the effect of generating
% a device-independent representation of a textbook, which
% can be converted to other formats and printed.  All intermediate
% representations (including DVI and Postscript), and all printed
% copies of the textbook are also covered by the GNU General
% Public License.

% This distribution includes a file named COPYING that contains the text
% of the GNU General Public License.  If it is missing, you can obtain
% it from www.gnu.org or by writing to the Free Software Foundation,
% Inc., 59 Temple Place - Suite 330, Boston, MA 02111-1307, USA.


\chapter{Linked lists}
\label{list}
\index{list}

\section{Embedded references}
\index{reference}
\index{embedded reference}
\index{reference!embedded}
\index{linked list}
\index{list!linked}
\index{node}
\index{cargo}

We have seen examples of attributes that refer to other objects, which
we called {\bf embedded references} (see Section~\ref{embedded}).  A
common data structure, the {\bf linked list}, takes advantage of this
feature.

Linked lists are made up of {\bf nodes}, where each node contains a
reference to the next node in the list.  In addition, each node
contains a unit of data called the {\bf cargo}.

A linked list is considered a {\bf recursive data
structure} because it has a recursive definition.

\begin{quote}
A linked list is either:
\begin{itemize}

\item the empty list, represented by {\tt None}, or

\item a node that contains a cargo object and a reference
to a linked list.

\end{itemize}

\end{quote}

\index{recursive data structure}
\index{data structure!recursive}

Recursive data structures lend themselves to
recursive methods.


\section{The {\tt Node} class}
\index{Node class}
\index{class!Node}

As usual when writing a new class, we'll start with the
initialization and {\tt \_\_str\_\_} methods so that we
can test the basic mechanism of creating and displaying the new
type:

\beforeverb
\begin{verbatim}
class Node:
  def __init__(self, cargo=None, next=None):
    self.cargo = cargo
    self.next  = next

  def __str__(self):
    return str(self.cargo)
\end{verbatim}
\afterverb
%
As usual, the parameters for the initialization method are optional. By
default, both the cargo and the link, {\tt next}, are set
to {\tt None}.

The string representation of a node is just the string representation
of the cargo.  Since any value can be passed to the {\tt str}
function, we can store any value in a list.

To test the implementation so far, we can create a {\tt Node}
and print it:

\beforeverb
\begin{verbatim}
>>> node = Node("test")
>>> print node
test
\end{verbatim}
\afterverb
%
To make it interesting, we need a list with more than
one node:

\beforeverb
\begin{verbatim}
>>> node1 = Node(1)
>>> node2 = Node(2)
>>> node3 = Node(3)
\end{verbatim}
\afterverb
%
This code creates three nodes, but we don't have a list yet
because the nodes are not {\bf linked}.  The state diagram
looks like this:

\beforefig
\centerline{\psfig{figure=illustrations/link1.eps}}
\afterfig

To link the nodes, we have to make the first node refer to the
second and the second node refer to the third:

\beforeverb
\begin{verbatim}
>>> node1.next = node2
>>> node2.next = node3
\end{verbatim}
\afterverb
%
The reference of the third node is {\tt None}, which indicates that
it is the end of the list.  Now the state diagram looks like this:

\beforefig
\centerline{\psfig{figure=illustrations/link2.eps}}
\afterfig

Now you know how to create nodes and link them into lists.  What
might be less clear at this point is why.


\section{Lists as collections}
\index{collection}

Lists are useful because they provide a way to assemble multiple
objects into a single entity, sometimes called a {\bf collection}.  In
the example, the first node of the list serves as a reference to the
entire list.

\index{list!printing}
\index{list!as argument}

To pass the list as an argument, we only have to pass a
reference to the first node.  For example, the function {\tt printList}
takes a single node as an argument.  Starting with the head of the
list, it prints each node until it gets to the end:

\beforeverb
\begin{verbatim}
def printList(node):
  while node:
    print node,
    node = node.next
  print
\end{verbatim}
\afterverb
%
To invoke this function, we pass a reference to the
first node:

\beforeverb
\begin{verbatim}
>>> printList(node1)
1 2 3
\end{verbatim}
\afterverb
%
Inside {\tt printList} we have a reference to the first node
of the list, but there is no variable that refers to the other
nodes.  We have to use the {\tt next} value from each node
to get to the next node.

To traverse a linked list, it is common to use a loop variable like
{\tt node} to refer to each of the nodes in succession.

\index{loop variable}
\index{list!traversal}
\index{traverse}

This diagram shows the nodes in the list and the values that
{\tt node} takes on:

\beforefig
\centerline{\psfig{figure=illustrations/link3.eps}}
\afterfig

\begin{quote}
{\em By convention, lists are often printed in brackets with commas
between the elements, as in {\tt [1, 2, 3]}.  As an exercise, modify
{\tt printList} so that it generates output in this format.}
\end{quote}


\section{Lists and recursion}
\label{listrecursion}
\index{list!traverse recursively}
\index{traverse}

It is natural to express many list operations using recursive methods.
For example, the following is a recursive algorithm for printing a list
backwards:

\begin{enumerate}

\item Separate the list into two pieces: the first node (called
the head); and the rest (called the tail).

\item Print the tail backward.

\item Print the head.

\end{enumerate}

Of course, Step 2, the recursive call, assumes that we have a way of
printing a list backward.  But if we assume that the recursive
call works---the leap of faith---then we can convince ourselves that
this algorithm works.

\index{leap of faith}
\index{list!printing backwards}

All we need are a base case and a way of proving that for
any list, we will eventually get to the base case.  Given the
recursive definition of a list, a natural base case is
the empty list, represented by {\tt None}:

\beforeverb
\begin{verbatim}
def printBackward(list):
  if list == None: return
  head = list
  tail = list.next
  printBackward(tail)
  print head,
\end{verbatim}
\afterverb
%
The first line handles the base case by doing nothing.  The
next two lines split the list into {\tt head} and {\tt tail}.
The last two lines print the list.  The comma at the end of the
last line keeps Python from printing a newline after each node.

We invoke this function as we invoked {\tt printList}:

\beforeverb
\begin{verbatim}
>>> printBackward(node1)
3 2 1
\end{verbatim}
\afterverb
%
The result is a backward list.

You might wonder why {\tt printList} and {\tt printBackward} are
functions and not methods in the {\tt Node} class.  The reason is that
we want to use {\tt None} to represent the empty list and it is not
legal to invoke a method on {\tt None}.  This limitation makes it
awkward to write list-manipulating code in a clean object-oriented
style.

Can we prove that {\tt printBackward} will always terminate?   In other
words, will it always reach the base case?  In fact, the answer
is no.  Some lists will make this function crash.


\section{Infinite lists}
\index{infinite list}
\index{list!infinite}
\index{loop!in list}
\index{list!loop}

There is nothing to prevent a node from referring back to
an earlier node in the list, including itself.  For example,
this figure shows a list with two nodes, one of which refers
to itself:

\beforefig
\centerline{\psfig{figure=illustrations/link4.eps}}
\afterfig

If we invoke {\tt printList} on this list, it will loop forever.
If we invoke {\tt printBackward}, it will recurse infinitely.
This sort of behavior makes infinite lists difficult to work
with.

Nevertheless, they are occasionally useful.  For example, we
might represent a number as a list of digits and use an infinite
list to represent a repeating fraction.

Regardless, it is problematic that we cannot prove that {\tt printList}
and {\tt printBackward} terminate.  The best we can do is the
hypothetical statement, ``If the list contains no loops, then these
functions will terminate.''  This sort of claim is called a {\bf
precondition}.  It imposes a constraint on one of the arguments and
describes the behavior of the function if the constraint is satisfied.
You will see more examples soon.

\index{precondition}


\section{The fundamental ambiguity theorem}
\index{ambiguity!fundamental theorem}
\index{theorem!fundamental ambiguity}

One part of {\tt printBackward} might have raised
an eyebrow:

\beforeverb
\begin{verbatim}
    head = list
    tail = list.next
\end{verbatim}
\afterverb
%
After the first assignment, {\tt head} and {\tt list} have the same
type and the same value.  So why did we create a new variable?

The reason is that the two variables play different roles.  We think
of {\tt head} as a reference to a single node, and we think of
{\tt list} as a reference to the first node of a list.  These
``roles'' are not part of the program; they are in the mind of the
programmer.

\index{variable!roles}
\index{role!variable}

In general we can't tell by looking at a program what role a
variable plays.
This ambiguity can be useful, but it can also make programs
difficult to read.  We often use variable names like {\tt node}
and {\tt list} to document how we intend to use a variable and
sometimes create additional variables to disambiguate.

We could have written {\tt printBackward} without {\tt head}
and {\tt tail}, which makes it more concise but possibly
less clear:

\beforeverb
\begin{verbatim}
def printBackward(list) :
  if list == None : return
  printBackward(list.next)
  print list,
\end{verbatim}
\afterverb
%
Looking at the two function calls, we have to remember that {\tt
printBackward} treats its argument as a collection and {\tt print}
treats its argument as a single object.

The {\bf fundamental ambiguity theorem} describes the ambiguity
that is inherent in a reference to a node:

\begin{quote}
{\bf A variable that refers to a node might treat the node as a single
object or as the first in a list of nodes.}
\end{quote}



\section{Modifying lists}
\index{list!modifying}
\index{modifying lists}

There are two ways to modify a linked list.  Obviously, we can change
the cargo of one of the nodes, but the more interesting operations are
the ones that add, remove, or reorder the nodes.

As an example, let's write a function that removes the second
node in the list and returns a reference to the removed node:

\beforeverb
\begin{verbatim}
def removeSecond(list):
  if list == None: return
  first = list
  second = list.next
  # make the first node refer to the third
  first.next = second.next
  # separate the second node from the rest of the list
  second.next = None
  return second
\end{verbatim}
\afterverb
%
Again, we are using temporary variables to make the code more
readable.  Here is how to use this function:

\beforeverb
\begin{verbatim}
>>> printList(node1)
1 2 3
>>> removed = removeSecond(node1)
>>> printList(removed)
2
>>> printList(node1)
1 3
\end{verbatim}
\afterverb
%
This state diagram shows the effect of the operation:

\beforefig
\centerline{\psfig{figure=illustrations/link5.eps}}
\afterfig

What happens if you invoke this function and pass a list with only one
element (a {\bf singleton})?  What happens if you pass the empty list
as an argument?  Is there a precondition for this function?  If so, fix
the function to handle a violation of the precondition in a reasonable
way.

\index{singleton}


\section{Wrappers and helpers}
\index{wrapper function}
\index{function!wrapper}
\index{helper function}
\index{function!helper}

It is often useful to divide a list operation into
two functions.  For example, to print a list
backward in the
format {\tt [3 2 1]} we can use the
{\tt printBackward} function to print {\tt 3 2 1} but we need
a separate function to print the brackets.
Let's call it {\tt printBackwardNicely}:

\beforeverb
\begin{verbatim}
def printBackwardNicely(list) :
  print "[",
  printBackward(list)
  print "]",
\end{verbatim}
\afterverb
%
Again, it is a good idea to check functions like this to see
if they work with special cases like an empty list or
a singleton.

\index{singleton}

When we use this function elsewhere in the program, we
invoke {\tt printBackwardNicely} directly, and it invokes
{\tt printBackward} on our behalf.  In that sense,
{\tt printBackwardNicely} acts as a {\bf wrapper}, and it uses
{\tt printBackward} as a {\bf helper}.


\section {The {\tt LinkedList} class}
\index{LinkedList}
\index{class!LinkedList}

There are some subtle problems with the way we have been
implementing lists.  In a reversal of cause and effect, we'll
propose an alternative implementation first and then explain what
problems it solves.

First, we'll create a new class called {\tt LinkedList}.  Its
attributes are an integer that contains the length of the list
and a reference to the first node.  {\tt LinkedList} objects
serve as handles for manipulating lists of {\tt Node} objects:

\beforeverb
\begin{verbatim}
class LinkedList :
  def __init__(self) :
    self.length = 0
    self.head   = None
\end{verbatim}
\afterverb
%
One nice thing about the {\tt LinkedList} class is that it provides
a natural place to put wrapper functions like
{\tt printBackwardNicely}, which we can make a
method of the {\tt LinkedList} class:

\beforeverb
\begin{verbatim}
class LinkedList:
  ...
  def printBackward(self):
    print "[",
    if self.head != None:
      self.head.printBackward()
    print "]",

class Node:
  ...
  def printBackward(self):
    if self.next != None:
      tail = self.next
      tail.printBackward()
    print self.cargo,
\end{verbatim}
\afterverb
%
Just to make things confusing, we renamed {\tt printBackwardNicely}.
Now there are two methods named {\tt printBackward}: one in the {\tt
Node} class (the helper); and one in the {\tt LinkedList} class (the
wrapper).  When the wrapper invokes {\tt self.head.printBackward},
it is invoking the helper, because {\tt self.head} is a
{\tt Node} object.

Another benefit of the {\tt LinkedList} class is that it
makes it easier to add or remove the first element of a list.  For
example, {\tt addFirst} is a method for {\tt LinkedList}s; it
takes an item of cargo as an argument and puts it at the beginning of the
list:

\beforeverb
\begin{verbatim}
class LinkedList:
  ...
  def addFirst(self, cargo):
    node = Node(cargo)
    node.next = self.head
    self.head = node
    self.length = self.length + 1
\end{verbatim}
\afterverb
%
As usual, you should check code like this to see if it handles
the special cases.  For example, what happens if the list is initially
empty?


\section {Invariants}
\index{invariant}
\index{object invariant}
\index{list!well-formed}

Some lists are ``well formed"; others are not.  For example, if a list
contains a loop, it will cause many of our methods to crash, so we
might want to require that lists contain no loops.  Another
requirement is that the {\tt length} value in the {\tt LinkedList}
object should be equal to the actual number of nodes in the list.

Requirements like these are called {\bf invariants} because, ideally,
they should be true of every object all the time.  Specifying
invariants for objects is a useful programming practice because it
makes it easier to prove the correctness of code, check the integrity
of data structures, and detect errors.

One thing that is sometimes confusing about invariants is that
there are times when they are violated.  For example, in the
middle of {\tt addFirst}, after we have added the node but
before we have incremented {\tt length}, the invariant is
violated.  This kind of violation is acceptable; in fact, it is
often impossible to modify an object without violating an
invariant for at least a little while.  Normally, we require
that every method that violates an invariant must restore
the invariant.

If there is any significant stretch of code in which the invariant
is violated, it is important for the comments to make that clear,
so that no operations are performed that depend on the invariant.

\index{documentation}


\section{Glossary}
\index{embedded reference}
\index{reference!embedded}
\index{recursive data structure}
\index{data structure!recursive}
\index{linked list}
\index{list!linked}
\index{node}
\index{cargo}
\index{link}
\index{precondition}
\index{invariant}
\index{wrapper}
\index{helper method}
\index{fundamental ambiguity theorem}
\index{singleton}

\begin{description}

\item[embedded reference:] A reference stored in an attribute of
an object.

\item[linked list:] A data structure that implements a collection using
a sequence of linked nodes.

\item[node:] An element of a list, usually implemented as an object
that contains a reference to another object of the same type.

\item[cargo:] An item of data contained in a node.

\item[link:] An embedded reference used to link one object to
another.

\item[precondition:] An assertion that must be true in order for a
method to work correctly.

\item[fundamental ambiguity theorem:] A reference to a list
node can be treated as a single
object or as the first in a list of nodes.

\item[singleton:] A linked list with a single node.

\item[wrapper:] A method that acts as a middleman between a
caller and a helper method, often making the method easier or
less error-prone to invoke.

\item[helper:] A method that is not invoked directly by a caller
but is used by another method to perform part of an operation.

\item[invariant:] An assertion that should be true of an object at
all times (except perhaps while the object is being modified).

\end{description}


\clearemptydoublepage
% LaTeX source for textbook ``How to think like a computer scientist''
% Copyright (c)  2001  Allen B. Downey, Jeffrey Elkner, and Chris Meyers.

% Permission is granted to copy, distribute and/or modify this
% document under the terms of the GNU Free Documentation License,
% Version 1.1  or any later version published by the Free Software
% Foundation; with the Invariant Sections being "Contributor List",
% with no Front-Cover Texts, and with no Back-Cover Texts. A copy of
% the license is included in the section entitled "GNU Free
% Documentation License".

% This distribution includes a file named fdl.tex that contains the text
% of the GNU Free Documentation License.  If it is missing, you can obtain
% it from www.gnu.org or by writing to the Free Software Foundation,
% Inc., 59 Temple Place - Suite 330, Boston, MA 02111-1307, USA.
%

\chapter{Stacks}

\section{Abstract data types}
\index{abstract data type|see{ADT}}
\index{ADT}
\index{encapsulation}

The data types you have seen so far are all concrete, in the
sense that we have completely specified how they are implemented.
For example, the {\tt Card} class represents a card using two
integers.  As we discussed at the time, that is not the only way
to represent a card; there are many alternative implementations.

An {\bf abstract data type}, or ADT, specifies a set of operations (or
methods) and the semantics of the operations (what they do), but it
does not specify the implementation of the operations.  That's
what makes it abstract.

Why is that useful?

\begin{itemize}

\item It simplifies the task of specifying an algorithm if you
can denote the operations you need without having to think at the
same time about how the operations are performed.

\item Since there are usually many ways to implement an ADT,
it might be useful to write an algorithm that can be used with
any of the possible implementations.

\item Well-known ADTs, such as the Stack ADT in this chapter,
are often implemented in standard libraries so they can be written
once and used by many programmers.

\item The operations on ADTs provide a common high-level language
for specifying and talking about algorithms.

\end{itemize}

When we talk about ADTs, we often distinguish the code that uses
the ADT, called the {\bf client} code, from the code that implements
the ADT, called the {\bf provider} code.

\index{client}
\index{provider}


\section{The Stack ADT}
\index{stack}
\index{collection}
\index{ADT!Stack}

In this chapter, we will look at one common ADT, the {\bf stack}.  A
stack is a collection, meaning that it is a data structure that
contains multiple elements.  Other collections we have seen include
dictionaries and lists.

\index{interface}

An ADT is defined by the operations that can be performed on it, which
is called an {\bf interface}.  The interface for a stack consists of
these operations:

\begin{description}

\item[{\tt \_\_init\_\_}:] Initialize a new empty stack.

\item[{\tt push}:] Add a new item to the stack.

\item[{\tt pop}:] Remove and return an item from the stack.  The item
that is returned is always the last one that was added.

\item[{\tt isEmpty}:] Check whether the stack is empty.

\end{description}

A stack is sometimes called a ``last in, first out'' or LIFO
data structure, because the last item added is the first to
be removed.


\section {Implementing stacks with Python lists}
\index{Stack}
\index{class!Stack}
\index{generic data structure}
\index{data structure!generic}

The list
operations that Python provides are similar to the operations that
define a stack.  The interface isn't exactly what it is supposed
to be, but we can write code to translate from the Stack ADT
to the built-in operations.

This code is called an {\bf implementation} of the Stack ADT.
In general, an implementation is a set of methods that satisfy
the syntactic and semantic requirements of an interface.

Here is an implementation of the Stack ADT that uses a Python list:

\beforeverb
\begin{verbatim}
class Stack :
  def __init__(self) :
    self.items = []

  def push(self, item) :
    self.items.append(item)

  def pop(self) :
    return self.items.pop()

  def isEmpty(self) :
    return (self.items == [])
\end{verbatim}
\afterverb
%
A {\tt Stack} object contains an attribute named {\tt items}
that is a list of items in the stack.  The initialization method
sets {\tt items} to the empty list.

To push a new item onto the stack, {\tt push} appends it onto {\tt
items}.  To pop an item off the stack, {\tt pop} uses the
homonymous\footnote{same-named} list method to remove and return the
last item on the list.

Finally, to check if the stack is empty, {\tt isEmpty} compares
{\tt items} to the empty list.

\index{veneer}

An implementation like this, in which the methods consist of simple
invocations of existing methods, is called a {\bf veneer}.  In real
life, veneer is a thin coating of good quality wood used in
furniture-making to hide lower quality wood underneath.  Computer
scientists use this metaphor to describe a small piece of code that
hides the details of an implementation and provides a simpler, or more
standard, interface.


\section{Pushing and popping}
\index{push}
\index{pop}
\index{generic data structure}
\index{data structure!generic}

A stack is a {\bf generic data structure}, which means that we can
add any type of item to it.  The following example pushes
two integers and a string onto the stack:

\beforeverb
\begin{verbatim}
>>> s = Stack()
>>> s.push(54)
>>> s.push(45)
>>> s.push("+")
\end{verbatim}
\afterverb
%
We can use {\tt isEmpty} and {\tt pop} to remove and print
all of the items on the stack:

\beforeverb
\begin{verbatim}
while not s.isEmpty() :
  print s.pop(),
\end{verbatim}
\afterverb
%
The output is {\tt + 45 54}.  In other words, we just used a stack
to print the items backward!  Granted, it's not the
standard format for printing a list, but by using a stack, it was
remarkably easy to do.

You should compare this bit of code to the implementation of {\tt
printBackward} in Section~\ref{listrecursion}.  There is a natural
parallel between the recursive version of {\tt printBackward} and the
stack algorithm here.  The difference is that {\tt printBackward} uses
the runtime stack to keep track of the nodes while it traverses the
list, and then prints them on the way back from the recursion.  The
stack algorithm does the same thing, except that it uses a {\tt Stack}
object instead of the runtime stack.



\section {Using a stack to evaluate postfix}
\index{postfix}
\index{infix}
\index{expression}

In most programming languages, mathematical expressions are
written with the operator between the two operands, as in
{\tt 1+2}.  This format is called {\bf infix}.  An alternative
used by some calculators is called {\bf postfix}.  In
postfix, the operator follows the operands, as in {\tt 1 2 +}.

The reason postfix is sometimes useful is that there is a
natural way to evaluate a postfix expression using a stack:

\begin{itemize}

\item Starting at the beginning of the expression, get one
term (operator or operand) at a time.

  \begin{itemize}

  \item If the term is an operand, push it on the stack.

  \item If the term is an operator, pop two operands off
  the stack, perform the operation on them, and push the
  result back on the stack.

  \end{itemize}

\item When you get to the end of the expression, there should
be exactly one operand left on the stack.  That operand is the
result.

\end{itemize}

\begin{quote}
{\em As an exercise, apply this algorithm to the expression
{\tt 1 2 + 3 *}.}
\end{quote}

This example demonstrates one of the advantages of postfix---there is
no need to use parentheses to control the order of operations.  To get
the same result in infix, we would have to write {\tt (1 + 2) * 3}.

\begin{quote}
{\em As an exercise, write a postfix expression that is equivalent to
{\tt 1 + 2 * 3}.}
\end{quote}


\section {Parsing}
\index{parse}
\index{token}
\index{delimiter}
\index{regular expression}

To implement the previous algorithm, we need
to be able to traverse a string and break it into operands and
operators.  This process is an example of {\bf parsing}, and the
results---the individual chunks of the string---are called {\bf
tokens}.  You might remember these words from Chapter 1.

Python provides a {\tt split} method in both the {\tt string} and {\tt
re} (regular expression) modules. The function {\tt string.split}
splits a string into a list using a single character as a {\bf delimiter}.
For example:

\beforeverb
\begin{verbatim}
>>> import string
>>> string.split("Now is the time"," ")
['Now', 'is', 'the', 'time']
\end{verbatim}
\afterverb
%
In this case, the delimiter is the space character, so the string
is split at each space.

The function {\tt re.split} is more powerful, allowing us to
provide a regular expression instead of a delimiter.
A regular expression is a way of specifying a set of strings.
For example, \verb+[A-z]+ is the set of all letters and
\verb+[0-9]+ is the set of all digits.  The \verb+^+ operator
negates a set, so \verb+[^0-9]+ is the set of every character that
is not a digit, which is exactly the set we want to use to
split up postfix expressions:

\beforeverb
\begin{verbatim}
>>> import re
>>> re.split("([^0-9])", "123+456*/")
['123', '+', '456', '*', '', '/', '']
\end{verbatim}
\afterverb
%
Notice that the order of the
arguments is different from {\tt string.split}; the delimiter comes
before the string.

The resulting list includes the operands {\tt 123} and {\tt 456} and
the operators {\tt *} and {\tt /}.  It also includes two empty
strings that are inserted as ``phantom operands,'' whenever an
operator appears without a number before or after it.


\section {Evaluating postfix}

To evaluate a postfix expression, we will use the parser from
the previous section and the algorithm from the section before that.
To keep things simple, we'll start with an evaluator that
only implements the operators {\tt +} and {\tt *}:

\adjustpage{-3}
\pagebreak

\beforeverb
\begin{verbatim}
def evalPostfix(expr):
  import re
  tokenList = re.split("([^0-9])", expr)
  stack = Stack()
  for token in tokenList:
    if  token == '' or token == ' ':
      continue
    if  token == '+':
      sum = stack.pop() + stack.pop()
      stack.push(sum)
    elif token == '*':
      product = stack.pop() * stack.pop()
      stack.push(product)
    else:
      stack.push(int(token))
  return stack.pop()
\end{verbatim}
\afterverb
%
The first condition takes care of spaces and empty strings.  The next
two conditions handle operators. We assume, for now, that anything
else must be an operand.  Of course, it would be better to check for
erroneous input and report an error message, but we'll get to that
later.

Let's test it by evaluating the postfix form of {\tt (56+47)*2}:

\beforeverb
\begin{verbatim}
>>> print evalPostfix ("56 47 + 2 *")
206
\end{verbatim}
\afterverb
%
That's close enough.


\section {Clients and providers}
\index{encapsulation}
\index{ADT}

One of the fundamental goals of an ADT is to separate the
interests of the provider, who writes the code that implements
the ADT, and the client, who uses the ADT.
The provider only has to worry
about whether the implementation is correct---in accord
with the specification of the ADT---and not how it will be used.

Conversely, the client {\em assumes} that the implementation of the
ADT is correct and doesn't worry about the details.  When you
are using one of Python's built-in types, you have the luxury
of thinking exclusively as a client.

Of course, when you implement an ADT, you also have
to write client code to test it.  In that case, you play both
roles, which can be confusing.  You should make some effort
to keep track of which role you are playing at any moment.


\section{Glossary}
\index{ADT}
\index{client}
\index{provider}
\index{infix}
\index{postfix}
\index{parse}
\index{token}
\index{delimiter}

\begin{description}

\item[abstract data type (ADT):]  A data type (usually a collection
of objects) that is defined by a set of operations but that can
be implemented in a variety of ways.

\item[interface:] The set of operations that define an ADT.

\item[implementation:] Code that satisfies the syntactic and semantic
requirements of an interface.

\item[client:]  A program (or the person who wrote it) that uses an ADT.

\item[provider:] The code (or the person
who wrote it) that implements an ADT.

\item[veneer:]  A class definition that implements an ADT with
method definitions that are invocations of other methods, sometimes
with simple transformations.  The veneer does no significant work,
but it improves or standardizes the interface seen by the client.

\item[generic data structure:] A kind of data structure that can
contain data of any type.

\item[infix:]  A way of writing mathematical expressions with the
operators between the operands.

\item[postfix:]  A way of writing mathematical expressions with the
operators after the operands.

\item[parse:]  To read a string of characters or tokens and analyze
its grammatical structure.

\item[token:]  A set of characters that are treated as a unit for
purposes of parsing, such as the words in a natural language.

\item[delimiter:]  A character that is used to separate tokens,
such as punctuation in a natural language.

\end{description}

\clearemptydoublepage
% LaTeX source for textbook ``How to think like a computer scientist''
% Copyright (c)  2001  Allen B. Downey, Jeffrey Elkner, and Chris Meyers.

% Permission is granted to copy, distribute and/or modify this
% document under the terms of the GNU Free Documentation License,
% Version 1.1  or any later version published by the Free Software
% Foundation; with the Invariant Sections being "Contributor List",
% with no Front-Cover Texts, and with no Back-Cover Texts. A copy of
% the license is included in the section entitled "GNU Free
% Documentation License".

% This distribution includes a file named fdl.tex that contains the text
% of the GNU Free Documentation License.  If it is missing, you can obtain
% it from www.gnu.org or by writing to the Free Software Foundation,
% Inc., 59 Temple Place - Suite 330, Boston, MA 02111-1307, USA.


\chapter{Queues}
\label{queue}
\index{queue}
\index{ADT!Queue}
\index{priority queue}
\index{ADT!Priority Queue}
\index{FIFO}
\index{queueing policy}
\index{priority queueing}

This chapter presents two ADTs: the Queue and the Priority Queue.
In real life, a {\bf queue} is a line of customers waiting for service
of some kind.  In most cases, the first customer in line is the
next customer to be served.  There are exceptions, though.
At airports, customers whose flights are leaving soon
are sometimes taken from the middle of the queue.  At
supermarkets, a polite customer might let someone with only a
few items go first.

The rule that determines who goes next is called the
{\bf queueing policy}.  The simplest queueing policy is
called {\bf FIFO}, for ``first-in-first-out.''  The most general
queueing policy is {\bf priority queueing}, in which each customer
is assigned a priority and the customer with the highest priority
goes first, regardless of the order of arrival.  We
say this is the most general policy because the priority
can be based on anything: what time a flight leaves; how many
groceries the customer has; or how important the customer is.
Of course, not all queueing policies are ``fair,'' but
fairness is in the eye of the beholder.

The Queue ADT and the Priority Queue ADT have the same set
of operations.  The difference
is in the semantics of the operations: a queue uses the FIFO
policy; and a priority queue (as the name suggests) uses the
priority queueing policy.

\adjustpage{1}

\section{The Queue ADT}
\index{ADT!Queue}
\index{Queue ADT}
\index{implementation!Queue}
\index{queue!List implementation}

The Queue ADT is defined by the following operations:

\begin{description}

\item[{\tt \_\_init\_\_}:] Initialize a new empty queue.

\item[{\tt insert}:] Add a new item to the queue.

\item[{\tt remove}:] Remove and return an item from the queue.  The item
that is returned is the first one that was added.

\item[{\tt isEmpty}:] Check whether the queue is empty.

\end{description}


\section{Linked Queue}
\index{linked queue}
\index{queue!linked implementation}

The first implementation of the Queue ADT we will look at is
called a {\bf linked queue} because it is made up of linked
{\tt Node} objects.  Here is the class definition:

\beforeverb
\begin{verbatim}
class Queue:
  def __init__(self):
    self.length = 0
    self.head = None

  def isEmpty(self):
    return (self.length == 0)

  def insert(self, cargo):
    node = Node(cargo)
    node.next = None
    if self.head == None:
      # if list is empty the new node goes first
      self.head = node
    else:
      # find the last node in the list
      last = self.head
      while last.next: last = last.next
      # append the new node
      last.next = node
    self.length = self.length + 1

  def remove(self):
    cargo = self.head.cargo
    self.head = self.head.next
    self.length = self.length - 1
    return cargo
\end{verbatim}
\afterverb
%
The methods {\tt isEmpty} and {\tt remove} are identical to the
{\tt LinkedList} methods {\tt isEmpty} and {\tt removeFirst}.
The {\tt insert} method is new and a bit more complicated.

We want to insert new items at the end of the list.
If the queue is empty, we just set {\tt
head} to refer to the new node.

Otherwise, we traverse the list to the last node and
tack the new node on the end.  We can identify the last node because
its {\tt next} attribute is {\tt None}.

There are two invariants for a properly formed {\tt Queue} object.
The value of {\tt length} should be the number of nodes in the
queue, and the last node should have {\tt next} equal to
{\tt None}.  Convince yourself that this method preserves
both invariants.


\section{Performance characteristics}
\index{performance}

Normally when we invoke a method, we are not concerned with the
details of its implementation.  But there is one ``detail''
we might want to know---the performance characteristics of the
method.  How long does it take, and how does the run time change
as the number of items in the collection increases?

First look at {\tt remove}.
There are no loops or function calls here, suggesting that
the runtime of this method is the same every time.  Such a method
is called a {\bf constant time} operation.
In reality, the method might be slightly faster
when the list is empty since it skips the body of the conditional,
but that difference is not significant.

\index{constant time}

The performance of {\tt insert} is very different.
In the general case, we have to
traverse the list to find the last element.

This traversal takes time proportional to the length of the
list.  Since the runtime is a linear function of the length,
this method is called {\bf linear time}.  Compared to
constant time, that's very bad.

\index{linear time}


\section{Improved Linked Queue}
\index{queue!improved implementation}
\index{improved queue}

We would like an implementation of the Queue ADT that can
perform all operations in constant time.  One way to
do that is to modify the Queue class so that it
maintains a reference to both the first and the last node,
as shown in the figure:

\beforefig
\centerline{\psfig{figure=illustrations/queue1.eps}}
\afterfig

The {\tt ImprovedQueue} implementation looks like this:

\beforeverb
\begin{verbatim}
class ImprovedQueue:
  def __init__(self):
    self.length = 0
    self.head   = None
    self.last   = None

  def isEmpty(self):
    return (self.length == 0)
\end{verbatim}
\afterverb
%
So far, the only change is the attribute {\tt last}. It is used in
{\tt insert} and {\tt remove} methods:

\beforeverb
\begin{verbatim}
class ImprovedQueue:
  ...
  def insert(self, cargo):
    node = Node(cargo)
    node.next = None
    if self.length == 0:
      # if list is empty, the new node is head and last
      self.head = self.last = node
    else:
      # find the last node
      last = self.last
      # append the new node
      last.next = node
      self.last = node
    self.length = self.length + 1
\end{verbatim}
\afterverb
%
Since {\tt last} keeps track of the last node, we don't have to search
for it.  As a result, this method is constant time.

There is a price to pay for that speed.  We have to add a special case
to {\tt remove} to set {\tt last} to {\tt None} when the last node is
removed:

\beforeverb
\begin{verbatim}
class ImprovedQueue:
  ...
  def remove(self):
    cargo     = self.head.cargo
    self.head = self.head.next
    self.length = self.length - 1
    if self.length == 0:
      self.last = None
    return cargo
\end{verbatim}
\afterverb
%
This implementation is more complicated than the
Linked Queue implementation, and it is more difficult to demonstrate
that it is correct.  The advantage is that we have achieved
the goal---both {\tt insert} and {\tt remove} are constant time
operations.

\begin{quote}
{\em As an exercise, write an implementation of the Queue ADT using
a Python list.  Compare the performance of this implementation to the
{\tt ImprovedQueue} for a range of queue lengths.}
\end{quote}


\section{Priority queue}
\index{priority queue!ADT}
\index{ADT!Priority Queue}

The Priority Queue ADT has the same interface as the Queue ADT, but
different semantics.  Again, the interface is:

\begin{description}

\item[{\tt \_\_init\_\_}:] Initialize a new empty queue.

\item[{\tt insert}:] Add a new item to the queue.

\item[{\tt remove}:] Remove and return an item from the queue.  The item
that is returned is the one with the highest priority.

\item[{\tt isEmpty}:] Check whether the queue is empty.

\end{description}

The semantic difference is that the item that is removed from the
queue is not necessarily the first one that was added.  Rather, it is
the item in the queue that has the highest priority.  What the
priorities are and how they compare to each other are not specified by
the Priority Queue implementation.  It depends on which items are in
the queue.

For example, if the items in the queue have names, we might choose
them in alphabetical order.  If they are bowling scores, we might go
from highest to lowest, but if they are golf scores, we would go from
lowest to highest.  As long as we can compare the items in the queue,
we can find and remove the one with the highest priority.

This implementation of Priority Queue has as an attribute
a Python list that
contains the items in the queue.

\beforeverb
\begin{verbatim}
class PriorityQueue:
  def __init__(self):
    self.items = []

  def isEmpty(self):
    return self.items == []

  def insert(self, item):
    self.items.append(item)
\end{verbatim}
\afterverb
%
The initialization method, {\tt isEmpty}, and {\tt insert} are all
veneers on list operations.  The only interesting method is {\tt
remove}:

\beforeverb
\begin{verbatim}
class PriorityQueue:
  ...
  def remove(self):
    maxi = 0
    for i in range(1,len(self.items)):
      if self.items[i] > self.items[maxi]:
        maxi = i
    item = self.items[maxi]
    self.items[maxi:maxi+1] = []
    return item
\end{verbatim}
\afterverb
%
At the beginning of each iteration, {\tt maxi} holds the index of the
biggest item (highest priority) we have seen {\em so far}.  Each time
through the loop, the program compares the {\tt i}-eth item to the champion.
If the new item is bigger, the value of {\tt maxi} is set to {\tt i}.

\index{traverse}

When the {\tt for} statement completes,
{\tt maxi} is the index of the biggest item.  This item
is removed from the list and returned.

Let's test the implementation:

\beforeverb
\begin{verbatim}
>>> q = PriorityQueue()
>>> q.insert(11)
>>> q.insert(12)
>>> q.insert(14)
>>> q.insert(13)
>>> while not q.isEmpty(): print q.remove()
14
13
12
11
\end{verbatim}
\afterverb
%
If the queue contains simple numbers or strings, they are
removed in numerical or alphabetical order, from highest to
lowest.  Python can find the biggest integer or string because
it can compare them using the built-in comparison operators.

If the queue contains an object type, it has to provide
a {\tt \_\_cmp\_\_} method.  When {\tt remove} uses the {\tt >}
operator to compare items, it invokes the {\tt \_\_cmp\_\_}
for one of the items and passes the other as an argument.  As
long as the {\tt \_\_cmp\_\_} method works correctly, the
Priority Queue will work.


\section{The {\tt Golfer} class}
\index{Golfer}
\index{class!Golfer}
\index{priority}
\index{operator overloading}
\index{overloading!operator}

As an example of an object with an unusual definition of priority, let's
implement a class called {\tt Golfer} that keeps track of the
names and scores of golfers.  As usual, we start by defining {\tt
\_\_init\_\_} and {\tt \_\_str\_\_}:

\beforeverb
\begin{verbatim}
class Golfer:
  def __init__(self, name, score):
    self.name = name
    self.score= score

  def __str__(self):
    return "%-16s: %d" % (self.name, self.score)
\end{verbatim}
\afterverb
%
{\tt \_\_str\_\_} uses the format operator to put the names
and scores in neat columns.

\index{format operator}
\index{operator!format}

Next we define a version of {\tt \_\_cmp\_\_} where the lowest
score gets highest priority.  As always, {\tt \_\_cmp\_\_} returns
1 if {\tt self} is ``greater than'' {\tt other}, -1 if {\tt self}
is ``less than'' other, and 0 if they are equal.

\beforeverb
\begin{verbatim}
class Golfer:
  ...
  def __cmp__(self, other):
    if self.score < other.score: return  1   # less is more
    if self.score > other.score: return -1
    return 0
\end{verbatim}
\afterverb
%
Now we are ready to test the priority queue with the {\tt Golfer} class:

\beforeverb
\begin{verbatim}
>>> tiger = Golfer("Tiger Woods",    61)
>>> phil  = Golfer("Phil Mickelson", 72)
>>> hal   = Golfer("Hal Sutton",     69)
>>>
>>> pq = PriorityQueue()
>>> pq.insert(tiger)
>>> pq.insert(phil)
>>> pq.insert(hal)
>>> while not pq.isEmpty(): print pq.remove()
Tiger Woods    : 61
Hal Sutton     : 69
Phil Mickelson : 72
\end{verbatim}
\afterverb

\begin{quote}
{\em As an exercise, write an implementation of the Priority Queue ADT 
using a linked list.  You should keep the list sorted so that removal
is a constant time operation.  Compare the performance of this
implementation with the Python list implementation.}
\end{quote}


\section{Glossary}
\index{queue}
\index{queueing policy}
\index{FIFO}
\index{priority queue}
\index{veneer}
\index{constant time}
\index{linear time}
\index{performance hazard}
\index{linked queue}
\index{circular buffer}
\index{abstract class}
\index{interface}

\begin{description}

\item[queue:]  An ordered set of objects waiting for a service of
some kind.

\item[Queue:]  An ADT that performs the operations one might perform
on a queue.

\item[queueing policy:]  The rules that determine which member
of a queue is removed next.

\item[FIFO:]  ``First In, First Out,'' a queueing policy in which
the first member to arrive is the first to be removed.

\item[priority queue:]  A queueing policy in which
each member has a priority determined by external factors.
The member with the highest priority is the first to be removed.

\item[Priority Queue:]  An ADT that defines the operations one
might perform on a priority queue.

\item[linked queue:]  An implementation of a queue using a linked
list.

\item[constant time:]  An operation whose runtime does not
depend on the size of the data structure.

\item[linear time:]  An operation whose runtime is a linear
function of the size of the data structure.

\end{description}

\clearemptydoublepage
% LaTeX source for textbook ``How to think like a computer scientist''
% Copyright (c)  2001  Allen B. Downey, Jeffrey Elkner, and Chris Meyers.

% Permission is granted to copy, distribute and/or modify this
% document under the terms of the GNU Free Documentation License,
% Version 1.1  or any later version published by the Free Software
% Foundation; with the Invariant Sections being "Contributor List",
% with no Front-Cover Texts, and with no Back-Cover Texts. A copy of
% the license is included in the section entitled "GNU Free
% Documentation License".

% This distribution includes a file named fdl.tex that contains the text
% of the GNU Free Documentation License.  If it is missing, you can obtain
% it from www.gnu.org or by writing to the Free Software Foundation,
% Inc., 59 Temple Place - Suite 330, Boston, MA 02111-1307, USA.
%

\chapter{Trees}
\index{tree}
\index{node}
\index{tree node}
\index{cargo}
\index{embedded reference}
\index{binary tree}

Like linked lists, trees are made up of nodes.  A common kind of tree
is a {\bf binary tree}, in which each node contains a reference to two
other nodes (possibly null).  These references are referred to as the
left and right subtrees.  Like list nodes, tree nodes also contain
cargo.  A state diagram for a tree looks like this:

\label{tree}
\beforefig
\centerline{\psfig{figure=illustrations/tree1.eps,height=1.7in}}
\afterfig

To avoid cluttering up the picture, we often omit the
{\tt Nones}.

The top of the tree (the node {\tt tree} refers to) is called the
{\bf root}.  In keeping with the tree metaphor, the other nodes are
called branches and the nodes at the tips with null references are
called {\bf leaves}.  It may seem odd that we draw the picture with
the root at the top and the leaves at the bottom, but that is not the
strangest thing.

\index{root node}
\index{leaf node}
\index{parent node}
\index{child node}
\index{level}

To make things worse, computer scientists mix in another
metaphor---the family tree.  The top node is sometimes called
a {\bf parent} and the nodes it refers to are its {\bf children}.
Nodes with the same parent are called {\bf siblings}.

Finally, there is a geometric vocabulary for talking
about trees.  We already mentioned left and right, but there is
also ``up'' (toward the parent/root) and ``down'' (toward the
children/leaves).  Also, all of the nodes that are the same
distance from the root comprise a {\bf level} of the tree.

We probably don't need three metaphors for talking about trees,
but there they are.

Like linked lists, trees are recursive data structures because
they are defined recursively.

\index{recursive data structure}
\index{data structure!recursive}

\begin{quote}
A tree is either:

\begin{itemize}

\item the empty tree, represented by {\tt None}, or

\item a node that contains an object reference and two
tree references.

\end{itemize}

\end{quote}

\index{tree!empty}


\section {Building trees}

The process of assembling a tree is similar
to the process of assembling a linked list.
Each constructor invocation builds a single node.

\beforeverb
\begin{verbatim}
class Tree:
  def __init__(self, cargo, left=None, right=None):
    self.cargo = cargo
    self.left  = left
    self.right = right

  def __str__(self):
    return str(self.cargo)
\end{verbatim}
\afterverb
%
The {\tt cargo} can
be any type, but the arguments for {\tt left} and {\tt right} should be
tree nodes.  {\tt left} and {\tt right} are optional; the default
value is {\tt None}.

To print a node, we just print the cargo.

One way to build a tree is from the bottom up.
Allocate the child nodes first:

\beforeverb
\begin{verbatim}
left = Tree(2)
right = Tree(3)
\end{verbatim}
\afterverb
%
Then create the parent node and link it to the children:

\beforeverb
\begin{verbatim}
tree = Tree(1, left, right);
\end{verbatim}
\afterverb
%
We can write this code more concisely by nesting constructor
invocations:

\beforeverb
\begin{verbatim}
>>> tree = Tree(1, Tree(2), Tree(3))
\end{verbatim}
\afterverb
%
Either way, the result is the tree at the beginning of the
chapter.


\section {Traversing trees}
\index{tree!traversal}
\index{traverse}
\index{recursion}

Any time you see a new data structure, your first
question should be, ``How do I traverse it?''  The most natural
way to traverse a tree is recursively.  For example, if the
tree contains integers as cargo, this function returns their sum:

\beforeverb
\begin{verbatim}
def total(tree):
  if tree == None: return 0
  return total(tree.left) + total(tree.right) + tree.cargo
\end{verbatim}
\afterverb
%
The base case is the empty tree, which contains no cargo, so
the sum is 0.
The recursive step
makes two recursive calls to find the sum of the child trees.
When the recursive calls complete,
we add the cargo of the parent and return the
total.


\section {Expression trees}
\index{tree!expression}
\index{expression tree}
\index{postfix}
\index{infix}
\index{binary operator}
\index{operator!binary}

A tree is a natural way to represent the structure of an expression.
Unlike other notations, it can represent the computation
unambiguously.  For example, the infix expression {\tt 1 + 2 * 3} is
ambiguous unless we know that the multiplication happens before the
addition.

This expression tree represents the same computation:

\beforefig
\centerline{\psfig{figure=illustrations/tree2.eps,height=2in}}
\afterfig

The nodes of an expression tree can be operands like {\tt 1} and
{\tt 2} or operators like {\tt +} and {\tt *}.  Operands are leaf nodes;
operator nodes contain references to their operands.  (All of these
operators are {\bf binary}, meaning they have exactly two operands.)

We can build this tree like this:

\beforeverb
\begin{verbatim}
>>> tree = Tree('+', Tree(1), Tree('*', Tree(2), Tree(3)))
\end{verbatim}
\afterverb
%
Looking at the figure, there is no question what the order of
operations is; the multiplication happens first in order to compute
the second operand of the addition.

Expression trees have many uses.  The example in this chapter uses
trees to translate expressions to postfix, prefix, and infix.
Similar trees are used inside compilers to parse, optimize, and
translate programs.


\section {Tree traversal}
\index{tree!traversal}
\index{traverse}
\index{recursion}
\index{preorder}
\index{postorder}
\index{inorder}

We can traverse an expression tree and print the contents like this:

\beforeverb
\begin{verbatim}
def printTree(tree):
  if tree == None: return
  print tree.cargo,
  printTree(tree.left)
  printTree(tree.right)
\end{verbatim}
\afterverb
%
\index{preorder}
\index{prefix}

In other words, to print a tree, first print the contents of
the root, then print the entire left subtree, and then print the
entire right subtree.  This way of traversing a tree is called
a {\bf preorder}, because the contents of the root appear {\em before}
the contents of the children.
For the previous example, the output is:

\beforeverb
\begin{verbatim}
>>> tree = Tree('+', Tree(1), Tree('*', Tree(2), Tree(3)))
>>> printTree(tree)
+ 1 * 2 3
\end{verbatim}
\afterverb
%
This format is different from both postfix and infix; it is another
notation called {\bf prefix}, in which the operators appear before
their operands.

You might suspect that if you traverse the tree in a
different order, you will get the expression in a different
notation.  For example, if you print the subtrees first and then
the root node, you get:

\beforeverb
\begin{verbatim}
def printTreePostorder(tree):
  if tree == None: return
  printTreePostorder(tree.left)
  printTreePostorder(tree.right)
  print tree.cargo,
\end{verbatim}
\afterverb
%
\index{postorder}
\index{inorder}
The result, {\tt 1 2 3 * +}, is in postfix!
This order of traversal is called {\bf postorder}.

Finally, to traverse a tree {\bf inorder},
you print the left tree, then the root, and then the right tree:

\beforeverb
\begin{verbatim}
def printTreeInorder(tree):
  if tree == None: return
  printTreeInorder(tree.left)
  print tree.cargo,
  printTreeInorder(tree.right)
\end{verbatim}
\afterverb
%
The result is {\tt 1 + 2 * 3}, which is the expression in infix.

To be fair, we should point out that we have omitted an
important complication.  Sometimes when we write an expression
in infix, we have to use parentheses to preserve the order of
operations.  So an inorder traversal is not quite sufficient to
generate an infix expression.

Nevertheless, with a few improvements, the expression tree and the
three recursive traversals provide a general way to translate
expressions from one format to another.

\begin{quote}
{\em As an exercise, modify {\tt printTreeInorder} so that it
puts parentheses around every operator and pair of operands.
Is the output correct and unambiguous?  Are the parentheses
always necessary? }
\end{quote}

If we do an inorder traversal and keep track of what level
in the tree we are on, we can generate a graphical representation
of a tree:

\beforeverb
\begin{verbatim}
def printTreeIndented(tree, level=0):
  if tree == None: return
  printTreeIndented(tree.right, level+1)
  print '  '*level + str(tree.cargo)
  printTreeIndented(tree.left, level+1)
\end{verbatim}
\afterverb
%
The parameter {\tt level} keeps track of where we are in the
tree.  By default, it is initially 0.  Each time we make a
recursive call, we pass {\tt level+1} because the child's level
is always one greater than the parent's.  Each item is indented by
two spaces per level.  The result for the example tree is:

\beforeverb
\begin{verbatim}
>>> printTreeIndented(tree)
    3
  *
    2
+
  1
\end{verbatim}
\afterverb
%
If you look at the output sideways, you see a simplified version
of the original figure.



\section{Building an expression tree}
\index{expression tree}
\index{tree!expression}
\index{parse}
\index{token}

In this section, we parse infix expressions and build the
corresponding expression trees.  For example, the expression
{\tt (3+7)*9} yields the following tree:

\beforefig
\centerline{\psfig{figure=illustrations/tree3.eps}}
\afterfig

Notice that we have simplified the diagram by leaving
out the names of the attributes.

The parser we will write handles expressions that include numbers,
parentheses, and the operators {\tt +} and {\tt *}.
We assume that the input string has already
been tokenized into a Python list.  The token list for
{\tt (3+7)*9} is:

\beforeverb
\begin{verbatim}
['(', 3, '+', 7, ')', '*', 9, 'end']
\end{verbatim}
\afterverb
%
The {\tt end} token is useful for preventing the parser from
reading past the end of the list.

\begin{quote}
{\em As an exercise, write a function that takes an expression
string and returns a token list.}
\end{quote}

The first function we'll write is {\tt getToken}, which takes a token
list and an expected token as arguments.  It compares the expected
token to the first token on the list: if they match, it removes the
token from the list and returns true; otherwise, it returns false:

\beforeverb
\begin{verbatim}
def getToken(tokenList, expected):
  if tokenList[0] == expected:
    del tokenList[0]
    return True
  else:
    return False
\end{verbatim}
\afterverb
%
Since {\tt tokenList} refers to a mutable object, the changes made
here are visible to any other variable that refers to the same object.

The next function, {\tt getNumber}, handles operands.
If the next token in {\tt tokenList} is a number,
{\tt getNumber} removes it and returns a leaf node containing
the number; otherwise, it returns {\tt None}.

\beforeverb
\begin{verbatim}
def getNumber(tokenList):
  x = tokenList[0]
  if not isinstance(x, int): return None
  del tokenList[0]
  return Tree (x, None, None)
\end{verbatim}
\afterverb
%
Before continuing, we should test {\tt getNumber} in isolation.  We
assign a list of numbers to {\tt tokenList}, extract the first,
print the result, and print what remains of the token list:

\beforeverb
\begin{verbatim}
>>> tokenList = [9, 11, 'end']
>>> x = getNumber(tokenList)
>>> printTreePostorder(x)
9
>>> print tokenList
[11, 'end']
\end{verbatim}
\afterverb
%
The next method we need is {\tt getProduct}, which builds an
expression tree for products.  A simple product has two numbers as
operands, like {\tt 3 * 7}.

Here is a version of {\tt getProduct} that handles
simple products.

\beforeverb
\begin{verbatim}
def getProduct(tokenList):
  a = getNumber(tokenList)
  if getToken(tokenList, '*'):
    b = getNumber(tokenList)
    return Tree ('*', a, b)
  else:
    return a
\end{verbatim}
\afterverb
%
Assuming that {\tt getNumber} succeeds and returns a singleton tree,
we assign the first operand to {\tt a}.
If the next character is {\tt *}, we get the second number
and build an expression tree with {\tt a}, {\tt b},
and the operator.

If the next character is anything else, then we just return
the leaf node with {\tt a}.  Here are two examples:

\beforeverb
\begin{verbatim}
>>> tokenList = [9, '*', 11, 'end']
>>> tree = getProduct(tokenList)
>>> printTreePostorder(tree)
9 11 *
\end{verbatim}
\afterverb

\beforeverb
\begin{verbatim}
>>> tokenList = [9, '+', 11, 'end']
>>> tree = getProduct(tokenList)
>>> printTreePostorder(tree)
9
\end{verbatim}
\afterverb
%
The second example implies that we consider a single
operand to be a kind of product.  This definition of
``product'' is counterintuitive, but it turns out to
be useful.  

Now we have to deal with compound products, like like {\tt 3 * 5 *
13}.  We treat this expression as a product of products, namely {\tt 3
* (5 * 13)}.  The resulting tree is:

\beforefig
\centerline{\psfig{figure=illustrations/tree4.eps}}
\afterfig

With a small change in {\tt getProduct}, we can handle
an arbitrarily long product:

\beforeverb
\begin{verbatim}
def getProduct(tokenList):
  a = getNumber(tokenList)
  if getToken(tokenList, '*'):
    b = getProduct(tokenList)       # this line changed
    return Tree ('*', a, b)
  else:
    return a
\end{verbatim}
\afterverb
%
In other words, a product can be either a singleton or a tree with
{\tt *} at the root, a number on the left, and a product on the right.
This kind of recursive definition should be starting to feel
familiar.

\index{product}
\index{definition!recursive}
\index{recursive definition}

Let's test the new version with a compound product:

\beforeverb
\begin{verbatim}
>>> tokenList = [2, '*', 3, '*', 5 , '*', 7, 'end']
>>> tree = getProduct(tokenList)
>>> printTreePostorder(tree)
2 3 5 7 * * *
\end{verbatim}
\afterverb
%
Next we will add the ability to parse sums.  Again, we
use a slightly counterintuitive definition of ``sum.''
For us, a sum can be a tree with {\tt +} at the root,
a product on the left, and a sum on the right.  Or, a sum
can be just a product.

\index{sum}

If you are willing to play along with this definition, it has a nice
property: we can represent any expression (without parentheses) as a
sum of products.  This property is the basis of our parsing algorithm.

{\tt getSum} tries to build a tree with a product on the left and a
sum on the right.  But if it doesn't find a {\tt +}, it just builds a
product.

\beforeverb
\begin{verbatim}
def getSum(tokenList):
  a = getProduct(tokenList)
  if getToken(tokenList, '+'):
    b = getSum(tokenList)
    return Tree ('+', a, b)
  else:
    return a
\end{verbatim}
\afterverb
%
Let's test it with {\tt 9 * 11 + 5 * 7}:

\beforeverb
\begin{verbatim}
>>> tokenList = [9, '*', 11, '+', 5, '*', 7, 'end']
>>> tree = getSum(tokenList)
>>> printTreePostorder(tree)
9 11 * 5 7 * +
\end{verbatim}
\afterverb
%
We are almost done, but we still have to handle parentheses.
Anywhere in an expression where there can be a number, there can
also be an entire sum
enclosed in parentheses.  We just need to modify {\tt getNumber} to
handle {\bf subexpressions}:

\index{subexpression}

\beforeverb
\begin{verbatim}
def getNumber(tokenList):
  if getToken(tokenList, '('):
    x = getSum(tokenList)         # get the subexpression
    getToken(tokenList, ')')      # remove the closing parenthesis
    return x
  else:
    x = tokenList[0]
    if not isinstance(x, int): return None
    tokenList[0:1] = []
    return Tree (x, None, None)    
\end{verbatim}
\afterverb
%
Let's test this code with {\tt 9 * (11 + 5) * 7}:

\beforeverb
\begin{verbatim}
>>> tokenList = [9, '*', '(', 11, '+', 5, ')', '*', 7, 'end']
>>> tree = getSum(tokenList)
>>> printTreePostorder(tree)
9 11 5 + 7 * *
\end{verbatim}
\afterverb
%
\adjustpage{-2}
\pagebreak

The parser handled the parentheses correctly; the addition happens
before the multiplication.

In the final version of the program, it would be a good idea
to give {\tt getNumber} a name
more descriptive of its new role.


\section{Handling errors}
\index{handling errors}
\index{error handling}

Throughout the parser, we've been assuming that expressions are
well-formed.  For example, when we reach the end of a
subexpression, we assume that the next character is a close
parenthesis.  If there is an error and the next character is something
else, we should deal with it.

\beforeverb
\begin{verbatim}
def getNumber(tokenList):
  if getToken(tokenList, '('):
    x = getSum(tokenList)       
    if not getToken(tokenList, ')'):
      raise ValueError, 'missing parenthesis'
    return x
  else:
    # the rest of the function omitted
\end{verbatim}
\afterverb
%
The {\tt raise} statement creates an exception; in this
case a {\tt ValueError}.  If the function that called
{\tt getNumber}, or one of the other functions in the
traceback, handles the exception, then the program
can continue.  Otherwise, Python will print an error message
and quit.

\begin{quote}
{\em As an exercise, find other places in these functions where errors
can occur and add appropriate {\tt raise} statements.
Test your code with improperly
formed expressions.}
\end{quote}


\section{The animal tree}
\index{animal game}
\index{game!animal}
\index{knowledge base}

In this section, we develop a small program that uses a tree
to represent a knowledge base.

The program interacts with the user to create a tree of questions
and animal names.  Here is a sample run:

\adjustpage{-3}
\pagebreak

\beforeverb
\begin{verbatim}
Are you thinking of an animal? y
Is it a bird? n
What is the animals name? dog
What question would distinguish a dog from a bird? Can it fly
If the animal were dog the answer would be? n

Are you thinking of an animal? y
Can it fly? n
Is it a dog? n
What is the animals name? cat
What question would distinguish a cat from a dog? Does it bark
If the animal were cat the answer would be? n

Are you thinking of an animal? y
Can it fly? n
Does it bark? y
Is it a dog? y
I rule!

Are you thinking of an animal? n
\end{verbatim}
\afterverb
%
Here is the tree this dialog builds:

\beforefig
\centerline{\psfig{figure=illustrations/tree5.eps}}
\afterfig

At the beginning of each round, the program starts at the top of the
tree and asks the first question.  Depending on the answer, it moves
to the left or right child and continues until it gets to a leaf
node.  At that point, it makes a guess.  If the guess is not correct,
it asks the user for the name of the new animal and a question that
distinguishes the (bad) guess from the new animal.  Then it adds a
node to the tree with the new question and the new animal.

Here is the code:

\adjustpage{-2}
\pagebreak

\beforeverb
\begin{verbatim}
def animal():
  # start with a singleton
  root = Tree("bird")

  # loop until the user quits
  while True:
    print
    if not yes("Are you thinking of an animal? "): break

    # walk the tree
    tree = root
    while tree.getLeft() != None:
      prompt = tree.getCargo() + "? "
      if yes(prompt):
        tree = tree.getRight()
      else:
        tree = tree.getLeft()

    # make a guess
    guess = tree.getCargo()
    prompt = "Is it a " + guess + "? "
    if yes(prompt):
      print "I rule!"
      continue

    # get new information
    prompt  = "What is the animal's name? "
    animal  = raw_input(prompt)
    prompt  = "What question would distinguish a %s from a %s? "
    question = raw_input(prompt % (animal,guess))

    # add new information to the tree
    tree.setCargo(question)
    prompt = "If the animal were %s the answer would be? "
    if yes(prompt % animal):
      tree.setLeft(Tree(guess))
      tree.setRight(Tree(animal))
    else:
      tree.setLeft(Tree(animal))
      tree.setRight(Tree(guess))
\end{verbatim}
\afterverb
%
The function {\tt yes} is a helper; it prints a prompt and then
takes input from the user.  If the response
begins with {\em y} or {\em Y}, the function returns true:

\beforeverb
\begin{verbatim}
def yes(ques):
  from string import lower
  ans = lower(raw_input(ques))
  return (ans[0] == 'y')
\end{verbatim}
\afterverb
%
The condition of the outer loop is {\tt True}, which means it will
continue until the {\tt break} statement executes, if the user
is not thinking of an animal.

The inner {\tt while} loop walks the tree from top to bottom,
guided by the user's responses.

When a new node is added to the tree, the new question replaces
the cargo, and the two children are the new animal and the original
cargo.

One shortcoming of the program is that when it exits, it forgets
everything you carefully taught it!

\begin{quote}
{\em As an exercise, think of various ways you might save
the knowledge tree in a file.  Implement the one you think
is easiest.}
\end{quote}


\section{Glossary}
\index{binary tree}
\index{node}
\index{root node}
\index{leaf node}
\index{parent node}
\index{child node}
\index{sibling node}
\index{level}
\index{prefix}
\index{preorder}
\index{postorder}
\index{inorder}
\index{binary operator}
\index{operator!binary}

\begin{description}

\item[binary tree:]  A tree in which each node refers to zero, one, or
two dependent nodes.

\item[root:]  The topmost node in a tree, with no parent.

\item[leaf:]  A bottom-most node in a tree, with no children.

\item[parent:]  The node that refers to a given node.

\item[child:]  One of the nodes referred to by a node.

\item[siblings:]  Nodes that share a common parent.

\item[level:]  The set of nodes equidistant from the root.

\item[binary operator:]  An operator that takes two operands.

\item[subexpression:]  An expression in parentheses that acts
as a single operand in a larger expression.

\item[preorder:]  A way to traverse a tree, visiting each node
before its children.

\item[prefix notation:]  A way of writing a mathematical expression
with each operator appearing before its operands.

\item[postorder:]  A way to traverse a tree, visiting the children
of each node before the node itself.

\item[inorder:]  A way to traverse a tree, visiting the left subtree,
then the root, and then the right subtree.

\end{description}







\clearemptydoublepage

\appendix
% LaTeX source for textbook ``How to think like a computer scientist''
% Copyright (c)  2001  Allen B. Downey, Jeffrey Elkner, and Chris Meyers.

% Permission is granted to copy, distribute and/or modify this
% document under the terms of the GNU Free Documentation License,
% Version 1.1  or any later version published by the Free Software
% Foundation; with the Invariant Sections being "Contributor List",
% with no Front-Cover Texts, and with no Back-Cover Texts. A copy of
% the license is included in the section entitled "GNU Free
% Documentation License".

% This distribution includes a file named fdl.tex that contains the text
% of the GNU Free Documentation License.  If it is missing, you can obtain
% it from www.gnu.org or by writing to the Free Software Foundation,
% Inc., 59 Temple Place - Suite 330, Boston, MA 02111-1307, USA.
%
\chapter{Debugging}
\index{debugging}

Different kinds of errors can occur
in a program, and it is useful to distinguish among them
in order to track them down more quickly:

\begin{itemize}

\item Syntax errors are produced by Python when it is
translating the source code into byte code.  They usually
indicate that there is something wrong with the syntax of the program.
Example: Omitting the colon at the end of a {\tt def} statement yields
the somewhat redundant message {\tt SyntaxError: invalid syntax}.

\item Runtime errors are produced by the runtime system if something
goes wrong while the program is running.  Most runtime error messages
include information about where the error occurred and what functions
were executing.
Example: An infinite recursion eventually causes
a runtime error of ``maximum recursion depth exceeded.''

\item Semantic errors are problems with a program that compiles and
runs but doesn't do the right thing.  Example: An expression may
not be evaluated in the order you expect, yielding an unexpected
result.

\end{itemize}

\index{compile-time error}
\index{syntax error}
\index{runtime error}
\index{semantic error}
\index{error!compile-time}
\index{error!syntax}
\index{error!runtime}
\index{error!semantic}
\index{exception}

The first step in debugging is to figure out which kind of
error you are dealing with.  Although the following sections are
organized by error type, some techniques are
applicable in more than one situation.


\section{Syntax errors}

\index{error messages}
\index{compiler}

Syntax errors are usually easy to fix once you figure out what they
are.  Unfortunately, the error messages are often not helpful.
The most common messages are {\tt SyntaxError: invalid syntax} and
{\tt SyntaxError: invalid token}, neither of which is very informative.

On the other hand, the message does tell you where in the program the
problem occurred.  Actually, it tells you where Python
noticed a problem, which is not necessarily where the error
is.  Sometimes the error is prior to the location of the error
message, often on the preceding line.

\index{incremental program development}

If you are building the program incrementally, you should have
a good idea about where the error is.  It will be in the last
line you added.

If you are copying code from a book, start by comparing
your code to the book's code very carefully.  Check every character.
At the same time, remember that the book might be wrong, so
if you see something that looks like a syntax error, it might be.

Here are some ways to avoid the most common syntax errors:

\index{syntax}

\begin{enumerate}

\item Make sure you are not using a Python keyword for a variable name.

\item Check that you have a colon at the end of the header of every
compound statement, including {\tt for}, {\tt while},
{\tt if}, and {\tt def} statements.

\item Check that indentation is consistent.  You may indent with either
spaces or tabs but it's best not to mix them.  Each level should be
nested the same amount.

\item Make sure that any strings in the code have matching
quotation marks.

\item If you have multiline strings with triple quotes (single or double), make
sure you have terminated the string properly.  An unterminated string
may cause an {\tt invalid token} error at the end of your program,
or it may treat the following part of the program as a string until it
comes to the next string.  In the second case, it might not produce an error
message at all!

\item An unclosed bracket---\verb+(+, \verb+{+, or \verb+[+---makes
Python continue with the next line as part of the current statement.
Generally, an error occurs almost immediately in the next line.

\item Check for the classic {\tt =} instead of {\tt ==} inside
a conditional.

\end{enumerate}

If nothing works, move on to the next section...


\subsection{I can't get my program to run no matter
what I do.}

If the compiler says there is an error and you don't see it, that
might be because you and the compiler are not looking at the same
code.  Check your programming environment to make sure that the
program you are editing is the one Python is trying to run.  If you
are not sure, try putting an obvious and deliberate syntax error at
the beginning of the program.  Now run (or import) it again.  If the
compiler doesn't find the new error, there is probably something wrong
with the way your environment is set up.  

If this happens, one approach is to start again with a new
program like ``Hello, World!,'' and make sure you can get a known
program to run.  Then gradually add the pieces of the new program
to the working one.



\section{Runtime errors}

Once your program is syntactically correct,
Python can import it and at least start running it.  What could
possibly go wrong?


\subsection{My program does absolutely nothing.}

This problem is most common when your file consists of functions and classes
but does not actually invoke anything to start execution.  This may be
intentional if you only plan to import this module to supply classes
and functions.

If it is not intentional, make sure that you
are invoking a function to start execution, or execute one from
the interactive prompt.  Also see the ``Flow of Execution'' section
below.


\subsection{My program hangs.}
\index{infinite loop}
\index{infinite recursion}
\index{hanging}

If a program stops and seems to be doing nothing, we
say it is ``hanging.''  Often that means that it is caught in
an infinite loop or an infinite recursion.

\begin{itemize}

\item If there is a particular loop that you suspect is the
problem, add a {\tt print} statement immediately before the loop that says
``entering the loop'' and another immediately after that says
``exiting the loop.''

Run the program.  If you get the first message and not the second,
you've got an infinite loop.  Go to the ``Infinite Loop'' section
below.

\item Most of the time, an infinite recursion will cause the program
to run for a while and then produce a ``RuntimeError: Maximum
recursion depth exceeded'' error.  If that happens, go to the
``Infinite Recursion'' section below.

If you are not getting this error but you suspect there is a problem
with a recursive method or function, you can still use the techniques
in the ``Infinite Recursion'' section.

\item If neither of those steps works, start testing other
loops and other recursive functions and methods.

\item If that doesn't work, then it is possible that
you don't understand the flow of execution in your program.
Go to the ``Flow of Execution'' section below.

\end{itemize}


\subsubsection{Infinite Loop}
\index{infinite loop}
\index{loop!infinite}
\index{condition}
\index{loop!condition}

If you think you have an infinite loop and you think you know
what loop is causing the problem, add a {\tt print} statement at
the end of the loop that prints the values of the variables in
the condition and the value of the condition.

For example:

\beforeverb
\begin{verbatim}
while x > 0 and y < 0 :
  # do something to x
  # do something to y

  print  "x: ", x
  print  "y: ", y
  print  "condition: ", (x > 0 and y < 0)
\end{verbatim}
\afterverb
%
Now when you run the program, you will see three lines of output
for each time through the loop.  The last time through the
loop, the condition should be {\tt false}.  If the loop keeps
going, you will be able to see the values of {\tt x} and {\tt y},
and you might figure out why they are not being updated correctly.


\subsubsection{Infinite Recursion}
\index{infinite recursion}
\index{recursion!infinite}

Most of the time, an infinite recursion will cause the program to run
for a while and then produce a {\tt Maximum recursion depth exceeded}
error.

If you suspect that a function or method is causing an infinite
recursion, start by checking to make sure that there is a base case.
In other words, there should be some condition that will cause the
function or method to return without making a recursive invocation.
If not, then you need to rethink the algorithm and identify a base
case.

If there is a base case but the program doesn't seem to be reaching
it, add a {\tt print} statement at the beginning of the function or method
that prints the parameters.  Now when you run the program, you will see
a few lines of output every time the function or method is invoked,
and you will see the parameters.  If the parameters are not moving
toward the base case, you will get some ideas about why not.


\subsubsection{Flow of Execution}
\index{flow of execution}
\index{execution!flow}

If you are not sure how the flow of execution is moving through
your program, add {\tt print} statements to the beginning of each
function with a message like ``entering function {\tt foo},'' where
{\tt foo} is the name of the function.

Now when you run the program, it will print a trace of each
function as it is invoked.


\subsection{When I run the program I get an exception.}
\index{exception}
\index{runtime error}

If something goes wrong during runtime, Python
prints a message that includes the name of the
exception, the line of the program where the problem occurred,
and a traceback.

\index{traceback}

The traceback identifies the function that is currently running,
and then the function that invoked it, and then the function that
invoked {\em that}, and so on.  In other words, it traces the
path of function invocations that got you to where you are.  It
also includes the line number in your file where each of these
calls occurs.

The first step is to examine the place in the program where
the error occurred and see if you can figure out what happened.
These are some of the most common runtime errors:

\begin{description}

\item[NameError:]  You are trying to use a variable that doesn't
exist in the current environment.
Remember that local variables are local.  You
cannot refer to them from outside the function where they are defined.

\index{NameError}
\index{TypeError}

\item[TypeError:] There are several possible causes:

\begin{itemize}

\item  You are trying to use a value improperly.  Example: indexing
a string, list, or tuple with something other than an integer.

\index{index}

\item There is a mismatch between the items in a format string and
the items passed for conversion.  This can happen if either the number
of items does not match or an invalid conversion is called for.

\index{format operator}
\index{operator!format}

\item You are passing the wrong number of arguments to a function or method.
For methods, look at the method definition and
check that the first parameter is {\tt self}.  Then look at the
method invocation; make sure you are invoking the method on an
object with the right type and providing the other arguments
correctly.

\end{itemize}

\item[KeyError:]  You are trying to access an element of a dictionary
using a key value that the dictionary does not contain.

\index{KeyError}
\index{dictionary}

\item[AttributeError:] You are trying to access an attribute or method
that does not exist.

\index{AttributeError}

\item[IndexError:] The index you are using
to access a list, string, or tuple is greater than
its length minus one.  Immediately before the site of the error,
add a {\tt print} statement to display
the value of the index and the length of the array.
Is the array the right size?  Is the index the right value?

\index{IndexError}

\end{description}


\subsection{I added so many {\tt print} statements I get inundated with
output.}
\index{print statement}
\index{statement!print}

One of the problems with using {\tt print} statements for debugging
is that you can end up buried in output.  There are two ways
to proceed: simplify the output or simplify the program.

To simplify the output, you can remove or comment out {\tt print}
statements that aren't helping, or combine them, or format
the output so it is easier to understand.

To simplify the program, there are several things you can do.  First,
scale down the problem the program is working on.  For example, if you
are sorting an array, sort a {\em small} array.  If the program takes
input from the user, give it the simplest input that causes the
problem.

Second, clean up the program.  Remove dead code and reorganize the
program to make it as easy to read as possible.  For example, if you
suspect that the problem is in a deeply nested part of the program,
try rewriting that part with simpler structure.  If you suspect a
large function, try splitting it into smaller functions and testing them
separately.

Often the process of finding the minimal test case leads you
to the bug.  If you find that a program works
in one situation but not in another,
that gives you a clue about what is going on.

Similarly, rewriting a piece of code can help you find subtle
bugs.  If you make a change that you think doesn't affect the
program, and it does, that can tip you off.


\section{Semantic errors}
\index{semantic error}
\index{error!semantic}

In some ways, semantic errors are the hardest to debug,
because the compiler and the runtime system provide no information
about what is wrong.  Only you know what the program is supposed to
do, and only you know that it isn't doing it.

The first step is to make a connection between the program
text and the behavior you are seeing.  You need a hypothesis
about what the program is actually doing.  One of the things
that makes that hard is that computers run so fast.

You will often wish that you could slow the program down to human
speed, and with some debuggers you can.  But the time it takes to
insert a few well-placed {\tt print} statements is often short compared to
setting up the debugger, inserting and removing breakpoints, and
``walking'' the program to where the error is occurring.

\subsection{My program doesn't work.}

You should ask yourself these questions:

\begin{itemize}

\item Is there something the program was supposed to do but
which doesn't seem to be happening?  Find the section of the code
that performs that function and make sure it is executing when
you think it should.

\item Is something happening that shouldn't?  Find code in
your program that performs that function and see if it is
executing when it shouldn't.

\item Is a section of code producing an effect that is not
what you expected?  Make sure that you understand the code in
question, especially if it involves invocations to functions or methods in
other Python modules.  Read the documentation for the functions you invoke.
Try them out by writing simple test cases and checking the results.

\end{itemize}

In order to program, you need to have a mental model of how
programs work.  If you write a program that doesn't do what you expect,
very often the problem is not in the program; it's in your mental
model.

\index{model!mental}
\index{mental model}

The best way to correct your mental model is to break the program
into its components (usually the functions and methods) and test
each component independently.  Once you find the discrepancy
between your model and reality, you can solve the problem.

Of course, you should be building and testing components as you
develop the program.  If you encounter a problem,
there should be only a small amount of new code
that is not known to be correct.


\subsection{I've got a big hairy expression and it doesn't
do what I expect.}
\index{expression!big and hairy}

Writing complex expressions is fine as long as they are readable,
but they can be hard to debug.  It is often a good idea to
break a complex expression into a series of assignments to
temporary variables.

For example:

\beforeverb
\begin{verbatim}
self.hands[i].addCard (self.hands[self.findNeighbor(i)].popCard())
\end{verbatim}
\afterverb
%
This can be rewritten as:

\beforeverb
\begin{verbatim}
neighbor = self.findNeighbor (i)
pickedCard = self.hands[neighbor].popCard()
self.hands[i].addCard (pickedCard)
\end{verbatim}
\afterverb
%
The explicit version is easier to read because the variable
names provide additional documentation, and it is easier to debug
because you can check the types of the intermediate variables
and display their values.

\index{temporary variable}
\index{variable!temporary}
\index{order of evaluation}
\index{precedence}

Another problem that can occur with big expressions is
that the order of evaluation may not be what you expect.
For example, if you are translating the expression
$\frac{x}{2 \pi}$ into Python, you might write:

\beforeverb
\begin{verbatim}
y = x / 2 * math.pi
\end{verbatim}
\afterverb
%
That is not correct because multiplication and division have
the same precedence and are evaluated from left to right.
So this expression computes $x \pi / 2$.

A good way to debug expressions is to add parentheses to make
the order of evaluation explicit:

\beforeverb
\begin{verbatim}
 y = x / (2 * math.pi)
\end{verbatim}
\afterverb
%
Whenever you are not sure of the order of evaluation, use
parentheses.  Not only will the program be correct (in the sense
of doing what you intended), it will also be more readable for
other people who haven't memorized the rules of precedence.


\subsection{I've got a function or method that doesn't return what I
expect.}
\index{return statement}
\index{statement!return}

If you have a {\tt return} statement with a complex expression,
you don't have a chance to print the {\tt return} value before
returning.  Again, you can use a temporary variable.  For
example, instead of:

\beforeverb
\begin{verbatim}
return self.hands[i].removeMatches()
\end{verbatim}
\afterverb
%
you could write:

\beforeverb
\begin{verbatim}
count = self.hands[i].removeMatches()
return count
\end{verbatim}
\afterverb
%
Now you have the opportunity to display the value of
{\tt count} before returning.


\subsection{I'm really, really stuck and I need help.}

First, try getting away from the computer for a few minutes.
Computers emit waves that affect the brain, causing these
effects:

\begin{itemize}

\item Frustration and/or rage.

\item Superstitious beliefs (``the computer hates me'') and
magical thinking (``the program only works when I wear my
hat backward'').

\item Random-walk programming (the attempt to program by writing
every possible program and choosing the one that does the right
thing).

\end{itemize}

If you find yourself suffering from any of these symptoms, get
up and go for a walk.  When you are calm, think about the program.
What is it doing?  What are some possible causes of that
behavior?  When was the last time you had a working program,
and what did you do next?

Sometimes it just takes time to find a bug.  We often find bugs
when we are away from the computer and let our minds wander.  Some
of the best places to find bugs are trains, showers, and in bed,
just before you fall asleep.


\subsection{No, I really need help.}

It happens.  Even the best programmers occasionally get stuck.
Sometimes you work on a program so long that you can't see the
error.  A fresh pair of eyes is just the thing.

Before you bring someone else in, make sure you have exhausted
the techniques described here.  Your program should be as simple
as possible, and you should be working on the smallest input
that causes the error.  You should have {\tt print} statements in the
appropriate places (and the output they produce should be
comprehensible).  You should understand the problem well enough
to describe it concisely.

When you bring someone in to help, be sure to give
them the information they need:

\begin{itemize}

\item If there is an error message, what is it
and what part of the program does it indicate?

\item What was the last thing you did before this error occurred?
What were the last lines of code that you wrote, or what is
the new test case that fails?

\item What have you tried so far, and what have you learned?

\end{itemize}

When you find the bug, take a second to think about what you
could have done to find it faster.  Next time you see something
similar, you will be able to find the bug more quickly.

Remember, the goal is not just to make the program
work.  The goal is to learn how to make the program work.

\clearemptydoublepage
% LaTeX source for textbook ``How to think like a computer scientist''
% Copyright (c)  2001  Allen B. Downey, Jeffrey Elkner, and Chris Meyers.

% Permission is granted to copy, distribute and/or modify this
% document under the terms of the GNU Free Documentation License,
% Version 1.1  or any later version published by the Free Software
% Foundation; with the Invariant Sections being "Contributor List",
% with no Front-Cover Texts, and with no Back-Cover Texts. A copy of
% the license is included in the section entitled "GNU Free
% Documentation License".

% This distribution includes a file named fdl.tex that contains the text
% of the GNU Free Documentation License.  If it is missing, you can obtain
% it from www.gnu.org or by writing to the Free Software Foundation,
% Inc., 59 Temple Place - Suite 330, Boston, MA 02111-1307, USA.
%

\chapter{Creating a new data type}
\label{overloading}
\index{data type!user-defined}

Object-oriented programming languages allow programmers to create new
data types that behave much like built-in data types.  We will explore
this capability by building a {\tt Fraction} class that works very much
like the built-in numeric types: integers, longs and floats.

Fractions, also known as rational numbers, are values that can be expressed
as a ratio of whole numbers, such as $5/6$. The top number is
called the numerator and the bottom number is called the denominator.

\index{rational}
\index{fraction}
\index{numerator}
\index{denominator}

We start by defining a {\tt Fraction} class with an initialization
method that provides the numerator and denominator as integers:

\adjustpage{-2}
\pagebreak

\beforeverb
\begin{verbatim}
class Fraction:
  def __init__(self, numerator, denominator=1):
    self.numerator = numerator
    self.denominator = denominator
\end{verbatim}
\afterverb
%
The denominator is optional.  A Fraction with just one
parameter represents a whole number.  If the numerator
is $n$, we build the Fraction
$n/1$.

The next step is to write a {\tt \_\_str\_\_} method that
displays fractions in a way that makes sense.  The form
``numerator/denominator'' is natural here:

\beforeverb
\begin{verbatim}
class Fraction:
  ...
  def __str__(self):
    return "%d/%d" % (self.numerator, self.denominator)
\end{verbatim}
\afterverb
%
To test what we have so far, we put it in a file named
{\tt Fraction.py} and import it into the Python interpreter.
Then we create a fraction object and print it.

\beforeverb
\begin{verbatim}
>>> from Fraction import Fraction
>>> spam = Fraction(5,6)
>>> print "The fraction is", spam
The fraction is 5/6
\end{verbatim}
\afterverb
%
As usual, the {\tt print} command invokes the {\tt \_\_str\_\_}
method implicitly.


\section {Fraction multiplication}
\index{multiplication!fraction}
\index{fraction!multiplication}

We would like to be able to apply the normal addition, subtraction,
multiplication, and division operations to fractions.  To do this, we
can overload the mathematical operators for {\tt Fraction} objects.

\index{overload}
\index{operator!overloading}
\index{mathematical operator}

We'll start with multiplication because it is the easiest to implement.
To multiply fractions, we create a new fraction with a numerator
that is the product of the original numerators and a denominator that
is a product of the original denominators.  {\tt \_\_mul\_\_} is the
name Python uses for a method that overloads the {\tt *} operator:

\beforeverb
\begin{verbatim}
class Fraction:
  ...
  def __mul__(self, other):
    return Fraction(self.numerator*other.numerator,
                    self.denominator*other.denominator)
\end{verbatim}
\afterverb
%
We can test this method by computing the product of two fractions:

\beforeverb
\begin{verbatim}
>>> print Fraction(5,6) * Fraction(3,4)
15/24
\end{verbatim}
\afterverb
%
It works, but we can do better!  We can extend the method to
handle multiplication by an integer.  We use the {\tt isinstance} function
to test if {\tt other} is an integer and convert it to a fraction if
it is.

\beforeverb
\begin{verbatim}
class Fraction:
  ...
  def __mul__(self, other):
    if isinstance(other, int):
      other = Fraction(other)
    return Fraction(self.numerator   * other.numerator,
                    self.denominator * other.denominator)
\end{verbatim}
\afterverb
%
Multiplying fractions and integers now works, but only if the fraction
is the left operand:

\beforeverb
\begin{verbatim}
>>> print Fraction(5,6) * 4
20/6
>>> print 4 * Fraction(5,6)
TypeError: __mul__ nor __rmul__ defined for these operands
\end{verbatim}
\afterverb
%
To evaluate a binary operator like multiplication, Python checks
the left operand first to see if it provides a {\tt \_\_mul\_\_}
that supports the type of the second operand.  In this case,
the built-in integer operator doesn't support fractions.

Next, Python checks the right operand to see if it provides
an {\tt \_\_rmul\_\_} method that supports the first type.  In
this case, we haven't provided {\tt \_\_rmul\_\_}, so it fails.

On the other hand, there is a simple way to provide
{\tt \_\_rmul\_\_}:

\beforeverb
\begin{verbatim}
class Fraction:
  ...
  __rmul__ = __mul__
\end{verbatim}
\afterverb
%
This assignment says that the {\tt \_\_rmul\_\_} is the same
as {\tt \_\_mul\_\_}.
Now if we evaluate {\tt 4 * Fraction(5,6)}, Python invokes
{\tt \_\_rmul\_\_} on the {\tt Fraction} object and passes 4
as a parameter:

\beforeverb
\begin{verbatim}
>>> print 4 * Fraction(5,6)
20/6
\end{verbatim}
\afterverb
%
Since {\tt \_\_rmul\_\_} is the same as {\tt \_\_mul\_\_}, and
{\tt \_\_mul\_\_} can handle an integer parameter, we're all set.


\section{Fraction addition}
\index{addition!fraction}
\index{fraction!addition}

Addition is more complicated than multiplication, but still not too
bad.  The sum of $a/b$ and $c/d$ is the fraction
{\tt (a*d+c*b)/(b*d)}.

Using the multiplication code as a model, we can write
{\tt \_\_add\_\_} and {\tt \_\_radd\_\_}:

\beforeverb
\begin{verbatim}
class Fraction:
  ...
  def __add__(self, other):
    if isinstance(other, int):
      other = Fraction(other)
    return Fraction(self.numerator   * other.denominator +
                    self.denominator * other.numerator,
                    self.denominator * other.denominator)

  __radd__ = __add__
\end{verbatim}
\afterverb
%
We can test these methods with {\tt Fraction}s and integers.

\beforeverb
\begin{verbatim}
>>> print Fraction(5,6) + Fraction(5,6)
60/36
>>> print Fraction(5,6) + 3
23/6
>>> print 2 + Fraction(5,6)
17/6
\end{verbatim}
\afterverb
%
The first two examples invoke {\tt \_\_add\_\_}; the last
invokes {\tt \_\_radd\_\_}.


\section{Euclid's algorithm}
\index{greatest common divisor}
\index{Euclid}
\index{pseudocode}
\index{reduce}

In the previous example, we computed the sum $5/6 + 5/6$ and got
$60/36$.  That is correct, but it's not the best way to represent the
answer.  To {\bf reduce} the fraction to its simplest terms, we have
to divide the numerator and denominator by their {\bf greatest common
divisor (GCD)}, which is 12.  The result is $5/3$.

In general, whenever we create a new {\tt Fraction} object, we should
reduce it by dividing the numerator and denominator by their GCD.  If
the fraction is already reduced, the GCD is 1.

Euclid of Alexandria (approx. 325--265 BCE) presented an algorithm
to find the GCD for two integers $m$ and $n$:

\begin{quote}
If $n$ divides $m$ evenly, then $n$ is the GCD.  Otherwise
the GCD is the GCD of $n$ and the remainder of $m$ divided by $n$.
\end{quote}

This recursive definition can be expressed concisely as a function:

\beforeverb
\begin{verbatim}
def gcd (m, n):
  if m % n == 0:
    return n
  else:
    return gcd(n, m%n)
\end{verbatim}
\afterverb
%
In the first line of the body, we use the modulus operator to
check divisibility.  On the last line, we use it to compute
the remainder after division.

Since all the operations we've written
create new {\tt Fraction}s for the result, we can reduce all results
by modifying the initialization method.

\beforeverb
\begin{verbatim}
class Fraction:
  def __init__(self, numerator, denominator=1):
    g = gcd (numerator, denominator)
    self.numerator   =   numerator / g
    self.denominator = denominator / g
\end{verbatim}
\afterverb
%
Now whenever we create a {\tt Fraction}, it is reduced to its simplest
form:

\beforeverb
\begin{verbatim}
>>> Fraction(100,-36)
-25/9
\end{verbatim}
\afterverb
%
A nice feature of {\tt gcd} is that if the fraction is
negative, the minus sign is always moved to the numerator.


\section{Comparing fractions}
\index{comparison!fraction}
\index{fraction!comparison}

Suppose we have two {\tt Fraction} objects, {\tt a} and {\tt b}, and we
evaluate {\tt a == b}.  The default implementation of {\tt ==}
tests for shallow equality, so it only returns true if {\tt a}
and {\tt b} are the same object.

More likely, we want to return true if $a$ and $b$ have
the same value---that is, deep equality.

We have to teach fractions how to compare themselves.  As we saw in
Section~\ref{comparecard}, we can overload all the comparison
operators at once by supplying a {\tt \_\_cmp\_\_} method.

By convention, the {\tt \_\_cmp\_\_} method returns a
negative number if {\tt self} is less than {\tt other}, zero
if they are the same, and a positive number if {\tt self} is greater
than {\tt other}.

The simplest way to compare fractions is to cross-multiply.
If $a/b > c/d$, then $ad > bc$.
With that in mind, here is the code for {\tt \_\_cmp\_\_}:

\beforeverb
\begin{verbatim}
class Fraction:
  ...
  def __cmp__(self, other):
    diff = (self.numerator  * other.denominator -
            other.numerator * self.denominator)
    return diff
\end{verbatim}
\afterverb
%
If {\tt self} is greater than {\tt other}, then {\tt diff}
will be positive.  If {\tt other} is greater, then {\tt diff}
will be negative.  If they are the same, {\tt diff} is zero.


\section {Taking it further}

Of course, we are not done.  We still have to implement
subtraction by overriding {\tt \_\_sub\_\_} and division
by overriding {\tt \_\_div\_\_}.

One way to handle those operations is to implement negation
by overriding
{\tt \_\_neg\_\_} and inversion by overriding {\tt \_\_invert\_\_}.
Then we can subtract by negating the second operand and adding,
and we can divide by inverting the second operand and
multiplying.

Next, we have to provide {\tt \_\_rsub\_\_} and {\tt \_\_rdiv\_\_}.
Unfortunately, we can't use the same trick we used for addition and
multiplication, because subtraction and division are not commutative.
We can't just set {\tt \_\_rsub\_\_} and {\tt \_\_rdiv\_\_} equal to
{\tt \_\_sub\_\_} and {\tt \_\_div\_\_}.  In these operations, the
order of the operands makes a difference.

To handle {\bf unary negation}, which is the use of the minus
sign with a single operand, we override {\tt \_\_neg\_\_}.

\index{unary operator}
\index{negation}

We can compute powers by overriding {\tt \_\_pow\_\_},
but the implementation is a little tricky.  If the exponent isn't
an integer, then it may not be possible to represent the result
as a {\tt Fraction}.  For example, {\tt Fraction(2) ** Fraction(1,2)}
is the square root of 2, which is an irrational number (it can't
be represented as a fraction).
So it's not easy to write the most general version of {\tt \_\_pow\_\_}.

\index{irrational}

There is one other extension to the {\tt Fraction} class that you might
want to think about.  So far, we have assumed that the numerator
and denominator are integers.

\begin{quote}
{\em As an exercise, finish the implementation of the {\tt Fraction}
class so that it handles subtraction, division and exponentiation.}
\end{quote}



\section{Glossary}

\begin{description}

\item[greatest common divisor (GCD):] The largest positive integer
that divides without a remainder into both the numerator and denominator
of a fraction.

\item[reduce:] To change a fraction into an equivalent form with a
GCD of 1.

\item[unary negation:] The operation that computes an additive
inverse, usually denoted with a leading minus sign.  Called 
``unary'' by contrast with the binary minus operation, which is
subtraction.


\index{greatest common divisor}
\index{reduce}
\index{unary negation}

\end{description}

\clearemptydoublepage
%% LaTeX source for textbook ``How to think like a computer scientist''
% Copyright (c)  2001  Allen B. Downey, Jeffrey Elkner, and Chris Meyers.

% Permission is granted to copy, distribute and/or modify this
% document under the terms of the GNU Free Documentation License,
% Version 1.1  or any later version published by the Free Software
% Foundation; with the Invariant Sections being "Contributor List",
% with no Front-Cover Texts, and with no Back-Cover Texts. A copy of
% the license is included in the section entitled "GNU Free
% Documentation License".

% This distribution includes a file named fdl.tex that contains the text
% of the GNU Free Documentation License.  If it is missing, you can obtain
% it from www.gnu.org or by writing to the Free Software Foundation,
% Inc., 59 Temple Place - Suite 330, Boston, MA 02111-1307, USA.
%
\chapter{Complete Python Listings}


\section{Point class}

\beforeverb
\begin{verbatim}
class Point:
  def __init__(self, x=0, y=0):
    self.x = x
    self.y = y

  def __str__(self):
    return '(' + str(self.x) + ', ' + str(self.y) + ')'

  def __add__(self, other):
    return Point(self.x + other.x, self.y + other.y)

  def __sub__(self, other):
    return Point(self.x - other.x, self.y - other.y)

  def __mul__(self, other):
    return self.x * other.x + self.y * other.y

  def __rmul__(self, other):
    return Point(other * self.x, other * self.y)

  def reverse(self):
    self.x, self.y = self.y, self.x

  def frontAndBack(front):
    from copy import copy
    back = copy(front)
    back.reverse()
    print str(front) + str(back)
\end{verbatim}
\afterverb


\section{Time class}

\beforeverb
\begin{verbatim}
class Time:
  def __init__(self, hours=0, minutes=0, seconds=0):
    self.hours = hours
    self.minutes = minutes
    self.seconds = seconds

  def __str__(self):
    return str(self.hours) + ":" + str(self.minutes) + ":" + str(self.seconds)

  def convertToSeconds(self):
    minutes = self.hours * 60 + self.minutes
    seconds = self.minutes * 60 + self.seconds
    return seconds

  def increment(self, secs):
    secs = secs + self.seconds

    self.hours = self.hours + secs/3600
    secs = secs % 3600
    self.minutes = self.minutes + secs/60
    secs = secs % 60
    self.seconds = secs

  def makeTime(secs):
    time = Time()
    time.hours = secs/3600
    secs = secs - time.hours * 3600
    time.minutes = secs/60
    secs = secs - time.minutes * 60
    time.seconds = secs
    return time
\end{verbatim}
\afterverb


\section {Cards, decks and games}

\beforeverb
\begin{verbatim}
import random

class Card:
  suitList = ["Clubs", "Diamonds", "Hearts", "Spades"]
  rankList = [ "narf", "Ace", "2", "3", "4", "5", "6", "7", "8", "9", "10",
                "Jack", "Queen", "King"]

  def __init__(self, suit=0, rank=0):
    self.suit = suit
    self.rank = rank

  def __str__(self):
    return self.rankList[self.rank] + " of " + self.suitList[self.suit]

  def __cmp__(self, other):
    # check the suits
    if self.suit > other.suit: return 1
    if self.suit < other.suit: return -1
    # suits are the same... check ranks
    if self.rank > other.rank: return 1
    if self.rank < other.rank: return -1
    # ranks are the same... it's a tie
    return 0

class Deck:
  def __init__(self):
    self.cards = []
    for suit in range(4):
      for rank in range(1, 14):
        self.cards.append(Card(suit, rank))

  def printDeck(self):
    for card in self.cards:
      print card

  def __str__(self):
    s = ""
    for i in range(len(self.cards)):
      s = s + " "*i + str(self.cards[i]) + "\n"
    return s

  def shuffle(self):
    import random
    nCards = len(self.cards)
    for i in range(nCards):
      j = random.randrange(i, nCards)
      [self.cards[i], self.cards[j]] = [self.cards[j], self.cards[i]]

  def removeCard(self, card):
    if card in self.cards:
      self.cards.remove(card)
      return 1
    else: return 0

  def popCard(self):
    return self.cards.pop()

  def isEmpty(self):
    return (len(self.cards) == 0)

  def deal(self, hands, nCards=999):
    nHands = len(hands)
    for i in range(nCards):
      if self.isEmpty(): break    # break if out of cards
      card = self.popCard()      # take the top card
      hand = hands[i % nHands]    # whose turn is next?
      hand.addCard(card)         # add the card to the hand

class Hand(Deck):
  def __init__(self, name=""):
    self.cards = []
    self.name = name

  def addCard(self,card) :
    self.cards.append(card)

  def __str__(self):
    s = "Hand " + self.name
    if self.isEmpty():
      s = s + " is empty\n"
    else:
      s = s + " contains\n"
    return s + Deck.__str__(self)

class CardGame:
  def __init__(self):
    self.deck = Deck()
    self.deck.shuffle()

class OldMaidHand(Hand):
  def removeMatches(self):
    count = 0
    originalCards = self.cards[:]
    for card in originalCards:
      match = Card(3 - card.suit, card.rank)
      if match in self.cards:
        self.cards.remove(card)
        self.cards.remove(match)
        print "Hand %s: %s matches %s" % (self.name,card,match)
        count = count+1
    return count

class OldMaidGame(CardGame):
  def play(self, names):
    # remove Queen of Clubs
    self.deck.removeCard(Card(0,12))

    # make hands base on names passed
    self.hands = []
    for name in names : self.hands.append(OldMaidHand(name))

    # deal the cards
    self.deck.deal(self.hands)
    print "---------- Cards have been dealt"
    self.printHands()

    # Remove initial matches
    matches = self.removeMatches()
    print "---------- Matches discarded, play begins"
    self.printHands()

    # Play until all 50 cards matched
    turn = 0
    numHands = len(self.hands)
    while matches < 25:
      matches = matches + self.playOneTurn(turn)
      turn = (turn + 1) % numHands

    print "---------- Game is Over"
    self.printHands ()

  def removeMatches(self):
    count = 0
    for hand in self.hands:
      count = count + hand.removeMatches()
    return count

  def playOneTurn(self, i):
    if self.hands[i].isEmpty():
      return 0
    neighbor = self.findNeighbor(i)
    pickedCard = self.hands[neighbor].popCard()
    self.hands[i].addCard(pickedCard)
    print "Hand", self.hands[i].name, "picked", pickedCard
    count = self.hands[i].removeMatches()
    self.hands[i].shuffle()
    return count

  def findNeighbor(self, i):
    numHands = len(self.hands)
    for next in range(1,numHands):
      neighbor = (i + next) % numHands
      if not self.hands[neighbor].isEmpty():
        return neighbor

  def printHands(self) :
    for hand in self.hands :
        print hand

\end{verbatim}
\afterverb


\section{Linked Lists}

\beforeverb
\begin{verbatim}
  def printList(node) :
    while node :
      print node,
      node = node.next
    print
  
  def printBackward(list) :
    if list == None : return
    head = list
    tail = list.next
    printBackward(tail)
    print head,
  
  def printBackwardNicely(list) :
    print "(",
    if list != None :
      head = list
      tail = list.next
      printBackward(tail)
      print head,
    print ")",
  
  def removeSecond(list) :
    if list == None : return
    first  = list
    second = list.next
    first.next = second.next
    second.next = None
    return second

class Node :

  def __init__(self, cargo=None) :
    self.cargo = cargo
    self.next  = None

  def __str__(self) :
    return str(self.cargo)

  def printBackward(self) :
    if self.next != None :
      tail = self.next
      tail.printBackward()
    print self.cargo,

class LinkedList :
  def __init__(self) :
    self.length = 0
    self.head   = None

  def printBackward(self) :
    print "(",
    if self.head != None :
      self.head.printBackward()
    print ")",

  def addFirst(self, cargo) :
    node = Node(cargo)
    node.next = self.head
    self.head = node
    self.length = self.length + 1
\end{verbatim}
\afterverb

\section{Stack class}

\beforeverb
\begin{verbatim}

class Stack:              # Python list implementation
  def __init__(self):
    self.items = []

  def push(self, item):
    self.items.append(item)

  def pop(self):
    return self.items.pop()

  def isEmpty(self):
    return(self.items == [])

  def evalPostfix(expr) :
    import re
    expr = re.split("([^0-9])", expr)
    stack = Stack()
    for token in expr :
      if  token == '' or token == ' ':
        continue
      if  token == '+' :
        sum = stack.pop() + stack.pop()
        stack.push(sum)
      elif token == '*' :
        product = stack.pop() * stack.pop()
        stack.push(product)
      else :
        stack.push(int(token))
    return stack.pop()

\end{verbatim}
\afterverb

\section {Queues and priority queues}
\beforeverb
\begin{verbatim}

class Queue :
  def __init__(self) :
    self.length = 0
    self.head   = None

  def empty(self) :
    return (self.length == 0)

  def insert(self, cargo) :
    node = Node(cargo)
    node.next = None
    if self.head == None :
        # If list is empty our new node is first
        self.head = node
    else :
        # Find the last node in the list
        last = self.head
        while last.next : last = last.next
        # Append our new node
        last.next = node
    self.length = self.length + 1

  def remove(self) :
    cargo = self.head.cargo
    self.head = self.head.next
    self.length = self.length - 1
    return cargo

class ImprovedQueue :
  def __init__(self) :
    self.length = 0
    self.head   = None
    self.last   = None

  def empty(self) :
    return (self.length == 0)

  def insert(self, cargo) :
    node = Node(cargo)
    node.next = None
    if self.length == 0 :
        # If list is empty our new node is first
        self.head = self.last = node
    else :
        # Find the last node in the list
        last = self.last
        # Append our new node
        last.next = node
        self.last = node
    self.length = self.length + 1

  def remove(self) :
    cargo    = self.head.cargo
    self.head = self.head.next
    self.length = self.length - 1
    if self.length == 0 : self.last = None
    return cargo

class PriorityQueue :
  def __init__(self) :
    self.items = []

  def empty(self) :
    return self.items == []

  def insert(self, item) :
    self.items.append(item)

  def remove(self) :
    maxi = 0
    for i in range(1,len(self.items)) :
       if self.items[i] > self.items[maxi] :
         maxi = i
    item = self.items[maxi]
    self.items[maxi:maxi+1] = []
    return item

class Golfer :
  def __init__(self, name, score) :
    self.name = name
    self.score= score

  def __str__(self) :
    return "%-15s: %d" % (self.name, self.score)

  def __cmp__(self, other) :
    if self.score < other.score : return  1   # less is more
    if self.score > other.score : return -1
    return 0

\end{verbatim}
\afterverb

\section{Trees}
\beforeverb
\begin{verbatim}
class Tree :
  def __init__(self, cargo, left=None, right=None) :
    self.cargo = cargo
    self.left  = left
    self.right = right

  def __str__(self) :
    return str(self.cargo)

  def getCargo(self): return self.cargo
  def getLeft (self): return self.left
  def getRight(self): return self.right

  def setCargo(self, cargo):  self.cargo = cargo
  def setLeft (self,  left):  self.left = left
  def setRight(self, right):  self.right = right

def total(tree) :
  if tree == None : return 0
  return total(tree.left) + total(tree.right) + tree.cargo

def printTree(tree) :
  if tree == None : return
  print tree.cargo,
  printTree(tree.left)
  printTree(tree.right)

def printTreePostorder(tree) :
  if tree == None : return
  printTreePostorder(tree.left)
  printTreePostorder(tree.right)
  print tree.cargo,

def printTreeInorder(tree) :
  if tree == None : return
  printTreeInorder(tree.left)
  print tree.cargo,
  printTreeInorder(tree.right)

def printTreeIndented(tree, level=0) :
  if tree == None : return
  printTreeIndented(tree.right, level+1)
  print '  '*level + str(tree.cargo)
  printTreeIndented(tree.left, level+1)
\end{verbatim}

\section{Expression trees}

\begin{verbatim}
def getToken(tokenList, expected) :
  if tokenList[0] == expected :
    tokenList[0:1] = []   # remove the token
    return 1
  else :
    return 0

def getProduct(tokenList) :
  a = getNumber(tokenList)
  if getToken(tokenList, '*') :
    b = getProduct(tokenList)
    return Tree('*', a, b)
  else :
    return a

def getSum(tokenList) :
  a = getProduct(tokenList)
  if getToken(tokenList, '+') :
    b = getSum(tokenList)
    return Tree('+', a, b)
  else :
    return a

def getNumber(tokenList) :
  if getToken(tokenList, '(') :
    x = getSum(tokenList)      # get subexpression
    getToken(tokenList, ')')    # eat the closing parenthesis
    return x
  else :
    x = tokenList[0]
    if not isinstance(x, int) : return None
    tokenList[0:1] = []   # remove the token
    return Tree(x, None, None)    # return a leaf with the number
\end{verbatim}


\section{Guess the animal}

\begin{verbatim}
def animal() :
  # start with a singleton
  root = Tree("bird")

  # loop until the user quits
  while 1 :
    print
    if not yes("Are you thinking of an animal? ") : break

    # walk the tree
    tree = root
    while tree.getLeft() != None :
      prompt = tree.getCargo() + "? "
      if yes(prompt):
        tree = tree.getRight()
      else:
        tree = tree.getLeft()

    # make a guess
    guess = tree.getCargo()
    prompt = "Is it a " + guess + "? "
    if yes(prompt) :
      print "I rule!"
      continue

    # get new information
    prompt  = "What is the animal\'s name? "
    animal  = raw_input(prompt)
    prompt  = "What question would distinguish a %s from a %s? "
    question = raw_input(prompt % (animal,guess))

    # add new information to the tree
    tree.setCargo(question)
    prompt = "If the animal were %s the answer would be? "
    if yes(prompt % animal) :
      tree.setLeft(Tree(guess))
      tree.setRight(Tree(animal))
    else :
      tree.setLeft(Tree(animal))
      tree.setRight(Tree(guess))


def yes(ques) :
  from string import lower
  ans = lower(raw_input(ques))
  return (ans[0:1] == 'y')
\end{verbatim}

\section{{\tt Fraction} class}

\beforeverb
\begin{verbatim}
class Fraction:
  def __init__(self, numerator, denominator=1):
    g = gcd(numerator, denominator)
    self.numerator   = numerator   / g
    self.denominator = denominator / g

  def __mul__(self, object):
    if isinstance(object, int):
      object = Fraction(object)
    return Fraction(self.numerator*object.numerator,
                 self.denominator*object.denominator)

  __rmul__ = __mul__

  def __add__(self, object):
    if isinstance(object, int):
      object = Fraction(object)

    return Fraction(self.numerator*object.denominator +
                    self.denominator*object.numerator,
                 self.denominator * object.denominator)

  __radd__ = __add__

  def __cmp__(self, object):
    if isinstance(object, int):
      object = Fraction(object)

    diff = (self.numerator*object.denominator -
            object.numerator*self.denominator)
    return diff

  def __repr__(self):
    return self.__str__()

  def __str__(self):
    return "%d/%d" % (self.numerator, self.denominator)

def gcd(m,n):
  "return the greatest common divisor of 2 integer arguments"
  if m % n == 0:
    return n
  else:
    return gcd(n,m%n)

\end{verbatim}
\afterverb

%\clearemptydoublepage
% LaTeX source for textbook ``How to think like a computer scientist''
% Copyright (c)  2001  Allen B. Downey, Jeffrey Elkner, and Chris Meyers.

% Permission is granted to copy, distribute and/or modify this
% document under the terms of the GNU Free Documentation License,
% Version 1.1  or any later version published by the Free Software
% Foundation; with the Invariant Sections being "Contributor List",
% with no Front-Cover Texts, and with no Back-Cover Texts. A copy of
% the license is included in the section entitled "GNU Free
% Documentation License".

% This distribution includes a file named fdl.tex that contains the text
% of the GNU Free Documentation License.  If it is missing, you can obtain
% it from www.gnu.org or by writing to the Free Software Foundation,
% Inc., 59 Temple Place - Suite 330, Boston, MA 02111-1307, USA.
%

\chapter{Recommendations for further reading}

So where do you go from here?  There are many directions to pursue,
extending your knowledge of Python specifically and
computer science in general.

The examples in this book have been deliberately simple, but they
may not have shown off Python's most exciting capabilities.
Here is a sampling of extensions to Python and suggestions for
projects that use them.

\begin{itemize}

\item GUI (graphical user interface) programming lets your
program use a windowing environment to interact with the user and
display graphics.

The oldest graphics package for Python
is Tkinter, which is based on Jon Ousterhout's Tcl and Tk scripting
languages.  Tkinter comes bundled with the Python distribution.

Another
popular platform is wxPython, which is essentially a Python veneer over
wxWindows, a C++ package which in turn implements windows using native
interfaces on Windows and Unix (including Linux) platforms.  The
windows and controls under wxPython tend to have a more native look
and feel than those of Tkinter and are somewhat simpler to
program.

Any type of GUI programming will lead you into event-driven
programming, where the user and not the programmer determines the flow of
execution.  This style of programming takes some getting used to,
sometimes forcing you to rethink the whole structure of a program.

\item Web programming integrates Python with the Internet.
For example, you can build web client programs that open and read
a remote web page (almost) as easily as you can open a file on
disk.  There are also Python modules that let you access remote files
via ftp, and modules to let you send and receive email.  Python is also
widely used for web server programs to handle input forms.

\item Databases are a bit like super files where data is stored in
predefined schemas, and relationships between data items let you access
the data in various ways.  Python has several modules to enable
users to connect to various database engines, both Open Source and
commercial.

\item Thread programming lets you run several threads of execution 
within a single program.  If you have had the experience of using a
web browser to scroll the beginning of a page while the browser
continues to load the rest of it, then you have a feel for what
threads can do.

\item When speed is paramount Python extensions may be written in a
compiled language like C or C++.  Such extensions
form the base of many of the modules in the Python
library. The mechanics of linking functions and data is somewhat
complex.  SWIG (Simplified Wrapper and Interface Generator) is a tool to
make the process much simpler.

\end{itemize}


\section{Python-related web sites and books}

Here are the authors' recommendations for Python resources
on the web:

\begin{itemize}

\item The Python home page at {\tt www.python.org} is the place to start 
your search for any Python related material.  You will find
help, documentation, links to other sites and SIG (Special Interest
Group) mailing lists that you can join.

\item The Open Book Project {\tt www.ibiblio.com/obp} contains not
only this book online but also similar books for Java and C++ by Allen
Downey.  In addition there are {\em
Lessons in Electric Circuits} by Tony R.  Kuphaldt, 
{\em Getting down
with ...}, a set of tutorials on a range of computer science topics,
written and edited by high school students, {\em Python for Fun},
a set of case studies in Python by Chris Meyers, and {\em The Linux
Cookbook} by Michael Stultz, with 300 pages of tips and techniques.

\item Finally if you go to Google and use the search 
string ``python -snake -monty'' you will get about 750,000 hits.

\end{itemize}

\adjustpage{-1}
\pagebreak

And here are some books that contain more material on the
Python language:

\begin{itemize}

\item {\em Core Python Programming} by Wesley Chun is a large book 
at about 750 pages.  The first part of the book covers the basic Python
language features.  The second part provides an easy-paced introduction
to more advanced topics including many of those mentioned above.

\item {\em Python Essential Reference} by David M. Beazley is a small
book, but it is packed with information both on the
language itself and the modules in the standard library.  It is also
very well indexed.

\item {\em Python Pocket Reference} by Mark Lutz really does fit in 
your pocket.  Although not as extensive as {\em Python Essential
Reference} it is a handy reference for the most commonly used functions
and modules.  Mark Lutz is also the author of {\em Programming Python},
one of the earliest (and largest) books on Python and not aimed at the
beginning programmer.  His later book {\em Learning Python} is smaller
and more accessible.

\item {\em Python Programming on Win32} by Mark Hammond and Andy 
Robinson is a ``must have'' for anyone seriously using Python to develop
Windows applications.  Among other things it covers the integration of
Python and COM, builds a small application with wxPython, and even
uses Python to script windows applications such as Word and Excel.

\end{itemize}

\section{Recommended general computer science books}

The following suggestions for further reading include many of the
authors' favorite books.  They deal with good programming practices
and computer science in general.

\begin{itemize}

\item {\em The Practice of Programming} by Kernighan and Pike covers not 
only the design and coding of algorithms and data structures, but also
debugging, testing and improving the performance of programs. The
examples are mostly C++ and Java, with none in Python.

\item {\em The Elements of Java Style} edited by Al Vermeulen is another
small book that discusses some of the finer points of good
programming, such as good use of naming conventions, comments, and
even whitespace and indentation (somewhat of a nonissue in
Python).  The book also covers programming by contract, using
assertions to catch errors by testing preconditions and
postconditions, and proper programming with threads and their
synchronization.

\item {\em Programming Pearls} by Jon Bentley is a classic book.
It consists of case studies that originally appeared in the author's
column in the {\em Communications of the ACM}.  The studies deal with
tradeoffs in programming and why it is often an especially bad idea to
run with your first idea for a program.  The book is a bit older than
those above (1986), so the examples are in older
languages.  There are lots of problems to solve, some with solutions
and others with hints.  This book was very popular and was followed by
a second volume.

\item {\em The New Turing Omnibus} by A.K Dewdney provides a gentle
introduction to 66 topics of computer science ranging from parallel
computing to computer viruses, from cat scans to genetic
algorithms.  All of the topics are short and entertaining.  An earlier
book by Dewdney {\em The Armchair Universe} is a collection from his
column {\em Computer Recreations} in {\em Scientific American}.  Both
books are a rich source of ideas for projects.

\item {\em Turtles, Termites and Traffic Jams} by Mitchel Resnick
is about the power of decentralization and how complex behavior can
arise from coordinated simple activity of a multitude of agents.  It
introduces the language StarLogo, which allows the user to write
programs for the agents.   Running the program
demonstrates complex aggregate behavior, which is often
counterintuitive.  Many of the programs in the book were developed by
students in middle school and high school.  Similar programs could be
written in Python using simple graphics and threads.

\item {\em G\"{o}del, Escher and Bach} by Douglas Hofstadter.  Put simply,
if you found magic in recursion you will also find it in this
bestselling book.  One of Hofstadter's themes involves ``strange loops''
where patterns evolve and ascend until they meet themselves again.  It
is Hofstadter's contention that such ``strange loops'' are an essential
part of what separates the animate from the inanimate.  He
demonstrates such patterns in the music of Bach, the pictures of
Escher and G\"{o}del's incompleteness theorem.

\end{itemize}

\clearemptydoublepage

%% fdl.tex
% This file is a chapter.  It must be included in a larger document to work
% properly.

\chapter{GNU Free Documentation License}

Version 1.1, March 2000\\

  Copyright $\copyright$ 2000  Free Software Foundation, Inc.\\
      59 Temple Place, Suite 330, Boston, MA  02111-1307  USA\\
  Everyone is permitted to copy and distribute verbatim copies
  of this license document, but changing it is not allowed.

\section*{Preamble}

The purpose of this License is to make a manual, textbook, or other
written document ``free'' in the sense of freedom: to assure everyone
the effective freedom to copy and redistribute it, with or without
modifying it, either commercially or noncommercially.  Secondarily,
this License preserves for the author and publisher a way to get
credit for their work, while not being considered responsible for
modifications made by others.

This License is a kind of ``copyleft,'' which means that derivative
works of the document must themselves be free in the same sense.  It
complements the GNU General Public License, which is a copyleft
license designed for free software.

We have designed this License in order to use it for manuals for free
software, because free software needs free documentation: a free
program should come with manuals providing the same freedoms that the
software does.  But this License is not limited to software manuals;
it can be used for any textual work, regardless of subject matter or
whether it is published as a printed book.  We recommend this License
principally for works whose purpose is instruction or reference.

\section{Applicability and Definitions}

This License applies to any manual or other work that contains a
notice placed by the copyright holder saying it can be distributed
under the terms of this License.  The ``Document,'' below, refers to any
such manual or work.  Any member of the public is a licensee, and is
addressed as ``you.''

A ``Modified Version'' of the Document means any work containing the
Document or a portion of it, either copied verbatim, or with
modifications and/or translated into another language.

A ``Secondary Section'' is a named appendix or a front-matter section of
the Document that deals exclusively with the relationship of the
publishers or authors of the Document to the Document's overall subject
(or to related matters) and contains nothing that could fall directly
within that overall subject.  (For example, if the Document is in part a
textbook of mathematics, a Secondary Section may not explain any
mathematics.)  The relationship could be a matter of historical
connection with the subject or with related matters, or of legal,
commercial, philosophical, ethical, or political position regarding
them.

The ``Invariant Sections'' are certain Secondary Sections whose titles
are designated, as being those of Invariant Sections, in the notice
that says that the Document is released under this License.

The ``Cover Texts'' are certain short passages of text that are listed,
as Front-Cover Texts or Back-Cover Texts, in the notice that says that
the Document is released under this License.

A ``Transparent'' copy of the Document means a machine-readable copy,
represented in a format whose specification is available to the
general public, whose contents can be viewed and edited directly and
straightforwardly with generic text editors or (for images composed of
pixels) generic paint programs or (for drawings) some widely available
drawing editor, and that is suitable for input to text formatters or
for automatic translation to a variety of formats suitable for input
to text formatters.  A copy made in an otherwise Transparent file
format whose markup has been designed to thwart or discourage
subsequent modification by readers is not Transparent.  A copy that is
not ``Transparent'' is called ``Opaque.''

Examples of suitable formats for Transparent copies include plain
ASCII without markup, Texinfo input format, \LaTeX~input format, SGML
or XML using a publicly available DTD, and standard-conforming simple
HTML designed for human modification.  Opaque formats include
PostScript, PDF, proprietary formats that can be read and edited only
by proprietary word processors, SGML or XML for which the DTD and/or
processing tools are not generally available, and the
machine-generated HTML produced by some word processors for output
purposes only.

The ``Title Page'' means, for a printed book, the title page itself,
plus such following pages as are needed to hold, legibly, the material
this License requires to appear in the title page.  For works in
formats which do not have any title page as such, ``Title Page'' means
the text near the most prominent appearance of the work's title,
preceding the beginning of the body of the text.


\section{Verbatim Copying}

You may copy and distribute the Document in any medium, either
commercially or noncommercially, provided that this License, the
copyright notices, and the license notice saying this License applies
to the Document are reproduced in all copies, and that you add no other
conditions whatsoever to those of this License.  You may not use
technical measures to obstruct or control the reading or further
copying of the copies you make or distribute.  However, you may accept
compensation in exchange for copies.  If you distribute a large enough
number of copies you must also follow the conditions in Section 3.

You may also lend copies, under the same conditions stated above, and
you may publicly display copies.


\section{Copying in Quantity}

If you publish printed copies of the Document numbering more than 100,
and the Document's license notice requires Cover Texts, you must enclose
the copies in covers that carry, clearly and legibly, all these Cover
Texts: Front-Cover Texts on the front cover, and Back-Cover Texts on
the back cover.  Both covers must also clearly and legibly identify
you as the publisher of these copies.  The front cover must present
the full title with all words of the title equally prominent and
visible.  You may add other material on the covers in addition.
Copying with changes limited to the covers, as long as they preserve
the title of the Document and satisfy these conditions, can be treated
as verbatim copying in other respects.

If the required texts for either cover are too voluminous to fit
legibly, you should put the first ones listed (as many as fit
reasonably) on the actual cover, and continue the rest onto adjacent
pages.

If you publish or distribute Opaque copies of the Document numbering
more than 100, you must either include a machine-readable Transparent
copy along with each Opaque copy, or state in or with each Opaque copy
a publicly accessible computer-network location containing a complete
Transparent copy of the Document, free of added material, which the
general network-using public has access to download anonymously at no
charge using public-standard network protocols.  If you use the latter
option, you must take reasonably prudent steps, when you begin
distribution of Opaque copies in quantity, to ensure that this
Transparent copy will remain thus accessible at the stated location
until at least one year after the last time you distribute an Opaque
copy (directly or through your agents or retailers) of that edition to
the public.

It is requested, but not required, that you contact the authors of the
Document well before redistributing any large number of copies, to give
them a chance to provide you with an updated version of the Document.


\section{Modifications}

You may copy and distribute a Modified Version of the Document under
the conditions of Sections 2 and 3 above, provided that you release
the Modified Version under precisely this License, with the Modified
Version filling the role of the Document, thus licensing distribution
and modification of the Modified Version to whoever possesses a copy
of it.  In addition, you must do these things in the Modified Version:

\begin{itemize}

\item Use in the Title Page (and on the covers, if any) a title distinct
    from that of the Document, and from those of previous versions
    (which should, if there were any, be listed in the History section
    of the Document).  You may use the same title as a previous version
    if the original publisher of that version gives permission.
\item List on the Title Page, as authors, one or more persons or entities
    responsible for authorship of the modifications in the Modified
    Version, together with at least five of the principal authors of the
    Document (all of its principal authors, if it has less than five).
\item State on the Title page the name of the publisher of the
    Modified Version, as the publisher.
\item Preserve all the copyright notices of the Document.
\item Add an appropriate copyright notice for your modifications
    adjacent to the other copyright notices.
\item Include, immediately after the copyright notices, a license notice
    giving the public permission to use the Modified Version under the
    terms of this License, in the form shown in the Addendum below.
\item Preserve in that license notice the full lists of Invariant Sections
    and required Cover Texts given in the Document's license notice.
\item Include an unaltered copy of this License.
\item Preserve the section entitled ``History,'' and its title, and add to
    it an item stating at least the title, year, new authors, and
    publisher of the Modified Version as given on the Title Page.  If
    there is no section entitled ``History'' in the Document, create one
    stating the title, year, authors, and publisher of the Document as
    given on its Title Page, then add an item describing the Modified
    Version as stated in the previous sentence.
\item Preserve the network location, if any, given in the Document for
    public access to a Transparent copy of the Document, and likewise
    the network locations given in the Document for previous versions
    it was based on.  These may be placed in the ``History'' section.
    You may omit a network location for a work that was published at
    least four years before the Document itself, or if the original
    publisher of the version it refers to gives permission.
\item In any section entitled ``Acknowledgements'' or ``Dedications,''
    preserve the section's title, and preserve in the section all the
    substance and tone of each of the contributor acknowledgements
    and/or dedications given therein.
\item Preserve all the Invariant Sections of the Document,
    unaltered in their text and in their titles.  Section numbers
    or the equivalent are not considered part of the section titles.
\item Delete any section entitled ``Endorsements.''  Such a section
    may not be included in the Modified Version.
\item Do not retitle any existing section as ``Endorsements''
    or to conflict in title with any Invariant Section.

\end{itemize}

If the Modified Version includes new front-matter sections or
appendices that qualify as Secondary Sections and contain no material
copied from the Document, you may at your option designate some or all
of these sections as invariant.  To do this, add their titles to the
list of Invariant Sections in the Modified Version's license notice.
These titles must be distinct from any other section titles.

You may add a section entitled ``Endorsements,'' provided it contains
nothing but endorsements of your Modified Version by various
parties---for example, statements of peer review or that the text has
been approved by an organization as the authoritative definition of a
standard.

You may add a passage of up to five words as a Front-Cover Text, and a
passage of up to 25 words as a Back-Cover Text, to the end of the list
of Cover Texts in the Modified Version.  Only one passage of
Front-Cover Text and one of Back-Cover Text may be added by (or
through arrangements made by) any one entity.  If the Document already
includes a cover text for the same cover, previously added by you or
by arrangement made by the same entity you are acting on behalf of,
you may not add another; but you may replace the old one, on explicit
permission from the previous publisher that added the old one.

The author(s) and publisher(s) of the Document do not by this License
give permission to use their names for publicity for or to assert or
imply endorsement of any Modified Version.


\section{Combining Documents}

You may combine the Document with other documents released under this
License, under the terms defined in Section 4 above for modified
versions, provided that you include in the combination all of the
Invariant Sections of all of the original documents, unmodified, and
list them all as Invariant Sections of your combined work in its
license notice.

The combined work need only contain one copy of this License, and
multiple identical Invariant Sections may be replaced with a single
copy.  If there are multiple Invariant Sections with the same name but
different contents, make the title of each such section unique by
adding at the end of it, in parentheses, the name of the original
author or publisher of that section if known, or else a unique number.
Make the same adjustment to the section titles in the list of
Invariant Sections in the license notice of the combined work.

In the combination, you must combine any sections entitled ``History''
in the various original documents, forming one section entitled
``History''; likewise combine any sections entitled ``Acknowledgements,''
and any sections entitled ``Dedications.'' You must delete all sections
entitled ``Endorsements.''


\section{Collections of Documents}

You may make a collection consisting of the Document and other documents
released under this License, and replace the individual copies of this
License in the various documents with a single copy that is included in
the collection, provided that you follow the rules of this License for
verbatim copying of each of the documents in all other respects.

You may extract a single document from such a collection, and distribute
it individually under this License, provided you insert a copy of this
License into the extracted document, and follow this License in all
other respects regarding verbatim copying of that document.



\section{Aggregation with Independent Works}

A compilation of the Document or its derivatives with other separate
and independent documents or works, in or on a volume of a storage or
distribution medium, does not as a whole count as a Modified Version
of the Document, provided no compilation copyright is claimed for the
compilation.  Such a compilation is called an ``aggregate,'' and this
License does not apply to the other self-contained works thus compiled
with the Document, on account of their being thus compiled, if they
are not themselves derivative works of the Document.

If the Cover Text requirement of Section 3 is applicable to these
copies of the Document, then if the Document is less than one quarter
of the entire aggregate, the Document's Cover Texts may be placed on
covers that surround only the Document within the aggregate.
Otherwise they must appear on covers around the whole aggregate.


\section{Translation}

Translation is considered a kind of modification, so you may
distribute translations of the Document under the terms of Section 4.
Replacing Invariant Sections with translations requires special
permission from their copyright holders, but you may include
translations of some or all Invariant Sections in addition to the
original versions of these Invariant Sections.  You may include a
translation of this License provided that you also include the
original English version of this License.  In case of a disagreement
between the translation and the original English version of this
License, the original English version will prevail.


\section{Termination}

You may not copy, modify, sublicense, or distribute the Document except
as expressly provided for under this License.  Any other attempt to
copy, modify, sublicense, or distribute the Document is void, and will
automatically terminate your rights under this License.  However,
parties who have received copies, or rights, from you under this
License will not have their licenses terminated so long as such
parties remain in full compliance.


\section{Future Revisions of This License}

The Free Software Foundation may publish new, revised versions
of the GNU Free Documentation License from time to time.  Such new
versions will be similar in spirit to the present version, but may
differ in detail to address new problems or concerns. See
http:///www.gnu.org/copyleft/.

Each version of the License is given a distinguishing version number.
If the Document specifies that a particular numbered version of this
License "or any later version" applies to it, you have the option of
following the terms and conditions either of that specified version or
of any later version that has been published (not as a draft) by the
Free Software Foundation.  If the Document does not specify a version
number of this License, you may choose any version ever published (not
as a draft) by the Free Software Foundation.

\section{Addendum: How to Use This License for Your Documents}

To use this License in a document you have written, include a copy of
the License in the document and put the following copyright and
license notices just after the title page:

\begin{quote}

       Copyright $\copyright$  YEAR  YOUR NAME.
       Permission is granted to copy, distribute and/or modify this document
       under the terms of the GNU Free Documentation License, Version 1.1
       or any later version published by the Free Software Foundation;
       with the Invariant Sections being LIST THEIR TITLES, with the
       Front-Cover Texts being LIST, and with the Back-Cover Texts being LIST.
       A copy of the license is included in the section entitled ``GNU
       Free Documentation License.''

\end{quote}

If you have no Invariant Sections, write ``with no Invariant Sections''
instead of saying which ones are invariant.  If you have no
Front-Cover Texts, write ``no Front-Cover Texts'' instead of
``Front-Cover Texts being LIST''; likewise for Back-Cover Texts.

If your document contains nontrivial examples of program code, we
recommend releasing these examples in parallel under your choice of
free software license, such as the GNU General Public License,
to permit their use in free software.
%\clearemptydoublepage

\printindex

\thispagestyle{empty} \blankpage
\thispagestyle{empty} \blankpage

\end{document}
